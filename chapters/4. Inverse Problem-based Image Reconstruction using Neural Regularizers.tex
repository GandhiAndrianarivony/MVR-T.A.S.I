%===========================================================
% Chapitre 4 — Méthodes basées sur l'apprentissage
%===========================================================
\chapter{Méthodes basées sur l'apprentissage}

\section{Fondements théoriques du deep learning appliqué à la reconstruction}
\subsection{Apprentissage de régularisation}
\subsection{Approches data-driven vs physics-based}
\subsection{Architectures CNN, U-Net, ResNet}

\section{Approches supervisées}
\subsection{Reconstruction directe image-à-image}
\subsection{Modèles guidés par sinogrammes (CT)}
\subsection{Méthodes itératives apprises}
\subsubsection{Networks unrollés}
\subsubsection{ISTA-Net}
\subsubsection{ADMM-Net}

\section{Approches non supervisées et auto-supervisées}
\subsection{Modèles génératifs (GAN, VAE)}
\subsection{Méthodes Noise2Noise, Noise2Void}
\subsection{Auto-supervision pour MRI sous-échantillonnée}

\section{Régularisation neuronale et méthodes hybrides}
\subsection{Deep Image Prior}
\subsection{Plug-and-Play (PnP)}
\subsection{Regularization by Denoising (RED)}
\subsection{Physics-Informed Neural Networks (PINNs)}

\section{Applications}
\subsection{Tomographie (CT)}
\subsubsection{Low-dose CT}
\subsubsection{Sparse-view CT}
\subsection{IRM (MRI)}
\subsubsection{Reconstruction accélérée}
\subsubsection{Super-résolution}
\subsection{OCT, microscopie, hyperspectral}

\section{Analyse expérimentale}
\subsection{Métriques (PSNR, SSIM, NMSE, FID)}
\subsection{Comparaison avec l'état de l'art}
\subsection{Analyse de robustesse}
