\chapter{OUTILS MATHÉMATIQUES UTILISÉS DANS LA RECONSTRUCTION D'IMAGES CT PAR LE COMPRESSED SENSING}

Un sujet très précis et intéressant ! La reconstruction d'images à partir de projections est un aspect crucial de l'imagerie médicale, en particulier dans des modalités telles que la tomodensitométrie (CT) et la tomographie par émission de positons (TEP).
Le scanner de tomodensitométrie (CT) diagnostique, voir la \Cref{fig :tomography_device}, est un véritable
chef-d'œuvre de la technologie moderne et constitue un exemple positif de l'influence des forces
du marché libre dans la stimulation de l'innovation. Tous les principaux fabricants de scanners
CT disposent d'équipes solides de recherche et développement, qui suivent et contribuent aux
travaux de recherche en science de l'imagerie en plus de leurs activités internes.

Les dispositifs CT peuvent être considérés comme la réalisation d'une caméra gigapixel en usage
clinique courant. Alors qu'un scanner CT diagnostique moderne typique fournit des volumes
composés de centaines d'images de coupes de taille $512 \times 512$ avec une résolution
submillimétrique, les scanners micro-CT peuvent produire des volumes atteignant
$2000$ voxels.

Dans le cas de l'imagerie cardiaque par CT, une application qui a largement stimulé les avancées
technologiques, les images volumiques peuvent être acquises avec une résolution temporelle
pouvant descendre jusqu'à 100 millisecondes. La vitesse et la résolution de l'imagerie CT en font
un outil indispensable pour l'imagerie cardiaque et l'évaluation des accidents vasculaires
cérébraux.

Elle est utilisée de manière routinière pour le diagnostic de diverses pathologies médicales
affectant l'ensemble des organes internes, et son utilisation est même envisagée comme outil de
dépistage du cancer du poumon. Les recherches actuelles visent globalement à améliorer
l'utilité clinique sans augmenter la dose d'irradiation de l'examen, ou à maintenir cette utilité
tout en réduisant l'exposition du patient aux rayonnements ionisants.

Les méthodes analytiques directes sont l'approche historique et mathématiquement élégante des problèmes inverses linéaires \cite{14}, particulièrement en tomographie. Leur positionnement répond à un impératif de rapidité de calcul dans des applications où le temps de reconstruction est critique (imagerie médicale clinique, contrôle non destructif industriel).

\section{Classification des approches de reconstruction}
Le paysage algorithmique de la tomographie se divise principalement en trois familles, distinguées par leur traitement de l'opérateur de projection.

\subsection{Méthodes analytiques}
Les méthodes analytiques de reconstruction reposent sur une formulation mathématique explicite
de l'inversion de l'opérateur direct reliant l'objet à ses projections. En tomodensitométrie (TDM),
cet opérateur est étroitement lié à la transformée de Radon. En exploitant ses propriétés, ces
méthodes permettent une reconstruction directe et rapide de l'image à partir des données de
projection.

L'exemple le plus emblématique est la Rétroprojection Filtrée (\emph{Filtered Backprojection},
FBP). Cette méthode analytique largement utilisée consiste à filtrer les projections avant de
les rétroprojeter sur la grille de l'image. La FBP est rapide et efficace, mais elle demeure
sensible au bruit et aux artefacts, en particulier lorsque le nombre de projections est limité
ou que la dose d'irradiation est réduite.


\subsubsection{Fondements de la tomographie assistée par ordinateur (CT)}
Imaginons qu'on ait un objet opaque constitué de différents matériaux, et que l'on souhaite savoir comment ces matériaux sont répartis à l'intérieur sans l'endommager (par exemple, l'objet peut être un malade à l'intérieur du corps duquel on aimerait voir). L'une des méthodes est le scanner  : on lance de fins faisceaux de rayons X à travers l'objet dans toutes les directions et on mesure quelle proportion de chaque faisceau a été absorbée.

La tomographie est un procédé permettant de créer l'image d'un objet en deux ou trois dimensions à partir de multiples "coupes" unidimensionnelles (Voir \Cref{fig :tomography_device}). Dans un scanner CT, ces coupes sont définies par des faisceaux de rayons $\mathbf{X}$ parallèles projetés à travers l'objet. En changeant l'orientation de la source et du détecteur (l'angle \(\theta\)), on obtient des informations sur la densité interne sous différents angles.\\
Le fonctionnement repose sur la mesure de l'intensité des rayons $\mathbf{X}$ :
\begin{itemize}
    \item[-] \textbf{Perte d'énergie}  : Lorsqu'un rayon $\mathbf{X}$ traverse un objet, il perd une partie de son énergie, ce qui réduit son intensité
    \item[-] \textbf{Coefficient d'atténuation}  : Cette perte dépend de la densité du milieu. Les objets denses (comme l'os) provoquent une variation d'intensité plus importante que les tissus moins denses. Cette caractéristique est appelée le coefficient d'atténuation. L'atténuation mesurée pour chaque faisceau, c'est-à-dire la différence entre l'intensité incidente et l'intensité détectée, correspond à une intégrale de ligne de la structure interne de l'objet. Cette relation entre l'objet et l'ensemble de ses intégrales de ligne est formalisée par la transformée de Radon. La reconstruction de l'image originale repose alors sur l'inversion de cette transformée, qui constitue le fondement théorique de la tomographie assistée par ordinateur
    \item[-] \textbf{Mesures}  : Le scanner enregistre l'intensité initiale émise ($I_{0}$) et l'intensité finale reçue ($I_{1}$) pour chaque faisceau afin de déduire la densité globale rencontrée sur le trajet
\end{itemize}

\begin{figure}[H]
    \centering
    \includegraphics[width=0.8\textwidth]{./images/ct_device.png}
    \caption{Principe de fonctionnement d'un appareil de tomodensitométrie}
    \label{fig :tomography_device}
\end{figure}
\medskip
\noindent
D'un point de vue mathématique, l'ensemble des mesures acquises par le scanner correspond à
l'application de la transformée de Radon à la fonction d'atténuation de l'objet. Le problème
fondamental de la tomographie consiste alors à reconstruire cette fonction d'atténuation à
partir de ses intégrales de ligne, c'est-à-dire à inverser la transformée de Radon.
Pour une fonction d'atténuation bidimensionnelle $A(x,y)$, la transformée de Radon est définie
par :
\begin{equation}
\mathcal{R}\{A\}(\theta, s) =
\int_{\mathbb{R}^2} A(x,y)\,
\delta(x\cos\theta + y\sin\theta - s)\,\mathrm{d}x\,\mathrm{d}y,
\end{equation}
où $\theta$ représente l'angle de projection, $s$ la position du détecteur, et $\delta(\cdot)$
la distribution de Dirac. Cette expression formalise le fait que chaque projection correspond
à une intégrale de la fonction d'atténuation le long d'une droite.

\subsubsection{Loi de Beer--Lambert et modélisation de l'atténuation des rayons X dans l'objet}
L'objet initial, considéré comme plan, est donné par une fonction d'atténuation qui, à chaque point du plan de coordonnées $(x, y)$, va associer un nombre $A(x, y)$ correspondant à la proportion des rayons qui sont absorbés par le matériau en ce point  : en un point d'un os, $A$ sera grand, et en un point de l'air, il sera faible.\vspace{10pt}\\
% \subsubsection{Loi de Beer--Lambert et modélisation de l'atténuation}
En supposant dans un premier temps que la fonction d'atténuation de notre objet est constante égale à $a$, pour tout rayon lumineux traversant notre objet, pour tout couple de points d'abscisses $x$ et $x+l$ sur ce rayon, les abscisses étant croissantes dans le sens du rayon, le rapport d'intensités lumineuses $\cfrac{I(x+l)}{I(x)}$ ne dépend que de $a$ et de la longueur $l$ traversée et pas du point $x$ (position).

En omettant provisoirement la dépendance par rapport à $a$ et en notant alors $p(l)$ ce rapport $\cfrac{I(x+l)}{I(x)}$ qui correspond à la proportion de photons non
absorbés sur une longueur $l$ à partir d'un point $x$, on voit que $p$ vérifie la propriété 
\[
    p(l_1+l_2) = p(l_1)p(l_2)
\] \vspace{10pt}\\
En Effet, la proportion de photons non absorbés sur une longueur $l_2$ à partir d'un point $x+l_1$ est $\cfrac{I(x+l_1+l_2)}{I(x+l_1)}=p(l_2)$. Donc $p(l_1)p(l_2)=\cfrac{I(x+l_1)}{I(x)} \times \cfrac{I(x+l_1+l_2)}{I(x+l_1)}=p(l_1+l_2)$. Ceci traduit juste le fait simple suivant  : les proportions de photons non
absorbés se multiplient lors de traversées successives de milieux absorbants. La bonne définition de l'atténuation est précisément : 
\begin{equation}
    p(l)=e^{-a\, l}
    \label{eq:init_loi_beer_lambert_attenuation}
\end{equation}\vspace{10pt}\\
Autrement dit, pour tout $x$ et $x+l$ sur un axe  : 
\begin{equation}
    I(x+l)=I(x)e^{-a\, l}
    \label{eq:init_loi_beer_lambert}
\end{equation}\vspace{10pt}\\
Notons que si le phénomène physique d'atténuation est spécifique de la tomographie à rayons X, les méthodes de reconstruction sont en revanche plus générales et sont appliquées également dans d'autres systèmes d'imagerie, dans lesquelles des équations analogues expriment une fonction à reconstruire en fonction de projections. C'est le cas par exemple de la TEP utilisée en médecine nucléaire.

\subsubsection{Reconstruction d'images en tomodensitométrie : inversion de Fourier, rétroprojection et FBP}
À partir de la modélisation physique de l'atténuation et de la formulation mathématique de la
transformée de Radon, plusieurs méthodes analytiques ont été proposées pour résoudre le
problème inverse de la reconstruction tomographique.
La reconstruction d'images en tomodensitométrie à partir des projections aux rayons X repose
classiquement sur trois grandes approches analytiques.\vspace{5pt} \\
La première est l'inversion directe de Fourier, fondée sur le théorème de la coupe de Fourier, selon lequel chaque projection acquise à
un angle donné correspond à une droite dans l'espace fréquentiel bidimensionnel de l'objet.
En théorie, l'acquisition d'un nombre suffisant de projections permet de remplir cet espace
fréquentiel et d'obtenir l'image par transformation de Fourier inverse. Toutefois, la nécessité
de rééchantillonner des données organisées sur une grille polaire vers une grille cartésienne
rend cette méthode complexe et peu utilisée en pratique. \vspace{5pt} \\
La seconde approche consiste à appliquer une rétroprojection directe des projections, ce qui conduit à une image floue,
équivalente à la convolution de l'image originale avec un noyau de type
$1/\sqrt{x^2+y^2}$. La restauration de l'image nécessite alors une déconvolution bidimensionnelle,
opération coûteuse en temps de calcul. \vspace{5pt} \\ 
La troisième méthode, la FBP, constitue l'approche la plus répandue en pratique. Elle repose sur le
filtrage préalable de chaque projection unidimensionnelle par un filtre rampe dans le domaine
fréquentiel, avant la rétroprojection. Ce traitement permet de compenser le flou inhérent à la
rétroprojection (\cite{9}) tout en conservant une complexité de calcul réduite, puisque le filtrage est
effectué en une dimension. Ces méthodes analytiques supposent néanmoins la disponibilité d'un
grand nombre de projections uniformément réparties, hypothèse qui n'est plus toujours valide
dans des contextes modernes tels que l'imagerie à faible dose ou à angles limités, où des
approches itératives basées sur le \emph{Compressed Sensing} sont alors privilégiées.

\subsection{Méthodes itératives et Compressed Sensing}

La réduction des données de projection et de la dose de rayonnement en tomodensitométrie n'est
pas simplement une préférence de calcul ; elle est dictée par la sécurité clinique, l'efficacité
opérationnelle et les contraintes physiques des systèmes d'imagerie par rayons \(\mathbf{X}\). 
Cette approche se justifie rigoureusement des points de vue médical, physique et systémique.

En premier lieu, la sécurité des patients est primordiale : les rayons \(\mathbf{X}\) ionisants
utilisés en CT peuvent endommager l'ADN et augmenter le risque de cancer radio-induit,
en particulier chez les populations pédiatriques ou lors d'expositions répétées. Le modèle
linéaire sans seuil (LNT), pierre angulaire de la radioprotection, postule que toute
réduction de dose diminue proportionnellement le risque à long terme. En pratique clinique
moderne, où les examens CT se multiplient (dépistage, suivi longitudinal, planification
radiothérapeutique), la réduction de la dose par acquisition est essentielle pour limiter
l'exposition cumulative.

D'un point de vue physique, la dose est approximativement proportionnelle au nombre de
projections et au produit courant-temps du tube. Réduire l'un ou l'autre diminue donc
directement l'exposition, mais augmente le bruit quantique des données, rendant la
reconstruction plus difficile. Opérationnellement, un plus petit nombre de projections
raccourcit la durée d'acquisition, réduisant les artefacts de mouvement et améliorant le
débit patient. Pour les populations sensibles (enfants) ou dans des contextes d'imagerie
répétée (radiothérapie guidée par l'image), cette réduction devient une obligation clinique
et éthique, conforme au principe ALARA (As Low As Reasonably Achievable). Le défi technique
réside dans la résolution du problème de reconstruction sous-déterminé qui en résulte,
nécessitant des méthodes avancées comme la reconstruction itérative ou le Compressed Sensing \cite{12}
pour préserver la qualité diagnostique malgré la réduction des données.

\subsubsection{Formulation générale du problème inverse}
Dans ce contexte, la reconstruction tomographique peut être formulée comme un
\textit{problème inverse régularisé}. De manière générale, un problème inverse \cite{10} consiste
à reconstruire un modèle à partir de mesures indirectes issues d'un processus physique
connu. Le \textit{problème direct} décrit la formation des données de projection à partir
de l'image inconnue, tandis que le \textit{problème inverse} vise à retrouver cette image à
partir des mesures acquises.\vspace{5pt}\\
En tomodensitométrie discrète, la relation entre l'image à reconstruire et les données
de projection peut s'écrire sous la forme d'un système linéaire :
\begin{equation}
    \mathbf{g} = \mathbf{H}\mathbf{f} + \boldsymbol{\varepsilon},
\end{equation}
où $\mathbf{f}$ représente l'image vectorisée, $\mathbf{g}$ les données de projection
(sinogramme), $\mathbf{H}$ la matrice du système modélisant la transformée de Radon discrète \cite{11},
et $\boldsymbol{\varepsilon}$ un terme représentant le bruit de mesure.

Dans un cadre idéal, l'opérateur $\mathbf{H}$ serait parfaitement inversible et les données
seraient complètes et exemptes de bruit. En pratique, ces conditions ne sont jamais réunies.
Le problème inverse est alors généralement \textit{mal posé au sens de Hadamard}, en raison
d'un manque d'unicité et surtout d'un fort manque de stabilité : de faibles perturbations du
bruit peuvent engendrer de grandes erreurs dans l'image reconstruite. Les méthodes itératives
abordent ce problème en recherchant une solution régularisée, obtenue par la minimisation
d'une fonction de coût combinant un terme d'adéquation aux données et un terme de
régularisation incorporant des connaissances \emph{a priori} sur l'image.

\subsubsection{Principe du Compressed Sensing}
Le Compressed Sensing fournit un cadre mathématique rigoureux pour la reconstruction de
signaux parcimonieux ou compressibles à partir de mesures linéaires sous-échantillonnées,
sous certaines conditions sur l'opérateur de mesure. Il permet de résoudre des problèmes
inverses sous-déterminés au moyen de méthodes d'optimisation favorisant la parcimonie ou
d'algorithmes gloutons, avec des garanties théoriques de stabilité et de robustesse au bruit.

Dans le cadre de l'imagerie tomographique, le Compressed Sensing exploite l'idée que de
nombreuses images médicales admettent une représentation parcimonieuse dans une base ou un
dictionnaire approprié, tel que les ondelettes ou le gradient de l'image. La reconstruction
est alors formulée comme un problème d'optimisation sous contrainte :
\begin{equation}
    \min_{\mathbf{f}} \; \|\mathbf{\Psi}\mathbf{f}\|_1
    \quad \text{sous la contrainte} \quad
    \|\mathbf{H}\mathbf{f} - \mathbf{g}\|_2 \leq \delta,
\end{equation}
où $\mathbf{\Psi}$ désigne un opérateur favorisant la parcimonie et $\delta$ un paramètre
lié au niveau de bruit des données.

Cette formulation permet de reconstruire des images de qualité acceptable à partir d'un
nombre de projections bien inférieur à celui requis par les méthodes analytiques classiques,
tout en assurant une certaine stabilité du problème inverse.

\subsubsection{Régularisation et variation totale}
Une régularisation particulièrement adaptée à l'imagerie tomographique est la variation
totale (TV) \cite{13}, qui favorise les images composées de régions quasi uniformes séparées par des
discontinuités nettes. La reconstruction TV est généralement formulée comme :
\begin{equation}
    \min_{\mathbf{f}} \;
    \frac{1}{2}\|\mathbf{H}\mathbf{f} - \mathbf{g}\|_2^2
    + \lambda \|\nabla \mathbf{f}\|_1,
\end{equation}
où $\nabla$ représente l'opérateur gradient discret et $\lambda$ un paramètre de
régularisation contrôlant le compromis entre fidélité aux données et stabilisation du
problème inverse.

Cette approche permet de réduire efficacement le bruit tout en préservant les contours,
qualité essentielle pour l'analyse diagnostique.

\subsubsection{Algorithmes itératifs}
La résolution des problèmes d'optimisation issus du Compressed Sensing et des méthodes
variationnelles repose sur des algorithmes itératifs, tels que les méthodes de descente de
gradient proximal, l'algorithme ISTA/FISTA, l'ADMM ou encore les méthodes de type primal-dual.
Ces algorithmes procèdent par mises à jour successives de l'image estimée, alternant entre la
réduction de l'erreur de projection et l'application de la régularisation.

Bien que plus coûteuses en temps de calcul que les méthodes analytiques, les méthodes
itératives offrent une qualité de reconstruction supérieure dans les scénarios de données
limitées et constituent aujourd'hui un axe majeur de recherche et de développement en
tomodensitométrie moderne.

\subsubsection{Comparaison avec les méthodes analytiques}
En résumé, les méthodes analytiques privilégient la rapidité et la simplicité au prix d'une
sensibilité accrue au bruit et aux artefacts, tandis que les méthodes itératives et basées sur
le Compressed Sensing s'inscrivent pleinement dans le cadre des problèmes inverses
régularisés. Elles exploitent des informations \emph{a priori} et des contraintes de parcimonie
pour améliorer la stabilité et la qualité de reconstruction dans des conditions d'acquisition
dégradées, répondant ainsi aux exigences de sécurité clinique et d'efficacité opérationnelle.


% ========================================================================================================
% TODO -> Phrase accrochage; mise en page; etc
% ========================================================================================================
% \section{Les traitements préalables à la reconstruction}

% \subsection{ Méthodes dans le domaine spatial}
% % =================== TODO ==================
% \subsection{ Filtrage linéaire}
% % =========================================


% \subsection{ Méthodes dans le domaine transformé}
\section{ Transformée de Fourier}
La transformée de Fourier joue un rôle fondamental dans la reconstruction d'images en tomodensitométrie (CT), en permettant une analyse fréquentielle de l'image et des projections. Cette décomposition facilite la distinction entre les structures globales, associées aux basses fréquences, et les détails fins, portés par les hautes fréquences, contribuant ainsi à une reconstruction plus fidèle. Son utilisation dans des algorithmes analytiques tels que la FBP permet de formuler le problème dans le domaine fréquentiel, où les opérations de filtrage sont plus simples et plus efficaces à mettre en œuvre numériquement. En particulier, l'application d'un filtre rampe aux projections unidimensionnelles permet de compenser le flou inhérent à la rétroprojection directe et de réduire des artefacts caractéristiques, tels que l'effet d'étoile. En pratique, les données d'atténuation, organisées sous forme de sinogramme, sont transformées dans le domaine fréquentiel afin de contrôler précisément leur contenu spectral avant la reconstruction, ce qui conduit à une amélioration globale de la qualité de l'image reconstruite.
\begin{definition}[Transformée de Fourier]
    Soit \( f \) une fonction absolument intégrable sur \( \mathbb{R} \).
    La transformée de Fourier de \( f \), notée \( \mathcal{F}f \), est définie
    pour tout nombre réel \( \xi \) par
    \[
    (\mathcal{F}f)(\xi)
    = \int_{-\infty}^{\infty} f(x)\, e^{-2\pi i \xi x}\, dx.
    \]
\end{definition}
La transformée de Fourier est fréquemment utilisée en analyse du signal et permet de transformer une fonction du temps en une fonction de la fréquence ; la variable $x$ représente le temps en secondes et la variable \( \xi \) représente la fréquence de la fonction en hertz. En fait, les transformées de Fourier indiquent des informations sur le signal. Elles montrent les détails concernant la composante de fréquence qui apparaît ou est présente
dans le signal. Elles ne donnent pas de détails sur la valeur exacte de la fréquence présente
à cet instant précis.\\

Il existe une définition alternative faisant intervenir la fréquence angulaire $w=2\pi \xi$, ce qui conduit à l'expression suivante.
\[(\mathcal{F}f)(w) = \int_{-\infty}^{\infty} f(x)\, e^{-i w x}\, dx\]
Comme pour la transformée de Radon, nous allons énumérer plusieurs propriétés de la transformée de Fourier.
\begin{proposition}
    Pour des constantes réelles $\alpha$ et $\beta$, et des fonctions absolument intégrables $f$ et $g$, on a:
    \begin{itemize}
        \item[(i)] Linéarité : $\mathcal{F}(\alpha f + \beta g)(w) = \alpha \mathcal{F}f(w) + \beta \mathcal{F}g(w)$
        \item[(ii)] $\mathcal{F}f(w) < +\infty$
    \end{itemize}
\end{proposition}

\begin{definition}[Transformée de Fourier inverse]
Soit \( f \) une fonction absolument intégrable.
La transformée de Fourier inverse de \( f \), notée \( \mathcal{F}^{-1}f \),
évaluée en \( x \), est définie par
\begin{equation}
    (\mathcal{F}^{-1}f)(x)
    = \cfrac{1}{2\pi}\int_{-\infty}^{\infty} f(w)\, e^{iw x}\, dw.
    \label{formula:fourier_inverse}
\end{equation}
\end{definition}
Ceci nous conduit immédiatement au théorème suivant.
\begin{proposition}[Théorème d'inversion de Fourier]
Soit $f$ une fonction absolument integrale sur $\mathbb{R}$.
Le théorème d'inversion de Fourier affirme que, pour tout \( x \),
\[
(\mathcal{F}^{-1} \circ \mathcal{F})f(x)=f(x)
\]
\end{proposition}
Jusqu'à présent, nous n'avons abordé la transformée de Fourier que dans une dimension. Il existe des définitions correspondantes en dimensions supérieures, mais, pour nos besoins, nous n'utiliserons que les analogues en deux dimensions.

\begin{definition}[Transformée de Fourier bidimensionnelle]
Soit \( g \) une fonction absolument intégrable définie sur \( \mathbb{R}^2 \).
La transformée de Fourier bidimensionnelle de \( g \), notée \( \mathcal{F}_2 g \),
est définie pour tout \((X,Y) \in \mathbb{R}^2\) par
\begin{equation}
    (\mathcal{F}_2 g)(X,Y) = \int_{-\infty}^{\infty} \int_{-\infty}^{\infty} 
    g(x,y)\, e^{-i (xX + yY)} \, dx\, dy.
    \label{eq:fourier_2d}
\end{equation}

\end{definition}

De manière similaire, nous définissons la transformée de Fourier inverse sur $\mathbb{R}^2$.
\begin{definition}[Transformée de Fourier bidimensionnelle inverse]
Soit \( g \) une fonction absolument intégrable définie sur \( \mathbb{R}^2 \).
La transformée de Fourier bidimensionnelle inverse de \( g \), évaluée en \((x,y)\)
et notée \( \mathcal{F}_2^{-1} g(x,y) \), est donnée par
\[
(\mathcal{F}_2^{-1} g)(x,y) = \cfrac{1}{4\pi^2}\int_{-\infty}^{\infty} \int_{-\infty}^{\infty} 
g(X,Y)\, e^{i (xX + yY)} \, dX\, dY.
\]
\end{definition}

% ================== TODO ==================
% \subsection{ Filtre de Wiener}
% 1. Filtre de Wiener
% ✅ Oui, tout à fait applicable
% Le filtre de Wiener est un filtre linéaire adaptatif.

% Il peut être appliqué :
%    - sur chaque projection (variable s)
%    - ou localement sur le sinogramme

% Intérêt :
%    - réduction du bruit additif (souvent gaussien)
%    - compromis bruit / flou optimal au sens MSE

% ⚠️ Limites :
%    - nécessite une estimation du bruit et du spectre du signal
%    - un mauvais modèle dégrade la reconstruction

% 📌 Très utilisé comme prétraitement des sinogrammes en CT à faible dose.

% \subsection{ Curvelets}
% 4. Curvelets
% ✅ Oui, très pertinent
% Les curvelets sont théoriquement bien adaptées :
%    - excellente représentation des singularités le long de courbes

% Les lignes du sinogramme correspondent à :
%    - des courbes liées aux bords de l'objet

% 📌 Très utilisé dans :
%    - CT basse dose
%    - Les méthodes variationnelles et itératives
% =================================================
\subsection{Transformée de Fourier à Court Terme (STFT)}

La STFT est plus avantageuse que la transformée de Fourier dans le sens où elle introduit la fenêtre glissante. En fait, la fenêtre sert à extraire une petite partie du signal donné. Mathématiquement, elle est représentée par :

\begin{equation}
    S f(u,\xi) = \int_{-\infty}^{\infty}f(t)\omega (t - u)exp(-j\xi t)dt \quad (3)
    \label{eq:stft}
\end{equation}



\section{Convolution}
La convolution joue un rôle clé dans la reconstruction d'image par rétroprojection filtrée (FBP), car elle intervient directement dans l'étape de filtrage des projections. Avant la rétroprojection, chaque projection mesurée est convoluée avec un filtre adapté afin de compenser le flou intrinsèque introduit par la rétroprojection simple. Cette opération permet de renforcer les hautes fréquences et d'améliorer la résolution de l'image reconstruite.

\textbf{Définition 8.1.}
Pour deux fonctions intégrables $f$ et $g$ définies sur $\mathbb{R}$,
nous définissons la convolution de $f$ et $g$, notée $f \star g$, par
\[
(f \star g)(x) = \int_{-\infty}^{\infty} f(t)\,g(x - t)\,dt,
\]
où $x \in \mathbb{R}$.

Nous pouvons facilement étendre cette définition à l'espace
bidimensionnel. Pour les fonctions polaires, nous prenons uniquement
l'intégrale par rapport à la variable radiale, tandis que pour les
fonctions cartésiennes nous intégrons par rapport aux deux variables.
Les définitions explicites sont données ci-dessous.

\begin{definition}
    Pour des fonctions polaires intégrables $f(t,\theta)$ et $g(t,\theta)$,
    nous définissons la convolution de $f$ et $g$ par
    \[
        (f \star g)(t,\theta)
        =
        \int_{-\infty}^{\infty}
        f(s,\theta)\,g(t - s,\theta)\,ds.
    \]
\end{definition}

Pour des fonctions intégrables $F$ et $G$ sur $\mathbb{R}^2$,
nous définissons la convolution de $F$ et $G$ par
\[
    (F \star G)(x,y)
    =
    \int_{-\infty}^{\infty}
    \int_{-\infty}^{\infty}
    F(s,t)\,G(x - s, y - t)\,ds\,dt.
\]

La convolution est une méthode mathématique permettant de moyenner
une fonction $f$ à l'aide du déplacement d'une autre fonction $g$.
Dans la convolution $f \star g$, la fonction $g$ est translatée à travers
la fonction $f$, et la fonction résultante dépend de la zone de recouvrement
au cours de cette translation.
En un certain sens, on peut voir $g$ comme un filtre utilisé pour effectuer
une moyenne de $f$ sur un intervalle donné.
La fonction de filtrage agit ainsi comme un lisseur pour les données bruitées
fournies par la fonction originale.

\begin{proposition}
    Pour des fonctions intégrables $f$, $g$, $h$ définies sur $\mathbb{R}$
    et des constantes $\alpha, \beta \in \mathbb{R}$ :
    
    \begin{itemize}
      \item[(i)] $f \star g = g \star f$ \quad (commutativité),
      \item[(ii)] $f \star (\alpha g + \beta h)
      = \alpha (f \star g) + \beta (f \star h)$ \quad (linéarité).
      \item[(iii)] $\mathcal{F}(f). \mathcal{F}(g)  = \mathcal{F}(f \star g)$
    \end{itemize}
\end{proposition}


% =================================================
\section{Transformée de Radon}
% =================================================
L'hypothèse fondamentale est que le détecteur mesure l'atténuation intégrée le long d'un rayon. 
\begin{definition}
    Pour un faisceau de rayons $\mathbf{X}$ d'énergie $\mathbf{E}$ donnée et un taux de propagation des photons $\mathbf{N}(x)$, l'intensité du faisceau $\mathbf{I}(x)$ à une distance $x$ de l'origine est définie comme \[\mathbf{I}(x) = \mathbf{N}(x) \mathbf{E}\]
\end{definition}

\begin{definition}
    La proportion de photons absorbés par millimètre de substance à une distance $x$ de l'origine est appelée le coefficient d'atténuation $\mathbf{A}(x)$ du milieu.
\end{definition}


Nous connaissons les intensités initiale et finale, $I_0$ et $I_1$ d'un faisceau unique. L'objectif est d'utiliser ces intensités pour déterminer le coefficient d'atténuation le long du trajet du faisceau. Heureusement, la loi de Beer-Lambert établit une relation entre ces deux grandeurs.

\begin{definition}[Loi de Beer-Lambert]
Pour un faisceau de rayons X monochromatique, non réfractif et de largeur nulle,
traversant un milieu homogène sur une distance \(x\) à partir de l'origine,
l'intensité \(I(x)\) est donnée par
\begin{equation}
    I(x) = I_0 e^{-\mathbf{A}(x)x}
    \label{eq:loi_beer_lambert}
\end{equation}
\end{definition}
En l'état, cette équation ne nous est pas particulièrement utile. Elle exprime le coefficient d'atténuation en un point donné en fonction de l'intensité en ce point, alors que nous ne connaissons la valeur de l'intensité qu'en des points situés à l'extérieur de l'objet. Ce que nous cherchons réellement est une relation entre le coefficient d'atténuation à l'intérieur de l'objet et la variation de l'intensité du faisceau. Pour cela, nous allons manipuler légèrement l'équation \eqref{eq:loi_beer_lambert}.\\
En passant à  la dérivée de la loi de Beer-Lambert, nous obtenons la relation suivante :
\[
    \frac{dI}{dx} = -\mathbf{A}(x)I(x)
\]
Soit $I(x_0)=I_0$ la valeur initiale de l'intensité du faisceau et $I(x_1)=I_1$ la valeur finale de l'intensité du faisceau. En utilisant cette relation, nous obtenons la relation suivante :

\[
    -\int_{x_0}^{x_1} \mathbf{A}(x)dx = \int_{x_0}^{x_1}\cfrac{dI}{I(x)}=ln(\frac{I_1}{I_0})
\]
ou encore \vspace{10pt}
\begin{equation}
    \int_{x_0}^{x_1} \mathbf{A}(x)dx = ln(\frac{I_0}{I_1})
    \label{eq:radon_transformation}
\end{equation}

$ln(\frac{I_0}{I_1})$ désigne les données de projection, communément appelées le sinogramme, qui résultent de l'acquisition des projections tomographiques.
\medskip
\noindent
Nous sommes maintenant prêts à introduire des outils mathématiques — en particulier la transformée de Radon — qui joueront un rôle central dans la détermination du coefficient d'atténuation dans l'équation \eqref{eq:loi_beer_lambert}.

L'écriture sous forme normale d'une équation de droite joue un rôle clé dans la transformée de Radon, car elle permet une paramétrisation naturelle et complète de toutes les droites du plan, ce qui est essentiel pour la définition mathématique et le calcul pratique de cette transformation.\\ 
Cette équation sous forme normale fournit :
\begin{itemize}
    \item[(i)] Une paramétrisation unique et continue de toutes les droites du plan. La forme normale (ou forme normale de Hesse) de l'équation de la droite s'écrit : $$x\,\cos(\theta)+y\,\sin(\theta)=\rho$$ où $\rho$ est la distance par rapport à l'origine et $\theta$ est l'angle par rapport à l'axe des abscisses.
    \item[(ii)] Une interprétation géométrique claire de $\rho$ et $\theta$. Chaque droite du plan correspond  à un unique couple ($\rho,\theta$). Cette paramétrisation évite les redondances et garantit qu'on parcourt toutes les droites une et une seule fois (à une convention près).
    \item[(iii)] Une mesure naturelle sur l'espace des droites, utilisée dans les formules d'inversion.
    \item[(iv)] Un formalisme adapté au théorème de coupe, reliant transformée de Radon et transformée de Fourier 2D. 
    \item[(v)] Une mesure naturelle sur l'espace des droites, utilisée dans les formules d'inversion.
\end{itemize}\vspace{10pt}
\subsection{\small Construction de l'orientation et de la distance}
Nous connaissons tous l'idée qu'une droite \( l \) dans \( \mathbb{R}^2 \) peut être représentée par l'équation 
\[
ax + by = c
\]
où \( a, b, c \in \mathbb{R} \) et \( a^2 + b^2 \neq 0 \).\\ On peut alors écrire cette équation d'une droite sous la forme \[w_1x + w_2y = t\]
où $\mathbf{w}:=(w_1, w_2) = (\cfrac{a}{\sqrt{a^2 + b^2}}, \cfrac{b}{\sqrt{a^2 + b^2}})$ et $t=\cfrac{c}{\sqrt{a^2 + b^2}}$, que nous pouvons voir comme un point situé sur le
cercle unitaire, pour \[\left(\cfrac{a}{\sqrt{a^2 + b^2}}\right)^{2} + \left(\cfrac{b}{\sqrt{a^2 + b^2}}\right)^{2} = 1\]
Cela implique que $\mathbf{w} := (\cos(\theta), \sin(\theta)) \text{ est un vecteur normal unitaire }$, $\theta \in [0, 2\pi)$ représente l'orientation, et $t$ est exactement la distance à l'origine. On a \[x\cos(\theta) + y\sin(\theta) = t\]
Notez que dans les équations ci-dessus, $t$ et $\theta$ sont fixes et déterminent une droite spécifique \( l \) dans le plan. On peut donc dire que $t$ et $\theta$ paramètrent une droite \( l_{t,\theta} \) et que $\mathbf{z}$ détermine des points spécifiques sur cette droite \( l \). Ou encore
\[l_{t,\theta} = \{ \mathbf{z} \in \mathbb{R}^2 : \langle z, (\cos \theta, \sin \theta) \rangle = t \}.\]
\begin{figure}[H]
    \centering
    \includegraphics[width=0.8\textwidth]{./images/l_t_theta.png}
    \caption{paramètrisation d'une droite \( l_{t,\theta} \) par \( t \) et \( \theta \)}
    \label{fig:l_t_theta}
\end{figure}
On voit  que $(t\, \cos(\theta), t\, \sin(\theta))$ est un point situé sur la droite \( l_{t,\theta} \) (\Cref{fig:l_t_theta}) et $(-\sin(\theta), \cos(\theta))$ est un vecteur perpendiculaire au vecteur unitaire $\mathbf{w}$.\\ En géométrie affine élémentaire, une ligne est un point plus une direction. Par conséquent, nous pouvons décrire un point particulier $(x, y)$ sur $l_{t, \theta}$ en termes de nombre réel s comme suit :
\begin{equation}
    l_{t, \theta} = \{(t\, \cos(\theta) - s\,\sin(\theta), t\,\sin(\theta) + s\,\cos(\theta)); s\in \mathbb{R}\}
    \label{set:l_t_theta}
\end{equation}
\begin{definition}[Transformée de Radon]
Soit \( f(t,\theta) \) une fonction définie sur \( \mathbb{R}^2 \) à support compact.
La transformée de Radon de \( f \), notée \( \mathcal{R}f \), est définie pour
\( t \in \mathbb{R} \) et \( \theta \in (0, 2\pi] \) par
\[
\mathcal{R}f(t,\theta) = \int_{-\infty}^{\infty} f(x(s),y(x))\mathrm{d}s
\]
\end{definition}

La transformée de Radon permet de déterminer la densité totale d'une fonction $f$ le long d'une droite donnée $l$. Cette droite $l$ est définie par un angle $\theta$  par rapport à l'axe 
$x$ et une distance $t$ par rapport à l'origine. Comme illustré à la \Cref{fig:radon}, si l'on calcule la transformée de Radon le long de plusieurs droites à des angles différents (ici $\theta_1$ et $\theta_2$), on peut déterminer plusieurs fonctions de densité pour notre objet. Intuitivement, on peut interpréter la transformée de Radon comme une version « étalée » de notre objet initial. Supposons que la région en forme de tache représentée à la \Cref{fig:radon} soit une tache d'encre; si l'on étale cette tache le long de lignes de direction $\theta_1$, on s'attend à ce que les régions les plus larges de la tache correspondent à des zones plus étendues que les régions plus petites, ce qui correspond exactement à ce que l'on observe.
\begin{figure}[H]
    \centering
    \includegraphics[width=0.8\textwidth]{./images/radon.png}
    \caption{Transformée de Radon pour $\theta_1$ et $\theta_2$.}
    \label{fig:radon}
\end{figure}
L'intégrale $\mathcal{R}f(t,\theta)$ représente le membre gauche de l'équation \eqref{eq:radon_transformation}. Rappelons que, dans cette équation, $\mathbf{A}(x)$ est inconnue et que $\ln(\frac{I_1}{I_0})$ correspond à une information mesurée.
Autrement dit, $\ln(\frac{I_1}{I_0})$ est la transformée de Radon, et la transformée de Radon représente donc des données connues issues de la mesure.

L'objectif est maintenant de trouver une formule d'inversion de la transformée de Radon qui nous permettra de reconstruire la fonction initiale $f$ (ou, dans le contexte de l'imagerie médicale, 
$\mathbf{A}(x)$). Pour ce faire, il sera utile de rappeler plusieurs propriétés de la transformée de Radon.
\begin{proposition}
    Soit $\alpha$ et $\beta$ deux réels et $f$ et $g$ deux fonctions continues sur $\mathbb{R}^2$ à support compact. On a
    \begin{itemize}
        \item[(i)] Linéarité : $\mathcal{R}(\alpha f + \beta g) = \alpha \mathcal{R}f + \beta \mathcal{R}g$
        \item[(ii)] Parité: $\mathcal{R}f(-t,-\theta) = \mathcal{R}f(t,\theta)$
        \item[(iii)] $\mathcal{R}f(t, \theta) = \int_{-\infty}^{\infty} f(x(s), y(s))\mathrm{d}s = \int_{-\infty}^{\infty} f(t\,cos(\theta)-s\,sin(\theta), t\,sin(\theta)+s\,cos(\theta))\mathrm{d}s$
        % \item[(iv)] Invariance par rotation : \(\mathcal{R}(f \circ R_{\psi}) = \mathcal{R}f(t,\theta - \psi)\)
        % \item[(v)] Relation avec la convolution : \(\mathcal{R}(f * g) = \mathcal{R}f * \mathcal{R}g\)
    \end{itemize}
\end{proposition}
Nous définissons en outre le domaine naturel de la transformée de Radon comme l'ensemble des fonctions $f$ sur $\mathbb{R}^2$ telles que \[\int_{-\infty}^{\infty} |f(x(s), y(s))|\mathrm{d}s < \infty\]

\subsection{Théorème de la Coupe Centrale}
Le théorème de la coupe centrale, également appelé théorème de projection-transforme de Fourier ou théorème de Fourier-Slice, est un résultat fondamental en traitement d'image et en tomographie. Il établit un lien profond entre la transformée de Radon (utilisée pour décrire les projections d'un objet) et la transformée de Fourier (utilisée pour analyser les fréquences spatiales). Ce théorème constitue la pierre angulaire mathématique de la plupart des méthodes de reconstruction tomographique moderne.

\begin{proposition}
    Soit \( g \) une fonction absolument integrale sur \( \mathbb{R}^2 \).
    Le théorème de la coupe centrale affirme que, pour tout $S \in \mathbb{R}$ et $\theta \in [0,2\pi]$, on a : \[\mathcal{F}_2 g(S\cos(\theta), S\sin(\theta)) = \mathcal{F}(\mathcal{R}g)(S, \theta)\]
\end{proposition}
\textbf{Preuve}: En utilisant la définition de la transformée de Fourier bidimensionnelle \eqref{eq:fourier_2d} on obtient 
\[
    \mathcal{F}_{2}g(S\,\cos(\theta), S\,\sin(\theta)) = \int_{-\infty}^{\infty} \int_{-\infty}^{\infty} g(x, y)\, e^{-iS (x\,\cos(\theta) + y\,\sin(\theta))}\, dx\, dy
\]
Nous effectuons maintenant un changement de variables conformément au système
de coordonnées que nous avons défini à la \textit{Construction de l'orientation et de la distance}.
Rappelons que, lors de la paramétrisation de la droite $\ell_{t,\theta}$,
nous avons montré que, pour $s\in\mathbb{R}$, on peut écrire :
\[
x(s)=t\cos\theta - s\sin\theta, 
\qquad
y(s)=t\sin\theta + s\cos\theta,
\qquad
t = x\cos\theta + y\sin\theta.
\]

En examinant le déterminant du Jacobien associé à $x(s)$ et $y(s)$, on obtient :
\[
\det
\begin{pmatrix}
\dfrac{\partial x}{\partial t} & \dfrac{\partial x}{\partial s} \\[6pt]
\dfrac{\partial y}{\partial t} & \dfrac{\partial y}{\partial s}
\end{pmatrix}
= 1.
\]

Nous en déduisons que
\[
ds\,dt = dx\,dy.
\]
et donc
\[
\int_{-\infty}^{\infty} \int_{-\infty}^{\infty} g(x, y)\, e^{-iS (x\,\cos(\theta) + y\,\sin(\theta))}\, dx\, dy = \int_{-\infty}^{\infty}\int_{-\infty}^{\infty}
g(t\cos\theta - s\sin\theta,\; t\sin\theta + s\cos\theta)\,
e^{-iSt}\,ds\,dt.
\]

Comme $e^{-iSt}$ ne dépend pas de la variable $s$, nous pouvons réarranger
l'intégrale précédente de la manière suivante :
\[
\int_{-\infty}^{\infty}
\left(
\int_{-\infty}^{\infty}
g(t\cos\theta - s\sin\theta,\; t\sin\theta + s\cos\theta)\,ds
\right)
e^{-iSt}\,dt.
\]

L'intégrale intérieure est exactement la transformée de Radon de $f$,
évaluée en $(t,\theta)$, ce qui implique que l'expression précédente devient :
\[
\int_{-\infty}^{\infty}
(Rg(t,\theta))\,e^{-iSt}\,dt.
\]

Cette dernière intégrale n'est autre que la transformée de Fourier de
$Rg(S,\theta)$, ce qui conclut la démonstration.
\hfill$\square$

% \section{Inversion analytique de la transformée de Radon}
\subsection{Rétroprojection filtrée (FBP)}
Nous sommes maintenant enfin prêts à effectuer une première tentative pour retrouver la fonction de coefficient d'atténuation.
Rappelons que, d'un point de vue physique, la transformée de Radon
$\mathcal{R}f(t,\theta)$ nous donne la densité totale de l'objet $f$ le long d'une droite
$\ell_{t,\theta}$.
Nous avons déterminé cette densité en mesurant les intensités initiale et finale
d'un faisceau de rayons $\mathbf{X}$ traversant l'objet le long de cette droite.
En procédant ainsi pour plusieurs droites différentes, nous sommes capables de
reconstruire une coupe unique de l'objet initial, et en faisant varier l'angle
$\theta$ de ces rayons $\mathbf{X}$, nous pouvons définir de nombreuses coupes.

Si nous sommes capables, d'une certaine manière, de « rétroprojeter » ces
densités sur le plan, nous pourrons peut-être reconstituer l'objet initial.
Intuitivement, on peut interpréter ce processus comme le fait de prendre les
données du sinogramme et de les « déflouter » pour les ramener dans le plan.
\begin{definition}
Soit $h = h(t,\theta)$. On définit la \emph{rétroprojection},
notée $\mathcal{B}h$, en un point $(x,y)$ par :
\[
\mathcal{B}h(x,y) = \frac{1}{\pi}\int_{0}^{\pi} h(x\cos\theta + y\sin\theta,\theta)\,d\theta.
\]

En appliquant cette formule de rétroprojection à la transformée de Radon, on
obtient :
\begin{equation}
    \mathcal{B}\mathcal{R}f(x,y) = \frac{1}{\pi}\int_{0}^{\pi}
    \mathcal{R}f(x\cos\theta + y\sin\theta,\theta)\,d\theta.
    \label{eq:FBP}
\end{equation}
\end{definition}
Nous sommes capables d'effectuer la rétroprojection sur les coupes que nous
avons mesurées. Comme illustré à la \Cref{fig:FBP}, effectuer une rétroprojection
selon seulement quelques directions $\theta$ constitue une méthode extrêmement
imprécise pour reconstituer ne serait-ce qu'un objet simple. Toutefois, même si
nous augmentons de manière significative le nombre de rétroprojections
(par exemple jusqu'à $1000$ directions), il subsiste encore une quantité
importante de bruit qui brouille l'image reconstruite.
En réalité, quel que soit le nombre de directions selon lesquelles nous tentons
d'effectuer la rétroprojection, nous ne serons jamais capables de reconstruire
parfaitement l'image à l'aide de la formule de rétroprojection donnée par
l'équation \eqref{eq:FBP}.
Pour que ce procédé soit réellement utile, il est nécessaire de dériver une
méthode permettant de filtrer une partie du bruit que la formule de
rétroprojection semble introduire dans l'image, afin d'obtenir une
représentation plus lisse de l'objet.

\begin{figure}[H]
    \centering
    \includegraphics[width=0.8\textwidth]{./images/fbp.png}
    \caption{Retroprojection d'un carré dans 5, 25, 100 et 1000 directions}
    \label{fig:FBP}
\end{figure}

Dans ce but, nous définissons une formule de \emph{rétroprojection filtrée}.
\begin{proposition}
    Soit $f$ une fonction absolument intégrable définie sur $\mathbb{R}^2$. Alors,
    \begin{equation}
        f(x,y)
        =
        \frac{1}{2}\,
        \mathcal{B}\!\left\{
        \mathcal{F}^{-1}
        \!\left[
        |S|\,
        \mathcal{F}\!\left(\mathcal{R}f\right)(S,\theta)
        \right]
        \right\}(x,y).
        \label{eq:FBP_filter}
    \end{equation}
\end{proposition}
\textit{Démonstration.}
Nous commençons par rappeler que, pour la transformée de Fourier bidimensionnelle
et son inverse, on a :
\begin{equation}
f(x,y) = \mathcal{F}_2^{-1}\,\mathcal{F}_2 f(x,y)
= \frac{1}{4\pi^2}
\int_{-\infty}^{\infty}\int_{-\infty}^{\infty}
\mathcal{F}_2 f(X,Y)\,e^{i(Xx+Yy)}\,dX\,dY.
\label{eq:fourier_2d_inverse}
\end{equation}

Nous allons maintenant effectuer un changement de variables des coordonnées
cartésiennes $(X,Y)$ vers les coordonnées polaires $(S,\theta)$, définies par
\[
X = S\cos\theta,
\qquad
Y = S\sin\theta,
\]
où $S \in \mathbb{R}$ et $\theta \in [0,\pi]$.
Ce changement de variables conduit au déterminant jacobien suivant :
\[\det
\begin{pmatrix}
    \dfrac{\partial X}{\partial s} & \dfrac{\partial X}{\partial \theta} \\[6pt]
    \dfrac{\partial Y}{\partial s} & \dfrac{\partial Y}{\partial \theta}
\end{pmatrix}
=|S|
\]
Ce qui nous dit que $dX\,dY = |S|\,dS\,d\theta$. En incorporant ce nouveau changement de variables, l'équation \eqref{eq:fourier_2d_inverse} devient :
\[
f(x,y) = \frac{1}{4\pi^{2}} \int_{0}^{\pi} \int_{-\infty}^{\infty}
\mathcal{F}_{2}f(S\cos\theta, S\sin\theta)\,
e^{iS(x\cos\theta + y\sin\theta)}\,|S|\,dS\,d\theta.
\]
Et en utilisant le théorème de la tranche centrale, nous voyons que l'équation ci-dessus est en fait égale à
\begin{equation}
    f(x,y) = \frac{1}{4\pi^{2}} \int_{0}^{\pi} \int_{-\infty}^{\infty}
    \mathcal{F}\bigl(\mathcal{R}f(S,\theta)\bigr)\,
    e^{iS(x\cos\theta + y\sin\theta)}\,|S|\,dS\,d\theta.
    \label{eq:fourier_radon}
\end{equation}
Prenons maintenant un regard plus attentif sur l'intégrale intérieure de l'équation \eqref{eq:fourier_radon} et en utilisant la définition de la Transformée de Fourier inverse, on a :
\[
    \begin{array}{rcl}
        \int_{-\infty}^{\infty}
        \mathcal{F}\bigl(\mathcal{R}f(S,\theta)\bigr)\,
        e^{iS(x\cos\theta + y\sin\theta)}\,|S|\,dS
        &=&
        2\pi \left(
        \frac{1}{2\pi} \int_{-\infty}^{\infty}
        \mathcal{F}\bigl(\mathcal{R}f(S,\theta)\bigr)\,
        e^{iS(x\cos\theta + y\sin\theta)}\,|S|\,dS
        \right)\\
        &=&
        2\pi\,\mathcal{F}^{-1}
        \Bigl(
        |S|\,\mathcal{F}\bigl(\mathcal{R}f\bigr)(S,\theta)
        \Bigr)
        \bigl(x\cos\theta + y\sin\theta,\theta\bigr)\\
    \end{array}
\]


Autrement dit, l'intégrale intérieure de l'équation (7.4) est égale à $2\pi$ fois l'inverse de la transformée de Fourier de
$|S|\,\mathcal{F}\bigl(\mathcal{R}f\bigr)(S,\theta)$
au point $(x\cos\theta + y\sin\theta,\theta)$.
Nous pouvons alors voir que l'équation (7.4) est en fait égale à
\[
\frac{1}{2\pi} \int_{0}^{\pi}
\mathcal{F}^{-1}
\Bigl(
|S|\,\mathcal{F}\bigl(\mathcal{R}f\bigr)(S,\theta)
\Bigr)
\bigl(x\cos\theta + y\sin\theta,\theta\bigr)
\,d\theta.
\]

Finalement, nous constatons que l'intégrale ci-dessus est égale à $\tfrac{1}{2}$ de la rétroprojection donnée dans la définition \eqref{eq:FBP} pour
$\mathcal{F}^{-1}\bigl[|S|\,\mathcal{F}(\mathcal{R}f)(S,\theta)\bigr]$.
Nous simplifions donc l'équation précédente pour obtenir
\[
\frac{1}{2}\,
\mathcal{B}
\Bigl\{
\mathcal{F}^{-1}
\bigl[|S|\,\mathcal{F}\bigl(\mathcal{R}f(S,\theta)\bigr)\bigr]
\Bigr\}(x,y).
\]

Ce qui nous conduit à la conclusion souhaitée :
\[
f(x,y)
=
\frac{1}{2}\,
\mathcal{B}
\Bigl\{
\mathcal{F}^{-1}
\bigl[|S|\,\mathcal{F}\bigl(\mathcal{R}f(S,\theta)\bigr)\bigr]
\Bigr\}(x,y).
\]
\hfill $\square$\\
Le facteur important dans cette formule est le multiplicateur $|S|$ qui apparaît entre la transformée de Fourier et son inverse. Sans ce facteur, ces deux termes s'annuleraient mutuellement et nous nous retrouverions avec la formule standard de rétroprojection pour la transformée de Radon que nous avons rencontrée précédemment et qui, comme nous l'avons vu, ne nous donne pas directement $f(x, y)$. Nous appelons ce $|S|$ supplémentaire un \textbf{filtre} de la transformée de Radon, ce qui nous donne le nom de la formule de \textbf{rétroprojection filtrée}.
\begin{proposition}
    Soit $f$ et $g$ deux fonctions intégrables définies sur $\mathbb{R}$, alors
    \[(\mathcal{B}g\star f)(X, Y) = \mathcal{B}(g\star \mathcal{R}f)(X, Y)\]
\end{proposition}
Considérons maintenant la relation \eqref{eq:FBP_filter} et 
supposons qu'il existe une fonction, notée $\varphi(t)$, dont la transformée de Fourier
soit égale à notre facteur de filtrage $|S|$. Autrement dit, supposons qu'il existe une
fonction $\varphi(t)$ telle que
\[
\mathcal{F}\varphi(S) = |S|.
\]
Plus simplement, supposons que nous connaissions une fonction dont la transformée de
Fourier est égale à la fonction valeur absolue. Nous pourrions alors réécrire la
rétroprojection sous la forme suivante :
\begin{equation}
    f(x,y) = \frac{1}{2}\,\mathcal{B}
    \left\{
    \mathcal{F}^{-1}
    \bigl[
    \mathcal{F}\varphi \cdot \mathcal{F}(\mathcal{R}f)(S,\theta)
    \bigr]
    \right\}(x,y).
    \label{eq:FBP_varphi}
\end{equation}

Cependant, le membre de droite de l'équation \eqref{eq:FBP_varphi} contient un produit de transforméesde Fourier, que nous savons être égal à la convolution des fonctions transformées
\[
    f(x,y)
    =
    \frac{1}{2}\,\mathcal{B}
    \left\{
    \mathcal{F}^{-1}
    \bigl[
    \mathcal{F}(\varphi \star \mathcal{R}f)(S,\theta)
    \bigr]
    \right\}(x,y).
\]

Mais ceci n'est rien d'autre que la transformée de Fourier inverse de la transformée
de Fourier, ce qui nous ramène à la fonction de départ. Cela nous conduit à la formule
de rétroprojection filtrée beaucoup plus simple :
\begin{equation}
    f(x,y) = \frac{1}{2}\,\mathcal{B}(\varphi \star \mathcal{R}f)(x,y).
    % 
    \label{eq:FBP_varphi_simple}
\end{equation}

L'équation \eqref{eq:FBP_varphi_simple} est bien plus élégante que notre formule initiale de rétroprojection filtrée et ne semble pas difficile à appliquer. Physiquement parlant, $\mathcal{R}f$ représente nos données mesurées et l'équation \eqref{eq:FBP_varphi_simple} requiert simplement de les filtrer à l'aide de notre nouvelle fonction $\varphi$, puis d'appliquer la formule de rétroprojection, qui est une intégrale relativement simple.

Malheureusement, il n'existe pas de fonction $\varphi$ dont la transformée de Fourier
soit exactement égale à la valeur absolue. Considérons la fonction $\mathcal{F}\varphi$ :
\[
\mathcal{F}\varphi(\omega)
=
\int_{-\infty}^{\infty}
\varphi(x)\,e^{-i\omega x}\,dx.
\]

Nous pouvons constater que, lorsque $\omega \to \infty$,
$\mathcal{F}\varphi(\omega) \to 0$ (remarquons l'exponentielle négative).
Cependant, pour la fonction valeur absolue $|\omega|$, lorsque $\omega \to \infty$,
$|\omega| \to \infty$.
Par conséquent, il est impossible de trouver une fonction $\varphi$ telle que,
pour tout $\omega$, $\mathcal{F}\varphi(\omega) = |\omega|$.

Toutefois, tout notre travail précédent n'est pas vain. Examinons plutôt le type de
fonctions sur lesquelles nous avons restreint notre étude. Nous ne considérons notre
fonction que sur un intervalle fini et supposons en fait qu'elle soit nulle en dehors
de cet intervalle. En étendant cette idée à la transformée de Fourier, nous constatons
que nous devons porter notre attention sur les \emph{fonctions à bande limitée}.

\begin{definition}
    Une fonction $\varphi$ est dite \emph{à bande limitée} s'il existe un réel $L > 0$ tel que
    \begin{equation}
        \mathcal{F}\varphi(\omega)
        =
        \int_{-\infty}^{\infty}
        \varphi(x)\,e^{-i\omega x}\,dx
        =
        0
        \quad \text{pour tout } \omega \notin [-L, L].
        % 
        \label{eq:FBP_varphi_banded}
    \end{equation}
\end{definition}

Le facteur de filtrage $|S|$ sert à amplifier le terme $\mathcal{F}(\mathcal{R}f)$ dans la formule de rétroprojection filtrée originale \eqref{eq:FBP_filter}. En pratique, $\mathcal{F}(\mathcal{R}f)$ est très sensible aux hautes fréquences.

En concentrant notre attention sur les basses fréquences à l'aide d'une fonction à bande limitée $\varphi$, nous sommes en mesure d'éviter ce problème. Notre objectif est de remplacer $S$ par ce que l'on appelle un \emph{filtre passe-bas} (noté $S'$), qui prend en compte les effets des basses fréquences tout en atténuant les hautes fréquences. Cette fonction $S'$ doit avoir un support compact et être de la forme
\[
S' = \mathcal{F}\varphi
\]
(sur un intervalle compact).

Le coût de l'utilisation de $S'(\omega)$ est que nous ne disposons plus de l'égalité présentée dans l'équation \eqref{eq:FBP_varphi_simple}. En revanche, nous obtenons :
\begin{equation}
    f(x,y) \approx \frac{1}{2}\,\mathcal{B}\!\left(\mathcal{F}^{-1} S' \star \mathcal{R}f \right)(x,y).
    \label{eq:FBP_varphi_approx}
\end{equation}

De manière générale, la plupart des filtres passe-bas sont de la forme
\[
S'(\omega) = |\omega| \cdot F(\omega) \cdot \Pi_L(\omega),
\]
où $L > 0$ définit la région sur laquelle le filtrage est effectué. Différentes fonctions $F$ déterminent les caractéristiques précises du filtre, et $\Pi_L(\omega)$ est définie comme suit :
\[
    \Pi_L(\omega) =
    \begin{cases}
        1 & \text{si } |\omega| \leq L, \\
        0 & \text{si } |\omega| > L.
    \end{cases}
\]

Nous introduisons maintenant deux filtres couramment utilisés en imagerie numérique et en traitement du signal : le filtre \emph{Ram-Lak} et le filtre \emph{Hann}.

\subsection*{Filtre Ram-Lak}

Le filtre Ram-Lak est défini par :
\[
S'(\omega) = |\omega| \cdot \Pi_L(\omega) =
\begin{cases}
|\omega| & \text{si } |\omega| \leq L, \\
0 & \text{si } |\omega| > L.
\end{cases}
\]

Le filtre Ram-Lak constitue la base de nombreux autres filtres utilisés en analyse du signal, car il remplace simplement la fonction $F(\omega)$ par la fonction constante égale à 1. D'autres filtres, tels que le filtre Hann, consistent généralement en des produits de fonctions sinus ou cosinus destinées à éliminer le bruit indésirable.

\subsection*{Filtre Hann}

Le filtre Hann est donné par :
\[
S'(\omega) = |\omega| \cdot \frac{1}{2}
\left( 1 + \cos\!\left( \frac{2\pi \omega}{L} \right) \right)
\cdot \Pi_L(\omega).
\]

Le filtre Hann utilise la fonction de Hann
\[
\frac{1}{2}\left( 1 + \cos\!\left( \frac{2\pi \omega}{L} \right) \right)
\]
comme fonction $F(\omega)$,
% et son efficacité est illustrée dans le sinogramme et la rétroprojection de la transformée de Radon de Johann, présentés à la Figure~5.
% ====== TODO ======
% Python implementation
% ==================

\section{Discrétisation des méthodes analytiques}
Ainsi nous avons traité presque exclusivement des intégrales continues pour la transformée de Radon, la transformée de Fourier et les formules de rétroprojection. En pratique, cependant, nous n'avons qu'un ensemble fini de données avec lesquelles travailler. Par conséquent, nous devrons former des versions discrètes de toutes les formules que nous avons utilisées dans notre rétroprojection filtrée.

Une fonction discrète est une fonction définie uniquement sur un ensemble dénombrable. Pour nos besoins, nous considérerons des fonctions discrètes définies sur des ensembles finis (l'ensemble étant composé des lignes sur lesquelles nous avons pris nos mesures d'intensité). Soit \(g_n\) la fonction discrète \(g\) à la valeur \(n\). Comme nous connaissons cette fonction discrète sur un ensemble fini, soit \(N\), nous pouvons dire que \(g = g_n : 0 \le n \le N - 1\). Si nous voulons étendre cette définition à tous les entiers, nous pouvons simplement « répéter » notre fonction encore et encore ; c'est-à-dire, nous pouvons la rendre périodique avec une période \(N\). Cette extension sera utile pour certaines des formules discrètes que nous rencontrerons.

Supposons que nous prenions des mesures à \(P\) angles différents \(\theta\) et que pour chaque angle nous ayons \(2M + 1\) faisceaux espacés d'une distance \(d\). Alors nous pouvons définir des valeurs particulières \(\theta_k\) et \(t_j\) comme

\[
\theta_k = \left\{ \frac{k \pi}{P} : 0 \le k \le P - 1 \right\},
\]

\[
t_j = \{ jd : -M \le j \le M \}.
\]

Ce qui nous permet de définir une ligne particulière comme \(l_{t_j, \theta_k}\). Nous définissons donc la transformée de Radon discrète comme suit :

\begin{definition}
Pour une fonction absolument intégrable \(f\) et \(0 \le k \le P\) et \(-M \le j \le M\), \((P, M > 0)\), nous définissons la transformée de Radon discrète de \(f\), notée \(\mathcal{R}_D f\), comme

\[
\mathcal{R}_D f_{j,k} = \mathcal{R} f(t_j, \theta_k).
\]

\end{definition}
Pour mettre en œuvre la formule de rétroprojection filtrée \eqref{eq:FBP_varphi_approx}, nous devons également définir la convolution de deux fonctions discrètes.

\begin{definition}
Pour deux fonctions discrètes \(N\)-périodiques \(f\) et \(g\), nous définissons la \textbf{convolution discrète} de \(f\) et \(g\), notée \(f \star g\), comme

\[
(f \star g)_m = \sum_{j=0}^{N-1} f_j \cdot g_{(m-j)}, \quad \text{pour } m \in \mathbb{Z}.
\]
Évidemment, nous aurons également besoin de la transformée de Fourier discrète.
\end{definition}

\begin{definition}[Transformée de Fourier discrète]
Étant donnée une fonction discrète $N$-périodique $f$, nous définissons la \textbf{transformée de Fourier discrète} de $f$, notée $\mathcal{F}_D f$, par
\begin{equation}
(\mathcal{F}_D f)_j = \sum_{k=0}^{N-1} f_k e^{i 2 \pi k j / N}, \quad \text{pour } j = 0, 1, \dots, (N-1).
\end{equation}
Il convient de noter que la $N$-périodicité de $f$ nous permet de remplacer les bornes de la sommation par tout ensemble d'entiers de longueur $(N-1)$. Avec cette définition, il n'est pas surprenant que nous définissions la transformée de Fourier discrète inverse de la manière suivante.
\end{definition}

\begin{definition}[Transformée de Fourier discrète inverse]
Étant donnée une fonction discrète $N$-périodique $g$, la \textbf{transformée de Fourier discrète inverse} de $g$, notée $\mathcal{F}_D^{-1} g$, est définie par
\begin{equation}
(\mathcal{F}_D^{-1} g)_n = \frac{1}{N} \sum_{k=0}^{N-1} g_k e^{i 2 \pi k n / N}, \quad \text{pour } n = 0, 1, \dots, (N-1).
\end{equation}
\end{definition}

Nous remarquons que plusieurs des mêmes propriétés de la transformée de Fourier que nous avons définies dans le cadre continu s'appliquent également au cas discret avec de légères modifications :

\begin{proposition}[Propriétés des fonctions discrètes $N$-périodiques]
Pour des fonctions discrètes $N$-périodiques $f$ et $g$ :
\begin{enumerate}
    \item $\mathcal{F}_D(f \star g) = (\mathcal{F}_D f) \cdot (\mathcal{F}_D g)$
    \item $\mathcal{F}_D(f \cdot g) = \frac{1}{N} (\mathcal{F}_D f) \star (\mathcal{F}_D g)$
    \item $\mathcal{F}_D^{-1}(\mathcal{F}_D f)_n = f_n \quad \text{pour tout } n \in \mathbb{Z}$
\end{enumerate}
\end{proposition}

Nous sommes maintenant prêts à aborder la discrétisation de la formule de rétroprojection elle-même. Rappelons que la formule de rétroprojection était définie comme une intégrale de $0$ à $\pi$ par rapport à $d\theta$. Dans le cas discret, nous avons remplacé ce $d\theta$ continu par $k\pi / P$ pour $0 \le k \le (P-1)$. Cela conduit à la définition suivante de la \textbf{rétroprojection discrète} :

\begin{definition}[Rétroprojection discrète]
Étant donnée une fonction discrète $h$, nous définissons la \textbf{rétroprojection discrète} de $h$, notée $\mathcal{B}_D h$, par
\begin{equation}
\mathcal{B}_D h(x,y) = \frac{1}{N} \sum_{k=0}^{N-1} h \big(x \cos \frac{k \pi}{N} + y \sin \frac{k \pi}{N}, k \pi / N \big).
\label{eq:discrete_backprojection}
\end{equation}
\end{definition}

Rappelons notre forme finale pour la formule filtrée de rétroprojection en équation \eqref{eq:FBP_varphi_approx} :
$$
f(x,y) \approx \frac{1}{2} \mathcal{B} (\mathcal{F}^{-1} S' \star \mathcal{R} f)(x,y).
$$

Pour former la version discrète de cette équation, nous voyons que nous devons appliquer la formule suivante:
\begin{equation}
f(x,y) \approx \frac{1}{2} \mathcal{B}_D \left( \mathcal{F}_D^{-1} \mathcal{S}' \ast \mathcal{R}_D f \right)(x,y).
\end{equation}

Nous rencontrons maintenant un léger problème. $\mathcal{R}_D f$ représente les données mesurées basées sur les intensités finales d'un seul faisceau de rayons X. Nous avons défini les emplacements des différents faisceaux (et donc des différentes coupes) en utilisant un système de coordonnées perpendiculaire aux coordonnées polaires basé sur des angles discrets $\theta$ et des distances $t$.  

En examinant l'équation \eqref{eq:discrete_backprojection}, nous voyons que nous devons sommer sur $h$ en différents points $(x,y)$ dans le système de coordonnées cartésien pour créer une grille de niveaux de gris rectangulaire qui représente notre objet original. Les systèmes de coordonnées polaires et cartésiens ne correspondent pas nécessairement parfaitement, et nous devons donc \emph{interpoler} les points de données manquants. L'interpolation consiste à créer une fonction continue (ou au minimum par morceaux continues) à partir d'un ensemble discret de valeurs. Il existe de nombreuses méthodes pour interpoler une fonction (spline cubique, Lagrange, etc.), chacune ayant ses avantages et inconvénients.  

Pour nos besoins, nous allons définir un type général d'interpolation basé sur une fonction de pondération $W$ qui détermine comment nous allons choisir nos points interpolés. Nous ne définissons pas de fonction de pondération particulière $W$, car les détails de l'interpolation ne sont pas aussi importants que le fait que nous pouvons remplir les "trous" dans nos données.

\begin{definition}
Pour une fonction de pondération donnée $W$ et une fonction discrète $N$-périodique $g$, l'\emph{interpolation $W$} de $g$ est définie par :
\begin{equation}
\mathcal{I}_W(g)(x) = \sum_n g(n) \cdot W\left(\frac{x}{d}-n\right), \quad \text{pour } -\infty < x < \infty.
\end{equation}
\end{definition}

Maintenant que nous avons couvert toutes les parties de l'équation \eqref{eq:FBP_varphi_simple} dans un cadre discret et traité le problème de l'interpolation, nous pouvons proposer un algorithme de reconstruction discret pour résoudre le coefficient d'atténuation à partir d'un ensemble de données discret.  

Nous interpolons ici la fonction $\left(\mathcal{F}_D^{-1}\mathcal{S}'\right) \ast \mathcal{R}_D f(jd,k\pi/N)$ (c'est-à-dire que nous remplissons les trous après le filtrage de la transformée de Radon). Définissons cette fonction interpolée comme $\mathcal{I}$. Cela conduit à la formule de reconstruction suivante :

\begin{equation}
\begin{aligned}
f(x_m,y_n) &\approx \frac{1}{2} \mathcal{B}_D \left( \left( \mathcal{F}_D^{-1}\mathcal{S}' \right) \ast \mathcal{R}_D f \right) (jd, k\pi/N) \\
&\approx \frac{1}{2} \mathcal{B}_D \mathcal{I}(x_m,y_n) \\
&= \frac{1}{2N} \sum_{k=0}^{N-1} \mathcal{I} \left( x_m \cos \frac{k\pi}{N} + y_n \sin \frac{k\pi}{N}, \frac{k\pi}{N} \right).
\end{aligned}
\end{equation}

L'équation précédente tient compte de la nature discrète de nos données réelles et traite les problèmes (comme le manque de données) qui surviennent lorsque l'on dispose d'un nombre fini de mesures.


% \section{Formulation linéaire -- Synthèse}
% D'accord, nous allons expliquer pas à pas comment passer de la formulation intégrale continue de la rétroprojection filtrée à la \textbf{formulation linéaire discrète \(g = Af\)} utilisée en pratique en tomographie.
% \subsection{Rétroprojection filtrée continue}
% On a la formule continue pour la reconstruction filtrée :
% \[
% f(x,y) = \frac{1}{2} \int_0^\pi \Big( (\mathcal{F}^{-1} S') \star \mathcal{R} f \Big)(x\cos\theta + y\sin\theta, \theta) \, d\theta
% \]

% Ici :
% \begin{itemize}
%     \item $f(x,y)$ : coefficient d'atténuation à reconstruire,
%     \item $\mathcal{R} f(t,\theta)$ : transformée de Radon (projection à l'angle $\theta$),
%     \item $\mathcal{F}^{-1} S'$ : filtre appliqué sur chaque projection,
%     \item $\star$ : convolution dans $t$.
% \end{itemize}

% C'est une \textbf{formule intégrale continue}, dépendante de coordonnées polaires.\vspace{5pt}\\
% Cette expression met en évidence la structure mathématique exacte de la reconstruction tomographique idéale. Toutefois, elle repose sur des intégrales et des fonctions continues qui ne peuvent pas être manipulées directement en pratique. Pour une implémentation numérique, il est indispensable de passer à une représentation discrète. Cette transition constitue le cœur des méthodes de reconstruction utilisées en imagerie médicale.

% \subsection{Discrétisation des coordonnées et des angles}

% Pour passer au discret :
% \begin{enumerate}
%     \item On ne mesure que $P$ angles : $\theta_k = k\pi/P$, $k = 0,\dots,P-1$,
%     \item On ne mesure que $2M+1$ faisceaux par angle, espacés de $d$ : $t_j = j d, j=-M,\dots,M$,
%     \item On obtient donc la \textbf{transformée de Radon discrète} :
%     \[
%     \mathcal{R}_D f_{j,k} = \mathcal{R} f(t_j, \theta_k).
%     \]
% \end{enumerate}\vspace{5pt}
% Cette étape traduit les contraintes physiques des systèmes d'acquisition réels, qui ne fournissent qu'un nombre fini de mesures. L'échantillonnage en angle et en position transforme ainsi le problème continu en un ensemble de données numériques exploitables. La qualité de la reconstruction dépend directement de la finesse de cette discrétisation. Elle conditionne notamment la résolution spatiale et les artefacts.

% \subsection{Convolution et filtrage discrets}

% On applique ensuite le filtre sur chaque projection :
% \[
% h_{j,k} = (\mathcal{F}_D^{-1} \mathcal{S}' \ast \mathcal{R}_D f)_{j,k}
% \]

% Ici, $\ast$ est la \textbf{convolution discrète} dans $t$ :
% \[
% (f \ast g)_m = \sum_{n=0}^{N-1} f_n \, g_{(m-n)}.
% \]\vspace{5pt}
% Le filtrage est une étape cruciale de la méthode FBP, car il corrige le flou introduit par la rétroprojection simple. La convolution discrète permet d'implémenter efficacement ce filtrage sur des données échantillonnées. Elle renforce les hautes fréquences nécessaires à une bonne résolution spatiale. Sans cette étape, la reconstruction serait fortement dégradée.

% \subsection{Discrétisation de la rétroprojection}

% La rétroprojection discrète est :
% \[
% f(x_m,y_n) \approx \frac{1}{2N} \sum_{k=0}^{N-1} h\Big( x_m \cos \frac{k\pi}{N} + y_n \sin \frac{k\pi}{N}, \frac{k\pi}{N} \Big).
% \]

% Comme les coordonnées cartésiennes $(x_m, y_n)$ ne tombent pas exactement sur les positions $t_j$, on \textbf{interpole} :
% \[
% h\big(x_m\cos\theta_k + y_n \sin\theta_k, \theta_k\big) \approx \sum_j h_{j,k} \, W\left(\frac{x_m \cos\theta_k + y_n \sin\theta_k - t_j}{d}\right),
% \]
% où $W$ est la fonction de pondération de l'interpolation (linéaire, spline, etc.).  
% Cela transforme chaque $f(x_m, y_n)$ en \textbf{combinaison linéaire des mesures $h_{j,k}$}.\vspace{5pt}\\
% La rétroprojection redistribue l'information des projections filtrées sur la grille de l'image reconstruite. Cette étape assure la cohérence géométrique entre les données mesurées et l'espace image. L'interpolation est indispensable pour relier les coordonnées polaires aux pixels cartésiens. Elle introduit également une approximation contrôlée du modèle continu.


% \subsection{Passage à la forme matricielle linéaire}

% Si on note :
% \begin{itemize}
%     \item $f$ le vecteur de tous les $f(x_m, y_n)$ sur la grille,
%     \item $g$ le vecteur de toutes les mesures projetées filtrées $h_{j,k}$,
%     \item $A$ la matrice représentant la \textbf{rétroprojection + interpolation},
% \end{itemize}

% alors :
% \[
% f_i = \sum_j A_{ij} \, g_j
% \]

% Chaque coefficient $A_{ij}$ représente le poids avec lequel la projection $g_j$ contribue au pixel $f_i$.  

% On obtient donc :
% \[
% \boxed{g = Af} \quad \text{ou souvent } f = A g \text{ selon la notation.}
% \]

% En pratique, $A$ est \textbf{très grande et creuse}, mais la reconstruction se réduit à un simple \textbf{produit matriciel}.\vspace{5pt}\\
% Cette écriture matricielle unifie l'ensemble du processus de reconstruction dans un cadre linéaire. Elle permet d'analyser la FBP avec les outils de l'algèbre linéaire et de l'optimisation. De nombreuses méthodes itératives modernes reposent sur cette formulation. Elle constitue ainsi un pont entre les approches analytiques et numériques.

% \subsection{Synthèse}

% Le passage de l'intégrale continue à $g = Af$ se fait en quatre étapes principales :
% \begin{enumerate}
%     \item \textbf{Échantillonnage discret} des angles et des faisceaux → $\mathcal{R}_D f$,
%     \item \textbf{Filtrage discret} via convolution et transformée de Fourier discrète,
%     \item \textbf{Rétroprojection discrète} et interpolation sur la grille cartésienne,
%     \item \textbf{Écriture linéaire} : chaque pixel reconstruit est une combinaison linéaire des mesures → matrice $A$.
% \end{enumerate}

% Ainsi, \textbf{toute la formule intégrale est transformée en somme discrète}, et la linéarité de la convolution et de la rétroprojection permet de la représenter par $A$.\vspace{5pt}\\
% Cette synthèse met en évidence la cohérence du passage du modèle physique continu vers une implémentation numérique concrète. La méthode FBP apparaît alors comme une succession d'opérations linéaires bien structurées. Cette vision est essentielle pour comprendre ses limites et ses extensions. Elle prépare naturellement l'introduction de méthodes de reconstruction plus avancées.
\section{Formulation linéaire -- Synthèse}

Ce chapitre explique pas à pas comment passer de la formulation intégrale continue de la rétroprojection filtrée à la \textbf{formulation linéaire discrète \( g = Af \)} utilisée en pratique en tomographie. Cette transition est essentielle pour relier la théorie mathématique aux algorithmes numériques implémentés dans les systèmes réels. Elle constitue également le point d’entrée vers les méthodes itératives modernes.\vspace{7pt}\\
D'accord, nous allons expliquer pas à pas comment passer de la formulation intégrale continue de la rétroprojection filtrée à la \textbf{formulation linéaire discrète \(g = Af\)} utilisée en pratique en tomographie.

\subsection{Rétroprojection filtrée continue}
Cette section présente l’expression analytique idéale de la reconstruction tomographique. Elle met en évidence le rôle central de la transformée de Radon, du filtrage et de l’intégration angulaire. Bien que mathématiquement élégante, cette formulation reste purement continue et ne peut pas être appliquée directement en pratique.\vspace{7pt}\\
On a la formule continue pour la reconstruction filtrée :
\[
f(x,y) = \frac{1}{2} \int_0^\pi \Big( (\mathcal{F}^{-1} S') \star \mathcal{R} f \Big)(x\cos\theta + y\sin\theta, \theta) \, d\theta
\]

Ici :
\begin{itemize}
    \item $f(x,y)$ : coefficient d'atténuation à reconstruire,
    \item $\mathcal{R} f(t,\theta)$ : transformée de Radon (projection à l'angle $\theta$),
    \item $\mathcal{F}^{-1} S'$ : filtre appliqué sur chaque projection,
    \item $\star$ : convolution dans $t$.
\end{itemize}

C'est une \textbf{formule intégrale continue}, dépendante de coordonnées polaires.
\subsection{Discrétisation des coordonnées et des angles}
La discrétisation permet d’adapter le modèle continu aux contraintes physiques des systèmes d’acquisition. Elle transforme les intégrales et variables continues en un ensemble fini de mesures exploitables numériquement. Cette étape conditionne directement la qualité et la résolution de la reconstruction finale. \vspace{7pt}\\
Pour passer au discret :
\begin{enumerate}
    \item On ne mesure que $P$ angles : $\theta_k = k\pi/P$, $k = 0,\dots,P-1$,
    \item On ne mesure que $2M+1$ faisceaux par angle, espacés de $d$ : $t_j = j d, j=-M,\dots,M$,
    \item On obtient donc la \textbf{transformée de Radon discrète} :
    \[
    \mathcal{R}_D f_{j,k} = \mathcal{R} f(t_j, \theta_k).
    \]
\end{enumerate}

\subsection{Convolution et filtrage discrets}
Le filtrage est une étape cruciale de la méthode FBP, car il compense le flou introduit par la rétroprojection simple. Il repose sur l’opération de convolution, qui permet de renforcer les hautes fréquences nécessaires à une bonne résolution spatiale. Sans ce filtrage, la reconstruction serait fortement dégradée.\vspace{7pt}\\
On applique ensuite le filtre sur chaque projection :
\[
h_{j,k} = (\mathcal{F}_D^{-1} \mathcal{S}' \ast \mathcal{R}_D f)_{j,k}
\]

Ici, $\ast$ est la \textbf{convolution discrète} dans $t$ :
\[
(f \ast g)_m = \sum_{n=0}^{N-1} f_n \, g_{(m-n)}.
\]

\subsection{Discrétisation de la rétroprojection}

La rétroprojection assure la redistribution des projections filtrées sur la grille de l’image reconstruite. Elle garantit la cohérence géométrique entre l’espace des mesures et l’espace image. Cette étape nécessite une interpolation, introduisant une approximation contrôlée du modèle continu.\vspace{7pt}\\
La rétroprojection discrète est :
\[
f(x_m,y_n) \approx \frac{1}{2N} \sum_{k=0}^{N-1} h\Big( x_m \cos \frac{k\pi}{N} + y_n \sin \frac{k\pi}{N}, \frac{k\pi}{N} \Big).
\]
Comme les coordonnées cartésiennes $(x_m, y_n)$ ne tombent pas exactement sur les positions $t_j$, on \textbf{interpole} :
\[
h\big(x_m\cos\theta_k + y_n \sin\theta_k, \theta_k\big) \approx \sum_j h_{j,k} \, W\left(\frac{x_m \cos\theta_k + y_n \sin\theta_k - t_j}{d}\right),
\]
où $W$ est la fonction de pondération de l'interpolation (linéaire, spline, etc.).\vspace{7pt}\\
Cela transforme chaque $f(x_m, y_n)$ en \textbf{combinaison linéaire des mesures $h_{j,k}$}.
\subsection{Passage à la forme matricielle linéaire}
La formulation matricielle permet de rassembler toutes les opérations précédentes dans un cadre linéaire unique. Elle constitue un lien direct entre la théorie continue et les algorithmes numériques. Cette écriture est essentielle pour l’analyse et l’extension vers des méthodes itératives.\vspace{7pt}\\
Si on note :
\begin{itemize}
    \item $f$ le vecteur de tous les $f(x_m, y_n)$ sur la grille,
    \item $g$ le vecteur de toutes les mesures projetées filtrées $h_{j,k}$,
    \item $A$ la matrice représentant la \textbf{rétroprojection + interpolation},
\end{itemize}

alors :
\[
f_i = \sum_j A_{ij} \, g_j
\]

On obtient :
\[
\boxed{g = Af} \quad \text{(ou } f = Ag \text{ selon la convention)}
\]

En pratique, $A$ est \textbf{très grande et creuse}, mais la reconstruction se réduit à un produit matriciel.

\begin{definition}[Matrice système $\Phi$ et poids $\varphi_{i,j}$]
Dans le cadre discret, le vecteur de données de projection à faisceau parallèle $\vec{g}$ est modélisé par une somme pondérée sur les pixels traversés par le rayon X :
\begin{equation}
g_i = \sum_{j=1}^{N} \varphi_{i,j} \cdot f_j, \quad \text{où } i = 1, 2, \cdots, M.
\end{equation}
Le coefficient de pondération $\varphi_{i,j}$ de la matrice système $\Phi$ est égal à la longueur d'intersection du $i$-ème rayon à travers le $j$-ème pixel.
\end{definition}

\begin{figure}[H]
    \centering
    \includegraphics[width=0.8\textwidth]{./images/projection à faisceau parallèle.png}
    \caption{Calcul du coefficient de poids $\varphi_{i,j}$ de la matrice système $\Phi$ à partir de la longueur d'intersection du $i$-ème rayon à travers le $j$-ème pixel.}
    \label{fig:phi}
\end{figure}

Le calcul direct de chaque $\varphi_{i,j}$ est coûteux. Pour accélérer la reconstruction, on peut pré-calculer et stocker ces poids, et exploiter les propriétés de symétrie des projections à faisceau parallèle pour réduire le nombre de calculs nécessaires.

\begin{figure}[H]
    \centering
    \includegraphics[width=0.8\textwidth]{./images/projection à faisceau parallèle-2.png}
    \caption{Mesures des rayons-\textbf{X} $a$, $b$, $c$ et $d$ pour des angles de rotation $\alpha$, $90-\alpha$, $90+\alpha$ et $180-\alpha$. Les propriétés de symétrie permettent de déduire les poids d'un rayon à partir d'un autre.}
    \label{fig:mesure-rayons-X}
\end{figure}


\subsection{Synthèse}
Cette synthèse met en évidence la cohérence globale du passage du modèle continu vers une implémentation discrète. Elle permet de comprendre la FBP comme une succession structurée d’opérations linéaires. Cette vision prépare naturellement l’introduction de méthodes de reconstruction plus avancées.\vspace{7pt}\\
Le passage de l'intégrale continue à $g = Af$ se fait en quatre étapes principales :
\begin{enumerate}
    \item \textbf{Échantillonnage discret} des angles et des faisceaux,
    \item \textbf{Filtrage discret} par convolution,
    \item \textbf{Rétroprojection discrète} avec interpolation,
    \item \textbf{Formulation linéaire matricielle}.
\end{enumerate}

\subsubsection*{Conclusion}
Ce chapitre a présenté les principaux outils mathématiques qui fondent la reconstruction d'images tomographiques, en mettant en évidence la transition fondamentale entre les méthodes analytiques classiques et les approches modernes basées sur le \textit{Compressed Sensing}.\vspace{7pt}\\
Les méthodes analytiques, historiquement les premières, reposent sur une formulation mathématique élégante et directe du problème inverse. La transformée de Radon en constitue la pierre angulaire, en modélisant le lien entre l'objet et ses projections. Son inversion, fondée sur le théorème de la coupe centrale et mise en œuvre dans l'algorithme de rétroprojection filtrée (FBP), permet une reconstruction rapide. Ces méthodes, dont la FBP demeure un standard clinique, supposent toutefois des données abondantes, complètes et faiblement bruitées.\vspace{7pt}\\
Cependant, les exigences actuelles de réduction de dose et de temps d'acquisition conduisent à des scénarios sous-échantillonnés, dans lesquels les méthodes analytiques atteignent leurs limites, se traduisant par des artefacts et une forte sensibilité au bruit. C'est dans ce contexte que les méthodes itératives et le cadre du \textit{Compressed Sensing} prennent tout leur sens. En formulant la reconstruction comme un problème inverse régularisé, elles exploitent des connaissances a priori sur l'image, telles que sa parcimonie dans un domaine approprié (ondelettes, gradient), afin de stabiliser l'inversion et d'obtenir des reconstructions de qualité à partir de données limitées. La variation totale (TV) constitue un exemple de régularisation particulièrement adapté, permettant de préserver les contours tout en réduisant le bruit.\vspace{7pt}\\
La transformée de Fourier et l'opération de convolution jouent un rôle transversal essentiel, aussi bien dans l'implémentation du filtrage pour la FBP que dans l'analyse fréquentielle des données. Par ailleurs, la discrétisation des opérateurs mathématiques et la formulation matricielle linéaire du problème, sous la forme
\[
g = H f,
\]
permettent d'établir un lien direct entre la théorie continue et son implémentation numérique, ouvrant ainsi la voie aux algorithmes d'optimisation itératifs.\vspace{7pt}\\
En résumé, ce chapitre dresse un panorama cohérent des approches algorithmiques de la reconstruction tomographique. Il montre que si les méthodes analytiques offrent rapidité et simplicité dans des conditions idéales, les méthodes itératives régularisées, appuyées par la théorie du \textit{Compressed Sensing}, constituent une réponse incontournable aux défis de l'imagerie moderne : reconstruire davantage d'information à partir de moins de données, sans compromettre la qualité diagnostique ni la sécurité du patient. La maîtrise de ces outils mathématiques apparaît ainsi comme un levier essentiel pour le développement des méthodes de reconstruction de demain.
