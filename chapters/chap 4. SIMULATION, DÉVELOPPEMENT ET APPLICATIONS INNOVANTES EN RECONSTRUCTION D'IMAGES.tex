\chapter{SIMULATION, DÉVELOPPEMENT ET APPLICATIONS INNOVANTES EN RECONSTRUCTION D'IMAGES CT}
\section{Collecte des données}
\subsection{Transformée de Radon et de la Génération de Sinogrammes}
La génération d'un sinogramme \Ref{fig:sinogram_generation} constitue une étape centrale en tomographie, notamment en imagerie médicale et industrielle. Elle repose sur un outil mathématique fondamental appelé transformée de Radon, qui permet de représenter une image bidimensionnelle sous forme d'un ensemble de projections unidimensionnelles. Ce processus consiste à mesurer l'intégrale de l'intensité de l'image le long de droites orientées selon différents angles, simulant ainsi le principe physique de l'acquisition tomographique. Cette transformation permet de passer de l'espace spatial de l'image à un espace de projections, appelé espace de Radon, qui contient l'information nécessaire à la reconstruction de l'image originale. L'implémentation numérique complète de ce processus, incluant la discrétisation, le calcul des projections et la construction du sinogramme, est présentée dans le code fourni en Annexe A.\vspace{5pt}\\
La transformée de Radon d'une fonction continue $f(x,y)$, représentant l'intensité de l'image, est définie comme l'intégrale de cette fonction le long d'une droite caractérisée par un angle $\theta$ et une distance $s$ par rapport à l'origine. Mathématiquement, elle s'écrit :

$$\mathcal{R}f(\theta, s) = \int_{-\infty}^{\infty} \int_{-\infty}^{\infty} f(x,y),\delta(x\cos\theta + y\sin\theta - s),dx,dy$$

où $\delta$ désigne la fonction delta de Dirac, qui permet de sélectionner uniquement les points appartenant à la droite considérée. Cette formulation exprime le fait que chaque valeur de la transformée correspond à une projection de l'image selon une direction donnée. D'un point de vue géométrique, pour un angle fixé, la transformée calcule la somme des intensités de tous les points situés le long de droites parallèles. En faisant varier l'angle, on obtient un ensemble complet de projections décrivant l'image sous différents points de vue.

\begin{figure}[H]
    \centering
    
    \begin{subfigure}[b]{0.32\textwidth}
        \centering
        \includegraphics[width=\textwidth]{images/Radon-transform-of-projections-at-different-angles2.png}
        % \caption{Image 1}
        % \label{fig:img1}
    \end{subfigure}
    \hfill
    \begin{subfigure}[b]{0.32\textwidth}
        \centering
        \includegraphics[width=\textwidth]{images/A-2-D-illustration-of-Radon-transform-geometry-On-the-right-Radon-transform-of-a-pixel.png}
        % \caption{Image 2}
        % \label{fig:img2}
    \end{subfigure}
    \hfill
    \begin{subfigure}[b]{0.32\textwidth}
        \centering
        \includegraphics[width=\textwidth]{images/Radon_transform.png}
        % \caption{Image 3}
        % \label{fig:img3}
    \end{subfigure}
    
    \caption{Processus de transformation de Radon}
    % \label{fig:three_images}
    
\end{figure}

\begin{figure}[H]
    \centering
    \includegraphics[width=1\textwidth, height=0.6\textwidth]{images/creation_de_sinogramme.png}
    \caption{Diagramme de génération d'un sinogramme}
    \label{fig:sinogram_generation}
\end{figure}

Dans la pratique, cette transformation continue doit être discrétisée afin d'être implémentée numériquement. Les angles de projection sont échantillonnés uniformément dans l'intervalle $[0,180^\circ[$ selon la relation :
$$\theta_k = \frac{180^\circ}{N} \cdot k, \quad k = 0, 1, ..., N-1$$
où $N$ représente le nombre total de projections. Pour chaque angle discret, l'image est projetée sur un ensemble de détecteurs qui échantillonnent la variable spatiale $s$. Le résultat de ce processus est organisé sous forme d'une matrice appelée sinogramme, notée $S \in \mathbb{R}^{M \times N}$, où $M$ correspond au nombre de détecteurs et $N$ au nombre d'angles. Chaque colonne du sinogramme contient la projection de l'image pour un angle donné, tandis que chaque ligne correspond à la variation des projections pour un détecteur donné lorsque l'angle change. L'algorithme détaillé de cette discrétisation ainsi que la structure matricielle utilisée pour stocker les données sont explicitement décrits dans le code de l'Annexe A, afin d'assurer la reproductibilité des résultats présentés.

\begin{figure}[H]
    \centering
    
    \begin{subfigure}[b]{0.32\textwidth}
        \centering
        \includegraphics[width=\textwidth]{images/A-Shepp-Logan-Phantom-and-reconstructed-Image-Sinogram-a-Original-image-b-radon.png}
        % \caption{Image 1}
        % \label{fig:img1}
    \end{subfigure}
    \hfill
    \begin{subfigure}[b]{0.32\textwidth}
        \centering
        \includegraphics[width=\textwidth]{images/Examples-of-reconstructed-images-and-sinograms-with-different-labels-for-a-body-part.png}
        % \caption{Image 2}
        % \label{fig:img2}
    \end{subfigure}
    \hfill
    \begin{subfigure}[b]{0.32\textwidth}
        \centering
        \includegraphics[width=\textwidth]{images/Examples-of-reconstructed-images-and-sinograms-with-different-labels-for-a-body-part.png}
        % \caption{Image 3}
        % \label{fig:img3}
    \end{subfigure}
    
    \caption{Images-Sinogrammes pairs}
    \label{fig:three_images}
    
\end{figure}

Dans les systèmes réels, les mesures sont inévitablement affectées par le bruit, provenant principalement des limitations électroniques des détecteurs, des fluctuations statistiques du signal et des perturbations environnementales. Ce bruit est généralement modélisé comme un bruit additif gaussien, ce qui conduit à l'expression :

$$S_{\text{bruité}}(s, \theta) = S(s, \theta) + \eta(s, \theta)$$

où $\eta(s,\theta)$ est une variable aléatoire suivant une distribution normale de moyenne nulle et de variance $\sigma^2$. Cette modélisation permet de reproduire de manière réaliste les conditions expérimentales et constitue une étape essentielle pour l'évaluation des algorithmes de reconstruction. Le mécanisme d'ajout de bruit, paramétré par un niveau de variance contrôlé, est également intégré dans l'implémentation fournie en Annexe A.

La transformée de Radon possède plusieurs propriétés mathématiques fondamentales qui expliquent son importance en tomographie. Elle est linéaire, ce qui signifie que la projection d'une combinaison linéaire de fonctions est égale à la combinaison linéaire de leurs projections. Elle présente également une propriété de symétrie qui réduit la plage angulaire nécessaire à l'acquisition. Plus important encore, elle est étroitement liée à la transformée de Fourier par le théorème de la coupe centrale, selon lequel la transformée de Fourier d'une projection correspond à une coupe radiale de la transformée de Fourier bidimensionnelle de l'image. Cette propriété constitue le fondement théorique de nombreux algorithmes de reconstruction, notamment la rétroprojection filtrée.

D'un point de vue physique, la génération d'un sinogramme correspond à la mesure de l'atténuation d'un rayonnement traversant un objet selon différentes directions. Chaque projection représente l'accumulation des interactions du rayonnement avec la matière le long de sa trajectoire. Cette représentation permet de transformer un problème de reconstruction bidimensionnelle en un ensemble structuré de mesures unidimensionnelles, facilitant ainsi l'analyse, la modélisation et le traitement numérique.

\subsection{Sous-Échantillonnage Angulaire en Tomographie pour le Compressed Sensing}

L'introduction de ce rapport présente le sous-échantillonnage angulaire comme une technique centrale en tomographie moderne, notamment dans le cadre du Compressed Sensing (CS). Cette approche permet de réduire la dose d'irradiation lors des acquisitions médicales tout en conservant une qualité d'image satisfaisante, grâce à des algorithmes de reconstruction avancés. Elle répond aux besoins croissants de sécurité des patients et d'acquisition rapide, tout en exploitant les propriétés structurelles des images médicales.\vspace{7pt}\\

En tomographie conventionnelle, le critère de Nyquist définit le nombre minimal d'angles nécessaires pour éviter les artefacts de repliement spectral dans le sinogramme. Formellement, on a :
\[
N_{\text{angles}} \geq \frac{\pi}{2} \cdot N_{\text{pixels}}
\]
Le sous-échantillonnage consiste à acquérir volontairement moins d'angles que cette limite. Ainsi, le nombre d'angles sous-échantillonnés s'écrit :
\[
N_{\text{sous-échantillonné}} = \alpha \cdot N_{\text{complet}}
\]
où \(\alpha \in [0,1]\) représente le facteur de sous-échantillonnage. Cette réduction entraîne une perte d'information, mais le Compressed Sensing permet de compenser cette limitation grâce à des hypothèses de parcimonie sur l'image.\vspace{7pt}\\

La sélection des angles pour le sous-échantillonnage suit généralement un processus aléatoire uniforme, sans remise, afin de garantir l'incohérence nécessaire au CS :
\[
\Theta_{\text{sous}} = \{\theta_{i_1}, \theta_{i_2}, ..., \theta_{i_M}\}
\]
où \(\{i_1, i_2, ..., i_M\}\) est un sous-ensemble aléatoire de \(\{0, 1, ..., N-1\}\), et \(M = \lfloor \alpha \cdot N \rfloor\) correspond au nombre d'angles acquis. Les indices sont triés pour préserver l'ordre angulaire, ce qui facilite l'interprétation du sinogramme (Voir Annexe A pour l'implémentation).\vspace{7pt}\\

Le Compressed Sensing repose sur trois concepts fondamentaux : la parcimonie, qui veut que l'image soit parcimonieuse dans un domaine de transformation \(\Psi\) (c'est-à-dire que seule une fraction des coefficients est significative) ; l'incohérence, qui exige que la matrice de mesure \(\Phi\) soit incohérente par rapport à la base dans laquelle l'image est parcimonieuse ; et la reconstruction non-linéaire, où le signal est reconstruit via une optimisation qui favorise la parcimonie, typiquement par la minimisation de la norme \(L_1\).  

Pour une image \(x \in \mathbb{R}^n\), le sinogramme sous-échantillonné \(y\) est obtenu par :
\[
y = \Phi x + \epsilon
\]
où \(y \in \mathbb{R}^m\) est le sinogramme (\(m \ll n\)), \(\Phi \in \mathbb{R}^{m \times n}\) représente l'opérateur de mesure (transformée de Radon partielle), et \(\epsilon\) est le bruit de mesure. Cette formulation traduit un problème inverse sous-déterminé.\vspace{7pt}\\

Pour garantir une reconstruction fiable, la matrice de mesure \(\Phi\) doit respecter la propriété d'isométrie restreinte (RIP) :
\[
(1-\delta_k)\|x\|_2^2 \leq \|\Phi x\|_2^2 \leq (1+\delta_k)\|x\|_2^2
\]
pour tous les vecteurs \(k\)-parcimonieux \(x\), avec \(\delta_k \in [0,1)\). Cette condition garantit que les distances entre vecteurs parcimonieux sont approximativement conservées après projection.\vspace{7pt}\\

Le théorème de la coupe centrale (Fourier Slice Theorem) établit que la transformée de Radon d'une image correspond à des lignes radiales dans son espace de Fourier. Ainsi, le sous-échantillonnage angulaire entraîne un sous-échantillonnage radial :
\[
\mathcal{F}_{1D}[\mathcal{R}f(\theta, s)](\omega) = \mathcal{F}_{2D}[f](\omega \cos\theta, \omega \sin\theta)
\]
En pratique, seules certaines lignes radiales sont acquises, ce qui crée des zones vides dans le plan de Fourier (Voir Annexe A pour l'implémentation).\vspace{7pt}\\

L'opérateur de Radon partiel \(\mathcal{R}_{\Theta}\) ne conserve qu'un sous-ensemble des angles :
\[
\mathcal{R}_{\Theta}f(s) = \{\mathcal{R}f(\theta, s) : \theta \in \Theta_{\text{sous}}\}
\]
On obtient ainsi un problème inverse sous-déterminé :
\[
y = \mathcal{R}_{\Theta}x
\]
Le taux de sous-échantillonnage est simplement le ratio entre le nombre d'angles acquis et le nombre total d'angles complets :
\[
\text{Taux} = \frac{M}{N} = \alpha
\]
où \(M\) est le nombre d'angles réellement acquis.\vspace{7pt}\\

La reconstruction consiste à résoudre l'optimisation :
\[
\hat{x} = \arg\min_{x} \|\Psi x\|_1 \quad \text{sous contrainte} \quad \|\mathcal{R}_{\Theta}x - y\|_2 \leq \epsilon
\]
où \(\Psi\) est un opérateur de transformation vers un domaine parcimonieux (i.e Bi-orthogonal Wavelet Transform, gradient, DCT), \(\|\cdot\|_1\) favorise la parcimonie, et \(\epsilon\) représente le niveau de bruit (Voir Annexe A pour l'implémentation).  

Dans la plupart des applications, la parcimonie est obtenue dans le domaine des ondelettes :
\[
x = \Psi c
\]
avec \(\|c\|_0 \ll n\) où \(\|c\|_0\) est le nombre de coefficients significatifs.\vspace{7pt}\\

Le pipeline complet (Voir Figure \ref{fig:sinogram_undersampling_pipeline}) du processus peut être visualisé comme suit : l'image originale \(x\) subit une transformée de Radon complète, puis un sous-échantillonnage angulaire aléatoire basé sur le facteur \(\alpha\) pour produire un sinogramme sous-échantillonné \(y\). Ce sinogramme est ensuite traité par un algorithme de reconstruction CS, qui utilise la minimisation L1 dans un domaine de parcimonie \(\Psi\) pour finalement produire l'image reconstruite \(\hat{x}\) (Voir Annexe A pour l'implémentation).

\begin{figure}[H]
    \centering
    \includegraphics[width=1\textwidth, height=1.2\textwidth]{images/undersample_sinogram.png}
    \caption{Pipeline de sous-échantillonnage angulaire}
    \label{fig:sinogram_undersampling_pipeline}
\end{figure}

Le nombre minimal d'angles pour une reconstruction fidèle est donné par :
\[
M \geq C \cdot k \cdot \log\left(\frac{n}{k}\right)
\]
où \(k\) est la parcimonie de l'image, \(n\) sa dimension et \(C\) une constante dépendant de l'opérateur de mesure. L'erreur de reconstruction est bornée par :
\[
\|\hat{x} - x\|_2 \leq C_1 \frac{\|x - x_k\|_1}{\sqrt{k}} + C_2 \epsilon
\]
avec \(x_k\) la meilleure approximation \(k\)-parcimonieuse de \(x\). Le processus reste robuste au bruit tant que :
\[
\frac{\|\epsilon\|_2}{\|y\|_2} \ll \frac{1}{\sqrt{\log n}}.
\]\vspace{7pt}\\

Dans les applications cliniques, le sous-échantillonnage angulaire permet de diminuer la dose d'irradiation proportionnellement au facteur \(\alpha\) :
\[
\text{Dose}_{\text{réduite}} = \alpha \cdot \text{Dose}_{\text{complète}}
\]
En imagerie cardiaque, l'acquisition sous-échantillonnée peut être synchronisée avec le cycle cardiaque :
\[
\Theta_{\text{cardio}} = \{\theta \in [0,180] : \text{phase cardiaque} = \text{cible}\}
\]
Pour l'imagerie dynamique (tomographie 4D), les angles peuvent varier dans le temps :
\[
y(t) = \mathcal{R}_{\Theta(t)}x(t)
\]



\subsection{Opérateur de Transformée en Ondelettes}

La transformation par ondelettes est un outil fondamental en traitement d'images et en optimisation, permettant de représenter une image dans une base multi-résolution. Elle établit une correspondance linéaire entre l'espace des pixels et l'espace des coefficients d'ondelettes. Le diagramme présenté à la Figure \ref{fig:wt_creation} illustre la construction de cet opérateur ainsi que ses transformations directe et adjointe. L'implémentation correspondante est fournie en Annexe~B.

\begin{figure}[H]
    \centering
    \includegraphics[width=1\textwidth, height=1.2\textwidth]{images/wt_creation.png}
    \caption{Opérateur de Transformée en Ondelettes}
    \label{fig:wt_creation}
\end{figure}

Mathématiquement, une image discrète de taille \(H \times W\) peut être représentée sous forme vectorielle \(x \in \mathbb{R}^n\), où \(n = H \times W\). L'opérateur de transformation en ondelettes est défini comme un opérateur linéaire :

\[
W : \mathbb{R}^n \rightarrow \mathbb{R}^m
\]

qui associe à chaque image \(x\) un vecteur de coefficients d'ondelettes :

\[
w = W x
\]

Ces coefficients représentent l'image dans une base multi-échelle composée de :

\[
w = \left\{ a_J, d_J^H, d_J^V, d_J^D, \dots, d_1^H, d_1^V, d_1^D \right\}
\]

où :

\begin{itemize}
\item \(a_J\) représente les coefficients d'approximation à basse résolution,
\item \(d_j^H, d_j^V, d_j^D\) représentent respectivement les coefficients de détail horizontal, vertical et diagonal au niveau \(j\).
\end{itemize}

Cette transformation correspond à une projection de l'image sur une base d'ondelettes :

\[
w_i = \langle x, \psi_i \rangle
\]

où \(\psi_i\) représente les fonctions de base d'ondelettes.

L'opérateur adjoint \(W^T\) permet la reconstruction de l'image à partir des coefficients :

\[
x = W^T w
\]

Dans le cas d'ondelettes orthogonales, on a la propriété importante :

\[
W^T = W^{-1}
\]

et donc :

\[
W^T W = I
\]

où \(I\) est l'opérateur identité.

Dans le cas général, \(W \in \mathbb{R}^{m \times n}\), avec :

\[
n = H \times W
\]
\[
m = \text{nombre total de coefficients}
\]

La transformation directe correspond à l'application :

\[
x \mapsto W x
\]

tandis que la transformation adjointe correspond à :

\[
w \mapsto W^T w
\]

Le diagramme de la Figure~\ref{fig:wt_creation} montre les différentes étapes de construction de cet opérateur :

\begin{itemize}
\item la définition de l'espace image \(\mathbb{R}^n\),
\item la construction de la structure des coefficients,
\item la définition de l'opérateur direct \(W\),
\item la définition de l'opérateur adjoint \(W^T\),
\item la création de l'opérateur linéaire complet.
\end{itemize}

Les coefficients sont organisés sous forme hiérarchique, permettant une représentation multi-échelle efficace. Pour assurer la reconstruction correcte, deux éléments essentiels sont conservés :

\begin{itemize}
\item la structure des coefficients,
\item les dimensions associées à chaque niveau de décomposition.
\end{itemize}

Cet opérateur possède plusieurs propriétés importantes :

\begin{itemize}
\item Linéarité :
\[
W(ax + by) = aWx + bWy
\]

\item Structure matricielle implicite :
\[
W \in \mathbb{R}^{m \times n}
\]

\item Adjoint bien défini :
\[
\langle Wx, w \rangle = \langle x, W^T w \rangle
\]

\item Conservation de l'énergie pour les bases orthogonales :
\[
\|Wx\|_2 = \|x\|_2
\]
\end{itemize}

En pratique, l'opérateur n'est pas construit explicitement sous forme matricielle, mais implémenté comme un opérateur linéaire implicite, comme illustré dans l'Annexe~B, afin d'éviter le stockage d'une matrice potentiellement très grande.

Cette structure permet d'intégrer efficacement la transformation en ondelettes dans des problèmes inverses, tels que :

\begin{itemize}
\item la reconstruction d'images,
\item le débruitage,
\item la compression,
\item les méthodes d'optimisation régularisées.
\end{itemize}
La transformation fournit ainsi une représentation compacte et multi-résolution de l'image, facilitant l'analyse et la manipulation des structures à différentes échelles.

\subsection{Reconstruction d'Images en Tomographie par Transformée de Radon}
La tomographie est une technique d'imagerie qui permet de visualiser l'intérieur d'un objet en trois dimensions sans avoir à le découper physiquement. Elle est largement utilisée en imagerie médicale, notamment dans les scanners à rayons X, mais aussi dans le contrôle non destructif des matériaux et en géophysique. Le principe fondamental de la tomographie repose sur l'acquisition d'un ensemble de projections de l'objet sous différents angles. Une projection peut être vue comme une ombre de l'objet, obtenue en mesurant l'atténuation d'un rayonnement qui traverse cet objet. Lorsque ces projections sont collectées pour de nombreux angles, elles forment un ensemble de données appelé sinogramme. Le défi principal consiste alors à reconstruire l'image originale de l'objet à partir de ce sinogramme.

La transformée de Radon est l'outil mathématique qui modélise ce processus d'acquisition des projections, et correspond à ce que l'on appelle la projection avant. Concrètement, si l'on considère une image bidimensionnelle représentant une coupe de l'objet, la transformée de Radon calcule, pour chaque angle donné, l'intégrale des valeurs de l'image le long de droites parallèles correspondant aux trajectoires des rayons. Chaque intégrale constitue un point dans la projection associée à cet angle. En répétant ce processus pour un ensemble d'angles, on obtient un sinogramme, dont l'axe horizontal représente la position des rayons sur le détecteur et l'axe vertical représente les angles de projection. D'un point de vue algébrique, cette opération peut être modélisée comme l'application d'un opérateur linéaire A qui transforme un vecteur x représentant l'image en un vecteur y représentant le sinogramme, selon la relation $y = A x$. Une implémentation pratique de cet opérateur sous forme sans matrice, utilisant la transformée de Radon, est présentée en Annexe C.

Pour reconstruire l'image originale à partir du sinogramme, il faut effectuer une opération inverse. Une première approche consiste à utiliser la rétroprojection, appelée aussi backprojection. Cette opération consiste à redistribuer les valeurs du sinogramme dans l'espace de l'image en suivant les mêmes trajectoires que celles empruntées par les rayons lors de l'acquisition. En pratique, chaque projection est projetée en retour sur l'image, et la somme de toutes ces contributions fournit une estimation de l'image originale. Mathématiquement, cette opération correspond à l'application de l'opérateur adjoint $A^T$ au sinogramme, ce qui donne une estimation de l'image selon la relation $x_{estimé} = A^T y$. Il est important de noter que l'opérateur adjoint n'est pas l'inverse exact de l'opérateur de projection, mais seulement son adjoint. Une implémentation de cet opérateur adjoint basée sur la rétroprojection non filtrée est également fournie en Annexe C.

Cependant, la rétroprojection simple produit généralement une image floue. Ce flou apparaît parce que le processus de projection favorise les basses fréquences de l'image, c'est-à-dire les variations lentes d'intensité, tandis que les hautes fréquences, correspondant aux détails fins et aux contours, sont atténuées. Lorsque l'on applique uniquement la rétroprojection simple, ce déséquilibre est amplifié, ce qui donne une image aux contours flous. Pour corriger ce problème, on utilise la rétroprojection filtrée, qui est la méthode standard en tomographie. Cette méthode consiste d'abord à appliquer un filtre aux projections du sinogramme, généralement un filtre passe-haut comme le filtre de Ram-Lak, afin de compenser la perte des hautes fréquences et de restaurer les détails. Ensuite, les projections filtrées sont rétroprojetées pour reconstruire l'image.
\begin{figure}[H]
\centering
\includegraphics[width=1\textwidth]{images/operateur_et_adjoint_CT_reconstruction.png}
\caption{Opérateur $A$ et son Adjoint $A^T$ pour la Reconstruction CT}
\end{figure}
Dans les applications réelles, la taille des images et des sinogrammes est très grande, ce qui rend impossible la construction explicite de la matrice $A$ représentant l'opérateur de Radon, car elle contiendrait des millions de lignes et de colonnes. Pour résoudre ce problème, on utilise une approche dite sans matrice, ou matrix-free. Dans cette approche, on ne construit pas la matrice elle-même, mais on définit uniquement deux fonctions : l'une qui calcule la projection avant d'une image, correspondant à l'application de $A x$, et l'autre qui calcule la rétroprojection d'un sinogramme, correspondant à l'application de $A^T$ y. Ces fonctions agissent comme une boîte noire qui encapsule le comportement de l'opérateur et de son adjoint, permettant de manipuler efficacement des problèmes de très grande taille sans stocker explicitement la matrice. Une implémentation complète de cet opérateur linéaire sous forme de LinearOperator est présentée en Annexe C.

Ainsi, la transformée de Radon modélise le processus d'acquisition des données, la rétroprojection simple correspond à l'opérateur adjoint et produit une image floue, tandis que la rétroprojection filtrée constitue la méthode standard de reconstruction en combinant filtrage et rétroprojection. Enfin, l'approche sans matrice permet de mettre en œuvre ces opérations de manière efficace, même pour des systèmes de très grande dimension, en évitant les limitations liées à la mémoire, comme illustré dans l'implémentation fournie en Annexe C.

\subsection{La Reconstruction Tomographique par Séparation de Variables (ADMM)}
La tomographie constitue un outil fondamental en imagerie médicale et dans diverses applications industrielles, permettant de reconstruire des structures internes à partir de mesures de projections acquises sous différents angles. La transformation de ces projections, regroupées sous forme de sinogramme, en une image cohérente représente un problème inverse complexe, particulièrement lorsque les données sont incomplètes, bruitées ou limitées en nombre. Les méthodes analytiques classiques atteignent rapidement leurs limites dans ces conditions, ce qui rend les approches itératives indispensables pour obtenir des reconstructions fiables.

Les méthodes itératives reposent sur un principe conceptuel simple mais efficace : partir d'une estimation initiale de l'image et l'affiner progressivement en comparant les projections simulées avec les mesures expérimentales. L'objectif est de déterminer l'image qui minimise la différence entre ses projections et les données observées tout en respectant certaines contraintes structurelles. Cette démarche s'inscrit naturellement dans un cadre d'optimisation, où l'on cherche à concilier fidélité aux données et régularisation, afin de garantir la stabilité et la qualité de la reconstruction. L'implémentation complète de cet algorithme itératif est présentée en Annexe D, offrant un exemple concret de mise en œuvre dans le contexte tomographique.

Parmi les approches itératives, la méthode de séparation de variables par directions alternées, connue sous le nom d'ADMM (Alternating Direction Method of Multipliers), constitue une solution particulièrement élégante et efficace. L'algorithme décompose un problème d'optimisation complexe en sous-problèmes plus simples, résolus de manière alternée. Dans le cadre de la reconstruction tomographique, cette séparation permet de traiter distinctement la fidélité aux données et la régularisation par parcimonie dans un domaine transformé, tel que les ondelettes. La mise en œuvre de ces mécanismes, incluant la mise à jour des variables primales et duales ainsi que la gestion de la représentation parcimonieuse, est explicitement détaillée dans l'implémentation fournie en Annexe D.

Le fonctionnement de l'algorithme repose sur trois étapes principales, illustrées dans la Figure~\ref{fig:ADMM_diagram_algorithm}, qui présente le flux complet de reconstruction par ADMM. La première consiste à mettre à jour l'image, en résolvant un système linéaire intégrant à la fois l'opérateur de projection et son adjoint, ainsi que l'opérateur de transformation. Cette étape garantit que l'image reconstruite est cohérente avec les mesures tout en restant proche d'une représentation transformée contrôlée. La deuxième étape consiste en la mise à jour de la représentation parcimonieuse, qui applique un seuillage doux afin de favoriser les coefficients significatifs et de réduire l'effet du bruit. La troisième étape met à jour la variable duale, qui assure la cohérence entre l'image reconstruite et sa représentation parcimonieuse, jouant un rôle crucial dans la stabilité et la convergence de l'algorithme. Toutes ces étapes sont implémentées et illustrées dans l'Annexe D.

La parcimonie constitue un a priori majeur en reconstruction d'images. En effet, les images naturelles et médicales possèdent une structure particulière qui permet de représenter l'information de manière concise dans un domaine approprié, tel que celui des ondelettes. L'encouragement de la parcimonie lors de la reconstruction conduit à des images plus nettes, avec des contours préservés, moins bruitées et capables d'être reconstruites de manière fiable même à partir d'un nombre limité de projections. Le choix du paramètre régulant l'équilibre entre fidélité aux données et régularisation parcimonieuse est déterminant pour obtenir des reconstructions à la fois précises et stables.

\begin{figure}[H]
\centering
\includegraphics[width=\textwidth]{images/admm_diagram.png}
\caption{Flux d'exécution de la reconstruction tomographique par ADMM}
\label{fig:ADMM_diagram_algorithm}
\end{figure}

Enfin, la convergence de l'algorithme est évaluée à l'aide de critères basés sur les résidus primal et dual, garantissant que les itérations successives atteignent un état stable et que l'image obtenue respecte à la fois les mesures et la parcimonie attendue. Cette approche itérative robuste et flexible est applicable dans de nombreux domaines, tels que la tomographie médicale (scanners et radiographies), le contrôle non destructif industriel, l'imagerie sismique et la microscopie électronique.

\section{Evaluation}
\section{Discussion}
\section{Applications}
\section{Innovation} % NOTRE MODELE