\chapter{Quelques modèles de reconstruction}
\section{Les traitements préalables à la reconstruction}

\subsection{\Large Méthodes dans le domaine spatial}
\subsubsection{\large Filtrage linéaire}
\subsubsection{\large Filtrage non linéaire}
\subsubsection{\large Filtres bilatéraux, médians, morphologiques}

\subsection{\Large Méthodes dans le domaine transformé}
\subsubsection{\large Transformée de Fourier}
\subsubsection{\large Filtre de Wiener}
\subsubsection{\large Transformée en cosinus discrète (DCT)}
\subsubsection{\large Ondelettes, curvelets, contourlets}
\subsubsection{\large Filtrage collaboratif BM3D}

% =================================================
\section{La transformée de Radon}
% =================================================
L'hypothèse fondamentale est que le détecteur mesure l'atténuation intégrée le long d'un rayon. 
\begin{definition}
    Pour un faisceau de rayons $\mathbf{X}$ d'énergie $\mathbf{E}$ donnée et un taux de propagation des photons $\mathbf{N}(x)$, l'intensité du faisceau $\mathbf{I}(x)$ à une distance $x$ de l'origine est définie comme \[\mathbf{I}(x) = \mathbf{N}(x) \mathbf{E}\]
\end{definition}

\begin{definition}
    La proportion de photons absorbés par millimètre de substance à une distance $x$ de l'origine est appelée le coefficient d'atténuation $\mathbf{A}(x)$ du milieu.
\end{definition}


Nous connaissons les intensités initiale et finale, $I_0$ et $I_1$ d'un faisceau unique. L'objectif est d'utiliser ces intensités pour déterminer le coefficient d'atténuation le long du trajet du faisceau. Heureusement, la loi de Beer-Lambert établit une relation entre ces deux grandeurs.

\begin{definition}[Loi de Beer-Lambert]
Pour un faisceau de rayons X monochromatique, non réfractif et de largeur nulle,
traversant un milieu homogène sur une distance \(x\) à partir de l'origine,
l'intensité \(I(x)\) est donnée par
\begin{equation}
    I(x) = I_0 e^{-\mathbf{A}(x)x}
    \label{eq:loi_beer_lambert}
\end{equation}
\end{definition}
En l'état, cette équation ne nous est pas particulièrement utile. Elle exprime le coefficient d'atténuation en un point donné en fonction de l'intensité en ce point, alors que nous ne connaissons la valeur de l'intensité qu'en des points situés à l'extérieur de l'objet. Ce que nous cherchons réellement est une relation entre le coefficient d'atténuation à l'intérieur de l'objet et la variation de l'intensité du faisceau. Pour cela, nous allons manipuler légèrement l'équation \eqref{eq:loi_beer_lambert}.\\
En passant à  la dérivée de la loi de Beer-Lambert, nous obtenons la relation suivante :
\[
    \frac{dI}{dx} = -\mathbf{A}(x)I(x)
\]
Soit $I(x_0)=I_0$ la valeur initiale de l'intensité du faisceau et $I(x_1)=I_1$ la valeur finale de l'intensité du faisceau. En utilisant cette relation, nous obtenons la relation suivante :

\[
    -\int_{x_0}^{x_1} \mathbf{A}(x)dx = \int_{x_0}^{x_1}\cfrac{dI}{I(x)}=ln(\frac{I_1}{I_0})
\]
ou encore \vspace{10pt}
\begin{equation}
    \int_{x_0}^{x_1} \mathbf{A}(x)dx = ln(\frac{I_0}{I_1})
\end{equation}

Nous sommes maintenant prêts à introduire des outils mathématiques — en particulier la transformée de Radon — qui joueront un rôle central dans la détermination du coefficient d'atténuation dans l'équation \eqref{eq:loi_beer_lambert}.

L'écriture sous forme normale d'une équation de droite joue un rôle clé dans la transformée de Radon, car elle permet une paramétrisation naturelle et complète de toutes les droites du plan, ce qui est essentiel pour la définition mathématique et le calcul pratique de cette transformation.\\ 
Cette équation sous forme normale fournit :
\begin{itemize}
    \item[(i)] Une paramétrisation unique et continue de toutes les droites du plan. La forme normale (ou forme normale de Hesse) de l'équation de la droite s'écrit : $$x\,\cos(\theta)+y\,\sin(\theta)=\rho$$ où $\rho$ est la distance par rapport à l'origine et $\theta$ est l'angle par rapport à l'axe des abscisses.
    \item[(ii)] Une interprétation géométrique claire de $\rho$ et $\theta$. Chaque droite du plan correspond  à un unique couple ($\rho,\theta$). Cette paramétrisation évite les redondances et garantit qu'on parcourt toutes les droites une et une seule fois (à une convention près).
    \item[(iii)] Une mesure naturelle sur l'espace des droites, utilisée dans les formules d'inversion.
    \item[(iv)] Un formalisme adapté au théorème de coupe, reliant transformée de Radon et transformée de Fourier 2D. 
    \item[(v)] Une mesure naturelle sur l'espace des droites, utilisée dans les formules d'inversion.
\end{itemize}\vspace{10pt}
\subsubsection{\small Construction de l'orientation et de la distance}
Nous connaissons tous l'idée qu'une droite \( l \) dans \( \mathbb{R}^2 \) peut être représentée par l'équation 
\[
ax + by = c
\]
où \( a, b, c \in \mathbb{R} \) et \( a^2 + b^2 \neq 0 \).\\ On peut alors écrire cette équation d'une droite sous la forme \[w_1x + w_2y = t\]
où $w:=(w_1, w_2) = (\cfrac{a}{\sqrt{a^2 + b^2}}, \cfrac{b}{\sqrt{a^2 + b^2}})$ et $t=\cfrac{c}{\sqrt{a^2 + b^2}}$, que nous pouvons voir comme un point situé sur le
cercle unitaire, pour \[\left(\cfrac{a}{\sqrt{a^2 + b^2}}\right)^{2} + \left(\cfrac{b}{\sqrt{a^2 + b^2}}\right)^{2} = 1\]
Cela implique que $w := (\cos(\theta), \sin(\theta)) \text{ est un vecteur normal unitaire }$, $\theta \in [0, 2\pi)$ représente l'orientation, et $t$ est exactement la distance à l'origine. On a \[x\cos(\theta) + y\sin(\theta) = t\]
Notez que dans les équations ci-dessus, $t$ et $\theta$ sont fixes et déterminent une droite spécifique \( l \) dans le plan. On peut donc dire que $t$ et $\theta$ paramètrent une droite \( l_{t,\theta} \) et que $z$ détermine des points spécifiques sur cette droite \( l \). Ou encore
\[l_{t,\theta} = \{ z \in \mathbb{R}^2 : \langle z, (\cos \theta, \sin \theta) \rangle = t \}.\]

\begin{definition}[Transformée de Radon]
Soit \( f(t,\theta) \) une fonction définie sur \( \mathbb{R}^2 \) à support compact.
La transformée de Radon de \( f \), notée \( \mathcal{R}f \), est définie pour
\( t \in \mathbb{R} \) et \( \theta \in (0, 2\pi] \) par
\[
\mathcal{R}f(t,\theta) = \int_{-\infty}^{\infty} f(x(s),y(x))\mathrm{d}s
\]
\end{definition}

La transformée de Radon permet de déterminer la densité totale d'une fonction $f$ le long d'une droite donnée $l$. Cette droite $l$ est définie par un angle $\theta$  par rapport à l'axe 
$x$ et une distance $t$ par rapport à l'origine. Comme illustré à la \Cref{fig:radon}, si l'on calcule la transformée de Radon le long de plusieurs droites à des angles différents (ici $\theta_1$ et $\theta_2$), on peut déterminer plusieurs fonctions de densité pour notre objet. Intuitivement, on peut interpréter la transformée de Radon comme une version « étalée » de notre objet initial. Supposons que la région en forme de tache représentée à la \Cref{fig:radon} soit une tache d'encre ; si l'on étale cette tache le long de lignes de direction $\theta_1$
, on s'attend à ce que les régions les plus larges de la tache correspondent à des zones plus étendues que les régions plus petites, ce qui correspond exactement à ce que l'on observe.
\begin{figure}[H]
    \centering
    \includegraphics[width=0.8\textwidth]{./images/radon.png}
    \caption{Transformée de Radon pour $\theta_1$ et $\theta_2$.}
    \label{fig:radon}
\end{figure}

L'objectif est maintenant de trouver une formule d'inversion de la transformée de Radon qui nous permettra de reconstruire la fonction initiale $f$ (ou, dans le contexte de l'imagerie médicale, 
$\mathbf{A}(x)$). Pour ce faire, il sera utile de rappeler plusieurs propriétés de la transformée de Radon.
\begin{proposition}
    Soit $\alpha$ et $\beta$ deux réels et $f$ et $g$ deux fonctions continues sur $\mathbb{R}^2$ à support compact. On a
    \begin{itemize}
        \item[(i)] Linéarité : $\mathcal{R}(\alpha f + \beta g) = \alpha \mathcal{R}f + \beta \mathcal{R}g$
        \item[(ii)] Parité: $\mathcal{R}f(-t,-\theta) = \mathcal{R}f(t,\theta)$
        \item[(iii)] $\mathcal{R}f(t, \theta) = \int_{-\infty}^{\infty} f(x(s), y(s))\mathrm{d}s = \int_{-\infty}^{\infty} f(t\,cos(\theta)-s\,sin(\theta), t\,sin(\theta)+s\,cos(\theta))\mathrm{d}s$
        % \item[(iv)] Invariance par rotation : \(\mathcal{R}(f \circ R_{\psi}) = \mathcal{R}f(t,\theta - \psi)\)
        % \item[(v)] Relation avec la convolution : \(\mathcal{R}(f * g) = \mathcal{R}f * \mathcal{R}g\)
    \end{itemize}
\end{proposition}
Nous définissons en outre le domaine naturel de la transformée de Radon comme l'ensemble des fonctions $f$ sur $\mathbb{R}^2$ telles que \[\int_{-\infty}^{\infty} |f(x(s), y(s))|\mathrm{d}s < \infty\]
\section{Transformée de Fourier}
\begin{definition}[Transformée de Fourier]
    Soit \( f \) une fonction absolument intégrable sur \( \mathbb{R} \).
    La transformée de Fourier de \( f \), notée \( \mathcal{F}f \), est définie
    pour tout nombre réel \( \xi \) par
    \[
    (\mathcal{F}f)(\xi)
    = \int_{-\infty}^{\infty} f(x)\, e^{-2\pi i \xi x}\, dx.
    \]
\end{definition}
La transformée de Fourier est fréquemment utilisée en analyse du signal et permet de transformer une fonction du temps en une fonction de la fréquence ; la variable $x$ représente le temps en secondes et la variable \( \xi \) représente la fréquence de la fonction en hertz.\\

Il existe une définition alternative faisant intervenir la fréquence angulaire $w=2\pi \xi$, ce qui conduit à l'expression suivante.
\[(\mathcal{F}f)(w) = \int_{-\infty}^{\infty} f(x)\, e^{-i w x}\, dx\]
Comme pour la transformée de Radon, nous allons énumérer plusieurs propriétés de la transformée de Fourier.
\begin{proposition}
    Pour des constantes réelles $\alpha$ et $\beta$, et des fonctions absolument intégrables $f$ et $g$, on a:
    \begin{itemize}
        \item[(i)] Linéarité : $\mathcal{F}(\alpha f + \beta g)(w) = \alpha \mathcal{F}f(w) + \beta \mathcal{F}g(w)$
        \item[(ii)] $\mathcal{F}f(w) < +\infty$
    \end{itemize}
\end{proposition}

\begin{definition}[Transformée de Fourier inverse]
Soit \( f \) une fonction absolument intégrable.
La transformée de Fourier inverse de \( f \), notée \( \mathcal{F}^{-1}f \),
évaluée en \( x \), est définie par
\[
(\mathcal{F}^{-1}f)(x)
= \cfrac{1}{2\pi}\int_{-\infty}^{\infty} f(w)\, e^{iw x}\, dw.
\]
\end{definition}
Ceci nous conduit immédiatement au théorème suivant.
\begin{proposition}[Théorème d'inversion de Fourier]
Soit $f$ une fonction absolument integrale sur $\mathbb{R}$.
Le théorème d'inversion de Fourier affirme que, pour tout \( x \),
\[
(\mathcal{F}^{-1} \circ \mathcal{F})f(x)=f(x)
\]
\end{proposition}
Jusqu'à présent, nous n'avons abordé la transformée de Fourier que dans une dimension. Il existe des définitions correspondantes en dimensions supérieures, mais, pour nos besoins, nous n'utiliserons que les analogues en deux dimensions.

\begin{definition}[Transformée de Fourier bidimensionnelle]
Soit \( g \) une fonction absolument intégrable définie sur \( \mathbb{R}^2 \).
La transformée de Fourier bidimensionnelle de \( g \), notée \( \mathcal{F}_2 g \),
est définie pour tout \((X,Y) \in \mathbb{R}^2\) par
\begin{equation}
    (\mathcal{F}_2 g)(X,Y) = \int_{-\infty}^{\infty} \int_{-\infty}^{\infty} 
    g(x,y)\, e^{-i (xX + yY)} \, dx\, dy.
    \label{eq:fourier_2d}
\end{equation}

\end{definition}

De manière similaire, nous définissons la transformée de Fourier inverse sur $\mathbb{R}^2$.
\begin{definition}[Transformée de Fourier bidimensionnelle inverse]
Soit \( g \) une fonction absolument intégrable définie sur \( \mathbb{R}^2 \).
La transformée de Fourier bidimensionnelle inverse de \( g \), évaluée en \((x,y)\)
et notée \( \mathcal{F}_2^{-1} g(x,y) \), est donnée par
\[
(\mathcal{F}_2^{-1} g)(x,y) = \cfrac{1}{4\pi^2}\int_{-\infty}^{\infty} \int_{-\infty}^{\infty} 
g(X,Y)\, e^{i (xX + yY)} \, dX\, dY.
\]
\end{definition}

\subsection{Le Théorème de la Coupe Centrale}
Le théorème de la coupe centrale, également appelé théorème de projection-transforme de Fourier ou théorème de Fourier-Slice, est un résultat fondamental en traitement d'image et en tomographie. Il établit un lien profond entre la transformée de Radon (utilisée pour décrire les projections d'un objet) et la transformée de Fourier (utilisée pour analyser les fréquences spatiales). Ce théorème constitue la pierre angulaire mathématique de la plupart des méthodes de reconstruction tomographique moderne.

\begin{proposition}
    Soit \( g \) une fonction absolument integrale sur \( \mathbb{R}^2 \).
    Le theorem de la coupe centrale affirme que, pour tout $S \in \mathbb{R}$ et $\theta \in [0,2\pi]$, on a : \[\mathcal{F}_2 g(S\cos(\theta), S\sin(\theta)) = \mathcal{F}(\mathcal{R}f)(S, \theta)\]
\end{proposition}
\textbf{Preuve}: En utilisant la définition de la transformée de Fourier bidimensionnelle \eqref{eq:fourier_2d} on obtient 
\[
    \mathcal{F}_{2}g(S\,\cos(\theta), S\,\sin(\theta)) = \int_{-\infty}^{\infty} \int_{-\infty}^{\infty} g(x, y)\, e^{-iS (x\,\cos(\theta) + y\,\sin(\theta))}\, dx\, dy
\]
\section{Inversion analytique de la transformée de Radon}
\section{Rétroprojection filtrée (FBP) : aspects algorithmiques}