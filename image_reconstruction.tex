\documentclass[12pt, a4paper, oneside]{extbook}
\usepackage[T1]{fontenc}
\usepackage[provide=*,french]{babel}
\usepackage[utf8]{inputenc}
\usepackage[top=2cm,right=3cm,left=2cm]{geometry}
\usepackage{soul}

\usepackage{amsmath, amssymb, amsthm, mathrsfs}
\usepackage{xcolor}
\usepackage{import}
\usepackage{bbm}
\usepackage{comment}

\usepackage[hidelinks]{hyperref}
\usepackage{cleveref}

\usepackage{graphicx}
\usepackage{caption}
\usepackage{subcaption}
\usepackage{tabularx}
\usepackage{titlesec}
\usepackage{matlab-prettifier}
\usepackage{float}
\usepackage{minted}
\usepackage{fancyhdr}
\usepackage{longtable}
\usepackage{algorithm}
\usepackage{algpseudocode} 
\usepackage{tocloft}
\usepackage{pdfpages}  % Pour insérer des PDF


% Show Prop, Lemma and other label near them
% \usepackage{showkeys}

\numberwithin{equation}{chapter}

% \setcounter{chapter}{-1}

% ================================================================
\theoremstyle{definition}
\newtheorem{definition}{Définition}[section]
\newtheorem{proposition}{Proposition}[section]
\newtheorem{lemme}{Lemme}[section]
\newtheorem{corollaire}[proposition]{Corollaire}
\newtheorem{property}[proposition]{Propriété}
% ================================================================

% =================================================================
% \title{\Huge Reconstruction d'images issues de CT}
% \author{
%     Réalisé par \\\Large Frederic Andrianarivony\vspace{2cm}\\
%     Option: \\ M2VR T.A.S.I
% }
% \date{\today} % You can also specify a date manually or leave it blank
% =================================================================
% CHAPTER STYLE
% numbering
\renewcommand{\thechapter}{\Roman{chapter}}

\renewcommand{\cftchappresnum}{CHAPITRE }
\renewcommand{\cftchapaftersnum}{: }
\setlength{\cftchapnumwidth}{8.5em}
% =================================================================
% FIGURES

% Numérotation des figures en chiffres arabes
\renewcommand{\thefigure}{\arabic{chapter}.\arabic{figure}}

% ===== LISTE DES FIGURES : STYLE PERSONNALISÉ =====
% Titre
% ===== LISTE DES FIGURES (babel french fix) =====
\addto\captionsfrench{%
  \renewcommand{\listfigurename}{\large LISTE DES FIGURES}
}

% Préfixe "Figure"
\renewcommand{\cftfigpresnum}{Figure }
\renewcommand{\cftfigaftersnum}{ : }

% Largeur pour "Figure 1.10 :"
\setlength{\cftfignumwidth}{6em}
% =================================================================
% SECTION STYLE
% font_size
\titleformat{\section}
  {\normalfont\fontsize{12}{14}\selectfont\bfseries}
  {\thesection}{1em}{}

\titleformat{\subsection}
  {\normalfont\fontsize{12}{14}\selectfont\bfseries}
  {\thesubsection}{1em}{}
% =================================================================
% DEFINITON STYLE
% text (it, gras, ...)
\begingroup
    \makeatletter
    \g@addto@macro\th@definition{\normalfont}
    \makeatother
\endgroup
% =================================================================
\let\cleardoublepage\clearpage
\setminted{frame=lines, framesep=2mm, fontsize=\scriptsize, linenos}
% ================================================================
% PAGE NUMBERING STYLE
\pagestyle{fancy}
\fancyhf{} % Clear all header and footer fields
\fancyfoot[C]{\thepage} % Page number in the center of the footer
\renewcommand{\headrulewidth}{0pt} % Remove header rule
\renewcommand{\footrulewidth}{0pt} % Remove footer rule if you want
\fancyhead{} % Clear all header content
% ================================================================



\begin{document}
    \includepdf[pages=-, fitpaper=true]{Couverture MVR TASI.pdf}

    % blan page
    \newpage
    \thispagestyle{empty}
    ~
    \newpage
    % Remerciements
    % Page de remerciements
\chapter*{\large REMERCIEMENTS} % * pour ne pas numéroter le chapitre
\thispagestyle{empty} % pas de numéro de page pour cette page
\begin{flushleft} % texte aligné à gauche
Je souhaite exprimer ma profonde gratitude à Dieu Tout-Puissant, source de sagesse et d'inspiration, pour son amour, sa miséricorde et sa guidance tout au long de mes études et pour la réalisation de ce mémoire.\vspace{8pt}\\

Je remercie chaleureusement Monsieur le Chef du Département Électronique de l'ESPA, Mr. … pour avoir accepté de présider ma soutenance, ainsi que Madame et Messieurs les membres du jury, composés de Madame …, Monsieur … et Monsieur …, pour leurs conseils, corrections et remarques pertinentes qui ont grandement enrichi ce travail.\vspace{8pt}\\

Je suis particulièrement reconnaissante à mon encadreur, Madame RAMAFIARISONA Malalatiana, pour son savoir, sa pédagogie et son accompagnement constants, qui m'ont permis de progresser et de mener ce projet à terme avec rigueur et confiance.\vspace{8pt}\\

Mes remerciements vont également à toute l'équipe pédagogique du Département Électronique de l'ESPA pour les connaissances et compétences transmises au cours de ce Master à visée de recherche.\vspace{8pt}\\

Enfin, je tiens à exprimer ma gratitude à ma famille, pour son soutien moral, matériel et affectif, qui a été une source constante de motivation et d'encouragement.\vspace{8pt}\\

Que la paix et la bénédiction de Dieu soient avec vous tous.\vspace{8pt}\\

\medskip
\hfill\textbf{Fréderic ANDRIANARIVONY}
\end{flushleft}
    \newpage

    % -- TODO --
    % Resumé

    \pagenumbering{roman}
    \setcounter{page}{1}
    % Table des matières
    \tableofcontents
    % -- TODO --
    % Liste des Abréviations

    % Liste des figures
    \listoffigures

    \newpage
    \pagenumbering{arabic}
    \setcounter{page}{1}

    % ================================================================
    % ADD CHAPTERS HERE
    %===========================================================
% Chapitre 1 — Introduction générale
%===========================================================

\chapter{Introduction générale}

\section{Historique et généralités}
Appareils photo numériques, scanners, images météo $\cdots$ les images ont naturellement envahi notre vie quotidienne. Leur traitement est désormais devenu commun dans beaucoup de domaines.
La reconstruction d'image est un processus fondamental en imagerie médicale, scientifique et numérique, visant à reconstituer une image 3D ou une représentation précise à partir de données acquises de manière indirecte, souvent sous forme de projections 2D.

\subsubsection{Origines et développement historique}
Les fondements théoriques de la reconstruction tomographique remontent au début du XXe siècle. En 1917, le mathématicien J. Radon a formulé une théorie mathématique permettant de reconstruire une fonction à partir de ses projections.
Cependant, ce n'est qu'avec les progrès des calculateurs numériques dans les années 1960-1970 que cette théorie a pu être appliquée concrètement. Godfrey Hounsfield, ingénieur britannique, a mis au point le premier scanner TDM (tomodensitométrie ou CT) en 1971, grâce à un financement indirect provenant de la société EMI, associée au succès des Beatles. Ce développement a marqué la naissance de l'imagerie médicale tomographique, suivie par l'IRM et la tomoscintigraphie.

\subsubsection{Évolution avec l'intelligence artificielle}
Depuis 2012, l'apprentissage profond a révolutionné le traitement d'images.
Contrairement aux méthodes traditionnelles basées sur des étapes séquentielles (extraction de caractéristiques → reconnaissance), les réseaux de neurones profonds apprennent directement à partir d'images brutes, en utilisant des vérités terrain (annotations manuelles).
Des architectures comme les réseaux antagonistes génératifs (GAN), les transformeurs ou les auto-encodeurs sont désormais utilisées pour la reconstruction d'images, la segmentation, le recalage ou la réduction de bruit

% Un sujet très précis et intéressant ! La reconstruction d'images à partir de projections est un aspect crucial de l'imagerie médicale, en particulier dans des modalités telles que la tomodensitométrie (CT) et la tomographie par émission de positons (TEP).



% ==============================================================================================================================================================
% \section{Cadres de reconstruction d'images}
% \subsection{Méthodes directes et régularisation algébrique}
% Pour les problèmes linéaires ($\mathcal{A}$ représenté par une matrice $\mathbf{A}$), la méthode des moindres carrés conduit naturellement à considérer le \textbf{pseudo-inverse} de Moore-Penrose :

% \begin{equation}
%     \mathbf{x}_{\text{LS}} = \mathbf{A}^\dagger \mathbf{y} = (\mathbf{A}^T\mathbf{A})^{-1}\mathbf{A}^T\mathbf{y}
%     \label{eq:pseudo_inverse}
% \end{equation}

% où $\mathbf{A}^\dagger$ désigne le pseudo-inverse. Cependant, cette approche naïve est généralement inapplicable en pratique à cause de la mal-positude :
% \begin{itemize}
%     \item[-] Si $\mathbf{A}^T\mathbf{A}$ est singulière ou mal conditionnée, son inversion amplifie démesurément le bruit.
%     \item[-] Dans le cas sous-déterminé ($m < n$), la solution des moindres carrés n'est pas unique.
% \end{itemize}

% La \textbf{régularisation} vise à stabiliser le problème en incorporant un terme de pénalité reflétant des connaissances a priori. La régularisation de Tikhonov, par exemple, résout :

% \begin{equation}
%     \min_{\mathbf{x}} \left\{ \|\mathbf{A}\mathbf{x} - \mathbf{y}\|_2^2 + \lambda \|\mathbf{\Gamma}\mathbf{x}\|_2^2 \right\}
%     \label{eq:tikhonov}
% \end{equation}

% où $\lambda > 0$ est un paramètre de régularisation et $\mathbf{\Gamma}$ un opérateur (souvent la dérivée première, imposant une régularité spatiale). La solution régularisée s'écrit alors :

% \begin{equation}
%     \mathbf{x}_{\lambda} = (\mathbf{A}^T\mathbf{A} + \lambda \mathbf{\Gamma}^T\mathbf{\Gamma})^{-1} \mathbf{A}^T\mathbf{y}
%     \label{eq:tikhonov_solution}
% \end{equation}

% Cette formulation peut être interprétée comme une version \textit{corrigée} du pseudo-inverse, où l'adjonction du terme $\lambda \mathbf{\Gamma}^T\mathbf{\Gamma}$ améliore le conditionnement numérique au prix d'un biais contrôlé.

% \subsection{Méthodes variationnelles et parcimonieuses}

% Les méthodes variationnelles généralisent l'approche de Tikhonov en considérant des fonctionnelles de régularisation plus sophistiquées :

% \begin{equation}
%     \min_{\mathbf{x}} \left\{ \mathcal{D}(\mathbf{A}\mathbf{x}, \mathbf{y}) + \lambda \mathcal{R}(\mathbf{x}) \right\}
%     \label{eq:variational}
% \end{equation}

% où $\mathcal{D}$ est un terme d'attache aux données (pas nécessairement quadratique) et $\mathcal{R}$ une pénalité reflétant des propriétés a priori (parcimonie, variation totale, etc.). Ces méthodes exploitent notamment la \textbf{parcimonie} des signaux d'intérêt dans des bases ou redondances appropriées.

% \subsection{Théorie du Compressive Sensing}

% Le \emph{Compressive Sensing} (CS) révolutionne l'acquisition en démontrant qu'un signal parcimonieux peut être exactement reconstruit à partir d'un nombre très réduit de mesures non adaptatives, pourvu que l'opérateur d'acquisition vérifie certaines propriétés (RIP, incohérence). Le problème de reconstruction en CS s'écrit typiquement :

% \begin{equation}
%     \min_{\mathbf{x}} \|\mathbf{\Psi}\mathbf{x}\|_1 \quad \text{sous la contrainte} \quad \|\mathbf{A}\mathbf{x} - \mathbf{y}\|_2 \leq \epsilon
%     \label{eq:cs}
% \end{equation}

% où $\mathbf{\Psi}$ est une transformée (ondelettes, DCT, etc.) dans laquelle le signal est parcimonieux. Cette approche fournit un cadre théorique solide pour les problèmes fortement sous-déterminés.

% \subsection{Méthodes d'apprentissage profond}

% Les méthodes d'apprentissage profond, et particulièrement les réseaux de neurones, offrent une alternative puissante aux approches variationnelles. Elles apprennent directement, à partir de grandes quantités de données, un opérateur de reconstruction :
% \begin{equation}
%     \hat{\mathbf{x}} = f_\theta(\mathbf{y})
%     \label{eq:deep_learning}
% \end{equation}
% où $f_\theta$ est un réseau de neurones paramétré par $\theta$. Ces méthodes peuvent être conçues pour imiter des algorithmes d'optimisation (\emph{unrolling}), apprendre des régularisations implicites, ou générer des reconstructions par inversion directe apprise.
% ==============================================================================================================================================================
    \chapter{Outils mathématiques pour la reconstruction d'image}

Un sujet très précis et intéressant ! La reconstruction d'images à partir de projections est un aspect crucial de l'imagerie médicale, en particulier dans des modalités telles que la tomodensitométrie (CT) et la tomographie par émission de positons (TEP).
Les méthodes analytiques directes sont l'approche historique et mathématiquement élégante des problèmes inverses linéaires, particulièrement en tomographie. Leur positionnement répond à un impératif de rapidité de calcul dans des applications où le temps de reconstruction est critique (imagerie médicale clinique, contrôle non destructif industriel).

\section{Classification des approches de reconstruction}
Le paysage algorithmique de la tomographie se divise principalement en trois familles, distinguées par leur traitement de l'opérateur de projection.

\subsection{\Large Les méthodes analytiques}
Elles reposent sur une formulation mathématique explicite de l'inversion de l'opérateur direct. En s'appuyant sur les propriétés de la transformée de Radon, elles permettent une reconstruction directe et rapide. L'exemple le plus emblématique reste la Rétroprojection Filtrée (FBP). C'est une méthode analytique largement utilisée qui consiste à filtrer les projections et à les rétroprojeter sur la grille d'image. FBP est rapide et efficace, mais peut être sensible au bruit et aux artefacts.

\subsubsection{\large La transformée de Radon}
Imaginons qu'on ait un objet opaque constitué de différents matériaux, et que l'on souhaite savoir comment ces matériaux sont répartis à l'intérieur sans l'endommager (par exemple, l'objet peut être un malade à l'intérieur du corps duquel on aimerait voir). L'une des méthodes est le scanner  : on lance de fins faisceaux de rayons X à travers l'objet dans toutes les directions et on mesure quelle proportion de chaque faisceau a été absorbée.

La tomographie est un procédé permettant de créer l'image d'un objet en deux ou trois dimensions à partir de multiples "coupes" unidimensionnelles (Voir \Cref{fig :tomography_device}). Dans un scanner CT (tomodensitométrie), ces coupes sont définies par des faisceaux de rayons $\mathbf{X}$ parallèles projetés à travers l'objet. En changeant l'orientation de la source et du détecteur (l'angle \(\theta\)), on obtient des informations sur la densité interne sous différents angles.\\
Le fonctionnement repose sur la mesure de l'intensité des rayons $\mathbf{X}$ :
\begin{itemize}
    \item[-] \textbf{Perte d'énergie}  : Lorsqu'un rayon $\mathbf{X}$ traverse un objet, il perd une partie de son énergie, ce qui réduit son intensité
    \item[-] \textbf{Coefficient d'atténuation}  : Cette perte dépend de la densité du milieu. Les objets denses (comme l'os) provoquent une variation d'intensité plus importante que les tissus moins denses. Cette caractéristique est appelée le coefficient d'atténuation ($A(x, y)$). L'atténuation mesurée pour chaque faisceau, c'est-à-dire la différence entre l'intensité incidente et l'intensité détectée, correspond à une intégrale de ligne de la structure interne de l'objet. Cette relation entre l'objet et l'ensemble de ses intégrales de ligne est formalisée par la transformée de Radon. La reconstruction de l'image originale repose alors sur l'inversion de cette transformée, qui constitue le fondement théorique de la tomographie assistée par ordinateur
    \item[-] \textbf{Mesures}  : Le scanner enregistre l'intensité initiale émise ($I_{0}$) et l'intensité finale reçue ($I_{1}$) pour chaque faisceau afin de déduire la densité globale rencontrée sur le trajet
\end{itemize}

\begin{figure}[H]
    \centering
    \includegraphics[width=0.8\textwidth]{./images/ct_device.png}
    \caption{Un appareil de tomographie est une sorte d'anneau dans lequel on place un objet ou une personne, qui sont alors traversés par un faisceau de rayons X « suivant toutes les directions » comme illustré sur l'image ci-dessus où l'on étudie la structure d'objets anciens.}
    \label{fig :tomography_device}
\end{figure}

\subsubsection{La fonction d'atténuation représentant l'objet étudié}
L'objet initial, considéré comme plan, est donné par une fonction d'atténuation qui, à chaque point du plan de coordonnées $(x, y)$, va associer un nombre $A(x, y)$ correspondant à la proportion des rayons qui sont absorbés par le matériau en ce point  : en un point d'un os, $A$ sera grand, et en un point de l'air, il sera faible.

\subsubsection{Une loi physique}
En supposant dans un premier temps que la fonction d'atténuation de notre objet est constante égale à $a$, pour tout rayon lumineux traversant notre objet, pour tout couple de points d'abscisses $x$ et $x+l$ sur ce rayon, les abscisses étant croissantes dans le sens du rayon, le rapport d'intensités lumineuses $\cfrac{I(x+l)}{I(x)}$ ne dépend que de $a$ et de la longueur $l$ traversée et pas du point $x$ (position).

En omettant provisoirement la dépendance par rapport à $a$ et en notant alors $p(l)$ ce rapport $\cfrac{I(x+l)}{I(x)}$ qui correspond à la proportion de photons non
absorbés sur une longueur $l$ à partir d'un point $x$, on voit que $p$ vérifie la propriété \[p(l_1+l_2) = p(l_1)p(l_2)\]
En Effet, la proportion de photons non absorbés sur une longueur $l_2$ à partir d'un point $x+l_1$ est $\cfrac{I(x+l_1+l_2)}{I(x+l_1)}=p(l_2)$. Donc $p(l_1)p(l_2)=\cfrac{I(x+l_1)}{I(x)} \times \cfrac{I(x+l_1+l_2)}{I(x+l_1)}=p(l_1+l_2)$. Ceci traduit juste le fait simple suivant  : les proportions de photons non
absorbés se multiplient lors de traversées successives de milieux absorbants. La
bonne définition de l'atténuation est précisément : 
\begin{equation}
    p(l)=e^{-a\, l}
    \label{eq:init_loi_beer_lambert_attenuation}
\end{equation}

Autrement dit, pour tout $x$ et $x+l$ sur un axe  : 
\begin{equation}
    I(x+l)=I(x)e^{-a\, l}
    \label{eq:init_loi_beer_lambert}
\end{equation}

Notons que si le phénomène physique d'atténuation est spécifique de la tomographie à rayons X, les méthodes de reconstruction sont en revanche plus générales et sont appliquées également dans d'autres systèmes d'imagerie, dans lesquelles des équations analogues expriment une fonction à reconstruire en fonction de projections. C'est le cas par exemple de la tomographie d'émission de simples photons utilisée en médecine nucléaire.

\subsection{\Large Les méthodes itératives}
Ces approches traitent la reconstruction comme un problème d'optimisation numérique. Elles cherchent à minimiser un critère d'erreur (souvent par les moindres carrés ou le maximum de vraisemblance) entre les projections mesurées et celles simulées à partir d'une image estimée. Elles sont particulièrement robustes face aux données bruitées ou incomplètes.
Un aspect important des sciences physiques consiste à inférer des paramètres physiques à partir de données. En général, les lois de la physique permettent de calculer les valeurs des données étant donné un modèle. C'est ce qu'on appelle le problème direct.

Le \textbf{problème inverse}, quant à lui, vise à reconstruire le modèle à partir d'un ensemble de mesures. Ce paradigme trouve une application centrale dans le domaine de la \textbf{reconstruction d'image}, où l'on s'efforce de retrouver une image -- représentant par exemple une distribution de densité, une structure anatomique ou une source astrophysique -- à partir de données acquises de manière indirecte, sous-échantillonnée ou bruitée. Que ce soit en tomographie, en imagerie par résonance magnétique (IRM) ou en astronomie, la reconstruction repose sur l'inversion d'un modèle direct qui décrit le processus physique d'acquisition.

Dans le cas idéal, il existe une théorie exacte qui prescrit comment les données doivent être transformées pour reproduire le modèle ou l'image recherchée. Pour certains problèmes bien conditionnés et avec des données complètes, une telle théorie existe, en supposant que des ensembles de données infinis et exempts de bruit seraient disponibles. Toutefois, la plupart des situations pratiques en reconstruction d'image se heurtent à la mal-positude du problème inverse, nécessitant des approches régularisées pour obtenir des solutions stables et physiquement plausibles.

\subsubsection*{\large Quelques exemples de problèmes inverse}
\subsubsection{Déconvolution}
Dans la déconvolution \cite{6}, on suppose que la mesure est une version convoluée de l'image réelle. L'opérateur est donc défini comme la convolution \( A : u \mapsto g \ast u \) avec un filtre \( g \). Dans le cas le plus simple, le filtre de convolution \( g \) est supposé connu. Un des exemples les plus connus est celui du débruitage, où le filtre utilisé est souvent le filtre gaussien, voir \Cref{fig:inverse_examples}.

\begin{figure}[H]
    \centering
    \includegraphics[width=0.8\textwidth]{./images/deconvolution.png}
    \caption{Dans le débruitage, l'objectif est de retrouver une image nette à partir d'une image floue, celle-ci étant obtenue par convolution avec un filtre gaussien.}
    \label{fig:inverse_examples}
\end{figure}
\subsubsection*{Computed Tomography (CT)}
Les examens par tomodensitométrie (CT) sont une méthode courante pour obtenir des images internes du corps humain. En gros, des rayons \(\mathbf{X}\) sont envoyés à travers le corps selon différentes directions. Pendant leur traversée, les rayons \(\mathbf{X}\) sont atténués en fonction de la densité des matériaux qu'ils rencontrent. Cette diminution d'intensité est ensuite mesurée sur le côté opposé du corps. L'ensemble de toutes ces mesures est appelé un \textit{sinogramme}.
L'opérateur linéaire utilisé pour décrire le processus de scan est appelé la \textit{transformée de Radon}, étudiée par Johann Radon bien avant son utilisation pratique.
\begin{figure}[H]
    \centering
    \includegraphics[width=0.8\textwidth]{./images/computed_tomography.png}
    \caption{\textbf{Exemple de tomodensitométrie (CT) avec le modèle de Shepp-Logan.} La colonne de gauche montre la rotation de la source de rayonnement, la colonne du milieu les mesures du détecteur pour un angle spécifique, et la colonne de gauche l'évolution du sinogramme.}
    \label{fig:computed_tomography}
\end{figure}

\subsubsection{\large Mal-positude au sens de Hadamard}
Lors de la résolution de problèmes inverses, nous devons faire face à certaines difficultés :
\begin{itemize}
    \item[-] \textbf{Le premier problème} survient s'il n'existe aucune solution au problème inverse. Cela peut se produire si la mesure est bruitée et que \(\mathbf{y}\) n'appartient pas à la plage de données supposée. Le problème de non-existence peut souvent être surmonté par une modélisation appropriée.

    \item[-] \textbf{Le deuxième problème} survient si la solution du problème inverse n'est pas \textbf{unique}, c'est-à-dire s'il existe plusieurs entrées \(\mathbf{x}\) qui génèrent la même mesure \(\mathbf{y}\).

    \item[-] \textbf{Le troisième problème, et le plus difficile}, survient si la résolution du problème inverse n'est pas \textbf{stable}, c'est-à-dire si le comportement de la solution ne varie pas continûment par rapport à la mesure \(\mathbf{y}\). Si le problème est instable, même de petites perturbations du bruit \(\varepsilon\) dans la mesure peuvent entraîner des artefacts importants dans la solution.
\end{itemize}

\subsubsection{\large Quelques approches itératives}
\begin{itemize}
    \item[-] Méthodes algébriques (ART, SIRT, SART)
    \item[-] Méthodes statistiques (ML-EM, OSEM, MAP)
    \item[-] \textbf{Méthodes variationnelles} (régularisation de Tikhonov, Total Variation)
    \item[-] \textbf{Compressive Sensing} (reconstruction à partir de mesures sous-échantillonnées)
\end{itemize}

\subsection{\Large Méthodes basées sur l'apprentissage profond}
Reconstruction par modèles neuronaux apprenant directement la relation données-image.
\subsubsection{Quelques exemples}
\begin{itemize}
    \item[-] Réseaux feed-forward (U-Net, FBPConvNet)
    \item[-] Architectures itératives unrolled
    \item[-] Modèles génératifs (GAN, diffusion models)
\end{itemize}\vspace{18pt}
Bref, le processus est fondamentalement linéaire et mal conditionné, ce qui signifie que de petites erreurs dans les données peuvent entraîner de grandes erreurs dans l'image reconstruite. Cela rend la qualité de l'acquisition et le traitement préalable (filtrage, correction d'aberrations) essentiels.
% \subsubsection{Le Sinogramme}
% Les données collectées sont compilées sous forme de sinogramme. Il s'agit d'une représentation graphique en niveaux de gris où  :
% \begin{itemize}
%     \item[-] L'axe horizontal représente l'angle de la mesure.
%     \item[-] L'axe vertical représente la distance du faisceau par rapport à l'origine. Une valeur de 0 (noir) indique aucun changement d'intensité, tandis qu'une valeur de 1 (blanc) signifie que le faisceau a été totalement absorbé.
% \end{itemize}

% Pour reconstruire l'image originale à partir d'un sinogramme, plusieurs outils mathématiques sont nécessaires  :
% \begin{itemize}
% \item[-] \textbf{Transformée de Radon}  : Elle est au cœur du processus de récupération de l'image à partir des projections.
% \item[-] Problème inverse  : Le défi consiste à utiliser les densités mesurées (données de sortie) pour retrouver le coefficient d'atténuation interne exact.
% \item[-] Outils de traitement  : La reconstruction utilise la transformée de Fourier, le théorème de la coupe centrale et la formule de rétroprojection (backprojection).
% \end{itemize}

    \chapter{Quelques modèles de reconstruction}
\section{Les traitements préalables à la reconstruction}

\subsection{\Large Méthodes dans le domaine spatial}
% =================== TODO ==================
\subsubsection{\large Filtrage linéaire}
% =========================================


% \subsection{\Large Méthodes dans le domaine transformé}
\section{\large Transformée de Fourier}
\begin{definition}[Transformée de Fourier]
    Soit \( f \) une fonction absolument intégrable sur \( \mathbb{R} \).
    La transformée de Fourier de \( f \), notée \( \mathcal{F}f \), est définie
    pour tout nombre réel \( \xi \) par
    \[
    (\mathcal{F}f)(\xi)
    = \int_{-\infty}^{\infty} f(x)\, e^{-2\pi i \xi x}\, dx.
    \]
\end{definition}
La transformée de Fourier est fréquemment utilisée en analyse du signal et permet de transformer une fonction du temps en une fonction de la fréquence ; la variable $x$ représente le temps en secondes et la variable \( \xi \) représente la fréquence de la fonction en hertz.\\

Il existe une définition alternative faisant intervenir la fréquence angulaire $w=2\pi \xi$, ce qui conduit à l'expression suivante.
\[(\mathcal{F}f)(w) = \int_{-\infty}^{\infty} f(x)\, e^{-i w x}\, dx\]
Comme pour la transformée de Radon, nous allons énumérer plusieurs propriétés de la transformée de Fourier.
\begin{proposition}
    Pour des constantes réelles $\alpha$ et $\beta$, et des fonctions absolument intégrables $f$ et $g$, on a:
    \begin{itemize}
        \item[(i)] Linéarité : $\mathcal{F}(\alpha f + \beta g)(w) = \alpha \mathcal{F}f(w) + \beta \mathcal{F}g(w)$
        \item[(ii)] $\mathcal{F}f(w) < +\infty$
    \end{itemize}
\end{proposition}

\begin{definition}[Transformée de Fourier inverse]
Soit \( f \) une fonction absolument intégrable.
La transformée de Fourier inverse de \( f \), notée \( \mathcal{F}^{-1}f \),
évaluée en \( x \), est définie par
\begin{equation}
    (\mathcal{F}^{-1}f)(x)
    = \cfrac{1}{2\pi}\int_{-\infty}^{\infty} f(w)\, e^{iw x}\, dw.
    \label{formula:fourier_inverse}
\end{equation}
\end{definition}
Ceci nous conduit immédiatement au théorème suivant.
\begin{proposition}[Théorème d'inversion de Fourier]
Soit $f$ une fonction absolument integrale sur $\mathbb{R}$.
Le théorème d'inversion de Fourier affirme que, pour tout \( x \),
\[
(\mathcal{F}^{-1} \circ \mathcal{F})f(x)=f(x)
\]
\end{proposition}
Jusqu'à présent, nous n'avons abordé la transformée de Fourier que dans une dimension. Il existe des définitions correspondantes en dimensions supérieures, mais, pour nos besoins, nous n'utiliserons que les analogues en deux dimensions.

\begin{definition}[Transformée de Fourier bidimensionnelle]
Soit \( g \) une fonction absolument intégrable définie sur \( \mathbb{R}^2 \).
La transformée de Fourier bidimensionnelle de \( g \), notée \( \mathcal{F}_2 g \),
est définie pour tout \((X,Y) \in \mathbb{R}^2\) par
\begin{equation}
    (\mathcal{F}_2 g)(X,Y) = \int_{-\infty}^{\infty} \int_{-\infty}^{\infty} 
    g(x,y)\, e^{-i (xX + yY)} \, dx\, dy.
    \label{eq:fourier_2d}
\end{equation}

\end{definition}

De manière similaire, nous définissons la transformée de Fourier inverse sur $\mathbb{R}^2$.
\begin{definition}[Transformée de Fourier bidimensionnelle inverse]
Soit \( g \) une fonction absolument intégrable définie sur \( \mathbb{R}^2 \).
La transformée de Fourier bidimensionnelle inverse de \( g \), évaluée en \((x,y)\)
et notée \( \mathcal{F}_2^{-1} g(x,y) \), est donnée par
\[
(\mathcal{F}_2^{-1} g)(x,y) = \cfrac{1}{4\pi^2}\int_{-\infty}^{\infty} \int_{-\infty}^{\infty} 
g(X,Y)\, e^{i (xX + yY)} \, dX\, dY.
\]
\end{definition}

% ================== TODO ==================
\subsubsection{\large Filtre de Wiener}
% 1. Filtre de Wiener
% ✅ Oui, tout à fait applicable
% Le filtre de Wiener est un filtre linéaire adaptatif.

% Il peut être appliqué :
%    - sur chaque projection (variable s)
%    - ou localement sur le sinogramme

% Intérêt :
%    - réduction du bruit additif (souvent gaussien)
%    - compromis bruit / flou optimal au sens MSE

% ⚠️ Limites :
%    - nécessite une estimation du bruit et du spectre du signal
%    - un mauvais modèle dégrade la reconstruction

% 📌 Très utilisé comme prétraitement des sinogrammes en CT à faible dose.

\subsubsection{\large Curvelets}
% 4. Curvelets
% ✅ Oui, très pertinent
% Les curvelets sont théoriquement bien adaptées :
%    - excellente représentation des singularités le long de courbes

% Les lignes du sinogramme correspondent à :
%    - des courbes liées aux bords de l'objet

% 📌 Très utilisé dans :
%    - CT basse dose
%    - Les méthodes variationnelles et itératives
% =================================================

\section{Convolution}
\textbf{Définition 8.1.}
Pour deux fonctions intégrables $f$ et $g$ définies sur $\mathbb{R}$,
nous définissons la convolution de $f$ et $g$, notée $f \star g$, par
\[
(f \star g)(x) = \int_{-\infty}^{\infty} f(t)\,g(x - t)\,dt,
\]
où $x \in \mathbb{R}$.

Nous pouvons facilement étendre cette définition à l'espace
bidimensionnel. Pour les fonctions polaires, nous prenons uniquement
l'intégrale par rapport à la variable radiale, tandis que pour les
fonctions cartésiennes nous intégrons par rapport aux deux variables.
Les définitions explicites sont données ci-dessous.

\begin{definition}
    Pour des fonctions polaires intégrables $f(t,\theta)$ et $g(t,\theta)$,
    nous définissons la convolution de $f$ et $g$ par
    \[
        (f \star g)(t,\theta)
        =
        \int_{-\infty}^{\infty}
        f(s,\theta)\,g(t - s,\theta)\,ds.
    \]
\end{definition}

Pour des fonctions intégrables $F$ et $G$ sur $\mathbb{R}^2$,
nous définissons la convolution de $F$ et $G$ par
\[
    (F \star G)(x,y)
    =
    \int_{-\infty}^{\infty}
    \int_{-\infty}^{\infty}
    F(s,t)\,G(x - s, y - t)\,ds\,dt.
\]

La convolution est une méthode mathématique permettant de moyenner
une fonction $f$ à l'aide du déplacement d'une autre fonction $g$.
Dans la convolution $f \star g$, la fonction $g$ est translatée à travers
la fonction $f$, et la fonction résultante dépend de la zone de recouvrement
au cours de cette translation.
En un certain sens, on peut voir $g$ comme un filtre utilisé pour effectuer
une moyenne de $f$ sur un intervalle donné.
La fonction de filtrage agit ainsi comme un lisseur pour les données bruitées
fournies par la fonction originale.

\begin{proposition}
    Pour des fonctions intégrables $f$, $g$, $h$ définies sur $\mathbb{R}$
    et des constantes $\alpha, \beta \in \mathbb{R}$ :
    
    \begin{itemize}
      \item[(i)] $f \star g = g \star f$ \quad (commutativité),
      \item[(ii)] $f \star (\alpha g + \beta h)
      = \alpha (f \star g) + \beta (f \star h)$ \quad (linéarité).
      \item[(iii)] $\mathcal{F}(f). \mathcal{F}(g)  = \mathcal{F}(f \star g)$
    \end{itemize}
\end{proposition}


% =================================================
\section{La transformée de Radon}
% =================================================
L'hypothèse fondamentale est que le détecteur mesure l'atténuation intégrée le long d'un rayon. 
\begin{definition}
    Pour un faisceau de rayons $\mathbf{X}$ d'énergie $\mathbf{E}$ donnée et un taux de propagation des photons $\mathbf{N}(x)$, l'intensité du faisceau $\mathbf{I}(x)$ à une distance $x$ de l'origine est définie comme \[\mathbf{I}(x) = \mathbf{N}(x) \mathbf{E}\]
\end{definition}

\begin{definition}
    La proportion de photons absorbés par millimètre de substance à une distance $x$ de l'origine est appelée le coefficient d'atténuation $\mathbf{A}(x)$ du milieu.
\end{definition}


Nous connaissons les intensités initiale et finale, $I_0$ et $I_1$ d'un faisceau unique. L'objectif est d'utiliser ces intensités pour déterminer le coefficient d'atténuation le long du trajet du faisceau. Heureusement, la loi de Beer-Lambert établit une relation entre ces deux grandeurs.

\begin{definition}[Loi de Beer-Lambert]
Pour un faisceau de rayons X monochromatique, non réfractif et de largeur nulle,
traversant un milieu homogène sur une distance \(x\) à partir de l'origine,
l'intensité \(I(x)\) est donnée par
\begin{equation}
    I(x) = I_0 e^{-\mathbf{A}(x)x}
    \label{eq:loi_beer_lambert}
\end{equation}
\end{definition}
En l'état, cette équation ne nous est pas particulièrement utile. Elle exprime le coefficient d'atténuation en un point donné en fonction de l'intensité en ce point, alors que nous ne connaissons la valeur de l'intensité qu'en des points situés à l'extérieur de l'objet. Ce que nous cherchons réellement est une relation entre le coefficient d'atténuation à l'intérieur de l'objet et la variation de l'intensité du faisceau. Pour cela, nous allons manipuler légèrement l'équation \eqref{eq:loi_beer_lambert}.\\
En passant à  la dérivée de la loi de Beer-Lambert, nous obtenons la relation suivante :
\[
    \frac{dI}{dx} = -\mathbf{A}(x)I(x)
\]
Soit $I(x_0)=I_0$ la valeur initiale de l'intensité du faisceau et $I(x_1)=I_1$ la valeur finale de l'intensité du faisceau. En utilisant cette relation, nous obtenons la relation suivante :

\[
    -\int_{x_0}^{x_1} \mathbf{A}(x)dx = \int_{x_0}^{x_1}\cfrac{dI}{I(x)}=ln(\frac{I_1}{I_0})
\]
ou encore \vspace{10pt}
\begin{equation}
    \int_{x_0}^{x_1} \mathbf{A}(x)dx = ln(\frac{I_0}{I_1})
    \label{eq:radon_transformation}
\end{equation}

Nous sommes maintenant prêts à introduire des outils mathématiques — en particulier la transformée de Radon — qui joueront un rôle central dans la détermination du coefficient d'atténuation dans l'équation \eqref{eq:loi_beer_lambert}.

L'écriture sous forme normale d'une équation de droite joue un rôle clé dans la transformée de Radon, car elle permet une paramétrisation naturelle et complète de toutes les droites du plan, ce qui est essentiel pour la définition mathématique et le calcul pratique de cette transformation.\\ 
Cette équation sous forme normale fournit :
\begin{itemize}
    \item[(i)] Une paramétrisation unique et continue de toutes les droites du plan. La forme normale (ou forme normale de Hesse) de l'équation de la droite s'écrit : $$x\,\cos(\theta)+y\,\sin(\theta)=\rho$$ où $\rho$ est la distance par rapport à l'origine et $\theta$ est l'angle par rapport à l'axe des abscisses.
    \item[(ii)] Une interprétation géométrique claire de $\rho$ et $\theta$. Chaque droite du plan correspond  à un unique couple ($\rho,\theta$). Cette paramétrisation évite les redondances et garantit qu'on parcourt toutes les droites une et une seule fois (à une convention près).
    \item[(iii)] Une mesure naturelle sur l'espace des droites, utilisée dans les formules d'inversion.
    \item[(iv)] Un formalisme adapté au théorème de coupe, reliant transformée de Radon et transformée de Fourier 2D. 
    \item[(v)] Une mesure naturelle sur l'espace des droites, utilisée dans les formules d'inversion.
\end{itemize}\vspace{10pt}
\subsubsection{\small Construction de l'orientation et de la distance}
Nous connaissons tous l'idée qu'une droite \( l \) dans \( \mathbb{R}^2 \) peut être représentée par l'équation 
\[
ax + by = c
\]
où \( a, b, c \in \mathbb{R} \) et \( a^2 + b^2 \neq 0 \).\\ On peut alors écrire cette équation d'une droite sous la forme \[w_1x + w_2y = t\]
où $\mathbf{w}:=(w_1, w_2) = (\cfrac{a}{\sqrt{a^2 + b^2}}, \cfrac{b}{\sqrt{a^2 + b^2}})$ et $t=\cfrac{c}{\sqrt{a^2 + b^2}}$, que nous pouvons voir comme un point situé sur le
cercle unitaire, pour \[\left(\cfrac{a}{\sqrt{a^2 + b^2}}\right)^{2} + \left(\cfrac{b}{\sqrt{a^2 + b^2}}\right)^{2} = 1\]
Cela implique que $\mathbf{w} := (\cos(\theta), \sin(\theta)) \text{ est un vecteur normal unitaire }$, $\theta \in [0, 2\pi)$ représente l'orientation, et $t$ est exactement la distance à l'origine. On a \[x\cos(\theta) + y\sin(\theta) = t\]
Notez que dans les équations ci-dessus, $t$ et $\theta$ sont fixes et déterminent une droite spécifique \( l \) dans le plan. On peut donc dire que $t$ et $\theta$ paramètrent une droite \( l_{t,\theta} \) et que $\mathbf{z}$ détermine des points spécifiques sur cette droite \( l \). Ou encore
\[l_{t,\theta} = \{ \mathbf{z} \in \mathbb{R}^2 : \langle z, (\cos \theta, \sin \theta) \rangle = t \}.\]
\begin{figure}[H]
    \centering
    \includegraphics[width=0.8\textwidth]{./images/l_t_theta.png}
    \caption{paramètrisation d'une droite \( l_{t,\theta} \) par \( t \) et \( \theta \)}
    \label{fig:l_t_theta}
\end{figure}
On voit  que $(t\, \cos(\theta), t\, \sin(\theta))$ est un point situé sur la droite \( l_{t,\theta} \) (\Cref{fig:l_t_theta}) et $(-\sin(\theta), \cos(\theta))$ est un vecteur perpendiculaire au vecteur unitaire $\mathbf{w}$.\\ En géométrie affine élémentaire, une ligne est un point plus une direction. Par conséquent, nous pouvons décrire un point particulier $(x, y)$ sur $l_{t, \theta}$ en termes de nombre réel s comme suit :
\begin{equation}
    l_{t, \theta} = \{(t\, \cos(\theta) - s\,\sin(\theta), t\,\sin(\theta) + s\,\cos(\theta)); s\in \mathbb{R}\}
    \label{set:l_t_theta}
\end{equation}
\begin{definition}[Transformée de Radon]
Soit \( f(t,\theta) \) une fonction définie sur \( \mathbb{R}^2 \) à support compact.
La transformée de Radon de \( f \), notée \( \mathcal{R}f \), est définie pour
\( t \in \mathbb{R} \) et \( \theta \in (0, 2\pi] \) par
\[
\mathcal{R}f(t,\theta) = \int_{-\infty}^{\infty} f(x(s),y(x))\mathrm{d}s
\]
\end{definition}

La transformée de Radon permet de déterminer la densité totale d'une fonction $f$ le long d'une droite donnée $l$. Cette droite $l$ est définie par un angle $\theta$  par rapport à l'axe 
$x$ et une distance $t$ par rapport à l'origine. Comme illustré à la \Cref{fig:radon}, si l'on calcule la transformée de Radon le long de plusieurs droites à des angles différents (ici $\theta_1$ et $\theta_2$), on peut déterminer plusieurs fonctions de densité pour notre objet. Intuitivement, on peut interpréter la transformée de Radon comme une version « étalée » de notre objet initial. Supposons que la région en forme de tache représentée à la \Cref{fig:radon} soit une tache d'encre; si l'on étale cette tache le long de lignes de direction $\theta_1$, on s'attend à ce que les régions les plus larges de la tache correspondent à des zones plus étendues que les régions plus petites, ce qui correspond exactement à ce que l'on observe.
\begin{figure}[H]
    \centering
    \includegraphics[width=0.8\textwidth]{./images/radon.png}
    \caption{Transformée de Radon pour $\theta_1$ et $\theta_2$.}
    \label{fig:radon}
\end{figure}
L'intégrale $\mathcal{R}f(t,\theta)$ représente le membre gauche de l'équation \eqref{eq:radon_transformation}. Rappelons que, dans cette équation, $\mathbf{A}(x)$ est inconnue et que $\ln(\frac{I_1}{I_0})$ correspond à une information mesurée.
Autrement dit, $\ln(\frac{I_1}{I_0})$ est la transformée de Radon, et la transformée de Radon représente donc des données connues issues de la mesure.

L'objectif est maintenant de trouver une formule d'inversion de la transformée de Radon qui nous permettra de reconstruire la fonction initiale $f$ (ou, dans le contexte de l'imagerie médicale, 
$\mathbf{A}(x)$). Pour ce faire, il sera utile de rappeler plusieurs propriétés de la transformée de Radon.
\begin{proposition}
    Soit $\alpha$ et $\beta$ deux réels et $f$ et $g$ deux fonctions continues sur $\mathbb{R}^2$ à support compact. On a
    \begin{itemize}
        \item[(i)] Linéarité : $\mathcal{R}(\alpha f + \beta g) = \alpha \mathcal{R}f + \beta \mathcal{R}g$
        \item[(ii)] Parité: $\mathcal{R}f(-t,-\theta) = \mathcal{R}f(t,\theta)$
        \item[(iii)] $\mathcal{R}f(t, \theta) = \int_{-\infty}^{\infty} f(x(s), y(s))\mathrm{d}s = \int_{-\infty}^{\infty} f(t\,cos(\theta)-s\,sin(\theta), t\,sin(\theta)+s\,cos(\theta))\mathrm{d}s$
        % \item[(iv)] Invariance par rotation : \(\mathcal{R}(f \circ R_{\psi}) = \mathcal{R}f(t,\theta - \psi)\)
        % \item[(v)] Relation avec la convolution : \(\mathcal{R}(f * g) = \mathcal{R}f * \mathcal{R}g\)
    \end{itemize}
\end{proposition}
Nous définissons en outre le domaine naturel de la transformée de Radon comme l'ensemble des fonctions $f$ sur $\mathbb{R}^2$ telles que \[\int_{-\infty}^{\infty} |f(x(s), y(s))|\mathrm{d}s < \infty\]

\subsection{Le Théorème de la Coupe Centrale}
Le théorème de la coupe centrale, également appelé théorème de projection-transforme de Fourier ou théorème de Fourier-Slice, est un résultat fondamental en traitement d'image et en tomographie. Il établit un lien profond entre la transformée de Radon (utilisée pour décrire les projections d'un objet) et la transformée de Fourier (utilisée pour analyser les fréquences spatiales). Ce théorème constitue la pierre angulaire mathématique de la plupart des méthodes de reconstruction tomographique moderne.

\begin{proposition}
    Soit \( g \) une fonction absolument integrale sur \( \mathbb{R}^2 \).
    Le théorème de la coupe centrale affirme que, pour tout $S \in \mathbb{R}$ et $\theta \in [0,2\pi]$, on a : \[\mathcal{F}_2 g(S\cos(\theta), S\sin(\theta)) = \mathcal{F}(\mathcal{R}g)(S, \theta)\]
\end{proposition}
\textbf{Preuve}: En utilisant la définition de la transformée de Fourier bidimensionnelle \eqref{eq:fourier_2d} on obtient 
\[
    \mathcal{F}_{2}g(S\,\cos(\theta), S\,\sin(\theta)) = \int_{-\infty}^{\infty} \int_{-\infty}^{\infty} g(x, y)\, e^{-iS (x\,\cos(\theta) + y\,\sin(\theta))}\, dx\, dy
\]
Nous effectuons maintenant un changement de variables conformément au système
de coordonnées que nous avons défini à la \textit{Construction de l'orientation et de la distance}.
Rappelons que, lors de la paramétrisation de la droite $\ell_{t,\theta}$,
nous avons montré que, pour $s\in\mathbb{R}$, on peut écrire :
\[
x(s)=t\cos\theta - s\sin\theta, 
\qquad
y(s)=t\sin\theta + s\cos\theta,
\qquad
t = x\cos\theta + y\sin\theta.
\]

En examinant le déterminant du Jacobien associé à $x(s)$ et $y(s)$, on obtient :
\[
\det
\begin{pmatrix}
\dfrac{\partial x}{\partial t} & \dfrac{\partial x}{\partial s} \\[6pt]
\dfrac{\partial y}{\partial t} & \dfrac{\partial y}{\partial s}
\end{pmatrix}
= 1.
\]

Nous en déduisons que
\[
ds\,dt = dx\,dy.
\]
et donc
\[
\int_{-\infty}^{\infty} \int_{-\infty}^{\infty} g(x, y)\, e^{-iS (x\,\cos(\theta) + y\,\sin(\theta))}\, dx\, dy = \int_{-\infty}^{\infty}\int_{-\infty}^{\infty}
g(t\cos\theta - s\sin\theta,\; t\sin\theta + s\cos\theta)\,
e^{-iSt}\,ds\,dt.
\]

Comme $e^{-iSt}$ ne dépend pas de la variable $s$, nous pouvons réarranger
l'intégrale précédente de la manière suivante :
\[
\int_{-\infty}^{\infty}
\left(
\int_{-\infty}^{\infty}
g(t\cos\theta - s\sin\theta,\; t\sin\theta + s\cos\theta)\,ds
\right)
e^{-iSt}\,dt.
\]

L'intégrale intérieure est exactement la transformée de Radon de $f$,
évaluée en $(t,\theta)$, ce qui implique que l'expression précédente devient :
\[
\int_{-\infty}^{\infty}
(Rg(t,\theta))\,e^{-iSt}\,dt.
\]

Cette dernière intégrale n'est autre que la transformée de Fourier de
$Rg(S,\theta)$, ce qui conclut la démonstration.
\hfill$\square$

% \section{Inversion analytique de la transformée de Radon}
\subsection{Rétroprojection filtrée (FBP)}
Nous sommes maintenant enfin prêts à effectuer une première tentative pour retrouver la fonction de coefficient d'atténuation.
Rappelons que, d'un point de vue physique, la transformée de Radon
$\mathcal{R}f(t,\theta)$ nous donne la densité totale de l'objet $f$ le long d'une droite
$\ell_{t,\theta}$.
Nous avons déterminé cette densité en mesurant les intensités initiale et finale
d'un faisceau de rayons $\mathbf{X}$ traversant l'objet le long de cette droite.
En procédant ainsi pour plusieurs droites différentes, nous sommes capables de
reconstruire une coupe unique de l'objet initial, et en faisant varier l'angle
$\theta$ de ces rayons $\mathbf{X}$, nous pouvons définir de nombreuses coupes.

Si nous sommes capables, d'une certaine manière, de « rétroprojeter » ces
densités sur le plan, nous pourrons peut-être reconstituer l'objet initial.
Intuitivement, on peut interpréter ce processus comme le fait de prendre les
données du sinogramme et de les « déflouter » pour les ramener dans le plan.
\begin{definition}
Soit $h = h(t,\theta)$. On définit la \emph{rétroprojection},
notée $\mathcal{B}h$, en un point $(x,y)$ par :
\[
\mathcal{B}h(x,y) = \frac{1}{\pi}\int_{0}^{\pi} h(x\cos\theta + y\sin\theta,\theta)\,d\theta.
\]

En appliquant cette formule de rétroprojection à la transformée de Radon, on
obtient :
\begin{equation}
    \mathcal{B}\mathcal{R}f(x,y) = \frac{1}{\pi}\int_{0}^{\pi}
    \mathcal{R}f(x\cos\theta + y\sin\theta,\theta)\,d\theta.
    \label{eq:FBP}
\end{equation}
\end{definition}
Nous sommes capables d'effectuer la rétroprojection sur les coupes que nous
avons mesurées. Comme illustré à la \Cref{fig:FBP}, effectuer une rétroprojection
selon seulement quelques directions $\theta$ constitue une méthode extrêmement
imprécise pour reconstituer ne serait-ce qu'un objet simple. Toutefois, même si
nous augmentons de manière significative le nombre de rétroprojections
(par exemple jusqu'à $1000$ directions), il subsiste encore une quantité
importante de bruit qui brouille l'image reconstruite.
En réalité, quel que soit le nombre de directions selon lesquelles nous tentons
d'effectuer la rétroprojection, nous ne serons jamais capables de reconstruire
parfaitement l'image à l'aide de la formule de rétroprojection donnée par
l'équation \eqref{eq:FBP}.
Pour que ce procédé soit réellement utile, il est nécessaire de dériver une
méthode permettant de filtrer une partie du bruit que la formule de
rétroprojection semble introduire dans l'image, afin d'obtenir une
représentation plus lisse de l'objet.

\begin{figure}[H]
    \centering
    \includegraphics[width=0.8\textwidth]{./images/fbp.png}
    \caption{Retroprojection d'un carré dans 5, 25, 100 et 1000 directions}
    \label{fig:FBP}
\end{figure}

Dans ce but, nous définissons une formule de \emph{rétroprojection filtrée}.
\begin{proposition}
    Soit $f$ une fonction absolument intégrable définie sur $\mathbb{R}^2$. Alors,
    \begin{equation}
        f(x,y)
        =
        \frac{1}{2}\,
        \mathcal{B}\!\left\{
        \mathcal{F}^{-1}
        \!\left[
        |S|\,
        \mathcal{F}\!\left(\mathcal{R}f\right)(S,\theta)
        \right]
        \right\}(x,y).
        \label{eq:FBP_filter}
    \end{equation}
\end{proposition}
\textit{Démonstration.}
Nous commençons par rappeler que, pour la transformée de Fourier bidimensionnelle
et son inverse, on a :
\begin{equation}
f(x,y) = \mathcal{F}_2^{-1}\,\mathcal{F}_2 f(x,y)
= \frac{1}{4\pi^2}
\int_{-\infty}^{\infty}\int_{-\infty}^{\infty}
\mathcal{F}_2 f(X,Y)\,e^{i(Xx+Yy)}\,dX\,dY.
\label{eq:fourier_2d_inverse}
\end{equation}

Nous allons maintenant effectuer un changement de variables des coordonnées
cartésiennes $(X,Y)$ vers les coordonnées polaires $(S,\theta)$, définies par
\[
X = S\cos\theta,
\qquad
Y = S\sin\theta,
\]
où $S \in \mathbb{R}$ et $\theta \in [0,\pi]$.
Ce changement de variables conduit au déterminant jacobien suivant :
\[\det
\begin{pmatrix}
    \dfrac{\partial X}{\partial s} & \dfrac{\partial X}{\partial \theta} \\[6pt]
    \dfrac{\partial Y}{\partial s} & \dfrac{\partial Y}{\partial \theta}
\end{pmatrix}
=|S|
\]
Ce qui nous dit que $dX\,dY = |S|\,dS\,d\theta$. En incorporant ce nouveau changement de variables, l'équation \eqref{eq:fourier_2d_inverse} devient :
\[
f(x,y) = \frac{1}{4\pi^{2}} \int_{0}^{\pi} \int_{-\infty}^{\infty}
\mathcal{F}_{2}f(S\cos\theta, S\sin\theta)\,
e^{iS(x\cos\theta + y\sin\theta)}\,|S|\,dS\,d\theta.
\]
Et en utilisant le théorème de la tranche centrale, nous voyons que l'équation ci-dessus est en fait égale à
\begin{equation}
    f(x,y) = \frac{1}{4\pi^{2}} \int_{0}^{\pi} \int_{-\infty}^{\infty}
    \mathcal{F}\bigl(\mathcal{R}f(S,\theta)\bigr)\,
    e^{iS(x\cos\theta + y\sin\theta)}\,|S|\,dS\,d\theta.
    \label{eq:fourier_radon}
\end{equation}
Prenons maintenant un regard plus attentif sur l'intégrale intérieure de l'équation \eqref{eq:fourier_radon} et en utilisant la définition de la Transformée de Fourier inverse, on a :
\[
    \begin{array}{rcl}
        \int_{-\infty}^{\infty}
        \mathcal{F}\bigl(\mathcal{R}f(S,\theta)\bigr)\,
        e^{iS(x\cos\theta + y\sin\theta)}\,|S|\,dS
        &=&
        2\pi \left(
        \frac{1}{2\pi} \int_{-\infty}^{\infty}
        \mathcal{F}\bigl(\mathcal{R}f(S,\theta)\bigr)\,
        e^{iS(x\cos\theta + y\sin\theta)}\,|S|\,dS
        \right)\\
        &=&
        2\pi\,\mathcal{F}^{-1}
        \Bigl(
        |S|\,\mathcal{F}\bigl(\mathcal{R}f\bigr)(S,\theta)
        \Bigr)
        \bigl(x\cos\theta + y\sin\theta,\theta\bigr)\\
    \end{array}
\]


Autrement dit, l'intégrale intérieure de l'équation (7.4) est égale à $2\pi$ fois l'inverse de la transformée de Fourier de
$|S|\,\mathcal{F}\bigl(\mathcal{R}f\bigr)(S,\theta)$
au point $(x\cos\theta + y\sin\theta,\theta)$.
Nous pouvons alors voir que l'équation (7.4) est en fait égale à
\[
\frac{1}{2\pi} \int_{0}^{\pi}
\mathcal{F}^{-1}
\Bigl(
|S|\,\mathcal{F}\bigl(\mathcal{R}f\bigr)(S,\theta)
\Bigr)
\bigl(x\cos\theta + y\sin\theta,\theta\bigr)
\,d\theta.
\]

Finalement, nous constatons que l'intégrale ci-dessus est égale à $\tfrac{1}{2}$ de la rétroprojection donnée dans la définition \eqref{eq:FBP} pour
$\mathcal{F}^{-1}\bigl[|S|\,\mathcal{F}(\mathcal{R}f)(S,\theta)\bigr]$.
Nous simplifions donc l'équation précédente pour obtenir
\[
\frac{1}{2}\,
\mathcal{B}
\Bigl\{
\mathcal{F}^{-1}
\bigl[|S|\,\mathcal{F}\bigl(\mathcal{R}f(S,\theta)\bigr)\bigr]
\Bigr\}(x,y).
\]

Ce qui nous conduit à la conclusion souhaitée :
\[
f(x,y)
=
\frac{1}{2}\,
\mathcal{B}
\Bigl\{
\mathcal{F}^{-1}
\bigl[|S|\,\mathcal{F}\bigl(\mathcal{R}f(S,\theta)\bigr)\bigr]
\Bigr\}(x,y).
\]
\hfill $\square$\\
Le facteur important dans cette formule est le multiplicateur $|S|$ qui apparaît entre la transformée de Fourier et son inverse. Sans ce facteur, ces deux termes s'annuleraient mutuellement et nous nous retrouverions avec la formule standard de rétroprojection pour la transformée de Radon que nous avons rencontrée précédemment et qui, comme nous l'avons vu, ne nous donne pas directement $f(x, y)$. Nous appelons ce $|S|$ supplémentaire un \textbf{filtre} de la transformée de Radon, ce qui nous donne le nom de la formule de \textbf{rétroprojection filtrée}.
\begin{proposition}
    Soit $f$ et $g$ deux fonctions intégrables définies sur $\mathbb{R}$, alors
    \[(\mathcal{B}g\star f)(X, Y) = \mathcal{B}(g\star \mathcal{R}f)(X, Y)\]
\end{proposition}
Considérons maintenant la relation \eqref{eq:FBP_filter} et 
supposons qu'il existe une fonction, notée $\varphi(t)$, dont la transformée de Fourier
soit égale à notre facteur de filtrage $|S|$. Autrement dit, supposons qu'il existe une
fonction $\varphi(t)$ telle que
\[
\mathcal{F}\varphi(S) = |S|.
\]
Plus simplement, supposons que nous connaissions une fonction dont la transformée de
Fourier est égale à la fonction valeur absolue. Nous pourrions alors réécrire la
rétroprojection sous la forme suivante :
\begin{equation}
    f(x,y) = \frac{1}{2}\,\mathcal{B}
    \left\{
    \mathcal{F}^{-1}
    \bigl[
    \mathcal{F}\varphi \cdot \mathcal{F}(\mathcal{R}f)(S,\theta)
    \bigr]
    \right\}(x,y).
    \label{eq:FBP_varphi}
\end{equation}

Cependant, le membre de droite de l'équation \eqref{eq:FBP_varphi} contient un produit de transforméesde Fourier, que nous savons être égal à la convolution des fonctions transformées
\[
    f(x,y)
    =
    \frac{1}{2}\,\mathcal{B}
    \left\{
    \mathcal{F}^{-1}
    \bigl[
    \mathcal{F}(\varphi \star \mathcal{R}f)(S,\theta)
    \bigr]
    \right\}(x,y).
\]

Mais ceci n'est rien d'autre que la transformée de Fourier inverse de la transformée
de Fourier, ce qui nous ramène à la fonction de départ. Cela nous conduit à la formule
de rétroprojection filtrée beaucoup plus simple :
\begin{equation}
    f(x,y) = \frac{1}{2}\,\mathcal{B}(\varphi \star \mathcal{R}f)(x,y).
    % \tag{8.2}
    \label{eq:FBP_varphi_simple}
\end{equation}

L'équation \eqref{eq:FBP_varphi_simple} est bien plus élégante que notre formule initiale de rétroprojection filtrée et ne semble pas difficile à appliquer. Physiquement parlant, $\mathcal{R}f$ représente nos données mesurées et l'équation \eqref{eq:FBP_varphi_simple} requiert simplement de les filtrer à l'aide de notre nouvelle fonction $\varphi$, puis d'appliquer la formule de rétroprojection, qui est une intégrale relativement simple.

Malheureusement, il n'existe pas de fonction $\varphi$ dont la transformée de Fourier
soit exactement égale à la valeur absolue. Considérons la fonction $\mathcal{F}\varphi$ :
\[
\mathcal{F}\varphi(\omega)
=
\int_{-\infty}^{\infty}
\varphi(x)\,e^{-i\omega x}\,dx.
\]

Nous pouvons constater que, lorsque $\omega \to \infty$,
$\mathcal{F}\varphi(\omega) \to 0$ (remarquons l'exponentielle négative).
Cependant, pour la fonction valeur absolue $|\omega|$, lorsque $\omega \to \infty$,
$|\omega| \to \infty$.
Par conséquent, il est impossible de trouver une fonction $\varphi$ telle que,
pour tout $\omega$, $\mathcal{F}\varphi(\omega) = |\omega|$.

Toutefois, tout notre travail précédent n'est pas vain. Examinons plutôt le type de
fonctions sur lesquelles nous avons restreint notre étude. Nous ne considérons notre
fonction que sur un intervalle fini et supposons en fait qu'elle soit nulle en dehors
de cet intervalle. En étendant cette idée à la transformée de Fourier, nous constatons
que nous devons porter notre attention sur les \emph{fonctions à bande limitée}.

\begin{definition}
    Une fonction $\varphi$ est dite \emph{à bande limitée} s'il existe un réel $L > 0$ tel que
    \begin{equation}
        \mathcal{F}\varphi(\omega)
        =
        \int_{-\infty}^{\infty}
        \varphi(x)\,e^{-i\omega x}\,dx
        =
        0
        \quad \text{pour tout } \omega \notin [-L, L].
        % \tag{8.3}
        \label{eq:FBP_varphi_banded}
    \end{equation}
\end{definition}

Le facteur de filtrage $|S|$ sert à amplifier le terme $\mathcal{F}(\mathcal{R}f)$ dans la formule de rétroprojection filtrée originale \eqref{eq:FBP_filter}. En pratique, $\mathcal{F}(\mathcal{R}f)$ est très sensible aux hautes fréquences.

En concentrant notre attention sur les basses fréquences à l'aide d'une fonction à bande limitée $\varphi$, nous sommes en mesure d'éviter ce problème. Notre objectif est de remplacer $S$ par ce que l'on appelle un \emph{filtre passe-bas} (noté $S'$), qui prend en compte les effets des basses fréquences tout en atténuant les hautes fréquences. Cette fonction $S'$ doit avoir un support compact et être de la forme
\[
S' = \mathcal{F}\varphi
\]
(sur un intervalle compact).

Le coût de l'utilisation de $S'(\omega)$ est que nous ne disposons plus de l'égalité présentée dans l'équation \eqref{eq:FBP_varphi_simple}. En revanche, nous obtenons :
\begin{equation}
    f(x,y) \approx \frac{1}{2}\,\mathcal{B}\!\left(\mathcal{F}^{-1} S' \star \mathcal{R}f \right)(x,y).
    \label{eq:FBP_varphi_approx}
\end{equation}

De manière générale, la plupart des filtres passe-bas sont de la forme
\[
S'(\omega) = |\omega| \cdot F(\omega) \cdot \Pi_L(\omega),
\]
où $L > 0$ définit la région sur laquelle le filtrage est effectué. Différentes fonctions $F$ déterminent les caractéristiques précises du filtre, et $\Pi_L(\omega)$ est définie comme suit :
\[
    \Pi_L(\omega) =
    \begin{cases}
        1 & \text{si } |\omega| \leq L, \\
        0 & \text{si } |\omega| > L.
    \end{cases}
\]

Nous introduisons maintenant deux filtres couramment utilisés en imagerie numérique et en traitement du signal : le filtre \emph{Ram-Lak} et le filtre \emph{Hann}.

\subsection*{Filtre Ram-Lak}

Le filtre Ram-Lak est défini par :
\[
S'(\omega) = |\omega| \cdot \Pi_L(\omega) =
\begin{cases}
|\omega| & \text{si } |\omega| \leq L, \\
0 & \text{si } |\omega| > L.
\end{cases}
\]

Le filtre Ram-Lak constitue la base de nombreux autres filtres utilisés en analyse du signal, car il remplace simplement la fonction $F(\omega)$ par la fonction constante égale à 1. D'autres filtres, tels que le filtre Hann, consistent généralement en des produits de fonctions sinus ou cosinus destinées à éliminer le bruit indésirable.

\subsection*{Filtre Hann}

Le filtre Hann est donné par :
\[
S'(\omega) = |\omega| \cdot \frac{1}{2}
\left( 1 + \cos\!\left( \frac{2\pi \omega}{L} \right) \right)
\cdot \Pi_L(\omega).
\]

Le filtre Hann utilise la fonction de Hann
\[
\frac{1}{2}\left( 1 + \cos\!\left( \frac{2\pi \omega}{L} \right) \right)
\]
comme fonction $F(\omega)$,
% et son efficacité est illustrée dans le sinogramme et la rétroprojection de la transformée de Radon de Johann, présentés à la Figure~5.
% ====== TODO ======
% Python implementation
% ==================


    \chapter{SIMULATION, DÉVELOPPEMENT ET APPLICATIONS INNOVANTES EN RECONSTRUCTION D'IMAGES}
\section{Algorithme de rétroprojection filtrée pour la reconstruction d'images en tomodensitométrie}

La tomodensitométrie (CT) est une modalité d'imagerie médicale permettant de visualiser les structures internes du corps à partir de mesures d'atténuation des rayons X. La qualité des images reconstruites dépend directement de l'algorithme utilisé pour transformer les projections en image. L'algorithme de rétroprojection filtrée, ou \emph{Filtered Backprojection} (FBP), est l'une des méthodes les plus utilisées dans ce contexte.

Le principe de FBP repose sur deux étapes principales : le filtrage et la rétroprojection. L'étape de filtrage vise à corriger le flou inhérent aux projections en amplifiant les composantes haute fréquence. Elle consiste à appliquer un filtre de rampe aux données de sinogramme obtenues à partir des mesures CT. Ensuite, l'étape de rétroprojection additionne les projections filtrées le long des trajectoires des rayons X pour former l'image reconstruite.

L'implémentation pratique de cet algorithme peut être illustrée par le code Python suivant. Il montre comment générer un sinogramme à partir d'une image, appliquer un filtre en rampe et réaliser la rétroprojection pour obtenir l'image finale.

\begin{minted}{python}
import numpy as np
import matplotlib.pyplot as plt

def generate_sinogram(image, theta):
    """
    Générer des données de sinogramme à partir d'une image et d'un tableau d'angles.
    
    Paramètres :
        image (ndarray) : image d'entrée
        theta (ndarray) : tableau d'angles en radians
        
    Renvoie :
        sinogram (ndarray) : données de sinogramme
    """
    sinogram = np.zeros((len(theta), image.shape[0]))
    for i, angle in enumerate(theta):
        rotated_image = np.rot90(image, -int(np.degrees(angle) / 90) % 4)
        sinogram[i] = np.sum(rotated_image, axis=0)
    return sinogram

def ramp_filter(sinogram):
    """
    Appliquer un filtre de rampe aux données de sinogramme.
    
    Paramètres :
        sinogram (ndarray) : données du sinogramme
        
    Renvoie :
        filtered_sinogram (ndarray) : sinogramme filtré
    """
    freq = np.fft.fftfreq(sinogram.shape[1])
    ramp = np.abs(freq)
    filtered_sinogram = np.real(np.fft.ifft(np.fft.fft(sinogram) * ramp))
    return filtered_sinogram

def backproject(filtered_sinogram, theta, image_shape):
    """
    Rétroprojeter les données filtrées pour reconstruire l'image.
    
    Paramètres :
        filtered_sinogram (ndarray) : données filtrées
        theta (ndarray) : tableau d'angles
        image_shape (tuple) : dimensions de l'image reconstruite
        
    Renvoie :
        reconstructed_image (ndarray) : image reconstruite
    """
    reconstructed_image = np.zeros(image_shape)
    for i, angle in enumerate(theta):
        reconstructed_image += np.rot90(
            np.tile(filtered_sinogram[i][:, np.newaxis], (1, image_shape[0])),
            int(np.degrees(angle) / 90) % 4
        )
    return reconstructed_image / len(theta)

def main():
    # Générer un exemple d'image
    image = np.zeros((256, 256))
    image[100:150, 100:150] = 1

    # Générer des données de sinogramme
    theta = np.linspace(0, np.pi, 180, endpoint=False)
    sinogram = generate_sinogram(image, theta)

    # Appliquer l'algorithme FBP
    filtered_sinogram = ramp_filter(sinogram)
    reconstructed_image = backproject(filtered_sinogram, theta, image.shape)

    # Afficher les résultats
    plt.figure(figsize=(12, 6))
    plt.subplot(1, 3, 1)
    plt.imshow(image, cmap='gray')
    plt.title("Image originale")

    plt.subplot(1, 3, 2)
    plt.imshow(sinogram, cmap='gray', aspect='auto')
    plt.title("Sinogramme")

    plt.subplot(1, 3, 3)
    plt.imshow(reconstructed_image, cmap='gray')
    plt.title("Image reconstruite")
    
    plt.show()

if __name__ == "__main__":
    main()
\end{minted}

Ce code illustre de manière pratique le fonctionnement de l’algorithme FBP. Il génère un exemple simple avec un objet carré, simule les mesures CT en produisant un sinogramme, applique le filtrage et la rétroprojection, puis affiche l'image originale, le sinogramme et l'image reconstruite. L'algorithme constitue ainsi une méthode efficace et largement utilisée pour la reconstruction d'images tomodensitométriques, combinant rapidité et fidélité visuelle.

\begin{figure}[H]
    \centering
    \includegraphics[width=0.8\textwidth]{./images/fbp_simulation.png}
    \caption{Simulation de l'algorithme FBP}
    \label{fig:l_t_theta}
\end{figure}


% =============================================================
\section{Compressive sensing using forward-backward}
Le compressed sensing est une technique révolutionnaire en traitement du signal qui permet d'acquérir et de reconstruire un signal en utilisant beaucoup moins de mesures que ce qu'exige le théorème d'échantillonnage de Nyquist-Shannon. Ceci est possible lorsque le signal est \textbf{creux} (sparse) dans une certaine base. Le compressed sensing s’inscrit dans un cadre théorique plus large visant à exploiter les structures intrinsèques des signaux naturels. En pratique, de nombreux signaux d’intérêt (images, signaux biomédicaux, signaux radar ou sismiques) ne sont pas arbitraires, mais présentent une forte redondance lorsqu’ils sont représentés dans une base appropriée (ondelettes, cosinus, dictionnaires appris, etc.). Cette observation remet en question l’approche classique de l’échantillonnage uniforme et ouvre la voie à des stratégies d’acquisition plus efficaces. \vspace{5pt}\\
L’idée fondamentale du compressed sensing repose sur deux piliers théoriques : la \emph{parcimonie} du signal et l’\emph{incohérence} entre la base de représentation du signal et le système de mesure. Lorsque ces conditions sont satisfaites, il devient possible de reconstruire exactement le signal original à partir d’un nombre de mesures très inférieur à sa dimension ambiante.\vspace{10pt}\\
Le problème d'optimisation s'écrit :
\begin{equation}
\min\limits_x \|Ax-y\|_2^2 + \tau \|x\|_1
\label{eq:main_problem}
\end{equation}
où :
\begin{itemize}
    \item $y \in \mathbb{R}^m$ : mesures acquises
    \item $A \in \mathbb{R}^{m \times n}$ : matrice de mesure ($m < n$)
    \item $x \in \mathbb{R}^n$ : signal à reconstruire
    \item $\tau > 0$ : paramètre de régularisation
\end{itemize}
Le problème d’optimisation présenté à l’équation~\eqref{eq:main_problem} correspond à une formulation dite \emph{régularisée}, largement utilisée en pratique. Le premier terme impose une cohérence entre les mesures acquises et le signal reconstruit, tandis que le second terme agit comme une contrainte indirecte sur la structure du signal recherché.

Cette formulation peut également être interprétée comme une relaxation convexe du problème initial, qui consisterait à minimiser directement le nombre de composantes non nulles du signal. En remplaçant la pseudo-norme $\ell_0$ par la norme $\ell_1$, on obtient un problème convexe, garantissant l’existence d’une solution globale et permettant l’utilisation d’algorithmes d’optimisation efficaces.



\subsubsection{Interprétation des termes}
L’équilibre entre ces deux termes est crucial pour la qualité de la reconstruction. Un poids trop faible accordé à la régularisation peut conduire à une solution bruitée et peu parcimonieuse, tandis qu’un poids excessif peut supprimer des composantes pertinentes du signal. Le choix du paramètre $\tau$ constitue donc un aspect fondamental du problème, souvent traité par validation croisée ou analyse de sensibilité.
\begin{itemize}
    \item $\|Ax-y\|_2^2$ : terme de fidélité aux données (L2-norm)
    \item $\|x\|_1$ : terme de régularisation favorisant la parcimonie (L1-norm)
    \item $\tau$ : contrôle le compromis entre fidélité et parcimonie
\end{itemize}

Afin d’illustrer concrètement les principes du compressed sensing, nous considérons dans la suite une simulation numérique. Cette approche permet de valider expérimentalement les résultats théoriques et d’évaluer les performances de reconstruction dans un cadre contrôlé, où le signal original est parfaitement connu.
\subsection{Configuration du problème}
\subsubsection{Paramètres de la simulation}
Nous allons reconstruire un signal de dimension $n=5000$ avec seulement $S=100$ composantes non nulles.

\begin{minted}{python}
import numpy as np

# Paramètres du problème
n = 5000          # Dimension du signal
S = 100           # Nombre de composantes non nulles (sparsity level)

# Nombre de mesures nécessaire (théorème de Candès et al.)
m = int(np.ceil(S * np.log(n)))
print(f'Dimension du signal : {n}')
print(f'Nombre de mesures : {m}')
print(f'Taux de compression : {n/m:.2f}')
print(f'Niveau de parcimonie : {S} ({S/n*100:.1f}\% de non-zéros)')
\end{minted}

\subsubsection{Génération de la matrice de mesure}
La matrice $A$ doit satisfaire la propriété d'isométrie restreinte (RIP). En pratique, une matrice gaussienne aléatoire convient.

\begin{minted}{python}
# Pour la reproductibilité des résultats
np.random.seed(1)

# Génération de la matrice de mesure (gaussienne i.i.d.)
A = np.random.normal(size=(m, n))
print(f'Forme de A : {A.shape}')

# Calcul de la norme opérateur pour le pas de gradient
norm_A = np.linalg.norm(A, ord=2)
print(f'Norme opérateur de A : {norm_A:.4f}')
\end{minted}
Le choix d’une matrice de mesure aléatoire de type gaussien est motivé par ses excellentes propriétés théoriques. En effet, ce type de matrice satisfait la propriété d’isométrie restreinte avec une forte probabilité, à condition que le nombre de mesures soit suffisant. Cette propriété garantit que les distances entre signaux parcimonieux sont approximativement préservées après projection.

\subsubsection{Création du signal parcimonieux}
Nous générons un signal avec exactement $S$ composantes non nulles.

\begin{minted}{python}
# Création d'un signal S-parcimonieux
x_true = np.zeros(n)

# Sélection aléatoire des positions des composantes non nulles
indices_non_nuls = np.random.permutation(n)[:S]
valeurs_non_nulles = np.random.normal(size=S)

x_true[indices_non_nuls] = valeurs_non_nulles
x_true = x_true / np.linalg.norm(x_true)  # Normalisation

print(f'Nombre de composantes non nulles : {np.sum(x_true != 0)}')
print(f'Norme L1 du signal : {np.linalg.norm(x_true, 1):.4f}')
print(f'Norme L2 du signal : {np.linalg.norm(x_true, 2):.4f}')
\end{minted}
La normalisation du signal permet de travailler dans un cadre numérique stable et facilite l’interprétation des métriques de reconstruction. Elle permet également de comparer les performances pour différentes configurations sans être influencé par l’échelle du signal.

\subsubsection{Simulation des mesures acquises}
\begin{minted}{python}
# Génération des mesures (acquisition compressée)
y = np.dot(A, x_true)
print(f'Forme du vecteur de mesures y : {y.shape}')

# Vérification de la cohérence énergétique
energie_signal = np.linalg.norm(x_true)**2
energie_mesures = np.linalg.norm(y)**2 / norm_A**2
print(f'Énergie du signal : {energie_signal:.4f}')
print(f'Énergie estimée à partir des mesures : {energie_mesures:.4f}')
\end{minted}

\subsection{Définition des fonctions objectif}
La séparation du problème en plusieurs fonctions objectif est particulièrement adaptée aux méthodes d’optimisation proximales. Elle permet de traiter séparément les termes différentiables et non différentiables, tout en conservant une structure algorithmique simple et efficace.

\subsubsection{Fonction de régularisation L1}
La norme L1 favorise la parcimonie de la solution.

\begin{minted}{python}
from pyunlocbox import functions

# Paramètre de régularisation
tau = 1.0

# Définition de la fonction L1
f1 = functions.norm_l1(lambda_=tau)

print("Fonction L1 définie avec tau =", tau)
print("Évaluation sur un vecteur test :", 
      f1._eval(np.array([1, -2, 0, 3])))
\end{minted}

\subsubsection{Fonction de fidélité L2}
Plusieurs méthodes pour définir le terme de fidélité aux données.

\subsubsection{Méthode 1 : Matrice explicite}

\begin{minted}{python}
# Méthode directe avec matrice A
f2 = functions.norm_l2(y=y, A=A)
\end{minted}

\subsubsection{Méthode 2 : Opérateurs fonctionnels}

\begin{minted}{python}
# Méthode avec opérateurs fonctionnels (utile pour les grandes matrices)
f3 = functions.norm_l2(y=y)
f3.A = lambda x: np.dot(A, x)      # Opérateur forward
f3.At = lambda x: np.dot(A.T, x)   # Opérateur adjoint
\end{minted}

\subsubsection{Méthode 3 : Définition manuelle}

\begin{minted}{python}
# Définition complètement manuelle
f4 = functions.func()
f4._eval = lambda x: np.linalg.norm(np.dot(A, x) - y)**2
f4._grad = lambda x: 2.0 * np.dot(A.T, np.dot(A, x) - y)

# Test d'équivalence
x_test = np.random.normal(size=n)
print(f"f2(x_test) = {f2._eval(x_test):.6f}")
print(f"f3(x_test) = {f3._eval(x_test):.6f}")
print(f"f4(x_test) = {f4._eval(x_test):.6f}")
\end{minted}

\subsection{Algorithme Forward-Backward}
L’algorithme forward-backward appartient à la famille des méthodes de descente proximale. Il est particulièrement bien adapté aux problèmes de grande dimension rencontrés en compressed sensing, car il ne nécessite que des opérations matricielles simples et l’évaluation d’opérateurs proximaux explicites.
L’utilisation de l’accélération de Nesterov (FISTA) permet d’améliorer significativement la vitesse de convergence, en particulier pour des problèmes mal conditionnés ou de grande taille.

\subsection{Principe mathématique}
L'algorithme forward-backward résout des problèmes de la forme :
\[
\min_x f(x) + g(x)
\]
où $f$ est différentiable et $g$ admet un opérateur proximal simple.

L'itération s'écrit :
\[
x_{k+1} = \text{prox}_{\gamma g}(x_k - \gamma \nabla f(x_k))
\]
avec $\gamma \in ]0, 2/\beta[$ où $\beta$ est la constante de Lipschitz de $\nabla f$.

\subsection{Calcul du pas optimal}

\begin{minted}{python}
# Calcul de la constante de Lipschitz
beta = 2.0 * norm_A**2  # Pour f(x) = ||Ax - y||^2
print(f"Constante de Lipschitz beta = {beta:.4f}")

# Pas optimal pour la convergence
step = 0.5 / norm_A**2  # gamma = 1/beta
print(f"Pas d'itération optimal : {step:.6f}")
\end{minted}

\subsection{Configuration du solveur}

\begin{minted}{python}
from pyunlocbox import solvers

# Instanciation de l'algorithme forward-backward
solver = solvers.forward_backward(
    step=step,           # Pas d'itération
    method='FISTA',      # Accélération de Nesterov (optionnel)
    tol=1e-10,           # Tolérance sur la condition d'arrêt
)

print("Solveur forward-backward configuré")
print(f"Méthode : {solver.method}")
print(f"Pas : {solver.step}")
\end{minted}

\section{Résolution du problème}

\subsection{Lancement de l'optimisation}

\begin{minted}{python}
# Point initial (vecteur nul)
x0 = np.zeros(n)

# Résolution du problème d'optimisation
ret = solvers.solve(
    [f1, f2],           # Liste des fonctions objectif
    x0,                 # Point initial
    solver,             # Algorithme d'optimisation
    rtol=1e-4,          # Tolérance relative
    maxit=300,          # Nombre maximum d'itérations
    verbosity='LOW'     # Niveau de verbosité
)

# Affichage des résultats
print("\n" + "="*50)
print("RÉSULTATS DE L'OPTIMISATION")
print("="*50)
print(f"Solution trouvée en {ret['iter']} itérations")
print(f"Critère d'arrêt : {ret['crit']}")
print(f"Valeur objective finale : {ret['objective'][-1].sum():.6f}")
print(f"Temps de calcul : {ret['time']:.2f} secondes")
\end{minted}

\subsubsection{Analyse de la solution}

\begin{minted}{python}
# Récupération de la solution
x_recon = ret['sol']

# Calcul des métriques de reconstruction
mse = np.linalg.norm(x_recon - x_true)**2 / n
psnr = 10 * np.log10(1.0 / mse) if mse > 0 else float('inf')
support_error = np.sum((x_recon != 0) != (x_true != 0))

print(f"\nANALYSE DE LA RECONSTRUCTION")
print(f"Erreur quadratique moyenne (MSE) : {mse:.2e}")
print(f"PSNR : {psnr:.2f} dB")
print(f"Erreur de support : {support_error} composantes")
print(f"Rapport de compression effectif : {n/m:.2f}")
\end{minted}
Les métriques utilisées permettent d’évaluer différents aspects de la reconstruction. L’erreur quadratique moyenne mesure la fidélité globale du signal reconstruit, tandis que l’erreur de support évalue la capacité de l’algorithme à identifier correctement les positions des composantes non nulles. Ces deux critères sont complémentaires et essentiels pour juger la qualité d’une méthode de compressed sensing.

\subsection{Visualisation des résultats}

\subsubsection{Comparaison signal original/reconstruit}

\begin{minted}{python}
import matplotlib.pyplot as plt

fig, axes = plt.subplots(2, 2, figsize=(12, 8))

# Signal original vs reconstruit (vue globale)
ax = axes[0, 0]
ax.plot(x_true, 'b-', alpha=0.6, linewidth=0.5, label='Original')
ax.plot(x_recon, 'r--', alpha=0.8, linewidth=0.5, label='Reconstruit')
ax.set_xlabel('Index')
ax.set_ylabel('Amplitude')
ax.set_title('Signal original vs reconstruit')
ax.legend()
ax.grid(True, alpha=0.3)

# Zoom sur les composantes non nulles
ax = axes[0, 1]
non_zero_indices = np.where(x_true != 0)[0]
zoom_indices = non_zero_indices[:min(50, len(non_zero_indices))]
ax.stem(zoom_indices, x_true[zoom_indices], 
        linefmt='b-', markerfmt='bo', basefmt=' ', label='Original')
ax.stem(zoom_indices, x_recon[zoom_indices], 
        linefmt='r--', markerfmt='rx', basefmt=' ', label='Reconstruit')
ax.set_xlabel('Index')
ax.set_ylabel('Amplitude')
ax.set_title('Zoom sur les composantes non nulles')
ax.legend()
ax.grid(True, alpha=0.3)

# Erreur de reconstruction
ax = axes[1, 0]
erreur = x_recon - x_true
ax.plot(erreur, 'g-', linewidth=0.5)
ax.axhline(y=0, color='k', linestyle='--', alpha=0.5)
ax.set_xlabel('Index')
ax.set_ylabel('Erreur')
ax.set_title('Erreur de reconstruction')
ax.grid(True, alpha=0.3)

# Histogramme des amplitudes
ax = axes[1, 1]
ax.hist(x_true, bins=50, alpha=0.5, label='Original', density=True)
ax.hist(x_recon, bins=50, alpha=0.5, label='Reconstruit', density=True)
ax.set_xlabel('Amplitude')
ax.set_ylabel('Densité')
ax.set_title('Distribution des amplitudes')
ax.legend()
ax.grid(True, alpha=0.3)

plt.tight_layout()
plt.savefig('reconstruction_results.png', dpi=150, bbox_inches='tight')
plt.show()
\end{minted}

\subsubsection{Convergence de l'algorithme}

\begin{minted}{python}
# Analyse de convergence
objective = np.array(ret['objective'])

fig, axes = plt.subplots(1, 2, figsize=(12, 4))

# Convergence des objectifs individuels
ax = axes[0]
ax.semilogy(objective[:, 0], 'b-', label='Terme L1 (parcimonie)')
ax.semilogy(objective[:, 1], 'r-', label='Terme L2 (fidélité)')
ax.semilogy(np.sum(objective, axis=1), 'k-', 
            linewidth=2, label='Objectif total')
ax.set_xlabel('Itération')
ax.set_ylabel('Valeur objective')
ax.set_title('Convergence des fonctions objectif')
ax.legend()
ax.grid(True, alpha=0.3)

# Taux de décroissance
ax = axes[1]
obj_total = np.sum(objective, axis=1)
rate = np.diff(np.log(obj_total)) / np.diff(range(len(obj_total)))
ax.plot(rate, 'g-')
ax.axhline(y=0, color='k', linestyle='--', alpha=0.5)
ax.set_xlabel('Itération')
ax.set_ylabel('Taux de décroissance logarithmique')
ax.set_title('Taux de convergence')
ax.grid(True, alpha=0.3)

plt.tight_layout()
plt.savefig('convergence_analysis.png', dpi=150, bbox_inches='tight')
plt.show()
\end{minted}

\subsection{Analyse de sensibilité}

\subsubsection{Influence du paramètre $\tau$}

\begin{minted}{python}
def reconstruire_avec_tau(tau_value):
    """Fonction helper pour tester différents tau"""
    f1_tau = functions.norm_l1(lambda_=tau_value)
    solver_tau = solvers.forward_backward(step=step)
    
    ret_tau = solvers.solve(
        [f1_tau, f2],
        x0,
        solver_tau,
        rtol=1e-4,
        maxit=200,
        verbosity='NONE'
    )
    
    x_tau = ret_tau['sol']
    mse_tau = np.linalg.norm(x_tau - x_true)**2 / n
    sparsity_tau = np.sum(np.abs(x_tau) > 1e-4) / n
    
    return mse_tau, sparsity_tau, ret_tau['iter']

# Test de différentes valeurs de tau
tau_values = np.logspace(-3, 2, 20)
results = []

print("Analyse de sensibilité au paramètre tau:")
print("-" * 50)
print(f"{'tau':>10} {'MSE':>12} {'Sparsity':>12} {'Iterations':>12}")
print("-" * 50)

for tau in tau_values:
    mse_val, sparsity_val, iterations = reconstruire_avec_tau(tau)
    results.append((tau, mse_val, sparsity_val, iterations))
    print(f"{tau:10.2e} {mse_val:12.2e} {sparsity_val:12.4f} {iterations:12d}")

results = np.array(results)

# Visualisation
fig, axes = plt.subplots(1, 3, figsize=(15, 4))

ax = axes[0]
ax.loglog(results[:, 0], results[:, 1], 'bo-')
ax.set_xlabel('$\\tau$')
ax.set_ylabel('MSE')
ax.set_title('Erreur vs $\\tau$')
ax.grid(True, alpha=0.3)

ax = axes[1]
ax.semilogx(results[:, 0], results[:, 2], 'ro-')
ax.set_xlabel('$\\tau$')
ax.set_ylabel('Taux de parcimonie')
ax.set_title('Parcimonie vs $\\tau$')
ax.grid(True, alpha=0.3)

ax = axes[2]
ax.semilogx(results[:, 0], results[:, 3], 'go-')
ax.set_xlabel('$\\tau$')
ax.set_ylabel('Itérations')
ax.set_title('Convergence vs $\\tau$')
ax.grid(True, alpha=0.3)

plt.tight_layout()
plt.savefig('sensitivity_analysis.png', dpi=150, bbox_inches='tight')
plt.show()
\end{minted}
Les résultats obtenus confirment l’efficacité du compressed sensing pour la reconstruction de signaux parcimonieux à partir d’un nombre réduit de mesures. L’algorithme forward-backward, combiné à une régularisation $\ell_1$, offre un compromis pertinent entre précision de reconstruction, parcimonie et complexité algorithmique.
Cette étude peut être étendue à des signaux bruités, à d’autres types de matrices de mesure ou encore à des régularisations plus sophistiquées, ouvrant ainsi la voie à de nombreuses applications pratiques.

    % %===========================================================
% Chapitre 3 — Méthodes variationnelles, parcimonieuses et Compressive Sensing
%===========================================================
\chapter{Reconstruction d'images : problèmes inverses et compressive sensing}

\section{Cadre des problèmes inverses}
\subsection{Le problème inverse : formulation générale}
Un problème inverse consiste à estimer une quantité inconnue (ici, une image) à partir d'observations indirectes, bruitées et souvent incomplètes. Formellement, ce processus peut être modélisé par :

\begin{equation}
    \mathbf{y} = \mathcal{A}\mathbf{x} + \mathbf{n}
    \label{eq:inverse_problem}
\end{equation}

où :
\begin{itemize}
    \item[-] $\mathbf{x} \in \mathbb{R}^n$ représente l'image à reconstruire (inconnue),
    \item[-] $\mathbf{y} \in \mathbb{R}^m$ correspond aux données acquises (observations),
    \item[-] $\mathcal{A} : \mathbb{R}^n \rightarrow \mathbb{R}^m$ est l'opérateur direct modélisant le processus de dégradation (flou, projection, sous-échantillonnage, etc.),
    \item[-] $\mathbf{n} \in \mathbb{R}^m$ désigne le bruit de mesure additif.
\end{itemize}

L'opérateur $\mathcal{A}$ est typiquement \textit{mal conditionné} voire non inversible ($m < n$ dans le cas sous-déterminé), rendant la reconstruction de $\mathbf{x}$ à partir de $\mathbf{y}$ intrinsèquement difficile.


\begin{definition}
    Un problème est dit \textbf{bien posé} au sens de Hadamard s'il vérifie simultanément trois conditions :
    \begin{enumerate}
        \item \textbf{Existence :} Il existe au moins un $\mathbf{x} \in \mathbb{R}^n$ tel que $\mathcal{A} \mathbf{x} = \mathbf{y}$.
        
        \item \textbf{Unicité :} Cette solution est unique ; c'est-à-dire que si $\mathbf{z} \in \mathbb{R}^n$ vérifie $\mathcal{A} \mathbf{z} = \mathbf{y}$, alors nécessairement $\mathbf{z} = \mathbf{x}$.
        
        \item \textbf{Stabilité :} La solution dépend continûment des données $\mathbf{y}$. Plus précisément, pour toute suite $\{\mathbf{x}_n\}_{n\in\mathbb{N}} \subset \mathbb{R}^n$ telle que $\mathcal{A} \mathbf{x}_n \to \mathbf{y}$ (convergence dans $\mathbb{R}^m$), on a également $\mathbf{x}_n \to \mathbf{x}$ (convergence dans $\mathbb{R}^n$).
    \end{enumerate}
    Si l'une de ces trois conditions n'est pas vérifiée, le problème \eqref{eq:inverse_problem} est dit \emph{mal posé} au sens de Hadamard.\vspace{5pt}
\end{definition}

La majorité des problèmes inverses en imagerie violent au moins une de ces conditions, les rendant \textbf{mal posés}. La mal-positude se manifeste notamment par une grande sensibilité au bruit : de infimes variations des données $\mathbf{y}$ peuvent provoquer des changements arbitrairement grands dans la solution estimée.\vspace{5pt}

En pratique, les observations $\mathbf{y}$ sont corrompues par différents types de bruit (photonique, électronique, quantique, etc.) et sont souvent acquises de manière incomplète ($m \ll n$) pour des raisons techniques ou temporelles. Cette sous-détermination aggrave la mal-positude et rend le problème \textit{indéterminé} (multiples solutions possibles).\vspace{5pt}

La résolution d'un problème inverse mal posé nécessite l'injection d'\textit{information a priori} sur la solution recherchée. Ces contraintes peuvent être de différentes natures :
\begin{itemize}
    \item[-] Contraintes physiques (positivité, support limité, etc.)
    \item[-] Propriétés statistiques (distribution du bruit, régularité spatiale)
    \item[-] Structures spécifiques (parcimonie dans une base appropriée, bas rang, etc.)
\end{itemize}
L'intégration judicieuse de ces informations constitue le cœur des méthodes modernes de reconstruction.

% ================================
\section{Approches variationnelles} 
% ================================
Lors de la résolution de problèmes inverses, l'objectif est de retrouver une quantité inconnue \(\mathbf{x}\) à partir de données mesurées \(\mathbf{y}\). Comme les exemples précédents l'ont montré, cette tâche peut s'avérer difficile. Même si l'opérateur \(\mathcal{A}\) admet un inverse bien défini sur son image, c'est-à-dire que \(\mathcal{A}^{-1}: \mathbb{R}^m \rightarrow \mathbb{R}^n\) existe, rien ne garantit que les données bruitées appartiennent encore à l'image de l'opérateur. On peut également n'avoir accès qu'à des mesures partielles, ce qui mène à des systèmes sous-déterminés et rend l'inversion directe impossible, même sur des données non bruitées.\vspace{5pt}\\
\textbf{Régularisation.} Une approche pour obtenir des solutions significatives dans les scénarios décrits est appelée \textit{régularisation}. Une régularisation \(\mathfrak{R}_{\alpha}:\mathbb{R}^m \rightarrow \mathbb{R}^n\) associe à tout point de données une solution favorable. Intuitivement, on espère que la régularisation étend approximativement la notion d'inverse au cadre bruité et potentiellement mal posé, c'est-à-dire
\begin{equation}
    \mathfrak{R}_{\alpha}(\mathcal{A}\mathbf{x}+\varepsilon)\approx \mathbf{x}.
\end{equation}
Une stratégie typique est la régularisation dite \textit{de type Tikhonov} ou \textit{variationnelle}, où la sortie de l'application de régularisation est définie comme la solution du problème suivant

\begin{equation}
    \underset{\mathbf{x}\in\mathbb{R}^n}{\arg\min}
    \underbrace{\left\|{\mathcal{A}\mathbf{x}-\mathbf{y}} \right\|_{L^{2}}^{2}}_{\text{fidélité aux données}}+\alpha\underbrace{{\mathcal{R}}(\mathbf{x}) }_{\text{régularisant}}.
\end{equation}


Les termes s'interprètent comme suit :

\begin{itemize}
    \item \textbf{Fidélité aux données :} Ce terme indique à quel point notre estimation \(\mathbf{x}\) correspond aux données observées \(\mathbf{y}\). Bien que nous souhaitions minimiser la fidélité aux données, il n'est pas toujours pertinent de l'annuler dans le cas bruité, car nous cherchons \(\hat{\mathbf{x}}\) tel que \({\mathcal{A}}\hat{\mathbf{x}}=\mathbf{y}-\varepsilon\), où potentiellement \({\mathcal{A}}\hat{\mathbf{x}}\neq \mathbf{y}\).

    \item \textbf{Régularisant :} Le régularisant nous permet d'incorporer des informations supplémentaires sur la solution recherchée. Dans le cas classique, nous savons déjà que \(\mathbf{x}\) doit être proche d'un certain point \(\mu\), c'est-à-dire que nous voulons pénaliser la distance entre la solution \(\mathbf{x}\) et \(\mu\). De plus, on souhaite souvent pénaliser certaines directions plus que d'autres, pour lesquelles nous considérons un opérateur unitaire \(\mathcal{Q}:\mathbb{R}^{n}\to\mathbb{R}^{n}\) et choisissons ensuite
        \[{\mathcal{R}}(\mathbf{x})=\|\mathbf{x}-\mu\|_{L^{2},\mathcal{Q}}^{2}:=\langle \mathbf{x}-\mu,\mathcal{Q}(\mathbf{x}-\mu)\rangle\,,\] 
        ce qui est appelé \textit{régularisation de Tikhonov}\footnote{parfois appelée « régularisation aux moindres carrés »}.

    \item \textbf{Paramètre de régularisation \(\alpha\) :} Le paramètre \(\alpha>0\) contrôle la force du régularisant. Dans certaines formulations, ce paramètre est inclus dans la définition du régularisant.
\end{itemize}

Le terme de régularisation $\mathcal{R}(\mathbf{x})$ joue un rôle central dans la qualité de reconstruction obtenue. Il encode des hypothèses structurelles sur l'image recherchée et conditionne à la fois l'aspect visuel de la solution et la complexité numérique de l'algorithme utilisé.\newpage

Le choix du régularisateur dépend :
\begin{itemize}
    \item[-] du type d'images traitées,
    \item[-] de la nature du bruit,
    \item[-] de la structure de l'opérateur $\mathbf{A}$,
    \item[-] des contraintes de temps de calcul.
\end{itemize}
Des régularisations hybrides, combinant par exemple variation totale et parcimonie, sont fréquemment utilisées pour améliorer la qualité de reconstruction.

%===========================================================
\section{Régularisation classique}
%===========================================================
La régularisation classique désigne l'ensemble des méthodes visant à stabiliser la résolution des problèmes inverses en introduisant des contraintes supplémentaires sur la solution. Ces contraintes imposent des propriétés de l'image reconstruite telles que la régularité, la parcimonie des gradients ou la similarité avec des voisins. Les principales familles comprennent la régularisation quadratique de Tikhonov, la variation totale, les approches non locales et les méthodes multi-échelles.

% -------------------------------
\subsection{Régularisation de Tikhonov ($L^2$)}
% -------------------------------

Une méthode bien connue et efficace est la \textit{régularisation de Tikhonov}. Dans cette approche, une solution du problème \eqref{eq:inverse_problem} est approchée par une solution du problème de minimisation

\begin{equation}
    \min \|\mathcal{A}\mathbf{x} - \mathbf{y}\|_2^2 + \alpha \|\mathbf{x} - \mathbf{x}^*\|_2^2 \quad 
    \label{eq:tikhonov_problem}
\end{equation}

où \(\alpha > 0\) est un petit paramètre, \(\mathbf{y} \in \mathbb{R}^m\) est une approximation du membre de gauche exact \(\mathbf{y}\) du problème \eqref{eq:inverse_problem}, et \(x^*\) est une estimation a priori de la solution inconnue.\\
% Dans le cas d'un opérateur linéaire \(\mathcal{A}\), les aspects de stabilité, de convergence et de vitesses de convergence (quand \(\delta \to 0\)) ont été largement étudiés [2]. 
% Le rôle de la régularisation de Tikhonov pour stabiliser les problèmes d'estimation de paramètres a été examiné dans [3].
% Dans cette étude, nous montrons qu'une solution de l'équation
% \begin{equation}
%     (\mathcal{A}^*\mathcal{A} + \alpha \mathbf{I})x(\alpha) = \mathcal{A}^*\mathbf{y}
%     \label{eq:tikhonov_solution}
% \end{equation}
% existe et est unique, et que \(\mathcal{R}_{xy}\) défini par \(x(\alpha)\) est un régularisateur de l'équation \eqref{eq:inverse_problem} sur \(\mathbb{R}^n\), à condition que l'équation \(Ax = 0\) n'ait que la solution nulle.

D'après l'article \cite{7}, la démonstration de l'existence, de l'unicité et de la stabilité de la solution du problème de minimisation de Tikhonov est principalement détaillée dans les sections 2 et 3.

\subsubsection{Existence de la solution}
L'existence d'une solution au problème de minimisation est établie à la fois par l'analyse fonctionnelle générale et par une construction opératorielle spécifique :
\begin{itemize}
    \item \textbf{Lemme 1 :} Les auteurs affirment qu'une solution $R_\alpha y$ du problème de minimisation $\min (\|Av - y\|^2_Y + \alpha M(v))$ \textbf{existe} . La preuve repose sur le fait que $M$ est une fonctionnelle semi-continue inférieurement et que l'ensemble $\{v \in X_M : M(v) \leq r\}$ est compact .
    \item \textbf{Système de fonctions propres :} Dans la section 3, le texte déclare explicitement : « Tout d'abord, nous prouvons l'\textbf{existence} et l'unicité ». Cela est démontré en montrant que la solution $x(\alpha)$ peut être représentée à l'aide d'un système complet de fonctions propres $\{L_k\}$ de l'opérateur autoadjoint $\mathcal{A}^*\mathcal{A}$ .
\end{itemize}

\subsubsection{Unicité de la solution}
L'article fournit deux justifications mathématiques distinctes pour l'unicité de la solution :
\begin{itemize}
    \item \textbf{Détermination des composantes :} Dans la section 3, les composantes de la solution $x_k(\alpha)$ sont montrées être \textbf{déterminées de manière unique} par la formule :
    \[ x_k(\alpha) = \frac{y^A_k}{\lambda_k + \alpha} \]
    où $\lambda_k$ sont les valeurs propres de $\mathcal{A}^*\mathcal{A}$
    \item \textbf{Opérateur injectif :} Les références prouvent en outre l'unicité en montrant que l'opérateur $(\mathcal{A}^*\mathcal{A} + \alpha I)$ est positif. Plus précisément, ils démontrent que le \textbf{noyau} de $(\mathcal{A}^*\mathcal{A} + \alpha I)$ est $\{0\}$, ce qui garantit que toute solution existante est nécessairement unique.
\end{itemize}

\subsubsection{Stabilité de la solution}
La stabilité est abordée comme la dépendance continue de la solution par rapport aux données et au paramètre $\alpha$ :
\begin{itemize}
    \item \textbf{Preuve de continuité :} Dans la section 3, les auteurs déclarent explicitement : « Nous montrons maintenant que $R_\alpha$, c'est-à-dire $x(\alpha)$, est \textbf{continu} » [5]. Ils fournissent une preuve montrant que lorsque la différence entre les paramètres $|\alpha' - \alpha|$ tend vers zéro, la distance entre les solutions correspondantes $\|x(\alpha) - x(\alpha')\|$ s'annule également.
    \item \textbf{Propriété de régularisation :} Par définition, la famille d'opérateurs $R_\alpha$ est montrée être un \textbf{régulariseur}, ce qui signifie qu'elle fournit une approximation stable qui converge ponctuellement vers la solution exacte lorsque $\alpha \to 0$, sous réserve que les données soient cohérentes ($y = \mathcal{A}x$).
\end{itemize}

% % -------------------------------
\subsection{Variation Totale (TV)}
% -------------------------------

Afin de pallier les limitations de la régularisation quadratique, la régularisation par variation totale a été introduite pour préserver les discontinuités. Dans sa forme isotrope, elle est définie par :
\begin{equation}
    \mathcal{R}_{\text{TV}}(\mathbf{x}) 
    = \sum_{i,j} \sqrt{(\nabla_x x_{i,j})^2 + (\nabla_y x_{i,j})^2}
    = \|\nabla \mathbf{x}\|_1.
\end{equation}

L'emploi de la norme $L^1$ du gradient conduit à :
\begin{itemize}
    \item la préservation des bords nets,
    \item la suppression efficace du bruit,
    \item la production d'images par morceaux quasi-constants.
\end{itemize}

La minimisation associée est cependant non lisse et nécessite des méthodes itératives de type primal-dual, ADMM ou Split-Bregman \cite{8}. Un inconvénient notable est l'apparition du phénomène dit de \emph{staircasing}, caractérisé par des paliers artificiels.

% -------------------------------
\subsection{Méthodes non locales}
% -------------------------------

Les méthodes non locales exploitent la redondance statistique présente dans l'image en considérant des similarités entre pixels spatialement éloignés. Plutôt que de ne considérer que le voisinage local, elles s'appuient sur une mesure de similarité entre patchs.

Une formulation typique repose sur la pénalisation :
\begin{equation}
    \mathcal{R}_{\text{NL}}(\mathbf{x}) =
    \sum_{i,j} w_{ij} \, (x_i - x_j)^2,
\end{equation}
où $w_{ij}$ représente un poids mesurant la similarité entre les régions autour des pixels $i$ et $j$.

Ces approches permettent :
\begin{itemize}
    \item meilleure préservation des textures,
    \item réduction du bruit sans sur-lissage,
    \item exploitation de structures répétitives.
\end{itemize}

Elles sont à la base des méthodes telles que \textit{Non-Local Means} et des approches basées patchs pour la débruitage et la défloutage.

% -------------------------------
\subsection{Approches multi-échelles (ondelettes, curvelets)}
% -------------------------------

Les approches multi-échelles s'appuient sur le fait que les images naturelles présentent des structures à plusieurs résolutions. Les représentations dans des bases telles que les ondelettes, les curvelets ou les contourlets permettent de capturer efficacement :

\begin{itemize}
    \item les singularités orientées,
    \item les contours,
    \item les textures fines.
\end{itemize}

La régularisation consiste alors à imposer la parcimonie des coefficients transformés :
\begin{equation}
    \mathcal{R}_{\text{MS}}(\mathbf{x}) = \|\mathbf{\Psi x}\|_1,
\end{equation}
où $\mathbf{\Psi}$ désigne une transformée multi-résolution.

Les ondelettes sont particulièrement adaptées aux singularités ponctuelles, tandis que les curvelets et shearlets offrent une meilleure représentation des structures anisotropes et courbes. Ces méthodes constituent un lien direct avec le \textit{Compressed Sensing} et les modèles parcimonieux modernes.


%===========================================================
\section{Modèles parcimonieux}
%===========================================================

Les modèles parcimonieux reposent sur l'hypothèse qu'une image ou un signal peut être représenté par un nombre réduit de coefficients significatifs dans une base appropriée ou un dictionnaire sur-complet. Cette propriété est à la base de nombreuses techniques modernes de reconstruction d'images, de compression et de \textit{Compressed Sensing}. L'objectif est de promouvoir des représentations compactes permettant de régulariser les problèmes inverses mal posés.

% -------------------------------
\subsection{Bases orthogonales vs dictionnaires}
% -------------------------------

Une représentation parcimonieuse peut être obtenue soit dans une \textbf{base orthogonale}, soit dans un \textbf{dictionnaire sur-complet}.

\paragraph{Bases orthogonales}

Une base orthogonale $\{\mathbf{\phi}_k\}_{k=1}^{N}$ de $\mathbb{R}^n$ permet d'écrire :
\begin{equation}
    \mathbf{x} = \sum_{k=1}^{N} \alpha_k \mathbf{\phi}_k,
\end{equation}
où les coefficients $\alpha_k$ sont uniques. Exemples courants :
\begin{itemize}
    \item ondelettes orthogonales,
    \item base de Fourier,
    \item transformée discrète du cosinus (DCT).
\end{itemize}

Les bases orthogonales ont l'avantage de garantir l'unicité des coefficients et de permettre des calculs rapides via des transformées rapides (FFT, DWT, DCT).

\paragraph{Dictionnaires sur-complets}

Un dictionnaire $\mathbf{D} \in \mathbb{R}^{N \times K}$ avec $K > n$ est dit sur-complet. Dans ce cas, un signal peut avoir plusieurs décompositions possibles :
\begin{equation}
    \mathbf{x} = \mathbf{D}\mathbf{\alpha},
\end{equation}
où $\mathbf{\alpha} \in \mathbb{R}^{K}$ est un vecteur de coefficients.

Les dictionnaires sur-complets offrent :
\begin{itemize}
    \item une meilleure capacité d'adaptation aux structures complexes,
    \item des représentations plus parcimonieuses,
    \item la possibilité d'apprentissage à partir des données.
\end{itemize}
Cependant, la décomposition n'est plus unique et nécessite la résolution de problèmes d'optimisation.

% -------------------------------
\subsection{Modèles sparse : $L^0$ et $L^1$}
% -------------------------------

La parcimonie consiste à rechercher une représentation comportant le moins de coefficients non nuls possible. Le problème fondamental est :
\begin{equation}
    \min_{\mathbf{\alpha}} \|\mathbf{\alpha}\|_0 
    \quad \text{sous la contrainte} \quad 
    \mathbf{x} = \mathbf{D}\mathbf{\alpha},
\end{equation}
où $\|\mathbf{\alpha}\|_0$ désigne le nombre de coefficients non nuls. Ce problème est combinatoire et NP-difficile.

Pour rendre la résolution praticable, on remplace la norme $L^0$ par la norme convexe $L^1$ :
\begin{equation}
    \min_{\mathbf{\alpha}} \|\mathbf{\alpha}\|_1 
    \quad \text{sous} \quad 
    \mathbf{x} = \mathbf{D}\mathbf{\alpha},
\end{equation}
ou, en présence de bruit,
\begin{equation}
    \min_{\mathbf{\alpha}} 
    \left\{
    \frac{1}{2}\|\mathbf{x} - \mathbf{D}\mathbf{\alpha}\|_2^2 + 
    \alpha \|\mathbf{\alpha}\|_1
    \right\}.
\end{equation}

La relaxation $L^1$ constitue la base mathématique des approches de \textit{Basis Pursuit}, \textit{LASSO} et du \textit{Compressed Sensing}.

% -------------------------------
\subsection{Algorithmes d'approximation parcimonieuse}
% -------------------------------

Plusieurs classes d'algorithmes permettent de résoudre les problèmes parcimonieux.

\paragraph{Méthodes gloutonnes}

Ces méthodes sélectionnent itérativement les atomes du dictionnaire qui expliquent au mieux le signal résiduel. Parmi elles :
\begin{itemize}
    \item Matching Pursuit (MP),
    \item Orthogonal Matching Pursuit (OMP),
    \item Stagewise OMP (StOMP).
\end{itemize}

Elles présentent un faible coût de calcul et sont adaptées aux dictionnaires de grande taille.

\paragraph{Méthodes par seuillage}

Ces méthodes reposent sur le seuillage doux ou dur des coefficients :
\begin{itemize}
    \item ISTA (Iterative Shrinkage-Thresholding Algorithm),
    \item FISTA (version accélérée),
    \item algorithmes proximal-gradient.
\end{itemize}

Elles sont particulièrement adaptées à la minimisation de fonctionnelles $L^2$–$L^1$ convexes.

% -------------------------------
\subsection{K-SVD : apprentissage de dictionnaire}
% -------------------------------

Le K-SVD est un algorithme d'apprentissage de dictionnaire visant à construire un dictionnaire sur-complet directement à partir d'un ensemble d'images ou de patchs. L'objectif est de résoudre :
\begin{equation}
    \min_{\mathbf{D},\mathbf{\alpha}_i}
    \sum_{i}
    \left\|
    \mathbf{x}_i - \mathbf{D}\mathbf{\alpha}_i
    \right\|_2^2
    \quad
    \text{sous la contrainte}
    \quad
    \|\mathbf{\alpha}_i\|_0 \leq T_0,
\end{equation}
où $\{\mathbf{x}_i\}$ sont les signaux d'apprentissage et $T_0$ fixe le niveau de parcimonie.

L'algorithme alterne :
\begin{enumerate}
    \item une étape de \textbf{codage parcimonieux} des coefficients,
    \item une étape de \textbf{mise à jour du dictionnaire} atome par atome via la décomposition en valeurs singulières (SVD).
\end{enumerate}

Le K-SVD permet de construire des dictionnaires adaptés aux structures réelles des images, conduisant à d'excellentes performances en :
\begin{itemize}
    \item débruitage,
    \item défloutage,
    \item inpainting,
    \item super-résolution.
\end{itemize}


% ============================================================
\section{Théorie du Compressive Sensing}
% ============================================================

Le compressed sensing fournit un cadre rigoureux pour la reconstruction de signaux parcimonieux ou compressibles à partir de mesures linéaires fortement sous-échantillonnées. Il offre une solution fondée sur des principes solides aux problèmes inverses sous-déterminés, via des méthodes d'optimisation favorisant la parcimonie ou des algorithmes d'approximation gloutons. Cette approche a un impact majeur en imagerie médicale, imagerie computationnelle, résolution de problèmes inverses et systèmes de communication modernes.
\begin{definition}
    Le \emph{compressed sensing} (CS) est un cadre mathématique et algorithmique qui permet la reconstruction de signaux de grande dimension à partir d'un nombre de mesures significativement inférieur à celui requis par les méthodes traditionnelles.
\end{definition}
\begin{definition}
    Soit $x\in \mathbb{R}^{n}$ un signal inconnu. On dit que $x$  est \(k\)-parcimonieux dans une base (ou dictionnaire) \(\Psi\) (ex: Fourier, wavelet, DCT) si
    \[
        x=\Psi \alpha, \qquad \text{où } \alpha \text{ possède au plus } k \ll n \text{ coefficients non nuls}.   
    \]
\end{definition}
% Le principe fondamental repose sur la \textbf{parcimonie}. 
En pratique, de nombreux signaux ne sont pas parcimonieux dans leur domaine original (canonique), mais le deviennent après l'application d'une transformation linéaire. 
On observe des mesures linéaires de la forme
\[
y = A x,
\]
où \(A \in \mathbb{R}^{m \times n}\) avec \(m \ll n\). La théorie classique de l'échantillonnage exige \(m = n\) mesures indépendantes pour une reconstruction exacte, tandis que le compressed sensing montre que
\[
m \gtrsim k \log(n/k)
\]
est suffisant pour une reconstruction exacte ou stable, sous des conditions appropriées sur \(A\), telles que la propriété d'isométrie restreinte (\emph{Restricted Isometry Property}, RIP) ou l'incohérence.\\

Les résultats classiques de l'échantillonnage, tels que le théorème de Nyquist--Shannon, imposent qu'un signal soit échantillonné à une fréquence proportionnelle à sa bande passante. Le compressed sensing remet en cause ce paradigme en observant que de nombreux signaux réels (images, données médicales, spectres) sont parcimonieux ou compressibles dans une base de transformation (par exemple ondelettes, Fourier, DCT). Par conséquent, leur dimension effective est bien plus faible que le nombre d'échantillons disponibles. Le compressed sensing exploite cette redondance pour réduire drastiquement les coûts d'acquisition. Le compressed sensing s'attaque au problème général suivant :
\begin{center}
    \vspace*{\fill}
        Comment reconstruire un signal parcimonieux de grande dimension à partir d'un ensemble sous-déterminé de mesures linéaires ?
    \vspace*{\fill}
\end{center}
Ce cadre permet de résoudre plusieurs limitations pratiques :
\paragraph{Réduction du nombre de mesures.}\text{}\\ 
De nombreux systèmes d'acquisition sont limités par le coût, le temps ou l'énergie. Le compressed sensing permet :
\begin{itemize}
    \item[-] une acquisition plus rapide des données,
    \item[-] une réduction de la complexité matérielle,
    \item[-] une diminution de la dose de radiation (par exemple en tomodensitométrie),
    \item[-] une réduction des coûts de stockage et de transmission.
\end{itemize}

\paragraph{Problèmes inverses mal posés (ill-posed inverse problems).}\text{}\\
Lorsque le nombre de mesures est insuffisant pour garantir une solution unique, le CS introduit une régularisation fondée sur la parcimonie, permettant une reconstruction stable. Les principales applications incluent :
\begin{itemize}
    \item[-] la tomographie (CT, IRM, PET),
    \item[-] l'imagerie à super-résolution,
    \item[-] la déconvolution,
    \item[-] les inversions géophysiques et les essais non destructifs.
\end{itemize}

\paragraph{Robustesse au bruit et aux données incomplètes.} \text{}\\
Le CS garantit une reconstruction stable même en présence de bruit, de corruptions ou d'observations manquantes.

\subsection{Reconstruction de signaux par Compressed Sensing}

\subsubsection{Reconstruction par optimisation}
La formulation canonique de la reconstruction est
\[
\min_{\alpha} \|\alpha\|_{1} \quad \text{s.c.} \quad y = A \Psi \alpha,
\]
ou, en présence de bruit,
\[
\min_{\alpha} \|\alpha\|_{1} \quad \text{s.c.} \quad \|A \Psi \alpha - y\|_{2} \le \epsilon.
\]
Cela correspond aux formulations de type \emph{Basis Pursuit} ou \emph{LASSO}. La minimisation de la norme \(\ell_1\) favorise la parcimonie tout en conservant un problème d'optimisation convexe et calculable efficacement.

\subsubsection{Algorithmes gloutons}

Des alternatives plus rapides incluent :
\begin{itemize}
    \item l'\emph{Orthogonal Matching Pursuit} (OMP),
    \item le \emph{Compressive Sampling Matching Pursuit} (CoSaMP),
    \item l'\emph{Iterative Hard Thresholding} (IHT).
\end{itemize}
Ces méthodes échangent une partie de la précision contre un coût computationnel réduit.

\subsection{Applications du Compressed Sensing}

\paragraph{Imagerie médicale.}
\begin{itemize}
    \item acquisition IRM accélérée,
    \item CT à dose réduite,
    \item échographie à haute cadence d'images.
\end{itemize}

\paragraph{Imagerie computationnelle.}
\begin{itemize}
    \item caméras à pixel unique,
    \item imagerie à ouverture codée,
    \item reconstruction hyperspectrale.
\end{itemize}

\paragraph{Télédétection et géophysique.}
\begin{itemize}
    \item inversion sismique parcimonieuse,
    \item imagerie radar et radar à synthèse d'ouverture (SAR).
\end{itemize}

\paragraph{Communications sans fil.}
\begin{itemize}
    \item estimation parcimonieuse de canaux,
    \item réduction des pilotes dans les systèmes MIMO massifs.
\end{itemize}

\paragraph{Apprentissage automatique et traitement du signal.}
\begin{itemize}
    \item régression parcimonieuse (LASSO),
    \item apprentissage de dictionnaires,
    \item ACP robuste et modèles de rang faible apparentés.
\end{itemize}


% Le compressed sensing (CS) est un cadre mathématique et algorithmique qui permet de reconstruire des signaux de grande dimension à partir d'un nombre de mesures bien inférieur à celui requis par les approches traditionnelles. Il exploite la parcimonie (sparsity) comme principal a priori structurel.
% \subsection{Hypothèse de parcimonie}
% Si un signal est parcimonieux ou compressible dans une certaine base, alors il peut être reconstruit exactement (ou avec une erreur contrôlée) à partir d'un nombre de mesures linéaires bien inférieur à sa dimension ambiante.
% \subsection{Incohérence et propriété de RIP}
% \subsection{Basis Pursuit et LASSO}
% \subsection{OMP et algorithmes gloutons}


%===========================================================
\section{Méthodes d'optimisation}
%===========================================================

Les problèmes inverses régularisés et les modèles parcimonieux conduisent le plus souvent à la minimisation de fonctionnelles non différentiables, voire contraintes. Le choix d'une méthode d'optimisation adaptée conditionne la qualité de la reconstruction, la rapidité de convergence et la robustesse numérique. Les approches modernes reposent notamment sur les méthodes proximales, les schémas de décomposition de type ADMM, ainsi que les méthodes de gradient projeté.

% -------------------------------
\subsection{Méthodes proximales}
% -------------------------------

Les méthodes proximales constituent le cadre de référence pour l'optimisation de fonctionnelles comportant des termes non lisses. Soit une fonction convexe propre et semi-continue inférieurement $f : \mathbb{R}^n \rightarrow \mathbb{R}\cup\{+\infty\}$. L'opérateur proximal associé est défini par :
\begin{equation}
    \mathrm{prox}_{\alpha f}(\mathbf{x}) =
    \arg\min_{\mathbf{z}} 
    \left\{
        f(\mathbf{z}) + \frac{1}{2\alpha}\|\mathbf{z} - \mathbf{x}\|_2^2
    \right\}.
\end{equation}

Cet opérateur permet de traiter naturellement des pénalités non différentiables telles que :
\begin{itemize}
    \item la norme $\ell_1$ (seuilage doux),
    \item la variation totale,
    \item les contraintes indicatrices de convexes fermés.
\end{itemize}

Les algorithmes emblématiques incluent :
\begin{itemize}
    \item Forward–Backward Splitting,
    \item FISTA (accéléré de Nesterov),
    \item Primal–Dual de Chambolle–Pock.
\end{itemize}

Ils sont particulièrement adaptés aux problèmes de la forme :
\begin{equation}
    \min_{\mathbf{x}} \; f(\mathbf{x}) + g(\mathbf{x}),
\end{equation}
où $f$ est différentiable à gradient lipschitzien et $g$ est convexe éventuellement non lisse.

% -------------------------------
\subsection{Méthodes ADMM}
% -------------------------------

L'Alternating Direction Method of Multipliers (ADMM) est une méthode de décomposition permettant de résoudre des problèmes séparables en introduisant des variables auxiliaires. On considère typiquement le problème :
\begin{equation}
    \min_{\mathbf{x},\mathbf{z}} \;
    f(\mathbf{x}) + g(\mathbf{z})
    \quad \text{sous la contrainte} \quad
    \mathbf{Kx} = \mathbf{z}.
\end{equation}

Le schéma itératif repose sur :
\begin{enumerate}
    \item minimisation alternée sur $\mathbf{x}$ et $\mathbf{z}$,
    \item mise à jour des multiplicateurs de Lagrange.
\end{enumerate}

Les avantages majeurs d'ADMM sont :
\begin{itemize}
    \item traitement naturel des contraintes linéaires,
    \item parallélisation possible des sous-problèmes,
    \item robustesse pour les grands problèmes mal conditionnés.
\end{itemize}

ADMM est aujourd'hui une référence pour la résolution de problèmes de variation totale, de débruitage parcimonieux et d'apprentissage de dictionnaire.

% -------------------------------
\subsection{Méthodes de gradient projeté}
% -------------------------------

Les méthodes de gradient projeté visent la résolution de problèmes d'optimisation sous contraintes convexes :
\begin{equation}
    \min_{\mathbf{x} \in C} f(\mathbf{x}),
\end{equation}
où $C$ est un ensemble convexe fermé. L'itération générique s'écrit :
\begin{equation}
    \mathbf{x}^{k+1} =
    \Pi_{C}\left(
        \mathbf{x}^k - \alpha_k \nabla f(\mathbf{x}^k)
    \right),
\end{equation}
où $\Pi_{C}$ désigne l'opérateur de projection sur $C$.

Ces méthodes sont notamment utilisées pour :
\begin{itemize}
    \item l'imposition de contraintes de positivité,
    \item la borne supérieure ou inférieure sur des intensités,
    \item les contraintes de norme sur des coefficients parcimonieux.
\end{itemize}

Elles constituent des schémas simples, peu coûteux en mémoire, et très utilisés en traitement d'images et en reconstruction tomographique lorsque la fonction coût est différentiable.

% \subsection{Unrolling des algorithmes (vers deep learning)}

% \section{Synthèse critique}
% \subsection{Avantages et limites}
% \subsection{Cas où le compressive sensing excelle}
% \subsection{Motivation des approches apprises}

    % %===========================================================
% Chapitre 4 — Méthodes basées sur l'apprentissage
%===========================================================
\chapter{Méthodes basées sur l'apprentissage}

\section{Fondements théoriques du deep learning appliqué à la reconstruction}
\subsection{Apprentissage de régularisation}
\subsection{Approches data-driven vs physics-based}
\subsection{Architectures CNN, U-Net, ResNet}

\section{Approches supervisées}
\subsection{Reconstruction directe image-à-image}
\subsection{Modèles guidés par sinogrammes (CT)}
\subsection{Méthodes itératives apprises}
\subsubsection{Networks unrollés}
\subsubsection{ISTA-Net}
\subsubsection{ADMM-Net}

\section{Approches non supervisées et auto-supervisées}
\subsection{Modèles génératifs (GAN, VAE)}
\subsection{Méthodes Noise2Noise, Noise2Void}
\subsection{Auto-supervision pour MRI sous-échantillonnée}

\section{Régularisation neuronale et méthodes hybrides}
\subsection{Deep Image Prior}
\subsection{Plug-and-Play (PnP)}
\subsection{Regularization by Denoising (RED)}
\subsection{Physics-Informed Neural Networks (PINNs)}

\section{Applications}
\subsection{Tomographie (CT)}
\subsubsection{Low-dose CT}
\subsubsection{Sparse-view CT}
\subsection{IRM (MRI)}
\subsubsection{Reconstruction accélérée}
\subsubsection{Super-résolution}
\subsection{OCT, microscopie, hyperspectral}

\section{Analyse expérimentale}
\subsection{Métriques (PSNR, SSIM, NMSE, FID)}
\subsection{Comparaison avec l'état de l'art}
\subsection{Analyse de robustesse}

    % ================================================================

    % ================================================================
    % ADD BIBILIOGRAPHY HERE
    \bibliographystyle{unsrt}
    \begin{thebibliography}{100} % 100 is a random guess of the total number of references
        \bibitem{1} Wei Zou, Jiajun Wang, David Dagan Feng, \emph{Image reconstruction of fluorescent molecular
            tomography based on the tree structured Schur complement decomposition}
        \bibitem{2}RANDRIANARISON N.Tsirilala, RANDRIAMITANTSOA P. Auguste, RAMAFIARISON H. Malalatiana, \emph{Performance of K-SVD algorithm in digital Optical Coherence Tomography}
        \bibitem{3} Linwei Fan, Fan Zhang, Hui Fan, and Caiming Zhang, \emph{Brief review of image denoising techniques}
        \bibitem{4} Michael S. Hansen, and Peter Kellman, \emph{Image reconstruction: an overview from clinicians}
        \bibitem{5} Joshua Trzasko*, Member, IEEE, and Armando Manduca, Member, IEEE, \emph{Highly Undersampled Magnetic Resonance Image Reconstruction via Homotopic `0-Minimization}
        \bibitem{6} Helmholtz Imaging, Deutsches Elektronen-Synchrotron DESY, Notkestr. 85, 22607 Hamburg, Germany, \emph{ntroduction to Regularization and Learning Methods for Inverse Problems}
        \bibitem{7} Bunyamin Yildiz *, Murat Subasi, Ali Sever, \emph{On a regularization problem}
        \bibitem{8} J. P. Bregman, \emph{Bregman Iteration for Correspondence Problems}
    \end{thebibliography}
    % ================================================================

\end{document}