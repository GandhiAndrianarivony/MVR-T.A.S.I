\documentclass[12pt, a4paper, oneside]{extbook}
\usepackage[T1]{fontenc}
\usepackage[provide=*,french]{babel}
\usepackage[utf8]{inputenc}
\usepackage[top=2cm,right=3cm,left=2cm]{geometry}
\usepackage{soul}

\usepackage{amsmath, amssymb, amsthm, mathrsfs}
\usepackage{xcolor}
\usepackage{import}
\usepackage{bbm}
\usepackage{comment}

\usepackage[hidelinks]{hyperref}
\usepackage{cleveref}

\usepackage{graphicx}
% \usepackage{caption}
\usepackage{appendix}
\usepackage[textfont=bf]{caption}
\usepackage[labelfont=bf]{caption}

\usepackage{subcaption}
\usepackage{tabularx}
\usepackage{titlesec}
\usepackage{matlab-prettifier}
\usepackage{float}
\usepackage{minted}
\usepackage{fancyhdr}
\usepackage{longtable}
\usepackage{algorithm}
\usepackage{algpseudocode} 
\usepackage{tocloft}
\usepackage{pdfpages}  % Pour insérer des PDF
\usepackage{nomencl}
\usepackage{etoolbox}
\makenomenclature


% Show Prop, Lemma and other label near them
% \usepackage{showkeys}

\numberwithin{equation}{chapter}

% \setcounter{chapter}{-1}

% ================================================================
\theoremstyle{definition}
\newtheorem{definition}{Définition}[section]
\newtheorem{proposition}{Proposition}[section]
\newtheorem{lemme}{Lemme}[section]
\newtheorem{corollaire}[proposition]{Corollaire}
\newtheorem{property}[proposition]{Propriété}
\newtheorem{remarque}{Remarque}[section]


% Afficher les numéros en arabic
\renewcommand{\thedefinition}{\arabic{chapter}.\arabic{definition}}
\renewcommand{\theproposition}{\arabic{chapter}.\arabic{proposition}}
\renewcommand{\thelemme}{\arabic{chapter}.\arabic{lemme}}
\renewcommand{\thecorollaire}{\arabic{chapter}.\arabic{corollaire}}
\renewcommand{\theproperty}{\arabic{chapter}.\arabic{property}}
% ================================================================

% =================================================================
% CHAPTER STYLE
% numbering
\renewcommand{\thechapter}{\Roman{chapter}}
\renewcommand{\cftchappresnum}{CHAPITRE }
\renewcommand{\cftchapaftersnum}{: }
\setlength{\cftchapnumwidth}{8.5em}

% font_size
\titleformat{\chapter}[display]
  {\rmfamily\bfseries} % <-- change ici la police : \rmfamily, \sffamily, \ttfamily
  {CHAPITRE \ \thechapter}{20pt}{\large}


\newcommand{\annexechapter}[1]{%
  \refstepcounter{chapter}%
  \chapter*{ANNEXE \ \thechapter\quad #1}%
  \addcontentsline{toc}{chapter}{ANNEXE \ \thechapter\quad #1}%
}
% =================================================================
% FIGURES STYLE
% Numérotation des figures en chiffres arabes
\renewcommand{\thefigure}{\arabic{chapter}.\arabic{figure}}
% ===== LISTE DES FIGURES : STYLE PERSONNALISÉ =====
\addto\captionsfrench{%
  \renewcommand{\listfigurename}{\large LISTE DES FIGURES}
}

% Préfixe "Figure"
\renewcommand{\cftfigpresnum}{Figure }
\renewcommand{\cftfigaftersnum}{ : }

% Largeur pour "Figure 1.10 :"
\setlength{\cftfignumwidth}{6em}
% =================================================================
% EQUATION STYLE
\renewcommand{\theequation}{\arabic{chapter}.\arabic{equation}}
% =================================================================
% NOMENCLATURE STYLE
\renewcommand{\nomname}{\large Liste des abréviations}
% =================================================================
% SECTION STYLE
% font_size
\titleformat{\section}
  {\normalfont\fontsize{12}{14}\selectfont\bfseries}
  {\thesection}{1em}{}

\titleformat{\subsection}
  {\normalfont\fontsize{12}{14}\selectfont\bfseries}
  {\thesubsection}{1em}{}

% subsubsection to a. b. c.
\setcounter{secnumdepth}{3}
\renewcommand{\thesubsubsection}{\alph{subsubsection}.}

% =================================================================
% DEFINITON STYLE
% text (it, gras, ...)
\begingroup
    \makeatletter
    \g@addto@macro\th@definition{\normalfont}
    \makeatother
\endgroup
% =================================================================
\let\cleardoublepage\clearpage
\setminted{frame=lines, framesep=2mm, fontsize=\scriptsize, linenos}
% ================================================================
% PAGE NUMBERING STYLE
\pagestyle{fancy}
\fancyhf{} % Clear all header and footer fields
\fancyfoot[C]{\thepage} % Page number in the center of the footer
\renewcommand{\headrulewidth}{0pt} % Remove header rule
\renewcommand{\footrulewidth}{0pt} % Remove footer rule if you want
\fancyhead{} % Clear all header content
% ================================================================



\begin{document}
    \includepdf[pages=-, fitpaper=true]{Couverture MVR TASI.pdf}

    % blan page
    \newpage
    \thispagestyle{empty}
    ~
    \newpage
    % Remerciements
    % Page de remerciements
\chapter*{\large REMERCIEMENTS} % * pour ne pas numéroter le chapitre
\thispagestyle{empty} % pas de numéro de page pour cette page
\begin{flushleft} % texte aligné à gauche
Je souhaite exprimer ma profonde gratitude à Dieu Tout-Puissant, source de sagesse et d'inspiration, pour son amour, sa miséricorde et sa guidance tout au long de mes études et pour la réalisation de ce mémoire.\vspace{8pt}\\

Je remercie chaleureusement Monsieur le Chef du Département Électronique de l'ESPA, Mr. … pour avoir accepté de présider ma soutenance, ainsi que Madame et Messieurs les membres du jury, composés de Madame …, Monsieur … et Monsieur …, pour leurs conseils, corrections et remarques pertinentes qui ont grandement enrichi ce travail.\vspace{8pt}\\

Je suis particulièrement reconnaissante à mon encadreur, Madame RAMAFIARISONA Malalatiana, pour son savoir, sa pédagogie et son accompagnement constants, qui m'ont permis de progresser et de mener ce projet à terme avec rigueur et confiance.\vspace{8pt}\\

Mes remerciements vont également à toute l'équipe pédagogique du Département Électronique de l'ESPA pour les connaissances et compétences transmises au cours de ce Master à visée de recherche.\vspace{8pt}\\

Enfin, je tiens à exprimer ma gratitude à ma famille, pour son soutien moral, matériel et affectif, qui a été une source constante de motivation et d'encouragement.\vspace{8pt}\\

Que la paix et la bénédiction de Dieu soient avec vous tous.\vspace{8pt}\\

\medskip
\hfill\textbf{Fréderic ANDRIANARIVONY}
\end{flushleft}
    \newpage

    % -- TODO --
    % Resumé

    \pagenumbering{roman}
    \setcounter{page}{1}
    % Table des matières
    \tableofcontents

    % Liste des Abréviations
    % commande />
    % makeindex image_reconstruction.nlo -s nomencl.ist -o image_reconstruction.nls
    \newpage
    % ====================================================
% Fichier: chapters/abreviations.tex
% makeindex image_reconstruction.nlo -s nomencl.ist -o image_reconstruction.nls
% ====================================================
\nomenclature{CT}{Computed tomography}
\nomenclature{TDM}{Tomodensitométrie}
\nomenclature{FBP}{Filtered back projection}
\nomenclature{TEP}{Tomographie par émission de positons}
\nomenclature{LNT}{Linear No-threshold Theory}
\nomenclature{ALARA}{As Low As Reasonably Achievable}
\nomenclature{TV}{Total Variation}
% ISTA/FISTA, l’ADMM
\nomenclature{ISTA}{Iterative Shrinkage-Thresholding Algorithm}
\nomenclature{FISTA}{FISTA: Accelerated proximal gradient method}
\nomenclature{ADMM}{Alternating direction method of multipliers}
\nomenclature{STFT}{Short-time Fourier transform}


% \nomenclature{ART}{Algebraic reconstruction method}
% \nomenclature{CART}{Circular-arc Radon transform}
% \nomenclature{CST}{Compton scattering tomography}
% \nomenclature{EQMR}{Erreur quadratique moyenne relative}
% \nomenclature{EM}{Expectation maximisation}
% \nomenclature{IRM}{Imagerie par résonance magnétique}
% \nomenclature{FFT}{Fast Fourier transform}
% \nomenclature{PSF}{Point spread function}
% \nomenclature{RSB}{Rapport signal sur bruit}
% \nomenclature{RSO}{Radar à synthèse d’ouverture}
% \nomenclature{SAR}{Synthetic aperture radar}
% \nomenclature{SIRT}{Simultaneous iterative reconstruction tomography}
% \nomenclature{SPECT}{Single photon emission computed tomography}
% \nomenclature{SVD}{Singular value decomposition}
% \nomenclature{TR}{Transformée de Radon}
% \nomenclature{TRAC}{Transformée de Radon sur des arcs de cercle}
% \nomenclature{TRS}{Transformée de Radon sphérique}
% \nomenclature{TRV}{Transformée de Radon sur des lignes V}
% \nomenclature{TRV1}{Première transformée de Radon sur des lignes V étudiée}
% \nomenclature{TRV2}{Deuxième transformée de Radon sur des lignes V étudiée}
% \nomenclature{TRVC}{Transformée de Radon sur des lignes V composées}
 % modularité : toutes les abréviations dans un fichier
    \printnomenclature[3cm]  

    % Liste des figures
    \newpage
    \listoffigures

    % TODO --- Introduction

    \newpage
    \pagenumbering{arabic}
    \setcounter{page}{1}

    % ================================================================
    % ADD CHAPTERS HERE
    %===========================================================
% Chapitre 1 — Introduction générale
%===========================================================

\chapter{RECONSTRUCTION D'IMAGE}
L'imagerie médicale occupe aujourd'hui une place centrale dans le diagnostic, le suivi thérapeutique et la recherche biomédicale. Des modalités telles que la tomodensitométrie (CT), l'imagerie par résonance magnétique (IRM) ou la tomographie par émission de positons (TEP) permettent d'explorer de manière non invasive l'anatomie et la physiologie du corps humain avec une précision toujours croissante. Toutefois, la qualité et la fiabilité de ces images reposent sur une étape fondamentale, souvent invisible pour l'utilisateur final : la reconstruction d'image.\vspace{5pt}\\
La reconstruction d'image médicale consiste à transformer des données de mesure brutes, acquises selon différents angles ou configurations physiques, en représentations visuelles exploitables par le clinicien. Contrairement à la photographie classique, les systèmes d'imagerie médicale ne produisent pas directement des images des organes. Ils mesurent plutôt des interactions physiques --- telles que l'atténuation des rayons X, la réponse à des champs magnétiques ou les émissions radioactives --- qui ne constituent que des projections partielles de la réalité anatomique ou fonctionnelle. La reconstruction vise alors à estimer, à partir de ces projections, la distribution spatiale des tissus ou des activités biologiques.\vspace{5pt}\\
En tomodensitométrie, les projections correspondent à l'atténuation des rayons X traversant le corps selon de multiples directions, tandis qu'en tomographie par émission de positons, elles résultent de la détection des photons émis par un traceur radioactif à l'intérieur du patient. Dans les deux cas, les données acquises sont incomplètes, bruitées et fortement corrélées, ce qui rend la reconstruction d'image intrinsèquement complexe. Ce processus s'inscrit dans le cadre des problèmes inverses, pour lesquels on cherche à retrouver les causes physiques à partir d'observations indirectes. Ces problèmes sont généralement mal posés au sens mathématique, plusieurs solutions pouvant conduire à des mesures expérimentales similaires.\vspace{5pt}\\
Historiquement, les premières méthodes de reconstruction, telles que la rétroprojection filtrée, ont permis le développement clinique rapide de l'imagerie tomographique grâce à leur simplicité et à leur efficacité computationnelle. Cependant, ces approches restent sensibles au bruit et aux artefacts, en particulier lorsque les conditions d'acquisition s'éloignent du cadre idéal. Afin de pallier ces limitations, des méthodes itératives ont été développées. Celles-ci reposent sur une modélisation plus fidèle du processus d'acquisition et permettent d'intégrer des connaissances a priori sur les propriétés des images médicales, au prix d'un coût de calcul plus élevé.\vspace{5pt}\\
Plus récemment, les méthodes basées sur des modèles physiques détaillés et l'intégration de techniques d'intelligence artificielle ont profondément renouvelé le domaine de la reconstruction d'image. Ces approches permettent de mieux prendre en compte la géométrie des systèmes d'acquisition, la physique des interactions rayonnement--matière ainsi que les caractéristiques des détecteurs. Elles ouvrent également la voie à des reconstructions plus rapides, plus robustes et de meilleure qualité, répondant à des enjeux majeurs tels que la réduction de la dose de rayonnement, la correction des artefacts et l'amélioration du contraste des images.\vspace{5pt}\\

\section{Modalités d'imagerie médicale et reconstruction d'images}

L'imagerie médicale regroupe plusieurs modalités utilisées pour le diagnostic et la surveillance de nombreuses pathologies. Chaque modalité repose sur des principes physiques distincts et présente des caractéristiques spécifiques en termes d'acquisition et de reconstruction des images. Comprendre ces différences est essentiel afin d'apprécier les forces, les limites et les défis associés à chaque technique de reconstruction.
\subsection{Tomodensitométrie (TDM)}

La tomodensitométrie repose sur l'utilisation de rayons X pour produire des images détaillées de l'intérieur du corps humain. Un tube à rayons X et un ensemble de détecteurs tournent autour du patient, permettant l'acquisition de projections sous de multiples angles. Ces projections correspondent à l'atténuation du faisceau de rayons X par les tissus traversés.\vspace{5pt}\\
Historiquement, la rétroprojection filtrée (\emph{Filtered Backprojection}, FBP) a constitué l'algorithme de reconstruction standard en TDM. Cette méthode consiste à filtrer les données de projection afin de corriger le flou inhérent à la rétroprojection, puis à les projeter dans l'espace image. Bien que rapide et robuste, la FBP est sensible au bruit, en particulier lorsque la dose de rayonnement est réduite.\vspace{5pt}\\
Les systèmes de tomodensitométrie modernes intègrent de plus en plus des méthodes de reconstruction itérative. Ces techniques consistent à ajuster progressivement une estimation de l'image en comparant les projections mesurées aux projections simulées à partir de l'image courante. La reconstruction itérative permet de réduire le bruit, d'améliorer le contraste et de maintenir une qualité d'image acceptable à faible dose, au prix d'un temps de calcul plus important.\vspace{5pt}\\
Un avantage majeur de la TDM réside dans la rapidité de la reconstruction, souvent réalisable en temps réel ou quasi temps réel. De plus, les valeurs reconstruites sont directement liées aux coefficients d'atténuation des tissus, ce qui confère à la tomodensitométrie une forte valeur quantitative.

\subsection{Imagerie par résonance magnétique (IRM)}

L'imagerie par résonance magnétique utilise un champ magnétique intense et des ondes radiofréquence pour produire des images des structures internes du corps. Elle repose sur les principes de la résonance magnétique nucléaire et permet d'obtenir des contrastes tissulaires riches sans exposition aux rayonnements ionisants.\vspace{5pt}\\
Les données IRM sont acquises dans le domaine fréquentiel, appelé espace \(k\). La reconstruction d'image consiste principalement à appliquer une transformée de Fourier inverse afin de convertir ces données fréquentielles en une image spatiale. Cette approche confère à l'IRM une base mathématique élégante et bien établie.\vspace{5pt}\\
Afin d'accélérer les temps d'acquisition, des techniques telles que le Fourier partiel et l'imagerie parallèle ont été développées. Des méthodes comme SENSE ou GRAPPA reposent sur le sous-échantillonnage de l'espace \(k\) et sur l'utilisation de multiples bobines de réception, combinées à des algorithmes de reconstruction capables de compenser les données manquantes.\vspace{5pt}\\
La reconstruction en IRM reste toutefois complexe en raison de plusieurs facteurs, notamment les inhomogénéités du champ magnétique, les artefacts liés aux mouvements du patient et la nécessité de combiner correctement les signaux issus de systèmes multi-bobines. Ces défis requièrent des modèles de reconstruction sophistiqués et une calibration précise du système.

\subsection{Tomographie par émission de positons (TEP)}

La tomographie par émission de positons est une modalité d'imagerie fonctionnelle qui repose sur l'injection d'un traceur radioactif dans l'organisme. Le traceur émet des positons qui, après annihilation avec des électrons, produisent deux photons gamma émis en coïncidence et détectés par le système TEP.\vspace{5pt}\\
Comme en tomodensitométrie, la rétroprojection filtrée a été historiquement utilisée pour la reconstruction des images TEP, en particulier dans les premiers systèmes. Toutefois, en raison du caractère bruité et statistique des données TEP, les méthodes itératives se sont rapidement imposées.\vspace{5pt}\\
La majorité des systèmes TEP modernes utilisent des algorithmes itératifs, tels que la maximisation d'espérance par sous-ensembles ordonnés (\emph{Ordered Subsets Expectation Maximization}, OSEM). Ces méthodes modélisent finement la physique de l'acquisition et permettent d'intégrer diverses corrections, notamment celles liées à l'atténuation, à la diffusion et aux coïncidences aléatoires.\vspace{5pt}\\
La reconstruction TEP présente des défis spécifiques, liés à la nécessité de corriger plusieurs phénomènes physiques. L'utilisation de scanners hybrides TEP-TDM ou TEP-IRM permet d'exploiter des informations anatomiques complémentaires afin d'améliorer la qualité et la quantification des images fonctionnelles.

\subsection{Comparaison des modalités}

Bien que la rétroprojection filtrée ait historiquement été utilisée en tomodensitométrie et en TEP, les méthodes de reconstruction itérative sont aujourd'hui privilégiées dans ces deux modalités en raison de leur capacité à améliorer la qualité d'image et à réduire les artefacts. À l'inverse, l'IRM repose principalement sur des reconstructions basées sur la transformée de Fourier, enrichies par des techniques d'accélération et de compensation des données manquantes.\vspace{5pt}\\
Les processus d'acquisition diffèrent également de manière significative. En TDM et en TEP, les données sont directement liées à des propriétés physiques mesurables, telles que l'atténuation des rayons X ou la distribution d'un traceur radioactif. En IRM, l'acquisition repose sur une manipulation complexe des champs magnétiques et des impulsions radiofréquence afin de coder l'information spatiale.\vspace{5pt}\\
Enfin, chaque modalité est sujette à des artefacts spécifiques nécessitant des stratégies de correction adaptées lors de la reconstruction. La TDM doit corriger des effets tels que le durcissement du faisceau, l'IRM est sensible aux inhomogénéités de champ et aux mouvements, tandis que la TEP requiert des corrections pour l'atténuation, la diffusion et les événements aléatoires.\vspace{5pt}\\
En résumé, bien que la reconstruction d'images soit une étape essentielle commune à ces trois modalités, la nature des données, les modèles physiques sous-jacents et les défis algorithmiques diffèrent considérablement. La compréhension de ces spécificités est indispensable pour optimiser la qualité des images et garantir une information diagnostique fiable et pertinente.

\section{Transformation de Radon et son rôle en reconstruction d'image}
\subsubsection{Origines et développement historique}
Les fondements théoriques de la reconstruction tomographique remontent au début du XXe siècle. En 1917, le mathématicien J. Radon a formulé une théorie mathématique permettant de reconstruire une fonction à partir de ses projections.
Cependant, ce n'est qu'avec les progrès des calculateurs numériques dans les années 1960-1970 que cette théorie a pu être appliquée concrètement. Godfrey Hounsfield, ingénieur britannique, a mis au point le premier scanner TDM (tomodensitométrie ou CT) en 1971, grâce à un financement indirect provenant de la société EMI, associée au succès des Beatles. Ce développement a marqué la naissance de l'imagerie médicale tomographique, suivie par l'IRM et la tomoscintigraphie.\vspace{5pt}\\
La transformation de Radon constitue un outil mathématique fondamental pour la reconstruction d'images à partir de projections. Elle est au cœur de nombreuses techniques d'imagerie médicale, en particulier la tomodensitométrie (TDM). Cette transformation permet de relier les mesures obtenues par les détecteurs à la distribution de densité des tissus à l'intérieur du corps.\vspace{5pt}\\
Elle consiste à intégrer l'information provenant de différentes lignes de projection afin de créer un ensemble de mesures représentant l'objet examiné sous plusieurs angles. Dans le cadre de la TDM, ces projections correspondent à l'atténuation des rayons X lorsqu'ils traversent le corps. Chaque angle de mesure fournit une “ombre” partielle de la structure interne, et la transformation de Radon permet de regrouper l'ensemble de ces informations.\vspace{5pt}\\
La transformée inverse de Radon joue un rôle central dans la reconstruction d'image. À partir des projections acquises, elle permet de reconstituer la distribution originale des tissus. L'une des méthodes les plus utilisées pour cette étape est la rétroprojection filtrée, qui combine les différentes projections après un traitement visant à corriger le flou inhérent aux mesures. Cette approche a longtemps constitué la base des reconstructions classiques en tomodensitométrie et demeure largement utilisée.\vspace{5pt}\\
L'application de la transformation de Radon et de sa transformée inverse a révolutionné le diagnostic médical. Elle permet aujourd'hui de visualiser de manière non invasive et détaillée les structures internes du corps, offrant des images transversales précises qui améliorent la qualité et la fiabilité des examens cliniques.\vspace{5pt}\\
En résumé, la transformation de Radon représente un composant essentiel de l'imagerie médicale moderne. Elle constitue le fondement de nombreuses techniques de reconstruction, transformant des données de projection en images exploitables, et reste un outil incontournable pour le diagnostic et le suivi des patients.
\section{Limites et artefacts de la rétroprojection filtrée (FBP)}

La rétroprojection filtrée, ou \emph{Filtered Backprojection} (FBP), est un algorithme de reconstruction largement utilisé dans les modalités d'imagerie médicale telles que la tomodensitométrie (CT) et la tomographie par émission de positons (TEP). Bien que cette méthode soit rapide et relativement efficace, elle présente plusieurs limitations et artefacts pouvant affecter la qualité des images reconstruites.

\subsection{Limites de la rétroprojection filtrée}

L'une des principales limites de la FBP réside dans le modèle d'atténuation qu'elle suppose. Celui-ci est souvent simplifié et peut ne pas représenter fidèlement les interactions complexes entre le rayonnement et les tissus du patient. Par ailleurs, l'algorithme a tendance à amplifier le bruit présent dans les données de projection, en particulier lorsque le filtrage appliqué n'est pas optimal. \vspace{5pt}\\
La FBP peut également introduire des artefacts de stries, surtout lorsque les projections sont incomplètes ou bruitées. La résolution des composantes haute fréquence est limitée, entraînant une perte de détails fins dans l'image reconstruite. De plus, l'algorithme suppose un échantillonnage uniforme des projections, condition rarement parfaitement respectée dans la pratique clinique.

\subsection{Artefacts associés à la FBP}

Plusieurs types d'artefacts peuvent apparaître dans les images reconstruites à l'aide de la FBP. Parmi les plus courants, on trouve les stries sombres ou lumineuses dues à des données incomplètes ou bruitées, ainsi que les artefacts en anneau, souvent liés à un mauvais calibrage des détecteurs. Le durcissement du faisceau peut également générer des bandes sombres ou des stries, conséquence de l'absorption préférentielle des photons de faible énergie.  \vspace{5pt}\\
Des zones sombres peuvent apparaître lorsque le nombre de photons détectés est insuffisant, notamment en présence d'implants métalliques ou de tissus très absorbants. Enfin, la FBP reste sensible aux mouvements du patient pendant l'examen, ce qui peut provoquer du flou ou des stries supplémentaires dans l'image finale.

\subsection{Stratégies pour atténuer les artefacts et limitations}

Plusieurs approches permettent de réduire l'impact des limitations et des artefacts associés à la FBP. L'optimisation des protocoles d'acquisition, par exemple l'augmentation du nombre de projections ou l'utilisation de détecteurs plus performants, contribue à améliorer la qualité des données de base.  \vspace{5pt}\\
Le pré-traitement des données, comme le filtrage du sinogramme ou la correction du durcissement du faisceau, peut également aider à atténuer les artefacts. L'emploi d'algorithmes de reconstruction plus avancés, tels que la reconstruction itérative ou basée sur un modèle, permet de surmonter certaines des limites inhérentes à la FBP. Enfin, des techniques de post-traitement d'image, incluant le filtrage et l'amélioration de contraste, complètent le processus pour améliorer la qualité visuelle et diagnostique des images reconstruites.\vspace{5pt}\\
En résumé, bien que la rétroprojection filtrée reste un algorithme de reconstruction largement utilisé et efficace, elle comporte des limites et génère certains artefacts susceptibles d'affecter la qualité des images. La compréhension de ces contraintes est essentielle pour développer des stratégies adaptées, tant au niveau de l'acquisition que du traitement et de la reconstruction, afin d'optimiser les résultats en imagerie médicale.

\section{Introduction au Compressed Sensing}

L’un des principes fondamentaux du traitement du signal repose sur la théorie
d’échantillonnage de Nyquist--Shannon, selon laquelle le nombre d’échantillons
nécessaires pour reconstruire un signal sans erreur est déterminé par sa bande
passante, c’est-à-dire la longueur du plus petit intervalle contenant le support du
spectre du signal considéré. Toutefois, au cours des dernières années, une théorie
alternative appelée \emph{Compressed Sensing} (ou échantillonnage compressif) a
émergé. Celle-ci démontre qu’il est possible de reconstruire des signaux et des images,
y compris à haute résolution, à partir d’un nombre de mesures bien inférieur à celui
classiquement requis.

L’objectif du Compressed Sensing est de tirer parti de la structure intrinsèque des
signaux afin de réduire significativement la quantité de données nécessaires lors du
processus d’acquisition. Un aspect particulièrement remarquable de cette théorie
réside dans ses nombreuses interactions avec divers domaines des sciences appliquées
et de l’ingénierie, tels que les statistiques, la théorie de l’information, la théorie du
codage ou encore l’informatique théorique, ce qui souligne son caractère fondamental
et transversal.

De manière générale, la parcimonie, et plus largement la compressibilité, joue depuis
longtemps un rôle central dans de nombreux domaines scientifiques. La parcimonie
permet d’obtenir des estimations efficaces, notamment à travers des méthodes de
seuillage ou de régularisation, dont les performances dépendent fortement du nombre
réduit de composantes significatives du signal. Elle constitue également le fondement
des techniques modernes de compression, où la qualité du codage dépend de la
capacité du signal à admettre une représentation parcimonieuse dans une base
appropriée. En outre, la parcimonie favorise la réduction de dimension ainsi que la
construction de modèles compacts et efficaces.

La contribution essentielle du Compressed Sensing réside dans le fait que la
parcimonie n’intervient plus uniquement au stade du traitement ou de la compression,
mais influence directement le processus d’acquisition des données. Cette approche
ouvre la voie à des protocoles d’acquisition plus efficaces, permettant de convertir des
signaux analogiques en une représentation numérique déjà compressée, tout en
optimisant l’utilisation des ressources disponibles.

En pratique, il est bien établi que les signaux naturels possèdent une structure
intrinsèque qui autorise une compression efficace avec une perte perceptuelle limitée.
Les méthodes de codage par transformée exploitent le fait que de nombreux signaux
admettent une représentation parcimonieuse dans une base fixe. Le schéma classique
consiste à acquérir l’ensemble des données, à calculer tous les coefficients de la
transformée, puis à ne conserver que les plus significatifs en éliminant les autres. Ce
processus, bien que largement utilisé, demeure peu efficace, puisqu’il implique une
acquisition massive de données dont une grande partie est finalement rejetée.

Cette observation soulève une question fondamentale : si la majorité des signaux sont
compressibles, pourquoi acquérir l’intégralité des données alors qu’une grande partie
sera inévitablement inutilisée~? Le Compressed Sensing apporte une réponse claire à
cette problématique en montrant qu’il est possible d’acquérir directement les données
sous une forme compressée, sans passer par une étape intermédiaire de
sur-acquisition suivie de compression.

\section{Reconstruction d'images par Compressive Sensing (CS)}

La \emph{Compressive Sensing} (CS) est une approche moderne de reconstruction d'images qui a émergé au milieu des années 2000 comme alternative aux méthodes classiques telles que la rétroprojection filtrée. Elle repose sur l’idée que de nombreuses images médicales peuvent être représentées de manière parcimonieuse dans un certain domaine de transformation, ce qui permet de reconstruire des images de qualité à partir d’un nombre réduit de mesures. Cette capacité à exploiter l’information essentielle rend la CS particulièrement pertinente dans des modalités où l’acquisition est longue, coûteuse ou comporte des risques pour le patient.

\subsection{Principes et avantages}

La CS réduit considérablement le nombre de projections ou d’échantillons nécessaires pour obtenir une image diagnostique exploitable. Elle est donc idéale pour des modalités telles que la tomodensitométrie, la TEP, ou l’IRM dynamique, où chaque mesure peut être longue ou exposer le patient à un rayonnement ou à un contraste. En exploitant la parcimonie naturelle des images médicales, la CS permet non seulement de diminuer la dose de rayonnement, mais aussi de réduire le temps global d’acquisition tout en maintenant une qualité diagnostique élevée.

Un avantage supplémentaire réside dans sa capacité à limiter le bruit. Contrairement à la rétroprojection filtrée (FBP), qui peut amplifier les fluctuations aléatoires des mesures, la CS intègre des contraintes de régularisation qui favorisent des reconstructions plus stables, visuellement plus propres et moins sensibles aux artefacts issus des mesures bruitées.

\vspace{0.2cm}
\noindent
\textbf{Contexte additionnel :}
\begin{itemize}
    \item \textbf{Transitions cliniques réussies :} Certains scanners commerciaux ont commencé à intégrer la CS, par exemple en tomodensitométrie cardiaque ou en imagerie mammaire, permettant une réduction de dose pouvant atteindre 50\% sans compromettre la qualité diagnostique.
    \item \textbf{Fondement théorique :} La CS repose sur deux concepts clés : la parcimonie du signal et l’incohérence entre la matrice d’acquisition et la base dans laquelle le signal est parcimonieux, comme les ondelettes ou la variation totale.
\end{itemize}

\subsection{Limites et défis}

Malgré ses avantages, la CS présente des limites importantes. Elle repose sur l’hypothèse que les images sont parcimonieuses, ce qui n’est pas toujours vrai pour certaines structures anatomiques complexes. Les algorithmes CS sont itératifs et demandent plus de ressources computationnelles que la FBP, ce qui peut freiner leur utilisation dans des contextes nécessitant une reconstruction quasi temps réel.

La qualité de reconstruction dépend fortement du choix de la régularisation et de l’algorithme d’optimisation. Des réglages inadéquats peuvent entraîner des artefacts, une perte de détails fins ou un lissage excessif des structures importantes, rendant l’interprétation diagnostique plus difficile.

\vspace{0.2cm}
\noindent
\textbf{Contexte additionnel :}
\begin{itemize}
    \item \textbf{Validation clinique :} Les images reconstruites par CS peuvent paraître différentes de celles obtenues par FBP, même lorsque les métriques quantitatives, telles que le rapport signal-sur-bruit, sont meilleures. Les radiologues doivent donc s’adapter à ce nouvel “aspect” des images.
    \item \textbf{Choix de la base parcimonieuse :} Le type de base choisi influence fortement la reconstruction : la variation totale favorise des contours nets, tandis que les ondelettes sont plus adaptées à la préservation des textures fines.
\end{itemize}

\subsection{Artefacts et comportements observés}

Les artefacts observés en CS sont différents de ceux classiques de la FBP. On retrouve notamment des effets de bloc ou des textures irrégulières, généralement dus à un sous-échantillonnage excessif ou à une régularisation trop forte. Les détails fins peuvent être atténués et des motifs répétitifs peuvent apparaître dans certaines zones.  

Les mouvements du patient ont un impact moindre que sur la FBP, mais ils peuvent encore dégrader les reconstructions, en particulier lorsque le nombre de mesures est limité ou que les projections sont bruitées.

\vspace{0.2cm}
\noindent
\textbf{Contexte additionnel :}
\begin{itemize}
    \item \textbf{Artefacts de type « piqûre » (staircasing) :} Typiques de la régularisation par variation totale, ces artefacts donnent un aspect en escalier aux zones de gradient doux et peuvent masquer des lésions subtiles.
    \item \textbf{Sensibilité aux hyperparamètres :} Contrairement à la FBP, qui ne nécessite presque aucun paramètre, la CS dépend fortement du poids de régularisation et des paramètres d’optimisation, souvent ajustés par validation croisée.
\end{itemize}

\subsection{Stratégies pour améliorer la reconstruction}

Plusieurs stratégies permettent d’atténuer les limitations et les artefacts de la CS. L’optimisation des acquisitions, notamment via un sous-échantillonnage structuré plutôt qu’aléatoire, améliore la fidélité des reconstructions. Les algorithmes itératifs peuvent être combinés avec des techniques de régularisation adaptées, comme la parcimonie dans un domaine transformé ou la minimisation de normes spécifiques, afin d’obtenir des reconstructions plus robustes.

Le post-traitement d’image, par filtrage ou amélioration adaptative du contraste, contribue également à améliorer la lisibilité et la qualité visuelle. Dans certaines modalités, comme l’IRM, des acquisitions adaptatives permettent de mieux préserver les informations structurelles tout en maintenant un niveau élevé de sous-échantillonnage.

\vspace{0.2cm}
\noindent
\textbf{Contexte additionnel :}
\begin{itemize}
    \item \textbf{Avancées algorithmiques :} Les méthodes d’optimisation moderne, comme les algorithmes ADMM (Alternating Direction Method of Multipliers), ont amélioré la vitesse de convergence et la robustesse des reconstructions CS.
    \item \textbf{Acquisitions structurées :} En IRM, des schémas d’échantillonnage variables dans l’espace de Fourier (k-space) permettent de mieux préserver les basses fréquences et d’optimiser la reconstruction des hautes fréquences sous-échantillonnées.
    \item \textbf{Intégration multi-modale :} La CS peut être combinée avec d’autres approches, comme les modèles physiques ou les méthodes statistiques, pour renforcer la qualité des reconstructions et réduire les artefacts dans les cas difficiles.
\end{itemize}

    \chapter{OUTILS MATHÉMATIQUES UTILISÉS DANS LA RECONSTRUCTION D'IMAGES CT PAR LE COMPRESSED SENSING}

Un sujet très précis et intéressant ! La reconstruction d'images à partir de projections est un aspect crucial de l'imagerie médicale, en particulier dans des modalités telles que la tomodensitométrie (CT) et la tomographie par émission de positons (TEP).
Le scanner de tomodensitométrie (CT) diagnostique, voir la \Cref{fig :tomography_device}, est un véritable
chef-d'œuvre de la technologie moderne et constitue un exemple positif de l'influence des forces
du marché libre dans la stimulation de l'innovation. Tous les principaux fabricants de scanners
CT disposent d'équipes solides de recherche et développement, qui suivent et contribuent aux
travaux de recherche en science de l'imagerie en plus de leurs activités internes.

Les dispositifs CT peuvent être considérés comme la réalisation d'une caméra gigapixel en usage
clinique courant. Alors qu'un scanner CT diagnostique moderne typique fournit des volumes
composés de centaines d'images de coupes de taille $512 \times 512$ avec une résolution
submillimétrique, les scanners micro-CT peuvent produire des volumes atteignant
$2000$ voxels.

Dans le cas de l'imagerie cardiaque par CT, une application qui a largement stimulé les avancées
technologiques, les images volumiques peuvent être acquises avec une résolution temporelle
pouvant descendre jusqu'à 100 millisecondes. La vitesse et la résolution de l'imagerie CT en font
un outil indispensable pour l'imagerie cardiaque et l'évaluation des accidents vasculaires
cérébraux.

Elle est utilisée de manière routinière pour le diagnostic de diverses pathologies médicales
affectant l'ensemble des organes internes, et son utilisation est même envisagée comme outil de
dépistage du cancer du poumon. Les recherches actuelles visent globalement à améliorer
l'utilité clinique sans augmenter la dose d'irradiation de l'examen, ou à maintenir cette utilité
tout en réduisant l'exposition du patient aux rayonnements ionisants.

Les méthodes analytiques directes sont l'approche historique et mathématiquement élégante des problèmes inverses linéaires \cite{14}, particulièrement en tomographie. Leur positionnement répond à un impératif de rapidité de calcul dans des applications où le temps de reconstruction est critique (imagerie médicale clinique, contrôle non destructif industriel).

\section{Classification des approches de reconstruction}
Le paysage algorithmique de la tomographie se divise principalement en trois familles, distinguées par leur traitement de l'opérateur de projection.

\subsection{Méthodes analytiques}
Les méthodes analytiques de reconstruction reposent sur une formulation mathématique explicite
de l'inversion de l'opérateur direct reliant l'objet à ses projections. En tomodensitométrie (TDM),
cet opérateur est étroitement lié à la transformée de Radon. En exploitant ses propriétés, ces
méthodes permettent une reconstruction directe et rapide de l'image à partir des données de
projection.

L'exemple le plus emblématique est la Rétroprojection Filtrée (\emph{Filtered Backprojection},
FBP). Cette méthode analytique largement utilisée consiste à filtrer les projections avant de
les rétroprojeter sur la grille de l'image. La FBP est rapide et efficace, mais elle demeure
sensible au bruit et aux artefacts, en particulier lorsque le nombre de projections est limité
ou que la dose d'irradiation est réduite.


\subsubsection{Fondements de la tomographie assistée par ordinateur (CT)}
Imaginons qu'on ait un objet opaque constitué de différents matériaux, et que l'on souhaite savoir comment ces matériaux sont répartis à l'intérieur sans l'endommager (par exemple, l'objet peut être un malade à l'intérieur du corps duquel on aimerait voir). L'une des méthodes est le scanner  : on lance de fins faisceaux de rayons X à travers l'objet dans toutes les directions et on mesure quelle proportion de chaque faisceau a été absorbée.

La tomographie est un procédé permettant de créer l'image d'un objet en deux ou trois dimensions à partir de multiples "coupes" unidimensionnelles (Voir \Cref{fig :tomography_device}). Dans un scanner CT, ces coupes sont définies par des faisceaux de rayons $\mathbf{X}$ parallèles projetés à travers l'objet. En changeant l'orientation de la source et du détecteur (l'angle \(\theta\)), on obtient des informations sur la densité interne sous différents angles.\\
Le fonctionnement repose sur la mesure de l'intensité des rayons $\mathbf{X}$ :
\begin{itemize}
    \item[-] \textbf{Perte d'énergie}  : Lorsqu'un rayon $\mathbf{X}$ traverse un objet, il perd une partie de son énergie, ce qui réduit son intensité
    \item[-] \textbf{Coefficient d'atténuation}  : Cette perte dépend de la densité du milieu. Les objets denses (comme l'os) provoquent une variation d'intensité plus importante que les tissus moins denses. Cette caractéristique est appelée le coefficient d'atténuation. L'atténuation mesurée pour chaque faisceau, c'est-à-dire la différence entre l'intensité incidente et l'intensité détectée, correspond à une intégrale de ligne de la structure interne de l'objet. Cette relation entre l'objet et l'ensemble de ses intégrales de ligne est formalisée par la transformée de Radon. La reconstruction de l'image originale repose alors sur l'inversion de cette transformée, qui constitue le fondement théorique de la tomographie assistée par ordinateur
    \item[-] \textbf{Mesures}  : Le scanner enregistre l'intensité initiale émise ($I_{0}$) et l'intensité finale reçue ($I_{1}$) pour chaque faisceau afin de déduire la densité globale rencontrée sur le trajet
\end{itemize}

\begin{figure}[H]
    \centering
    \includegraphics[width=0.8\textwidth]{./images/ct_device.png}
    \caption{Principe de fonctionnement d'un appareil de tomodensitométrie}
    \label{fig :tomography_device}
\end{figure}
\medskip
\noindent
D'un point de vue mathématique, l'ensemble des mesures acquises par le scanner correspond à
l'application de la transformée de Radon à la fonction d'atténuation de l'objet. Le problème
fondamental de la tomographie consiste alors à reconstruire cette fonction d'atténuation à
partir de ses intégrales de ligne, c'est-à-dire à inverser la transformée de Radon.
Pour une fonction d'atténuation bidimensionnelle $A(x,y)$, la transformée de Radon est définie
par :
\begin{equation}
\mathcal{R}\{A\}(\theta, s) =
\int_{\mathbb{R}^2} A(x,y)\,
\delta(x\cos\theta + y\sin\theta - s)\,\mathrm{d}x\,\mathrm{d}y,
\end{equation}
où $\theta$ représente l'angle de projection, $s$ la position du détecteur, et $\delta(\cdot)$
la distribution de Dirac. Cette expression formalise le fait que chaque projection correspond
à une intégrale de la fonction d'atténuation le long d'une droite.

\subsubsection{Loi de Beer--Lambert et modélisation de l'atténuation des rayons X dans l'objet}
L'objet initial, considéré comme plan, est donné par une fonction d'atténuation qui, à chaque point du plan de coordonnées $(x, y)$, va associer un nombre $A(x, y)$ correspondant à la proportion des rayons qui sont absorbés par le matériau en ce point  : en un point d'un os, $A$ sera grand, et en un point de l'air, il sera faible.\vspace{10pt}\\
% \subsubsection{Loi de Beer--Lambert et modélisation de l'atténuation}
En supposant dans un premier temps que la fonction d'atténuation de notre objet est constante égale à $a$, pour tout rayon lumineux traversant notre objet, pour tout couple de points d'abscisses $x$ et $x+l$ sur ce rayon, les abscisses étant croissantes dans le sens du rayon, le rapport d'intensités lumineuses $\cfrac{I(x+l)}{I(x)}$ ne dépend que de $a$ et de la longueur $l$ traversée et pas du point $x$ (position).

En omettant provisoirement la dépendance par rapport à $a$ et en notant alors $p(l)$ ce rapport $\cfrac{I(x+l)}{I(x)}$ qui correspond à la proportion de photons non
absorbés sur une longueur $l$ à partir d'un point $x$, on voit que $p$ vérifie la propriété 
\[
    p(l_1+l_2) = p(l_1)p(l_2)
\] \vspace{10pt}\\
En Effet, la proportion de photons non absorbés sur une longueur $l_2$ à partir d'un point $x+l_1$ est $\cfrac{I(x+l_1+l_2)}{I(x+l_1)}=p(l_2)$. Donc $p(l_1)p(l_2)=\cfrac{I(x+l_1)}{I(x)} \times \cfrac{I(x+l_1+l_2)}{I(x+l_1)}=p(l_1+l_2)$. Ceci traduit juste le fait simple suivant  : les proportions de photons non
absorbés se multiplient lors de traversées successives de milieux absorbants. La bonne définition de l'atténuation est précisément : 
\begin{equation}
    p(l)=e^{-a\, l}
    \label{eq:init_loi_beer_lambert_attenuation}
\end{equation}\vspace{10pt}\\
Autrement dit, pour tout $x$ et $x+l$ sur un axe  : 
\begin{equation}
    I(x+l)=I(x)e^{-a\, l}
    \label{eq:init_loi_beer_lambert}
\end{equation}\vspace{10pt}\\
Notons que si le phénomène physique d'atténuation est spécifique de la tomographie à rayons X, les méthodes de reconstruction sont en revanche plus générales et sont appliquées également dans d'autres systèmes d'imagerie, dans lesquelles des équations analogues expriment une fonction à reconstruire en fonction de projections. C'est le cas par exemple de la TEP utilisée en médecine nucléaire.

\subsubsection{Reconstruction d'images en tomodensitométrie : inversion de Fourier, rétroprojection et FBP}
À partir de la modélisation physique de l'atténuation et de la formulation mathématique de la
transformée de Radon, plusieurs méthodes analytiques ont été proposées pour résoudre le
problème inverse de la reconstruction tomographique.
La reconstruction d'images en tomodensitométrie à partir des projections aux rayons X repose
classiquement sur trois grandes approches analytiques.\vspace{5pt} \\
La première est l'inversion directe de Fourier, fondée sur le théorème de la coupe de Fourier, selon lequel chaque projection acquise à
un angle donné correspond à une droite dans l'espace fréquentiel bidimensionnel de l'objet.
En théorie, l'acquisition d'un nombre suffisant de projections permet de remplir cet espace
fréquentiel et d'obtenir l'image par transformation de Fourier inverse. Toutefois, la nécessité
de rééchantillonner des données organisées sur une grille polaire vers une grille cartésienne
rend cette méthode complexe et peu utilisée en pratique. \vspace{5pt} \\
La seconde approche consiste à appliquer une rétroprojection directe des projections, ce qui conduit à une image floue,
équivalente à la convolution de l'image originale avec un noyau de type
$1/\sqrt{x^2+y^2}$. La restauration de l'image nécessite alors une déconvolution bidimensionnelle,
opération coûteuse en temps de calcul. \vspace{5pt} \\ 
La troisième méthode, la FBP, constitue l'approche la plus répandue en pratique. Elle repose sur le
filtrage préalable de chaque projection unidimensionnelle par un filtre rampe dans le domaine
fréquentiel, avant la rétroprojection. Ce traitement permet de compenser le flou inhérent à la
rétroprojection (\cite{9}) tout en conservant une complexité de calcul réduite, puisque le filtrage est
effectué en une dimension. Ces méthodes analytiques supposent néanmoins la disponibilité d'un
grand nombre de projections uniformément réparties, hypothèse qui n'est plus toujours valide
dans des contextes modernes tels que l'imagerie à faible dose ou à angles limités, où des
approches itératives basées sur le \emph{Compressed Sensing} sont alors privilégiées.

\subsection{Méthodes itératives et Compressed Sensing}

La réduction des données de projection et de la dose de rayonnement en tomodensitométrie n'est
pas simplement une préférence de calcul ; elle est dictée par la sécurité clinique, l'efficacité
opérationnelle et les contraintes physiques des systèmes d'imagerie par rayons \(\mathbf{X}\). 
Cette approche se justifie rigoureusement des points de vue médical, physique et systémique.

En premier lieu, la sécurité des patients est primordiale : les rayons \(\mathbf{X}\) ionisants
utilisés en CT peuvent endommager l'ADN et augmenter le risque de cancer radio-induit,
en particulier chez les populations pédiatriques ou lors d'expositions répétées. Le modèle
linéaire sans seuil (LNT), pierre angulaire de la radioprotection, postule que toute
réduction de dose diminue proportionnellement le risque à long terme. En pratique clinique
moderne, où les examens CT se multiplient (dépistage, suivi longitudinal, planification
radiothérapeutique), la réduction de la dose par acquisition est essentielle pour limiter
l'exposition cumulative.

D'un point de vue physique, la dose est approximativement proportionnelle au nombre de
projections et au produit courant-temps du tube. Réduire l'un ou l'autre diminue donc
directement l'exposition, mais augmente le bruit quantique des données, rendant la
reconstruction plus difficile. Opérationnellement, un plus petit nombre de projections
raccourcit la durée d'acquisition, réduisant les artefacts de mouvement et améliorant le
débit patient. Pour les populations sensibles (enfants) ou dans des contextes d'imagerie
répétée (radiothérapie guidée par l'image), cette réduction devient une obligation clinique
et éthique, conforme au principe ALARA (As Low As Reasonably Achievable). Le défi technique
réside dans la résolution du problème de reconstruction sous-déterminé qui en résulte,
nécessitant des méthodes avancées comme la reconstruction itérative ou le Compressed Sensing \cite{12}
pour préserver la qualité diagnostique malgré la réduction des données.

\subsubsection{Formulation générale du problème inverse}
Dans ce contexte, la reconstruction tomographique peut être formulée comme un
\textit{problème inverse régularisé}. De manière générale, un problème inverse \cite{10} consiste
à reconstruire un modèle à partir de mesures indirectes issues d'un processus physique
connu. Le \textit{problème direct} décrit la formation des données de projection à partir
de l'image inconnue, tandis que le \textit{problème inverse} vise à retrouver cette image à
partir des mesures acquises.\vspace{5pt}\\
En tomodensitométrie discrète, la relation entre l'image à reconstruire et les données
de projection peut s'écrire sous la forme d'un système linéaire :
\begin{equation}
    \mathbf{g} = \mathbf{H}\mathbf{f} + \boldsymbol{\varepsilon},
\end{equation}
où $\mathbf{f}$ représente l'image vectorisée, $\mathbf{g}$ les données de projection
(sinogramme), $\mathbf{H}$ la matrice du système modélisant la transformée de Radon discrète \cite{11},
et $\boldsymbol{\varepsilon}$ un terme représentant le bruit de mesure.

Dans un cadre idéal, l'opérateur $\mathbf{H}$ serait parfaitement inversible et les données
seraient complètes et exemptes de bruit. En pratique, ces conditions ne sont jamais réunies.
Le problème inverse est alors généralement \textit{mal posé au sens de Hadamard}, en raison
d'un manque d'unicité et surtout d'un fort manque de stabilité : de faibles perturbations du
bruit peuvent engendrer de grandes erreurs dans l'image reconstruite. Les méthodes itératives
abordent ce problème en recherchant une solution régularisée, obtenue par la minimisation
d'une fonction de coût combinant un terme d'adéquation aux données et un terme de
régularisation incorporant des connaissances \emph{a priori} sur l'image.

\subsubsection{Principe du Compressed Sensing}
Le Compressed Sensing fournit un cadre mathématique rigoureux pour la reconstruction de
signaux parcimonieux ou compressibles à partir de mesures linéaires sous-échantillonnées,
sous certaines conditions sur l'opérateur de mesure. Il permet de résoudre des problèmes
inverses sous-déterminés au moyen de méthodes d'optimisation favorisant la parcimonie ou
d'algorithmes gloutons, avec des garanties théoriques de stabilité et de robustesse au bruit.

Dans le cadre de l'imagerie tomographique, le Compressed Sensing exploite l'idée que de
nombreuses images médicales admettent une représentation parcimonieuse dans une base ou un
dictionnaire approprié, tel que les ondelettes ou le gradient de l'image. La reconstruction
est alors formulée comme un problème d'optimisation sous contrainte :
\begin{equation}
    \min_{\mathbf{f}} \; \|\mathbf{\Psi}\mathbf{f}\|_1
    \quad \text{sous la contrainte} \quad
    \|\mathbf{H}\mathbf{f} - \mathbf{g}\|_2 \leq \delta,
\end{equation}
où $\mathbf{\Psi}$ désigne un opérateur favorisant la parcimonie et $\delta$ un paramètre
lié au niveau de bruit des données.

Cette formulation permet de reconstruire des images de qualité acceptable à partir d'un
nombre de projections bien inférieur à celui requis par les méthodes analytiques classiques,
tout en assurant une certaine stabilité du problème inverse.

\subsubsection{Régularisation et variation totale}
Une régularisation particulièrement adaptée à l'imagerie tomographique est la variation
totale (TV) \cite{13}, qui favorise les images composées de régions quasi uniformes séparées par des
discontinuités nettes. La reconstruction TV est généralement formulée comme :
\begin{equation}
    \min_{\mathbf{f}} \;
    \frac{1}{2}\|\mathbf{H}\mathbf{f} - \mathbf{g}\|_2^2
    + \lambda \|\nabla \mathbf{f}\|_1,
\end{equation}
où $\nabla$ représente l'opérateur gradient discret et $\lambda$ un paramètre de
régularisation contrôlant le compromis entre fidélité aux données et stabilisation du
problème inverse.

Cette approche permet de réduire efficacement le bruit tout en préservant les contours,
qualité essentielle pour l'analyse diagnostique.

\subsubsection{Algorithmes itératifs}
La résolution des problèmes d'optimisation issus du Compressed Sensing et des méthodes
variationnelles repose sur des algorithmes itératifs, tels que les méthodes de descente de
gradient proximal, l'algorithme ISTA/FISTA, l'ADMM ou encore les méthodes de type primal-dual.
Ces algorithmes procèdent par mises à jour successives de l'image estimée, alternant entre la
réduction de l'erreur de projection et l'application de la régularisation.

Bien que plus coûteuses en temps de calcul que les méthodes analytiques, les méthodes
itératives offrent une qualité de reconstruction supérieure dans les scénarios de données
limitées et constituent aujourd'hui un axe majeur de recherche et de développement en
tomodensitométrie moderne.

\subsubsection{Comparaison avec les méthodes analytiques}
En résumé, les méthodes analytiques privilégient la rapidité et la simplicité au prix d'une
sensibilité accrue au bruit et aux artefacts, tandis que les méthodes itératives et basées sur
le Compressed Sensing s'inscrivent pleinement dans le cadre des problèmes inverses
régularisés. Elles exploitent des informations \emph{a priori} et des contraintes de parcimonie
pour améliorer la stabilité et la qualité de reconstruction dans des conditions d'acquisition
dégradées, répondant ainsi aux exigences de sécurité clinique et d'efficacité opérationnelle.


% ========================================================================================================
% TODO -> Phrase accrochage; mise en page; etc
% ========================================================================================================
% \section{Les traitements préalables à la reconstruction}

% \subsection{ Méthodes dans le domaine spatial}
% % =================== TODO ==================
% \subsection{ Filtrage linéaire}
% % =========================================


% \subsection{ Méthodes dans le domaine transformé}
\section{ Transformée de Fourier}
La transformée de Fourier joue un rôle fondamental dans la reconstruction d'images en tomodensitométrie (CT), en permettant une analyse fréquentielle de l'image et des projections. Cette décomposition facilite la distinction entre les structures globales, associées aux basses fréquences, et les détails fins, portés par les hautes fréquences, contribuant ainsi à une reconstruction plus fidèle. Son utilisation dans des algorithmes analytiques tels que la FBP permet de formuler le problème dans le domaine fréquentiel, où les opérations de filtrage sont plus simples et plus efficaces à mettre en œuvre numériquement. En particulier, l'application d'un filtre rampe aux projections unidimensionnelles permet de compenser le flou inhérent à la rétroprojection directe et de réduire des artefacts caractéristiques, tels que l'effet d'étoile. En pratique, les données d'atténuation, organisées sous forme de sinogramme, sont transformées dans le domaine fréquentiel afin de contrôler précisément leur contenu spectral avant la reconstruction, ce qui conduit à une amélioration globale de la qualité de l'image reconstruite.
\begin{definition}[Transformée de Fourier]
    Soit \( f \) une fonction absolument intégrable sur \( \mathbb{R} \).
    La transformée de Fourier de \( f \), notée \( \mathcal{F}f \), est définie
    pour tout nombre réel \( \xi \) par
    \[
    (\mathcal{F}f)(\xi)
    = \int_{-\infty}^{\infty} f(x)\, e^{-2\pi i \xi x}\, dx.
    \]
\end{definition}
La transformée de Fourier est fréquemment utilisée en analyse du signal et permet de transformer une fonction du temps en une fonction de la fréquence ; la variable $x$ représente le temps en secondes et la variable \( \xi \) représente la fréquence de la fonction en hertz. En fait, les transformées de Fourier indiquent des informations sur le signal. Elles montrent les détails concernant la composante de fréquence qui apparaît ou est présente
dans le signal. Elles ne donnent pas de détails sur la valeur exacte de la fréquence présente
à cet instant précis.\\

Il existe une définition alternative faisant intervenir la fréquence angulaire $w=2\pi \xi$, ce qui conduit à l'expression suivante.
\[(\mathcal{F}f)(w) = \int_{-\infty}^{\infty} f(x)\, e^{-i w x}\, dx\]
Comme pour la transformée de Radon, nous allons énumérer plusieurs propriétés de la transformée de Fourier.
\begin{proposition}
    Pour des constantes réelles $\alpha$ et $\beta$, et des fonctions absolument intégrables $f$ et $g$, on a:
    \begin{itemize}
        \item[(i)] Linéarité : $\mathcal{F}(\alpha f + \beta g)(w) = \alpha \mathcal{F}f(w) + \beta \mathcal{F}g(w)$
        \item[(ii)] $\mathcal{F}f(w) < +\infty$
    \end{itemize}
\end{proposition}

\begin{definition}[Transformée de Fourier inverse]
Soit \( f \) une fonction absolument intégrable.
La transformée de Fourier inverse de \( f \), notée \( \mathcal{F}^{-1}f \),
évaluée en \( x \), est définie par
\begin{equation}
    (\mathcal{F}^{-1}f)(x)
    = \cfrac{1}{2\pi}\int_{-\infty}^{\infty} f(w)\, e^{iw x}\, dw.
    \label{formula:fourier_inverse}
\end{equation}
\end{definition}
Ceci nous conduit immédiatement au théorème suivant.
\begin{proposition}[Théorème d'inversion de Fourier]
Soit $f$ une fonction absolument integrale sur $\mathbb{R}$.
Le théorème d'inversion de Fourier affirme que, pour tout \( x \),
\[
(\mathcal{F}^{-1} \circ \mathcal{F})f(x)=f(x)
\]
\end{proposition}
Jusqu'à présent, nous n'avons abordé la transformée de Fourier que dans une dimension. Il existe des définitions correspondantes en dimensions supérieures, mais, pour nos besoins, nous n'utiliserons que les analogues en deux dimensions.

\begin{definition}[Transformée de Fourier bidimensionnelle]
Soit \( g \) une fonction absolument intégrable définie sur \( \mathbb{R}^2 \).
La transformée de Fourier bidimensionnelle de \( g \), notée \( \mathcal{F}_2 g \),
est définie pour tout \((X,Y) \in \mathbb{R}^2\) par
\begin{equation}
    (\mathcal{F}_2 g)(X,Y) = \int_{-\infty}^{\infty} \int_{-\infty}^{\infty} 
    g(x,y)\, e^{-i (xX + yY)} \, dx\, dy.
    \label{eq:fourier_2d}
\end{equation}

\end{definition}

De manière similaire, nous définissons la transformée de Fourier inverse sur $\mathbb{R}^2$.
\begin{definition}[Transformée de Fourier bidimensionnelle inverse]
Soit \( g \) une fonction absolument intégrable définie sur \( \mathbb{R}^2 \).
La transformée de Fourier bidimensionnelle inverse de \( g \), évaluée en \((x,y)\)
et notée \( \mathcal{F}_2^{-1} g(x,y) \), est donnée par
\[
(\mathcal{F}_2^{-1} g)(x,y) = \cfrac{1}{4\pi^2}\int_{-\infty}^{\infty} \int_{-\infty}^{\infty} 
g(X,Y)\, e^{i (xX + yY)} \, dX\, dY.
\]
\end{definition}

% ================== TODO ==================
% \subsection{ Filtre de Wiener}
% 1. Filtre de Wiener
% ✅ Oui, tout à fait applicable
% Le filtre de Wiener est un filtre linéaire adaptatif.

% Il peut être appliqué :
%    - sur chaque projection (variable s)
%    - ou localement sur le sinogramme

% Intérêt :
%    - réduction du bruit additif (souvent gaussien)
%    - compromis bruit / flou optimal au sens MSE

% ⚠️ Limites :
%    - nécessite une estimation du bruit et du spectre du signal
%    - un mauvais modèle dégrade la reconstruction

% 📌 Très utilisé comme prétraitement des sinogrammes en CT à faible dose.

% \subsection{ Curvelets}
% 4. Curvelets
% ✅ Oui, très pertinent
% Les curvelets sont théoriquement bien adaptées :
%    - excellente représentation des singularités le long de courbes

% Les lignes du sinogramme correspondent à :
%    - des courbes liées aux bords de l'objet

% 📌 Très utilisé dans :
%    - CT basse dose
%    - Les méthodes variationnelles et itératives
% =================================================
\subsection{Transformée de Fourier à Court Terme (STFT)}

La STFT est plus avantageuse que la transformée de Fourier dans le sens où elle introduit la fenêtre glissante. En fait, la fenêtre sert à extraire une petite partie du signal donné. Mathématiquement, elle est représentée par :

\begin{equation}
    S f(u,\xi) = \int_{-\infty}^{\infty}f(t)\omega (t - u)exp(-j\xi t)dt \quad (3)
    \label{eq:stft}
\end{equation}



\section{Convolution}
La convolution joue un rôle clé dans la reconstruction d'image par rétroprojection filtrée (FBP), car elle intervient directement dans l'étape de filtrage des projections. Avant la rétroprojection, chaque projection mesurée est convoluée avec un filtre adapté afin de compenser le flou intrinsèque introduit par la rétroprojection simple. Cette opération permet de renforcer les hautes fréquences et d'améliorer la résolution de l'image reconstruite.

\textbf{Définition 8.1.}
Pour deux fonctions intégrables $f$ et $g$ définies sur $\mathbb{R}$,
nous définissons la convolution de $f$ et $g$, notée $f \star g$, par
\[
(f \star g)(x) = \int_{-\infty}^{\infty} f(t)\,g(x - t)\,dt,
\]
où $x \in \mathbb{R}$.

Nous pouvons facilement étendre cette définition à l'espace
bidimensionnel. Pour les fonctions polaires, nous prenons uniquement
l'intégrale par rapport à la variable radiale, tandis que pour les
fonctions cartésiennes nous intégrons par rapport aux deux variables.
Les définitions explicites sont données ci-dessous.

\begin{definition}
    Pour des fonctions polaires intégrables $f(t,\theta)$ et $g(t,\theta)$,
    nous définissons la convolution de $f$ et $g$ par
    \[
        (f \star g)(t,\theta)
        =
        \int_{-\infty}^{\infty}
        f(s,\theta)\,g(t - s,\theta)\,ds.
    \]
\end{definition}

Pour des fonctions intégrables $F$ et $G$ sur $\mathbb{R}^2$,
nous définissons la convolution de $F$ et $G$ par
\[
    (F \star G)(x,y)
    =
    \int_{-\infty}^{\infty}
    \int_{-\infty}^{\infty}
    F(s,t)\,G(x - s, y - t)\,ds\,dt.
\]

La convolution est une méthode mathématique permettant de moyenner
une fonction $f$ à l'aide du déplacement d'une autre fonction $g$.
Dans la convolution $f \star g$, la fonction $g$ est translatée à travers
la fonction $f$, et la fonction résultante dépend de la zone de recouvrement
au cours de cette translation.
En un certain sens, on peut voir $g$ comme un filtre utilisé pour effectuer
une moyenne de $f$ sur un intervalle donné.
La fonction de filtrage agit ainsi comme un lisseur pour les données bruitées
fournies par la fonction originale.

\begin{proposition}
    Pour des fonctions intégrables $f$, $g$, $h$ définies sur $\mathbb{R}$
    et des constantes $\alpha, \beta \in \mathbb{R}$ :
    
    \begin{itemize}
      \item[(i)] $f \star g = g \star f$ \quad (commutativité),
      \item[(ii)] $f \star (\alpha g + \beta h)
      = \alpha (f \star g) + \beta (f \star h)$ \quad (linéarité).
      \item[(iii)] $\mathcal{F}(f). \mathcal{F}(g)  = \mathcal{F}(f \star g)$
    \end{itemize}
\end{proposition}


% =================================================
\section{Transformée de Radon}
% =================================================
L'hypothèse fondamentale est que le détecteur mesure l'atténuation intégrée le long d'un rayon. 
\begin{definition}
    Pour un faisceau de rayons $\mathbf{X}$ d'énergie $\mathbf{E}$ donnée et un taux de propagation des photons $\mathbf{N}(x)$, l'intensité du faisceau $\mathbf{I}(x)$ à une distance $x$ de l'origine est définie comme \[\mathbf{I}(x) = \mathbf{N}(x) \mathbf{E}\]
\end{definition}

\begin{definition}
    La proportion de photons absorbés par millimètre de substance à une distance $x$ de l'origine est appelée le coefficient d'atténuation $\mathbf{A}(x)$ du milieu.
\end{definition}


Nous connaissons les intensités initiale et finale, $I_0$ et $I_1$ d'un faisceau unique. L'objectif est d'utiliser ces intensités pour déterminer le coefficient d'atténuation le long du trajet du faisceau. Heureusement, la loi de Beer-Lambert établit une relation entre ces deux grandeurs.

\begin{definition}[Loi de Beer-Lambert]
Pour un faisceau de rayons X monochromatique, non réfractif et de largeur nulle,
traversant un milieu homogène sur une distance \(x\) à partir de l'origine,
l'intensité \(I(x)\) est donnée par
\begin{equation}
    I(x) = I_0 e^{-\mathbf{A}(x)x}
    \label{eq:loi_beer_lambert}
\end{equation}
\end{definition}
En l'état, cette équation ne nous est pas particulièrement utile. Elle exprime le coefficient d'atténuation en un point donné en fonction de l'intensité en ce point, alors que nous ne connaissons la valeur de l'intensité qu'en des points situés à l'extérieur de l'objet. Ce que nous cherchons réellement est une relation entre le coefficient d'atténuation à l'intérieur de l'objet et la variation de l'intensité du faisceau. Pour cela, nous allons manipuler légèrement l'équation \eqref{eq:loi_beer_lambert}.\\
En passant à  la dérivée de la loi de Beer-Lambert, nous obtenons la relation suivante :
\[
    \frac{dI}{dx} = -\mathbf{A}(x)I(x)
\]
Soit $I(x_0)=I_0$ la valeur initiale de l'intensité du faisceau et $I(x_1)=I_1$ la valeur finale de l'intensité du faisceau. En utilisant cette relation, nous obtenons la relation suivante :

\[
    -\int_{x_0}^{x_1} \mathbf{A}(x)dx = \int_{x_0}^{x_1}\cfrac{dI}{I(x)}=ln(\frac{I_1}{I_0})
\]
ou encore \vspace{10pt}
\begin{equation}
    \int_{x_0}^{x_1} \mathbf{A}(x)dx = ln(\frac{I_0}{I_1})
    \label{eq:radon_transformation}
\end{equation}

$ln(\frac{I_0}{I_1})$ désigne les données de projection, communément appelées le sinogramme, qui résultent de l'acquisition des projections tomographiques.
\medskip
\noindent
Nous sommes maintenant prêts à introduire des outils mathématiques — en particulier la transformée de Radon — qui joueront un rôle central dans la détermination du coefficient d'atténuation dans l'équation \eqref{eq:loi_beer_lambert}.

L'écriture sous forme normale d'une équation de droite joue un rôle clé dans la transformée de Radon, car elle permet une paramétrisation naturelle et complète de toutes les droites du plan, ce qui est essentiel pour la définition mathématique et le calcul pratique de cette transformation.\\ 
Cette équation sous forme normale fournit :
\begin{itemize}
    \item[(i)] Une paramétrisation unique et continue de toutes les droites du plan. La forme normale (ou forme normale de Hesse) de l'équation de la droite s'écrit : $$x\,\cos(\theta)+y\,\sin(\theta)=\rho$$ où $\rho$ est la distance par rapport à l'origine et $\theta$ est l'angle par rapport à l'axe des abscisses.
    \item[(ii)] Une interprétation géométrique claire de $\rho$ et $\theta$. Chaque droite du plan correspond  à un unique couple ($\rho,\theta$). Cette paramétrisation évite les redondances et garantit qu'on parcourt toutes les droites une et une seule fois (à une convention près).
    \item[(iii)] Une mesure naturelle sur l'espace des droites, utilisée dans les formules d'inversion.
    \item[(iv)] Un formalisme adapté au théorème de coupe, reliant transformée de Radon et transformée de Fourier 2D. 
    \item[(v)] Une mesure naturelle sur l'espace des droites, utilisée dans les formules d'inversion.
\end{itemize}\vspace{10pt}
\subsection{\small Construction de l'orientation et de la distance}
Nous connaissons tous l'idée qu'une droite \( l \) dans \( \mathbb{R}^2 \) peut être représentée par l'équation 
\[
ax + by = c
\]
où \( a, b, c \in \mathbb{R} \) et \( a^2 + b^2 \neq 0 \).\\ On peut alors écrire cette équation d'une droite sous la forme \[w_1x + w_2y = t\]
où $\mathbf{w}:=(w_1, w_2) = (\cfrac{a}{\sqrt{a^2 + b^2}}, \cfrac{b}{\sqrt{a^2 + b^2}})$ et $t=\cfrac{c}{\sqrt{a^2 + b^2}}$, que nous pouvons voir comme un point situé sur le
cercle unitaire, pour \[\left(\cfrac{a}{\sqrt{a^2 + b^2}}\right)^{2} + \left(\cfrac{b}{\sqrt{a^2 + b^2}}\right)^{2} = 1\]
Cela implique que $\mathbf{w} := (\cos(\theta), \sin(\theta)) \text{ est un vecteur normal unitaire }$, $\theta \in [0, 2\pi)$ représente l'orientation, et $t$ est exactement la distance à l'origine. On a \[x\cos(\theta) + y\sin(\theta) = t\]
Notez que dans les équations ci-dessus, $t$ et $\theta$ sont fixes et déterminent une droite spécifique \( l \) dans le plan. On peut donc dire que $t$ et $\theta$ paramètrent une droite \( l_{t,\theta} \) et que $\mathbf{z}$ détermine des points spécifiques sur cette droite \( l \). Ou encore
\[l_{t,\theta} = \{ \mathbf{z} \in \mathbb{R}^2 : \langle z, (\cos \theta, \sin \theta) \rangle = t \}.\]
\begin{figure}[H]
    \centering
    \includegraphics[width=0.8\textwidth]{./images/l_t_theta.png}
    \caption{paramètrisation d'une droite \( l_{t,\theta} \) par \( t \) et \( \theta \)}
    \label{fig:l_t_theta}
\end{figure}
On voit  que $(t\, \cos(\theta), t\, \sin(\theta))$ est un point situé sur la droite \( l_{t,\theta} \) (\Cref{fig:l_t_theta}) et $(-\sin(\theta), \cos(\theta))$ est un vecteur perpendiculaire au vecteur unitaire $\mathbf{w}$.\\ En géométrie affine élémentaire, une ligne est un point plus une direction. Par conséquent, nous pouvons décrire un point particulier $(x, y)$ sur $l_{t, \theta}$ en termes de nombre réel s comme suit :
\begin{equation}
    l_{t, \theta} = \{(t\, \cos(\theta) - s\,\sin(\theta), t\,\sin(\theta) + s\,\cos(\theta)); s\in \mathbb{R}\}
    \label{set:l_t_theta}
\end{equation}
\begin{definition}[Transformée de Radon]
Soit \( f(t,\theta) \) une fonction définie sur \( \mathbb{R}^2 \) à support compact.
La transformée de Radon de \( f \), notée \( \mathcal{R}f \), est définie pour
\( t \in \mathbb{R} \) et \( \theta \in (0, 2\pi] \) par
\[
\mathcal{R}f(t,\theta) = \int_{-\infty}^{\infty} f(x(s),y(x))\mathrm{d}s
\]
\end{definition}

La transformée de Radon permet de déterminer la densité totale d'une fonction $f$ le long d'une droite donnée $l$. Cette droite $l$ est définie par un angle $\theta$  par rapport à l'axe 
$x$ et une distance $t$ par rapport à l'origine. Comme illustré à la \Cref{fig:radon}, si l'on calcule la transformée de Radon le long de plusieurs droites à des angles différents (ici $\theta_1$ et $\theta_2$), on peut déterminer plusieurs fonctions de densité pour notre objet. Intuitivement, on peut interpréter la transformée de Radon comme une version « étalée » de notre objet initial. Supposons que la région en forme de tache représentée à la \Cref{fig:radon} soit une tache d'encre; si l'on étale cette tache le long de lignes de direction $\theta_1$, on s'attend à ce que les régions les plus larges de la tache correspondent à des zones plus étendues que les régions plus petites, ce qui correspond exactement à ce que l'on observe.
\begin{figure}[H]
    \centering
    \includegraphics[width=0.8\textwidth]{./images/radon.png}
    \caption{Transformée de Radon pour $\theta_1$ et $\theta_2$.}
    \label{fig:radon}
\end{figure}
L'intégrale $\mathcal{R}f(t,\theta)$ représente le membre gauche de l'équation \eqref{eq:radon_transformation}. Rappelons que, dans cette équation, $\mathbf{A}(x)$ est inconnue et que $\ln(\frac{I_1}{I_0})$ correspond à une information mesurée.
Autrement dit, $\ln(\frac{I_1}{I_0})$ est la transformée de Radon, et la transformée de Radon représente donc des données connues issues de la mesure.

L'objectif est maintenant de trouver une formule d'inversion de la transformée de Radon qui nous permettra de reconstruire la fonction initiale $f$ (ou, dans le contexte de l'imagerie médicale, 
$\mathbf{A}(x)$). Pour ce faire, il sera utile de rappeler plusieurs propriétés de la transformée de Radon.
\begin{proposition}
    Soit $\alpha$ et $\beta$ deux réels et $f$ et $g$ deux fonctions continues sur $\mathbb{R}^2$ à support compact. On a
    \begin{itemize}
        \item[(i)] Linéarité : $\mathcal{R}(\alpha f + \beta g) = \alpha \mathcal{R}f + \beta \mathcal{R}g$
        \item[(ii)] Parité: $\mathcal{R}f(-t,-\theta) = \mathcal{R}f(t,\theta)$
        \item[(iii)] $\mathcal{R}f(t, \theta) = \int_{-\infty}^{\infty} f(x(s), y(s))\mathrm{d}s = \int_{-\infty}^{\infty} f(t\,cos(\theta)-s\,sin(\theta), t\,sin(\theta)+s\,cos(\theta))\mathrm{d}s$
        % \item[(iv)] Invariance par rotation : \(\mathcal{R}(f \circ R_{\psi}) = \mathcal{R}f(t,\theta - \psi)\)
        % \item[(v)] Relation avec la convolution : \(\mathcal{R}(f * g) = \mathcal{R}f * \mathcal{R}g\)
    \end{itemize}
\end{proposition}
Nous définissons en outre le domaine naturel de la transformée de Radon comme l'ensemble des fonctions $f$ sur $\mathbb{R}^2$ telles que \[\int_{-\infty}^{\infty} |f(x(s), y(s))|\mathrm{d}s < \infty\]

\subsection{Théorème de la Coupe Centrale}
Le théorème de la coupe centrale, également appelé théorème de projection-transforme de Fourier ou théorème de Fourier-Slice, est un résultat fondamental en traitement d'image et en tomographie. Il établit un lien profond entre la transformée de Radon (utilisée pour décrire les projections d'un objet) et la transformée de Fourier (utilisée pour analyser les fréquences spatiales). Ce théorème constitue la pierre angulaire mathématique de la plupart des méthodes de reconstruction tomographique moderne.

\begin{proposition}
    Soit \( g \) une fonction absolument integrale sur \( \mathbb{R}^2 \).
    Le théorème de la coupe centrale affirme que, pour tout $S \in \mathbb{R}$ et $\theta \in [0,2\pi]$, on a : \[\mathcal{F}_2 g(S\cos(\theta), S\sin(\theta)) = \mathcal{F}(\mathcal{R}g)(S, \theta)\]
\end{proposition}
\textbf{Preuve}: En utilisant la définition de la transformée de Fourier bidimensionnelle \eqref{eq:fourier_2d} on obtient 
\[
    \mathcal{F}_{2}g(S\,\cos(\theta), S\,\sin(\theta)) = \int_{-\infty}^{\infty} \int_{-\infty}^{\infty} g(x, y)\, e^{-iS (x\,\cos(\theta) + y\,\sin(\theta))}\, dx\, dy
\]
Nous effectuons maintenant un changement de variables conformément au système
de coordonnées que nous avons défini à la \textit{Construction de l'orientation et de la distance}.
Rappelons que, lors de la paramétrisation de la droite $\ell_{t,\theta}$,
nous avons montré que, pour $s\in\mathbb{R}$, on peut écrire :
\[
x(s)=t\cos\theta - s\sin\theta, 
\qquad
y(s)=t\sin\theta + s\cos\theta,
\qquad
t = x\cos\theta + y\sin\theta.
\]

En examinant le déterminant du Jacobien associé à $x(s)$ et $y(s)$, on obtient :
\[
\det
\begin{pmatrix}
\dfrac{\partial x}{\partial t} & \dfrac{\partial x}{\partial s} \\[6pt]
\dfrac{\partial y}{\partial t} & \dfrac{\partial y}{\partial s}
\end{pmatrix}
= 1.
\]

Nous en déduisons que
\[
ds\,dt = dx\,dy.
\]
et donc
\[
\int_{-\infty}^{\infty} \int_{-\infty}^{\infty} g(x, y)\, e^{-iS (x\,\cos(\theta) + y\,\sin(\theta))}\, dx\, dy = \int_{-\infty}^{\infty}\int_{-\infty}^{\infty}
g(t\cos\theta - s\sin\theta,\; t\sin\theta + s\cos\theta)\,
e^{-iSt}\,ds\,dt.
\]

Comme $e^{-iSt}$ ne dépend pas de la variable $s$, nous pouvons réarranger
l'intégrale précédente de la manière suivante :
\[
\int_{-\infty}^{\infty}
\left(
\int_{-\infty}^{\infty}
g(t\cos\theta - s\sin\theta,\; t\sin\theta + s\cos\theta)\,ds
\right)
e^{-iSt}\,dt.
\]

L'intégrale intérieure est exactement la transformée de Radon de $f$,
évaluée en $(t,\theta)$, ce qui implique que l'expression précédente devient :
\[
\int_{-\infty}^{\infty}
(Rg(t,\theta))\,e^{-iSt}\,dt.
\]

Cette dernière intégrale n'est autre que la transformée de Fourier de
$Rg(S,\theta)$, ce qui conclut la démonstration.
\hfill$\square$

% \section{Inversion analytique de la transformée de Radon}
\subsection{Rétroprojection filtrée (FBP)}
Nous sommes maintenant enfin prêts à effectuer une première tentative pour retrouver la fonction de coefficient d'atténuation.
Rappelons que, d'un point de vue physique, la transformée de Radon
$\mathcal{R}f(t,\theta)$ nous donne la densité totale de l'objet $f$ le long d'une droite
$\ell_{t,\theta}$.
Nous avons déterminé cette densité en mesurant les intensités initiale et finale
d'un faisceau de rayons $\mathbf{X}$ traversant l'objet le long de cette droite.
En procédant ainsi pour plusieurs droites différentes, nous sommes capables de
reconstruire une coupe unique de l'objet initial, et en faisant varier l'angle
$\theta$ de ces rayons $\mathbf{X}$, nous pouvons définir de nombreuses coupes.

Si nous sommes capables, d'une certaine manière, de « rétroprojeter » ces
densités sur le plan, nous pourrons peut-être reconstituer l'objet initial.
Intuitivement, on peut interpréter ce processus comme le fait de prendre les
données du sinogramme et de les « déflouter » pour les ramener dans le plan.
\begin{definition}
Soit $h = h(t,\theta)$. On définit la \emph{rétroprojection},
notée $\mathcal{B}h$, en un point $(x,y)$ par :
\[
\mathcal{B}h(x,y) = \frac{1}{\pi}\int_{0}^{\pi} h(x\cos\theta + y\sin\theta,\theta)\,d\theta.
\]

En appliquant cette formule de rétroprojection à la transformée de Radon, on
obtient :
\begin{equation}
    \mathcal{B}\mathcal{R}f(x,y) = \frac{1}{\pi}\int_{0}^{\pi}
    \mathcal{R}f(x\cos\theta + y\sin\theta,\theta)\,d\theta.
    \label{eq:FBP}
\end{equation}
\end{definition}
Nous sommes capables d'effectuer la rétroprojection sur les coupes que nous
avons mesurées. Comme illustré à la \Cref{fig:FBP}, effectuer une rétroprojection
selon seulement quelques directions $\theta$ constitue une méthode extrêmement
imprécise pour reconstituer ne serait-ce qu'un objet simple. Toutefois, même si
nous augmentons de manière significative le nombre de rétroprojections
(par exemple jusqu'à $1000$ directions), il subsiste encore une quantité
importante de bruit qui brouille l'image reconstruite.
En réalité, quel que soit le nombre de directions selon lesquelles nous tentons
d'effectuer la rétroprojection, nous ne serons jamais capables de reconstruire
parfaitement l'image à l'aide de la formule de rétroprojection donnée par
l'équation \eqref{eq:FBP}.
Pour que ce procédé soit réellement utile, il est nécessaire de dériver une
méthode permettant de filtrer une partie du bruit que la formule de
rétroprojection semble introduire dans l'image, afin d'obtenir une
représentation plus lisse de l'objet.

\begin{figure}[H]
    \centering
    \includegraphics[width=0.8\textwidth]{./images/fbp.png}
    \caption{Retroprojection d'un carré dans 5, 25, 100 et 1000 directions}
    \label{fig:FBP}
\end{figure}

Dans ce but, nous définissons une formule de \emph{rétroprojection filtrée}.
\begin{proposition}
    Soit $f$ une fonction absolument intégrable définie sur $\mathbb{R}^2$. Alors,
    \begin{equation}
        f(x,y)
        =
        \frac{1}{2}\,
        \mathcal{B}\!\left\{
        \mathcal{F}^{-1}
        \!\left[
        |S|\,
        \mathcal{F}\!\left(\mathcal{R}f\right)(S,\theta)
        \right]
        \right\}(x,y).
        \label{eq:FBP_filter}
    \end{equation}
\end{proposition}
\textit{Démonstration.}
Nous commençons par rappeler que, pour la transformée de Fourier bidimensionnelle
et son inverse, on a :
\begin{equation}
f(x,y) = \mathcal{F}_2^{-1}\,\mathcal{F}_2 f(x,y)
= \frac{1}{4\pi^2}
\int_{-\infty}^{\infty}\int_{-\infty}^{\infty}
\mathcal{F}_2 f(X,Y)\,e^{i(Xx+Yy)}\,dX\,dY.
\label{eq:fourier_2d_inverse}
\end{equation}

Nous allons maintenant effectuer un changement de variables des coordonnées
cartésiennes $(X,Y)$ vers les coordonnées polaires $(S,\theta)$, définies par
\[
X = S\cos\theta,
\qquad
Y = S\sin\theta,
\]
où $S \in \mathbb{R}$ et $\theta \in [0,\pi]$.
Ce changement de variables conduit au déterminant jacobien suivant :
\[\det
\begin{pmatrix}
    \dfrac{\partial X}{\partial s} & \dfrac{\partial X}{\partial \theta} \\[6pt]
    \dfrac{\partial Y}{\partial s} & \dfrac{\partial Y}{\partial \theta}
\end{pmatrix}
=|S|
\]
Ce qui nous dit que $dX\,dY = |S|\,dS\,d\theta$. En incorporant ce nouveau changement de variables, l'équation \eqref{eq:fourier_2d_inverse} devient :
\[
f(x,y) = \frac{1}{4\pi^{2}} \int_{0}^{\pi} \int_{-\infty}^{\infty}
\mathcal{F}_{2}f(S\cos\theta, S\sin\theta)\,
e^{iS(x\cos\theta + y\sin\theta)}\,|S|\,dS\,d\theta.
\]
Et en utilisant le théorème de la tranche centrale, nous voyons que l'équation ci-dessus est en fait égale à
\begin{equation}
    f(x,y) = \frac{1}{4\pi^{2}} \int_{0}^{\pi} \int_{-\infty}^{\infty}
    \mathcal{F}\bigl(\mathcal{R}f(S,\theta)\bigr)\,
    e^{iS(x\cos\theta + y\sin\theta)}\,|S|\,dS\,d\theta.
    \label{eq:fourier_radon}
\end{equation}
Prenons maintenant un regard plus attentif sur l'intégrale intérieure de l'équation \eqref{eq:fourier_radon} et en utilisant la définition de la Transformée de Fourier inverse, on a :
\[
    \begin{array}{rcl}
        \int_{-\infty}^{\infty}
        \mathcal{F}\bigl(\mathcal{R}f(S,\theta)\bigr)\,
        e^{iS(x\cos\theta + y\sin\theta)}\,|S|\,dS
        &=&
        2\pi \left(
        \frac{1}{2\pi} \int_{-\infty}^{\infty}
        \mathcal{F}\bigl(\mathcal{R}f(S,\theta)\bigr)\,
        e^{iS(x\cos\theta + y\sin\theta)}\,|S|\,dS
        \right)\\
        &=&
        2\pi\,\mathcal{F}^{-1}
        \Bigl(
        |S|\,\mathcal{F}\bigl(\mathcal{R}f\bigr)(S,\theta)
        \Bigr)
        \bigl(x\cos\theta + y\sin\theta,\theta\bigr)\\
    \end{array}
\]


Autrement dit, l'intégrale intérieure de l'équation (7.4) est égale à $2\pi$ fois l'inverse de la transformée de Fourier de
$|S|\,\mathcal{F}\bigl(\mathcal{R}f\bigr)(S,\theta)$
au point $(x\cos\theta + y\sin\theta,\theta)$.
Nous pouvons alors voir que l'équation (7.4) est en fait égale à
\[
\frac{1}{2\pi} \int_{0}^{\pi}
\mathcal{F}^{-1}
\Bigl(
|S|\,\mathcal{F}\bigl(\mathcal{R}f\bigr)(S,\theta)
\Bigr)
\bigl(x\cos\theta + y\sin\theta,\theta\bigr)
\,d\theta.
\]

Finalement, nous constatons que l'intégrale ci-dessus est égale à $\tfrac{1}{2}$ de la rétroprojection donnée dans la définition \eqref{eq:FBP} pour
$\mathcal{F}^{-1}\bigl[|S|\,\mathcal{F}(\mathcal{R}f)(S,\theta)\bigr]$.
Nous simplifions donc l'équation précédente pour obtenir
\[
\frac{1}{2}\,
\mathcal{B}
\Bigl\{
\mathcal{F}^{-1}
\bigl[|S|\,\mathcal{F}\bigl(\mathcal{R}f(S,\theta)\bigr)\bigr]
\Bigr\}(x,y).
\]

Ce qui nous conduit à la conclusion souhaitée :
\[
f(x,y)
=
\frac{1}{2}\,
\mathcal{B}
\Bigl\{
\mathcal{F}^{-1}
\bigl[|S|\,\mathcal{F}\bigl(\mathcal{R}f(S,\theta)\bigr)\bigr]
\Bigr\}(x,y).
\]
\hfill $\square$\\
Le facteur important dans cette formule est le multiplicateur $|S|$ qui apparaît entre la transformée de Fourier et son inverse. Sans ce facteur, ces deux termes s'annuleraient mutuellement et nous nous retrouverions avec la formule standard de rétroprojection pour la transformée de Radon que nous avons rencontrée précédemment et qui, comme nous l'avons vu, ne nous donne pas directement $f(x, y)$. Nous appelons ce $|S|$ supplémentaire un \textbf{filtre} de la transformée de Radon, ce qui nous donne le nom de la formule de \textbf{rétroprojection filtrée}.
\begin{proposition}
    Soit $f$ et $g$ deux fonctions intégrables définies sur $\mathbb{R}$, alors
    \[(\mathcal{B}g\star f)(X, Y) = \mathcal{B}(g\star \mathcal{R}f)(X, Y)\]
\end{proposition}
Considérons maintenant la relation \eqref{eq:FBP_filter} et 
supposons qu'il existe une fonction, notée $\varphi(t)$, dont la transformée de Fourier
soit égale à notre facteur de filtrage $|S|$. Autrement dit, supposons qu'il existe une
fonction $\varphi(t)$ telle que
\[
\mathcal{F}\varphi(S) = |S|.
\]
Plus simplement, supposons que nous connaissions une fonction dont la transformée de
Fourier est égale à la fonction valeur absolue. Nous pourrions alors réécrire la
rétroprojection sous la forme suivante :
\begin{equation}
    f(x,y) = \frac{1}{2}\,\mathcal{B}
    \left\{
    \mathcal{F}^{-1}
    \bigl[
    \mathcal{F}\varphi \cdot \mathcal{F}(\mathcal{R}f)(S,\theta)
    \bigr]
    \right\}(x,y).
    \label{eq:FBP_varphi}
\end{equation}

Cependant, le membre de droite de l'équation \eqref{eq:FBP_varphi} contient un produit de transforméesde Fourier, que nous savons être égal à la convolution des fonctions transformées
\[
    f(x,y)
    =
    \frac{1}{2}\,\mathcal{B}
    \left\{
    \mathcal{F}^{-1}
    \bigl[
    \mathcal{F}(\varphi \star \mathcal{R}f)(S,\theta)
    \bigr]
    \right\}(x,y).
\]

Mais ceci n'est rien d'autre que la transformée de Fourier inverse de la transformée
de Fourier, ce qui nous ramène à la fonction de départ. Cela nous conduit à la formule
de rétroprojection filtrée beaucoup plus simple :
\begin{equation}
    f(x,y) = \frac{1}{2}\,\mathcal{B}(\varphi \star \mathcal{R}f)(x,y).
    % 
    \label{eq:FBP_varphi_simple}
\end{equation}

L'équation \eqref{eq:FBP_varphi_simple} est bien plus élégante que notre formule initiale de rétroprojection filtrée et ne semble pas difficile à appliquer. Physiquement parlant, $\mathcal{R}f$ représente nos données mesurées et l'équation \eqref{eq:FBP_varphi_simple} requiert simplement de les filtrer à l'aide de notre nouvelle fonction $\varphi$, puis d'appliquer la formule de rétroprojection, qui est une intégrale relativement simple.

Malheureusement, il n'existe pas de fonction $\varphi$ dont la transformée de Fourier
soit exactement égale à la valeur absolue. Considérons la fonction $\mathcal{F}\varphi$ :
\[
\mathcal{F}\varphi(\omega)
=
\int_{-\infty}^{\infty}
\varphi(x)\,e^{-i\omega x}\,dx.
\]

Nous pouvons constater que, lorsque $\omega \to \infty$,
$\mathcal{F}\varphi(\omega) \to 0$ (remarquons l'exponentielle négative).
Cependant, pour la fonction valeur absolue $|\omega|$, lorsque $\omega \to \infty$,
$|\omega| \to \infty$.
Par conséquent, il est impossible de trouver une fonction $\varphi$ telle que,
pour tout $\omega$, $\mathcal{F}\varphi(\omega) = |\omega|$.

Toutefois, tout notre travail précédent n'est pas vain. Examinons plutôt le type de
fonctions sur lesquelles nous avons restreint notre étude. Nous ne considérons notre
fonction que sur un intervalle fini et supposons en fait qu'elle soit nulle en dehors
de cet intervalle. En étendant cette idée à la transformée de Fourier, nous constatons
que nous devons porter notre attention sur les \emph{fonctions à bande limitée}.

\begin{definition}
    Une fonction $\varphi$ est dite \emph{à bande limitée} s'il existe un réel $L > 0$ tel que
    \begin{equation}
        \mathcal{F}\varphi(\omega)
        =
        \int_{-\infty}^{\infty}
        \varphi(x)\,e^{-i\omega x}\,dx
        =
        0
        \quad \text{pour tout } \omega \notin [-L, L].
        % 
        \label{eq:FBP_varphi_banded}
    \end{equation}
\end{definition}

Le facteur de filtrage $|S|$ sert à amplifier le terme $\mathcal{F}(\mathcal{R}f)$ dans la formule de rétroprojection filtrée originale \eqref{eq:FBP_filter}. En pratique, $\mathcal{F}(\mathcal{R}f)$ est très sensible aux hautes fréquences.

En concentrant notre attention sur les basses fréquences à l'aide d'une fonction à bande limitée $\varphi$, nous sommes en mesure d'éviter ce problème. Notre objectif est de remplacer $S$ par ce que l'on appelle un \emph{filtre passe-bas} (noté $S'$), qui prend en compte les effets des basses fréquences tout en atténuant les hautes fréquences. Cette fonction $S'$ doit avoir un support compact et être de la forme
\[
S' = \mathcal{F}\varphi
\]
(sur un intervalle compact).

Le coût de l'utilisation de $S'(\omega)$ est que nous ne disposons plus de l'égalité présentée dans l'équation \eqref{eq:FBP_varphi_simple}. En revanche, nous obtenons :
\begin{equation}
    f(x,y) \approx \frac{1}{2}\,\mathcal{B}\!\left(\mathcal{F}^{-1} S' \star \mathcal{R}f \right)(x,y).
    \label{eq:FBP_varphi_approx}
\end{equation}

De manière générale, la plupart des filtres passe-bas sont de la forme
\[
S'(\omega) = |\omega| \cdot F(\omega) \cdot \Pi_L(\omega),
\]
où $L > 0$ définit la région sur laquelle le filtrage est effectué. Différentes fonctions $F$ déterminent les caractéristiques précises du filtre, et $\Pi_L(\omega)$ est définie comme suit :
\[
    \Pi_L(\omega) =
    \begin{cases}
        1 & \text{si } |\omega| \leq L, \\
        0 & \text{si } |\omega| > L.
    \end{cases}
\]

Nous introduisons maintenant deux filtres couramment utilisés en imagerie numérique et en traitement du signal : le filtre \emph{Ram-Lak} et le filtre \emph{Hann}.

\subsection*{Filtre Ram-Lak}

Le filtre Ram-Lak est défini par :
\[
S'(\omega) = |\omega| \cdot \Pi_L(\omega) =
\begin{cases}
|\omega| & \text{si } |\omega| \leq L, \\
0 & \text{si } |\omega| > L.
\end{cases}
\]

Le filtre Ram-Lak constitue la base de nombreux autres filtres utilisés en analyse du signal, car il remplace simplement la fonction $F(\omega)$ par la fonction constante égale à 1. D'autres filtres, tels que le filtre Hann, consistent généralement en des produits de fonctions sinus ou cosinus destinées à éliminer le bruit indésirable.

\subsection*{Filtre Hann}

Le filtre Hann est donné par :
\[
S'(\omega) = |\omega| \cdot \frac{1}{2}
\left( 1 + \cos\!\left( \frac{2\pi \omega}{L} \right) \right)
\cdot \Pi_L(\omega).
\]

Le filtre Hann utilise la fonction de Hann
\[
\frac{1}{2}\left( 1 + \cos\!\left( \frac{2\pi \omega}{L} \right) \right)
\]
comme fonction $F(\omega)$,
% et son efficacité est illustrée dans le sinogramme et la rétroprojection de la transformée de Radon de Johann, présentés à la Figure~5.
% ====== TODO ======
% Python implementation
% ==================

\section{Discrétisation des méthodes analytiques}
Ainsi nous avons traité presque exclusivement des intégrales continues pour la transformée de Radon, la transformée de Fourier et les formules de rétroprojection. En pratique, cependant, nous n'avons qu'un ensemble fini de données avec lesquelles travailler. Par conséquent, nous devrons former des versions discrètes de toutes les formules que nous avons utilisées dans notre rétroprojection filtrée.

Une fonction discrète est une fonction définie uniquement sur un ensemble dénombrable. Pour nos besoins, nous considérerons des fonctions discrètes définies sur des ensembles finis (l'ensemble étant composé des lignes sur lesquelles nous avons pris nos mesures d'intensité). Soit \(g_n\) la fonction discrète \(g\) à la valeur \(n\). Comme nous connaissons cette fonction discrète sur un ensemble fini, soit \(N\), nous pouvons dire que \(g = g_n : 0 \le n \le N - 1\). Si nous voulons étendre cette définition à tous les entiers, nous pouvons simplement « répéter » notre fonction encore et encore ; c'est-à-dire, nous pouvons la rendre périodique avec une période \(N\). Cette extension sera utile pour certaines des formules discrètes que nous rencontrerons.

Supposons que nous prenions des mesures à \(P\) angles différents \(\theta\) et que pour chaque angle nous ayons \(2M + 1\) faisceaux espacés d'une distance \(d\). Alors nous pouvons définir des valeurs particulières \(\theta_k\) et \(t_j\) comme

\[
\theta_k = \left\{ \frac{k \pi}{P} : 0 \le k \le P - 1 \right\},
\]

\[
t_j = \{ jd : -M \le j \le M \}.
\]

Ce qui nous permet de définir une ligne particulière comme \(l_{t_j, \theta_k}\). Nous définissons donc la transformée de Radon discrète comme suit :

\begin{definition}
Pour une fonction absolument intégrable \(f\) et \(0 \le k \le P\) et \(-M \le j \le M\), \((P, M > 0)\), nous définissons la transformée de Radon discrète de \(f\), notée \(\mathcal{R}_D f\), comme

\[
\mathcal{R}_D f_{j,k} = \mathcal{R} f(t_j, \theta_k).
\]

\end{definition}
Pour mettre en œuvre la formule de rétroprojection filtrée \eqref{eq:FBP_varphi_approx}, nous devons également définir la convolution de deux fonctions discrètes.

\begin{definition}
Pour deux fonctions discrètes \(N\)-périodiques \(f\) et \(g\), nous définissons la \textbf{convolution discrète} de \(f\) et \(g\), notée \(f \star g\), comme

\[
(f \star g)_m = \sum_{j=0}^{N-1} f_j \cdot g_{(m-j)}, \quad \text{pour } m \in \mathbb{Z}.
\]
Évidemment, nous aurons également besoin de la transformée de Fourier discrète.
\end{definition}

\begin{definition}[Transformée de Fourier discrète]
Étant donnée une fonction discrète $N$-périodique $f$, nous définissons la \textbf{transformée de Fourier discrète} de $f$, notée $\mathcal{F}_D f$, par
\begin{equation}
(\mathcal{F}_D f)_j = \sum_{k=0}^{N-1} f_k e^{i 2 \pi k j / N}, \quad \text{pour } j = 0, 1, \dots, (N-1).
\end{equation}
Il convient de noter que la $N$-périodicité de $f$ nous permet de remplacer les bornes de la sommation par tout ensemble d'entiers de longueur $(N-1)$. Avec cette définition, il n'est pas surprenant que nous définissions la transformée de Fourier discrète inverse de la manière suivante.
\end{definition}

\begin{definition}[Transformée de Fourier discrète inverse]
Étant donnée une fonction discrète $N$-périodique $g$, la \textbf{transformée de Fourier discrète inverse} de $g$, notée $\mathcal{F}_D^{-1} g$, est définie par
\begin{equation}
(\mathcal{F}_D^{-1} g)_n = \frac{1}{N} \sum_{k=0}^{N-1} g_k e^{i 2 \pi k n / N}, \quad \text{pour } n = 0, 1, \dots, (N-1).
\end{equation}
\end{definition}

Nous remarquons que plusieurs des mêmes propriétés de la transformée de Fourier que nous avons définies dans le cadre continu s'appliquent également au cas discret avec de légères modifications :

\begin{proposition}[Propriétés des fonctions discrètes $N$-périodiques]
Pour des fonctions discrètes $N$-périodiques $f$ et $g$ :
\begin{enumerate}
    \item $\mathcal{F}_D(f \star g) = (\mathcal{F}_D f) \cdot (\mathcal{F}_D g)$
    \item $\mathcal{F}_D(f \cdot g) = \frac{1}{N} (\mathcal{F}_D f) \star (\mathcal{F}_D g)$
    \item $\mathcal{F}_D^{-1}(\mathcal{F}_D f)_n = f_n \quad \text{pour tout } n \in \mathbb{Z}$
\end{enumerate}
\end{proposition}

Nous sommes maintenant prêts à aborder la discrétisation de la formule de rétroprojection elle-même. Rappelons que la formule de rétroprojection était définie comme une intégrale de $0$ à $\pi$ par rapport à $d\theta$. Dans le cas discret, nous avons remplacé ce $d\theta$ continu par $k\pi / P$ pour $0 \le k \le (P-1)$. Cela conduit à la définition suivante de la \textbf{rétroprojection discrète} :

\begin{definition}[Rétroprojection discrète]
Étant donnée une fonction discrète $h$, nous définissons la \textbf{rétroprojection discrète} de $h$, notée $\mathcal{B}_D h$, par
\begin{equation}
\mathcal{B}_D h(x,y) = \frac{1}{N} \sum_{k=0}^{N-1} h \big(x \cos \frac{k \pi}{N} + y \sin \frac{k \pi}{N}, k \pi / N \big).
\label{eq:discrete_backprojection}
\end{equation}
\end{definition}

Rappelons notre forme finale pour la formule filtrée de rétroprojection en équation \eqref{eq:FBP_varphi_approx} :
$$
f(x,y) \approx \frac{1}{2} \mathcal{B} (\mathcal{F}^{-1} S' \star \mathcal{R} f)(x,y).
$$

Pour former la version discrète de cette équation, nous voyons que nous devons appliquer la formule suivante:
\begin{equation}
f(x,y) \approx \frac{1}{2} \mathcal{B}_D \left( \mathcal{F}_D^{-1} \mathcal{S}' \ast \mathcal{R}_D f \right)(x,y).
\end{equation}

Nous rencontrons maintenant un léger problème. $\mathcal{R}_D f$ représente les données mesurées basées sur les intensités finales d'un seul faisceau de rayons X. Nous avons défini les emplacements des différents faisceaux (et donc des différentes coupes) en utilisant un système de coordonnées perpendiculaire aux coordonnées polaires basé sur des angles discrets $\theta$ et des distances $t$.  

En examinant l'équation \eqref{eq:discrete_backprojection}, nous voyons que nous devons sommer sur $h$ en différents points $(x,y)$ dans le système de coordonnées cartésien pour créer une grille de niveaux de gris rectangulaire qui représente notre objet original. Les systèmes de coordonnées polaires et cartésiens ne correspondent pas nécessairement parfaitement, et nous devons donc \emph{interpoler} les points de données manquants. L'interpolation consiste à créer une fonction continue (ou au minimum par morceaux continues) à partir d'un ensemble discret de valeurs. Il existe de nombreuses méthodes pour interpoler une fonction (spline cubique, Lagrange, etc.), chacune ayant ses avantages et inconvénients.  

Pour nos besoins, nous allons définir un type général d'interpolation basé sur une fonction de pondération $W$ qui détermine comment nous allons choisir nos points interpolés. Nous ne définissons pas de fonction de pondération particulière $W$, car les détails de l'interpolation ne sont pas aussi importants que le fait que nous pouvons remplir les "trous" dans nos données.

\begin{definition}
Pour une fonction de pondération donnée $W$ et une fonction discrète $N$-périodique $g$, l'\emph{interpolation $W$} de $g$ est définie par :
\begin{equation}
\mathcal{I}_W(g)(x) = \sum_n g(n) \cdot W\left(\frac{x}{d}-n\right), \quad \text{pour } -\infty < x < \infty.
\end{equation}
\end{definition}

Maintenant que nous avons couvert toutes les parties de l'équation \eqref{eq:FBP_varphi_simple} dans un cadre discret et traité le problème de l'interpolation, nous pouvons proposer un algorithme de reconstruction discret pour résoudre le coefficient d'atténuation à partir d'un ensemble de données discret.  

Nous interpolons ici la fonction $\left(\mathcal{F}_D^{-1}\mathcal{S}'\right) \ast \mathcal{R}_D f(jd,k\pi/N)$ (c'est-à-dire que nous remplissons les trous après le filtrage de la transformée de Radon). Définissons cette fonction interpolée comme $\mathcal{I}$. Cela conduit à la formule de reconstruction suivante :

\begin{equation}
\begin{aligned}
f(x_m,y_n) &\approx \frac{1}{2} \mathcal{B}_D \left( \left( \mathcal{F}_D^{-1}\mathcal{S}' \right) \ast \mathcal{R}_D f \right) (jd, k\pi/N) \\
&\approx \frac{1}{2} \mathcal{B}_D \mathcal{I}(x_m,y_n) \\
&= \frac{1}{2N} \sum_{k=0}^{N-1} \mathcal{I} \left( x_m \cos \frac{k\pi}{N} + y_n \sin \frac{k\pi}{N}, \frac{k\pi}{N} \right).
\end{aligned}
\end{equation}

L'équation précédente tient compte de la nature discrète de nos données réelles et traite les problèmes (comme le manque de données) qui surviennent lorsque l'on dispose d'un nombre fini de mesures.


% \section{Formulation linéaire -- Synthèse}
% D'accord, nous allons expliquer pas à pas comment passer de la formulation intégrale continue de la rétroprojection filtrée à la \textbf{formulation linéaire discrète \(g = Af\)} utilisée en pratique en tomographie.
% \subsection{Rétroprojection filtrée continue}
% On a la formule continue pour la reconstruction filtrée :
% \[
% f(x,y) = \frac{1}{2} \int_0^\pi \Big( (\mathcal{F}^{-1} S') \star \mathcal{R} f \Big)(x\cos\theta + y\sin\theta, \theta) \, d\theta
% \]

% Ici :
% \begin{itemize}
%     \item $f(x,y)$ : coefficient d'atténuation à reconstruire,
%     \item $\mathcal{R} f(t,\theta)$ : transformée de Radon (projection à l'angle $\theta$),
%     \item $\mathcal{F}^{-1} S'$ : filtre appliqué sur chaque projection,
%     \item $\star$ : convolution dans $t$.
% \end{itemize}

% C'est une \textbf{formule intégrale continue}, dépendante de coordonnées polaires.\vspace{5pt}\\
% Cette expression met en évidence la structure mathématique exacte de la reconstruction tomographique idéale. Toutefois, elle repose sur des intégrales et des fonctions continues qui ne peuvent pas être manipulées directement en pratique. Pour une implémentation numérique, il est indispensable de passer à une représentation discrète. Cette transition constitue le cœur des méthodes de reconstruction utilisées en imagerie médicale.

% \subsection{Discrétisation des coordonnées et des angles}

% Pour passer au discret :
% \begin{enumerate}
%     \item On ne mesure que $P$ angles : $\theta_k = k\pi/P$, $k = 0,\dots,P-1$,
%     \item On ne mesure que $2M+1$ faisceaux par angle, espacés de $d$ : $t_j = j d, j=-M,\dots,M$,
%     \item On obtient donc la \textbf{transformée de Radon discrète} :
%     \[
%     \mathcal{R}_D f_{j,k} = \mathcal{R} f(t_j, \theta_k).
%     \]
% \end{enumerate}\vspace{5pt}
% Cette étape traduit les contraintes physiques des systèmes d'acquisition réels, qui ne fournissent qu'un nombre fini de mesures. L'échantillonnage en angle et en position transforme ainsi le problème continu en un ensemble de données numériques exploitables. La qualité de la reconstruction dépend directement de la finesse de cette discrétisation. Elle conditionne notamment la résolution spatiale et les artefacts.

% \subsection{Convolution et filtrage discrets}

% On applique ensuite le filtre sur chaque projection :
% \[
% h_{j,k} = (\mathcal{F}_D^{-1} \mathcal{S}' \ast \mathcal{R}_D f)_{j,k}
% \]

% Ici, $\ast$ est la \textbf{convolution discrète} dans $t$ :
% \[
% (f \ast g)_m = \sum_{n=0}^{N-1} f_n \, g_{(m-n)}.
% \]\vspace{5pt}
% Le filtrage est une étape cruciale de la méthode FBP, car il corrige le flou introduit par la rétroprojection simple. La convolution discrète permet d'implémenter efficacement ce filtrage sur des données échantillonnées. Elle renforce les hautes fréquences nécessaires à une bonne résolution spatiale. Sans cette étape, la reconstruction serait fortement dégradée.

% \subsection{Discrétisation de la rétroprojection}

% La rétroprojection discrète est :
% \[
% f(x_m,y_n) \approx \frac{1}{2N} \sum_{k=0}^{N-1} h\Big( x_m \cos \frac{k\pi}{N} + y_n \sin \frac{k\pi}{N}, \frac{k\pi}{N} \Big).
% \]

% Comme les coordonnées cartésiennes $(x_m, y_n)$ ne tombent pas exactement sur les positions $t_j$, on \textbf{interpole} :
% \[
% h\big(x_m\cos\theta_k + y_n \sin\theta_k, \theta_k\big) \approx \sum_j h_{j,k} \, W\left(\frac{x_m \cos\theta_k + y_n \sin\theta_k - t_j}{d}\right),
% \]
% où $W$ est la fonction de pondération de l'interpolation (linéaire, spline, etc.).  
% Cela transforme chaque $f(x_m, y_n)$ en \textbf{combinaison linéaire des mesures $h_{j,k}$}.\vspace{5pt}\\
% La rétroprojection redistribue l'information des projections filtrées sur la grille de l'image reconstruite. Cette étape assure la cohérence géométrique entre les données mesurées et l'espace image. L'interpolation est indispensable pour relier les coordonnées polaires aux pixels cartésiens. Elle introduit également une approximation contrôlée du modèle continu.


% \subsection{Passage à la forme matricielle linéaire}

% Si on note :
% \begin{itemize}
%     \item $f$ le vecteur de tous les $f(x_m, y_n)$ sur la grille,
%     \item $g$ le vecteur de toutes les mesures projetées filtrées $h_{j,k}$,
%     \item $A$ la matrice représentant la \textbf{rétroprojection + interpolation},
% \end{itemize}

% alors :
% \[
% f_i = \sum_j A_{ij} \, g_j
% \]

% Chaque coefficient $A_{ij}$ représente le poids avec lequel la projection $g_j$ contribue au pixel $f_i$.  

% On obtient donc :
% \[
% \boxed{g = Af} \quad \text{ou souvent } f = A g \text{ selon la notation.}
% \]

% En pratique, $A$ est \textbf{très grande et creuse}, mais la reconstruction se réduit à un simple \textbf{produit matriciel}.\vspace{5pt}\\
% Cette écriture matricielle unifie l'ensemble du processus de reconstruction dans un cadre linéaire. Elle permet d'analyser la FBP avec les outils de l'algèbre linéaire et de l'optimisation. De nombreuses méthodes itératives modernes reposent sur cette formulation. Elle constitue ainsi un pont entre les approches analytiques et numériques.

% \subsection{Synthèse}

% Le passage de l'intégrale continue à $g = Af$ se fait en quatre étapes principales :
% \begin{enumerate}
%     \item \textbf{Échantillonnage discret} des angles et des faisceaux → $\mathcal{R}_D f$,
%     \item \textbf{Filtrage discret} via convolution et transformée de Fourier discrète,
%     \item \textbf{Rétroprojection discrète} et interpolation sur la grille cartésienne,
%     \item \textbf{Écriture linéaire} : chaque pixel reconstruit est une combinaison linéaire des mesures → matrice $A$.
% \end{enumerate}

% Ainsi, \textbf{toute la formule intégrale est transformée en somme discrète}, et la linéarité de la convolution et de la rétroprojection permet de la représenter par $A$.\vspace{5pt}\\
% Cette synthèse met en évidence la cohérence du passage du modèle physique continu vers une implémentation numérique concrète. La méthode FBP apparaît alors comme une succession d'opérations linéaires bien structurées. Cette vision est essentielle pour comprendre ses limites et ses extensions. Elle prépare naturellement l'introduction de méthodes de reconstruction plus avancées.
\section{Formulation linéaire -- Synthèse}

Ce chapitre explique pas à pas comment passer de la formulation intégrale continue de la rétroprojection filtrée à la \textbf{formulation linéaire discrète \( g = Af \)} utilisée en pratique en tomographie. Cette transition est essentielle pour relier la théorie mathématique aux algorithmes numériques implémentés dans les systèmes réels. Elle constitue également le point d’entrée vers les méthodes itératives modernes.\vspace{7pt}\\
D'accord, nous allons expliquer pas à pas comment passer de la formulation intégrale continue de la rétroprojection filtrée à la \textbf{formulation linéaire discrète \(g = Af\)} utilisée en pratique en tomographie.

\subsection{Rétroprojection filtrée continue}
Cette section présente l’expression analytique idéale de la reconstruction tomographique. Elle met en évidence le rôle central de la transformée de Radon, du filtrage et de l’intégration angulaire. Bien que mathématiquement élégante, cette formulation reste purement continue et ne peut pas être appliquée directement en pratique.\vspace{7pt}\\
On a la formule continue pour la reconstruction filtrée :
\[
f(x,y) = \frac{1}{2} \int_0^\pi \Big( (\mathcal{F}^{-1} S') \star \mathcal{R} f \Big)(x\cos\theta + y\sin\theta, \theta) \, d\theta
\]

Ici :
\begin{itemize}
    \item $f(x,y)$ : coefficient d'atténuation à reconstruire,
    \item $\mathcal{R} f(t,\theta)$ : transformée de Radon (projection à l'angle $\theta$),
    \item $\mathcal{F}^{-1} S'$ : filtre appliqué sur chaque projection,
    \item $\star$ : convolution dans $t$.
\end{itemize}

C'est une \textbf{formule intégrale continue}, dépendante de coordonnées polaires.
\subsection{Discrétisation des coordonnées et des angles}
La discrétisation permet d’adapter le modèle continu aux contraintes physiques des systèmes d’acquisition. Elle transforme les intégrales et variables continues en un ensemble fini de mesures exploitables numériquement. Cette étape conditionne directement la qualité et la résolution de la reconstruction finale. \vspace{7pt}\\
Pour passer au discret :
\begin{enumerate}
    \item On ne mesure que $P$ angles : $\theta_k = k\pi/P$, $k = 0,\dots,P-1$,
    \item On ne mesure que $2M+1$ faisceaux par angle, espacés de $d$ : $t_j = j d, j=-M,\dots,M$,
    \item On obtient donc la \textbf{transformée de Radon discrète} :
    \[
    \mathcal{R}_D f_{j,k} = \mathcal{R} f(t_j, \theta_k).
    \]
\end{enumerate}

\subsection{Convolution et filtrage discrets}
Le filtrage est une étape cruciale de la méthode FBP, car il compense le flou introduit par la rétroprojection simple. Il repose sur l’opération de convolution, qui permet de renforcer les hautes fréquences nécessaires à une bonne résolution spatiale. Sans ce filtrage, la reconstruction serait fortement dégradée.\vspace{7pt}\\
On applique ensuite le filtre sur chaque projection :
\[
h_{j,k} = (\mathcal{F}_D^{-1} \mathcal{S}' \ast \mathcal{R}_D f)_{j,k}
\]

Ici, $\ast$ est la \textbf{convolution discrète} dans $t$ :
\[
(f \ast g)_m = \sum_{n=0}^{N-1} f_n \, g_{(m-n)}.
\]

\subsection{Discrétisation de la rétroprojection}

La rétroprojection assure la redistribution des projections filtrées sur la grille de l’image reconstruite. Elle garantit la cohérence géométrique entre l’espace des mesures et l’espace image. Cette étape nécessite une interpolation, introduisant une approximation contrôlée du modèle continu.\vspace{7pt}\\
La rétroprojection discrète est :
\[
f(x_m,y_n) \approx \frac{1}{2N} \sum_{k=0}^{N-1} h\Big( x_m \cos \frac{k\pi}{N} + y_n \sin \frac{k\pi}{N}, \frac{k\pi}{N} \Big).
\]
Comme les coordonnées cartésiennes $(x_m, y_n)$ ne tombent pas exactement sur les positions $t_j$, on \textbf{interpole} :
\[
h\big(x_m\cos\theta_k + y_n \sin\theta_k, \theta_k\big) \approx \sum_j h_{j,k} \, W\left(\frac{x_m \cos\theta_k + y_n \sin\theta_k - t_j}{d}\right),
\]
où $W$ est la fonction de pondération de l'interpolation (linéaire, spline, etc.).\vspace{7pt}\\
Cela transforme chaque $f(x_m, y_n)$ en \textbf{combinaison linéaire des mesures $h_{j,k}$}.
\subsection{Passage à la forme matricielle linéaire}
La formulation matricielle permet de rassembler toutes les opérations précédentes dans un cadre linéaire unique. Elle constitue un lien direct entre la théorie continue et les algorithmes numériques. Cette écriture est essentielle pour l’analyse et l’extension vers des méthodes itératives.\vspace{7pt}\\
Si on note :
\begin{itemize}
    \item $f$ le vecteur de tous les $f(x_m, y_n)$ sur la grille,
    \item $g$ le vecteur de toutes les mesures projetées filtrées $h_{j,k}$,
    \item $A$ la matrice représentant la \textbf{rétroprojection + interpolation},
\end{itemize}

alors :
\[
f_i = \sum_j A_{ij} \, g_j
\]

On obtient :
\[
\boxed{g = Af} \quad \text{(ou } f = Ag \text{ selon la convention)}
\]

En pratique, $A$ est \textbf{très grande et creuse}, mais la reconstruction se réduit à un produit matriciel.

\begin{definition}[Matrice système $\Phi$ et poids $\varphi_{i,j}$]
Dans le cadre discret, le vecteur de données de projection à faisceau parallèle $\vec{g}$ est modélisé par une somme pondérée sur les pixels traversés par le rayon X :
\begin{equation}
g_i = \sum_{j=1}^{N} \varphi_{i,j} \cdot f_j, \quad \text{où } i = 1, 2, \cdots, M.
\end{equation}
Le coefficient de pondération $\varphi_{i,j}$ de la matrice système $\Phi$ est égal à la longueur d'intersection du $i$-ème rayon à travers le $j$-ème pixel.
\end{definition}

\begin{figure}[H]
    \centering
    \includegraphics[width=0.8\textwidth]{./images/projection à faisceau parallèle.png}
    \caption{Calcul du coefficient de poids $\varphi_{i,j}$ de la matrice système $\Phi$ à partir de la longueur d'intersection du $i$-ème rayon à travers le $j$-ème pixel.}
    \label{fig:phi}
\end{figure}

Le calcul direct de chaque $\varphi_{i,j}$ est coûteux. Pour accélérer la reconstruction, on peut pré-calculer et stocker ces poids, et exploiter les propriétés de symétrie des projections à faisceau parallèle pour réduire le nombre de calculs nécessaires.

\begin{figure}[H]
    \centering
    \includegraphics[width=0.8\textwidth]{./images/projection à faisceau parallèle-2.png}
    \caption{Mesures des rayons-\textbf{X} $a$, $b$, $c$ et $d$ pour des angles de rotation $\alpha$, $90-\alpha$, $90+\alpha$ et $180-\alpha$. Les propriétés de symétrie permettent de déduire les poids d'un rayon à partir d'un autre.}
    \label{fig:mesure-rayons-X}
\end{figure}


\subsection{Synthèse}
Cette synthèse met en évidence la cohérence globale du passage du modèle continu vers une implémentation discrète. Elle permet de comprendre la FBP comme une succession structurée d’opérations linéaires. Cette vision prépare naturellement l’introduction de méthodes de reconstruction plus avancées.\vspace{7pt}\\
Le passage de l'intégrale continue à $g = Af$ se fait en quatre étapes principales :
\begin{enumerate}
    \item \textbf{Échantillonnage discret} des angles et des faisceaux,
    \item \textbf{Filtrage discret} par convolution,
    \item \textbf{Rétroprojection discrète} avec interpolation,
    \item \textbf{Formulation linéaire matricielle}.
\end{enumerate}

\subsubsection*{Conclusion}
Ce chapitre a présenté les principaux outils mathématiques qui fondent la reconstruction d'images tomographiques, en mettant en évidence la transition fondamentale entre les méthodes analytiques classiques et les approches modernes basées sur le \textit{Compressed Sensing}.\vspace{7pt}\\
Les méthodes analytiques, historiquement les premières, reposent sur une formulation mathématique élégante et directe du problème inverse. La transformée de Radon en constitue la pierre angulaire, en modélisant le lien entre l'objet et ses projections. Son inversion, fondée sur le théorème de la coupe centrale et mise en œuvre dans l'algorithme de rétroprojection filtrée (FBP), permet une reconstruction rapide. Ces méthodes, dont la FBP demeure un standard clinique, supposent toutefois des données abondantes, complètes et faiblement bruitées.\vspace{7pt}\\
Cependant, les exigences actuelles de réduction de dose et de temps d'acquisition conduisent à des scénarios sous-échantillonnés, dans lesquels les méthodes analytiques atteignent leurs limites, se traduisant par des artefacts et une forte sensibilité au bruit. C'est dans ce contexte que les méthodes itératives et le cadre du \textit{Compressed Sensing} prennent tout leur sens. En formulant la reconstruction comme un problème inverse régularisé, elles exploitent des connaissances a priori sur l'image, telles que sa parcimonie dans un domaine approprié (ondelettes, gradient), afin de stabiliser l'inversion et d'obtenir des reconstructions de qualité à partir de données limitées. La variation totale (TV) constitue un exemple de régularisation particulièrement adapté, permettant de préserver les contours tout en réduisant le bruit.\vspace{7pt}\\
La transformée de Fourier et l'opération de convolution jouent un rôle transversal essentiel, aussi bien dans l'implémentation du filtrage pour la FBP que dans l'analyse fréquentielle des données. Par ailleurs, la discrétisation des opérateurs mathématiques et la formulation matricielle linéaire du problème, sous la forme
\[
g = H f,
\]
permettent d'établir un lien direct entre la théorie continue et son implémentation numérique, ouvrant ainsi la voie aux algorithmes d'optimisation itératifs.\vspace{7pt}\\
En résumé, ce chapitre dresse un panorama cohérent des approches algorithmiques de la reconstruction tomographique. Il montre que si les méthodes analytiques offrent rapidité et simplicité dans des conditions idéales, les méthodes itératives régularisées, appuyées par la théorie du \textit{Compressed Sensing}, constituent une réponse incontournable aux défis de l'imagerie moderne : reconstruire davantage d'information à partir de moins de données, sans compromettre la qualité diagnostique ni la sécurité du patient. La maîtrise de ces outils mathématiques apparaît ainsi comme un levier essentiel pour le développement des méthodes de reconstruction de demain.

    \chapter{NOTRE MODÈLE}
% ---- TODO -----
% \section{Limites de la méthode analytique}
% ============================================================
\section{Théorie du Compressed Sensing}
% ============================================================

En tomodensitométrie (CT), la réduction du nombre de projections et de la dose de rayonnement
constitue un enjeu majeur de sécurité clinique et de performance opérationnelle. La diminution
de l'exposition aux rayons $\mathbf{X}$ vise à limiter les risques biologiques associés aux
rayonnements ionisants, en particulier dans les contextes d'examens répétés ou pour les
populations sensibles. Toutefois, cette réduction conduit inévitablement à une acquisition
de données incomplètes et bruitées, rendant la reconstruction d'image plus difficile.\vspace{5pt}\\
D'un point de vue mathématique, cette situation se traduit par un problème inverse
sous-déterminé, pour lequel les méthodes analytiques classiques, telles que la
rétroprojection filtrée, deviennent instables ou génèrent des artefacts importants.
Le \emph{Compressed Sensing} (CS) fournit un cadre théorique et algorithmique permettant
d'aborder cette problématique en exploitant des propriétés structurelles des images CT.

\begin{definition}
    Le \emph{compressed sensing} (CS) est un cadre mathématique et algorithmique permettant la
    reconstruction de signaux de grande dimension à partir d'un nombre de mesures
    significativement inférieur à celui requis par les méthodes d'échantillonnage classiques,
    sous réserve que le signal présente une structure de parcimonie adaptée.
\end{definition}

\begin{definition}
    Soit $x \in \mathbb{R}^{n}$ un signal inconnu. On dit que $x$ est
    \(k\)-parcimonieux dans une base (ou un dictionnaire) \(\Psi\)
    (par exemple ondelettes, DCT) si
    \[
        x = \Psi \alpha, \qquad \text{où } \alpha \text{ possède au plus } k \ll n
        \text{ coefficients non nuls}.
    \]
\end{definition}

Dans le cas des images CT, bien que la distribution d'atténuation ne soit pas parcimonieuse
dans le domaine spatial, elle est souvent compressible dans des bases multi-échelles ou via
le gradient de l'image. Cette propriété constitue le fondement de l'application du
compressed sensing à la reconstruction tomographique.

Les mesures acquises lors d'un examen CT peuvent être modélisées par un ensemble de relations
linéaires :
\[
    \mathbf{y} = \mathbf{A}\mathbf{x},
\]
où $\mathbf{A} \in \mathbb{R}^{m \times n}$ représente l'opérateur de projection discrétisé
(assimilable à la transformée de Radon discrète) et $m \ll n$ lorsque le nombre de projections
est réduit.

Contrairement au cadre classique de l'échantillonnage, qui impose un nombre de mesures au
moins égal à la dimension du signal, le compressed sensing montre que
\[
    m \gtrsim k \log(n/k)
\]
peut être suffisant pour une reconstruction stable, sous des conditions appropriées sur
l'opérateur $\mathbf{A}$, telles que l'incohérence ou la propriété d'isométrie restreinte
(\emph{Restricted Isometry Property}, RIP).

\subsection{Le problème inverse en tomodensitométrie}

La reconstruction CT s'inscrit dans le cadre général des problèmes inverses, où l'objectif
est d'estimer une image à partir de mesures indirectes, bruitées et incomplètes. Ce problème
peut être formulé sous la forme :
\begin{equation}
    \mathbf{y} = \mathcal{A}\mathbf{x} + \mathbf{n},
    \label{eq:inverse_problem}
\end{equation}
où :
\begin{itemize}
    \item[-] $\mathbf{x} \in \mathbb{R}^n$ représente la distribution d'atténuation à reconstruire,
    \item[-] $\mathbf{y} \in \mathbb{R}^m$ correspond aux données de projection (sinogramme),
    \item[-] $\mathcal{A}$ modélise le processus de projection CT,
    \item[-] $\mathbf{n}$ représente le bruit de mesure, principalement de nature quantique.
\end{itemize}

Lorsque le nombre de projections est réduit, l'opérateur $\mathcal{A}$ devient non inversible
et le problème est sous-déterminé. Cette situation est inhérente aux stratégies de réduction
de dose et ne peut être évitée sans compromettre la sécurité du patient.

\subsection{Mal-positude et conséquences pratiques}

\begin{definition}
    Un problème est dit \textbf{bien posé} au sens de Hadamard s'il vérifie l'existence,
    l'unicité et la stabilité de la solution. Si l'une de ces conditions n'est pas satisfaite,
    le problème est dit \emph{mal posé}.
\end{definition}

Dans le contexte de la reconstruction CT à faible dose, la condition d'unicité est violée
du fait de la sous-détermination, et la condition de stabilité est fortement compromise par
la présence de bruit. De faibles fluctuations du sinogramme peuvent ainsi engendrer des
artefacts marqués dans l'image reconstruite.

\subsection{Régularisation par parcimonie et Compressed Sensing}

Pour rendre le problème inverse traitable, il est nécessaire d'introduire des informations
a priori sur la solution recherchée. Le compressed sensing propose d'utiliser la parcimonie
ou la compressibilité de l'image CT dans une représentation appropriée comme mécanisme de
régularisation.\vspace{5pt}\\
Cette hypothèse restreint l'ensemble des solutions admissibles et permet de transformer un
problème inverse mal posé en un problème d'optimisation bien conditionné, pour lequel une
solution stable et physiquement plausible peut être obtenue malgré la réduction du nombre
de projections.\vspace{5pt}\\
Jusqu'à présent, le compressed sensing a été présenté comme un cadre
théorique exploitant la parcimonie pour résoudre des problèmes inverses
sous-déterminés. En pratique, cette hypothèse de parcimonie est intégrée
au processus de reconstruction via des formulations variationnelles.
Ces formulations constituent un cadre général permettant d'unifier les
approches classiques de régularisation et les méthodes issues du
compressed sensing.

\subsection{Formulation variationnelle des problèmes inverses}
% ============================================================================================

Dans de nombreux problèmes d'imagerie, et en particulier en tomodensitométrie
à faible dose, l'objectif est de reconstruire une image inconnue
$\mathbf{x} \in \mathbb{R}^n$ à partir d'un ensemble de mesures
$\mathbf{y} \in \mathbb{R}^m$ obtenues par un système d'acquisition indirect.
Ce processus est généralement modélisé par une relation linéaire de la forme
\[
\mathbf{y} = \mathcal{A}\mathbf{x} + \boldsymbol{\varepsilon},
\]
où $\mathcal{A}$ représente l'opérateur direct du système CT et
$\boldsymbol{\varepsilon}$ un terme de bruit.

Lorsque les données sont bruitées et/ou acquises de manière incomplète
($m \ll n$), l'opérateur $\mathcal{A}$ devient non inversible ou mal conditionné.
Dans ce cas, une inversion directe est soit impossible, soit extrêmement
instable, et de petites perturbations des données peuvent engendrer de fortes
dégradations de la solution reconstruite. Ce phénomène est caractéristique des
problèmes inverses mal posés.

\begin{definition}
Un \emph{problème inverse} consiste à estimer une quantité inconnue
$\mathbf{x}$ à partir d'observations indirectes $\mathbf{y}$, reliées par un
opérateur $\mathcal{A}$, lorsque l'inversion directe de cet opérateur est
impossible ou instable.
\end{definition}

\paragraph{Principe de la régularisation.}
Afin de rendre le problème inverse traitable, il est nécessaire d'introduire
des informations a priori sur la solution recherchée. Cette démarche est
connue sous le nom de \emph{régularisation}.

\begin{definition}
Une régularisation est une application
$\mathfrak{R}_{\alpha} : \mathbb{R}^m \rightarrow \mathbb{R}^n$ qui associe à
des données observées $\mathbf{y}$ une solution stable $\hat{\mathbf{x}}$,
en incorporant des hypothèses supplémentaires sur la structure de la solution.
\end{definition}

Intuitivement, une méthode de régularisation vise à étendre la notion d'inverse
au cadre bruité et mal posé, de sorte que
\[
\mathfrak{R}_{\alpha}(\mathcal{A}\mathbf{x} + \boldsymbol{\varepsilon})
\approx \mathbf{x},
\]
même lorsque $\boldsymbol{\varepsilon} \neq \mathbf{0}$ ou que
$\mathcal{A}$ n'est pas inversible.

\paragraph{Formulation variationnelle.}
Une approche largement utilisée pour implémenter la régularisation consiste à
formuler le problème inverse comme un problème d'optimisation variationnelle,
dans lequel on recherche une solution équilibrant fidélité aux données et
conformité aux a priori. Cette formulation s'écrit généralement sous la forme

\begin{equation}
    \hat{\mathbf{x}} =
    \underset{\mathbf{x} \in \mathbb{R}^n}{\arg\min}
    \left\{
    \underbrace{\left\| \mathcal{A}\mathbf{x} - \mathbf{y} \right\|_{2}^{2}}_{\text{fidélité aux données}}
    + \alpha
    \underbrace{\mathcal{R}(\mathbf{x})}_{\text{terme de régularisation}}
    \right\}.
\end{equation}

Les différents termes de cette formulation jouent des rôles complémentaires :

\begin{itemize}
    \item \textbf{Fidélité aux données :}
    Ce terme impose la cohérence entre l'image reconstruite $\mathbf{x}$ et les
    mesures observées $\mathbf{y}$. Dans un contexte bruité, il n'est pas souhaitable
    de l'annuler strictement, car cela conduirait à une reconstruction amplifiant
    le bruit.

    \item \textbf{Terme de régularisation :}
    Le régularisant $\mathcal{R}(\mathbf{x})$ encode les informations a priori
    disponibles sur la solution recherchée, telles que la régularité, la
    parcimonie ou des contraintes physiques. Il permet de restreindre l'ensemble
    des solutions admissibles et d'améliorer la stabilité du problème.

    \item \textbf{Paramètre de régularisation $\alpha$ :}
    Le paramètre $\alpha > 0$ contrôle le compromis entre fidélité aux données
    et influence de l'a priori. Un choix inadéquat peut conduire soit à une
    reconstruction bruitée (faible $\alpha$), soit à une image excessivement
    lissée (grand $\alpha$).
\end{itemize}

\paragraph{Cas particulier : régularisation de Tikhonov.}
Une régularisation classique consiste à choisir un régularisant quadratique,
conduisant à la régularisation dite de Tikhonov. Par exemple, en supposant que
la solution recherchée soit proche d'un modèle de référence $\boldsymbol{\mu}$,
on peut définir
\[
\mathcal{R}(\mathbf{x})
= \| \mathbf{x} - \boldsymbol{\mu} \|_{L^{2},\mathcal{Q}}^{2}
:= \langle \mathbf{x} - \boldsymbol{\mu},
\mathcal{Q}(\mathbf{x} - \boldsymbol{\mu}) \rangle,
\]
où $\mathcal{Q}$ est un opérateur positif définissant une pondération
directionnelle.

Bien que cette approche soit mathématiquement simple et numériquement stable,
elle favorise des solutions lisses et ne permet pas de promouvoir des structures
parcimonieuses. Dans le contexte de la reconstruction CT à faible dose, elle est
souvent insuffisante pour préserver les contours et les détails fins.

\medskip
\noindent
Le compressed sensing s'inscrit naturellement dans ce cadre variationnel en
choisissant des régularisants non quadratiques conçus pour promouvoir la
parcimonie ou la compressibilité de l'image, tels que les normes $\ell_1$ ou la
variation totale. Ces choix conduisent à des problèmes d'optimisation
non différentiables, nécessitant des algorithmes itératifs spécifiques, qui
seront abordés dans les sections suivantes.

% ============================================================================================
\subsection{Formulation du problème}

Dans le cadre de la tomographie par rayons X (CT), la reconstruction d’image à partir d’un nombre limité de projections conduit à un problème inverse sous-déterminé. Le cadre du \emph{Compressed Sensing} (CS) permet de résoudre ce problème en exploitant la parcimonie intrinsèque des images CT dans un domaine approprié, typiquement le domaine du gradient.

\begin{definition}[Image et représentation parcimonieuse]
Considérons une image $f$, vue comme un vecteur colonne de dimension $n \times 1$ dans $\mathbb{R}^n$, dont les éléments individuels $f_j$, pour $j = 1, 2, \ldots, n$, représentent les $n$ valeurs de pixels de l'image. On développe le vecteur $f$ dans une base orthonormée $\Psi$ comme suit :
\[
f = \Psi \mathbf{x},
\]
où $\Psi$ est la matrice $n \times n$ $[\boldsymbol{\psi}_1, \ldots, \boldsymbol{\psi}_n]$, dont les vecteurs $\{\boldsymbol{\psi}_i\}_{i=1}^{n}$ constituent les colonnes, et où $\mathbf{x}$ est un vecteur colonne de dimension $n \times 1$.  

Si la majorité des composantes du vecteur $\mathbf{x}$ sont nulles ou quasi nulles, on dira que $f$ est \textbf{parcimonieuse} dans le domaine $\Psi$, et que $\mathbf{x}$ constitue sa \textbf{représentation parcimonieuse}.
\end{definition}

Dans le cas des images CT, la parcimonie ne s’exprime généralement pas directement dans le domaine spatial, mais plutôt dans le domaine du gradient. Les images CT sont en effet caractérisées par des régions quasi homogènes séparées par des discontinuités nettes, ce qui rend leur gradient parcimonieux.

Considérons l'exemple du fantôme de Shepp--Logan représenté à la \Cref{fig:shepp-logan} et de son équivalent en gradient à la \Cref{fig:shepp-logan-gradient}. On note l'intensité d'un pixel d'une image bidimensionnelle par $f_{h,w}$, où $h = 1,2,\ldots,H$ et $w = 1,2,\ldots,W$ ; $H$ et $W$ désignent respectivement la hauteur et la largeur de l'image 2D, et $W \times H = n$.

\begin{definition}[Module du gradient]
Si les valeurs des pixels sont notées $f_{h,w}$, le module du gradient discret est défini comme suit :
\begin{equation}
\left| \nabla f_{h,w} \right|
=
\sqrt{
\left( f_{h+1,w} - f_{h,w} \right)^2
+
\left( f_{h,w+1} - f_{h,w} \right)^2
}.
\label{eq:gradient-modulus}
\end{equation}
\end{definition}

La \emph{variation totale} (Total Variation, TV) de l’image est alors définie comme la somme du module du gradient sur l’ensemble des pixels :
\[
\mathrm{TV}(f) = \sum_{h,w} \left| \nabla f_{h,w} \right|.
\]
La minimisation de la variation totale correspond à la minimisation de la norme $\ell_1$ du gradient et constitue une pénalisation standard dans le cadre du Compressed Sensing appliqué au CT.

\begin{figure}[H]
    \centering
    \includegraphics[width=0.8\textwidth]{./images/shepp-logan phantom.png}
    \caption{Fantôme de Shepp--Logan}
    \label{fig:shepp-logan}
\end{figure}

\begin{figure}[H]
    \centering
    \includegraphics[width=0.8\textwidth]{./images/shepp-logan phantom gradient.png}
    \caption{Gradient du fantôme de Shepp--Logan}
    \label{fig:shepp-logan-gradient}
\end{figure}

\begin{proposition}[Modèle d'acquisition en tomographie CT]
En imagerie CT réaliste, les données de projection à faisceau parallèle, également appelées \emph{sinogramme}, sont modélisées par un système linéaire discret :
\begin{equation}
\mathbf{g} = \Phi \mathbf{f},
\end{equation}
où $\mathbf{g} \in \mathbb{R}^m$ est le vecteur des mesures de projection, et $\Phi \in \mathbb{R}^{m \times n}$ est la matrice système décrivant la géométrie d’acquisition CT.
\end{proposition}

En introduisant la représentation parcimonieuse de l’image, le modèle devient :
\begin{equation}
\mathbf{g} = \Phi \mathbf{f} = \Phi \Psi \mathbf{x} = \Phi' \mathbf{x},
\label{eq:4}
\end{equation}
où $\Phi' = \Phi \Psi$.

Lorsque le nombre de projections est limité, on a $m \ll n$, ce qui rend le système sous-déterminé.

\begin{proposition}[Reconstruction CT par Compressed Sensing]
La reconstruction de l’image consiste alors à résoudre le problème d’optimisation suivant :
\begin{equation}
\mathbf{x}
=
\arg\min_{\tilde{\mathbf{x}}}
\left\| \tilde{\mathbf{x}} \right\|_{1}
\quad
\text{sous la contrainte}
\quad
\left\| \Phi^{'}\tilde{\mathbf{x}} - \mathbf{g} \right\|_{2} \leq \varepsilon,
\end{equation}
où $\varepsilon$ modélise le bruit présent dans les mesures.
\end{proposition}

Dans le cas particulier du CT, cette formulation est équivalente à une minimisation de la variation totale de l’image sous contrainte de fidélité aux données.

\subsection{Algorithmes de reconstruction itérative en Compressed Sensing}
\subsubsection{Descente de gradient}

\begin{definition}[Descente de gradient pour la minimisation de la variation totale]
Afin de minimiser la norme $\ell_1$ du gradient (variation totale), une méthode de descente de gradient est employée. La mise à jour de l'image $f$ s'effectue selon :
\begin{equation}
f^{\text{suivant}} = f^{\text{courant}} - \alpha \,\vec{\Delta}^{\,\text{courant}},
\end{equation}
où $\alpha$ est un pas de descente. Le terme $\vec{\Delta}$ correspond au gradient régularisé de la variation totale.
\end{definition}


% =======================================================================================================
% \subsection{Formulation du problème}
% \begin{definition}[Image et représentation parcimonieuse]
% Considérons une image $f$, vue comme un vecteur colonne de dimension $n \times 1$ dans $\mathbb{R}^n$, dont les éléments individuels $f_j$, pour $j = 1, 2, \ldots, n$, représentent les $n$ valeurs de pixels de l'image. On développe le vecteur $f$ dans une base orthonormée $\Psi$ comme suit :
% \[
% f = \Psi \mathbf{x},
% \]
% où $\Psi$ est la matrice $n \times n$ $[\boldsymbol{\psi}_1, \ldots, \boldsymbol{\psi}_n]$, dont les vecteurs $\{\boldsymbol{\psi}_i\}_{i=1}^{n}$ de dimension $n \times 1$ constituent les colonnes, et où $\mathbf{x}$ est également un vecteur colonne de dimension $n \times 1$. Si toutes les composantes du vecteur $\mathbf{x}$, à l'exception de quelques-unes, sont nulles ou quasi nulles, on dira que $f$ est \textbf{parcimonieuse} dans le domaine $\Psi$ et que $\mathbf{x}$ est sa \textbf{représentation parcimonieuse}.
% \end{definition}

% Considérons l'exemple du fantôme de Shepp-Logan représenté à la \Cref{fig:shepp-logan} et de son équivalent en gradient à la \Cref{fig:shepp-logan-gradient}. On note l'intensité d'un pixel d'une image bidimensionnelle par $f_{h,w}$, où $h = 1,2,\ldots,H$ et $w = 1,2,\ldots,W$ ; $H$ et $W$ désignent respectivement la hauteur et la largeur de l'image 2D, et $W \times H = n$.

% \begin{definition}[Module du gradient]
% Si les valeurs des pixels sont notées $f_{h,w}$, le module du gradient est défini comme suit :
% \begin{equation}
% \left| \nabla f_{h,w} \right|
% =
% \sqrt{
% \left( f_{h+1,w} - f_{h,w} \right)^2
% +
% \left( f_{h,w+1} - f_{h,w} \right)^2
% }
% \label{eq:gradient-modulus}
% \end{equation}
% \end{definition}

% \begin{figure}[H]
%     \centering
%     \includegraphics[width=0.8\textwidth]{./images/shepp-logan phantom.png}
%     \caption{Shepp-Logan phantom}
%     \label{fig:shepp-logan}
% \end{figure}

% \begin{figure}[H]
%     \centering
%     \includegraphics[width=0.8\textwidth]{./images/shepp-logan phantom gradient.png}
%     \caption{Shepp-Logan phantom gradient}
%     \label{fig:shepp-logan-gradient}
% \end{figure}

% \begin{proposition}[Modèle d'acquisition en tomographie]
% En imagerie CT réaliste, on suppose que les données de projection à faisceau parallèle échantillonnées de l'image $f$ sont modélisées par un système linéaire discret
% \begin{equation}
% \mathbf{g} = \Phi \mathbf{f},
% \end{equation}
% où le vecteur $\mathbf{g}$ est de longueur $m$, ses mesures individuelles étant notées $g_i$, pour $i = 1,2,\ldots,m$, et où $\Phi$ est la matrice système $m \times n$ produisant l'ensemble discret des mesures de projection pour un balayage à faisceau parallèle de l'objet.
% En substituant $\Psi \mathbf{x}$ à $\mathbf{f}$, ce modèle s'écrit :
% \begin{equation}
% \mathbf{g} = \Phi \mathbf{f} = \Phi \Psi \mathbf{x} = \Phi' \mathbf{x},
% \label{eq:4}
% \end{equation}
% où $\Phi' = \Phi \Psi$ est une matrice de dimension $m \times n$.
% \end{proposition}

% \begin{proposition}[Problème de reconstruction par minimisation $\ell_1$]
% Pour une image parcimonieuse, puisque $m << n$ dans \eqref{eq:4}, il existe une infinité de vecteurs $\tilde{\mathbf{x}}$ satisfaisant $\mathbf{g} = \Phi' \tilde{\mathbf{x}}$. Par conséquent, la reconstruction d'image vise à déterminer le vecteur $\mathbf{x}$ dans le domaine transformé en résolvant le programme d'optimisation suivant :
% \begin{equation}
% \mathbf{x}
% =
% \arg\min_{\tilde{\mathbf{x}}}
% \left\| \tilde{\mathbf{x}} \right\|_{1}
% \quad
% \text{sous la contrainte}
% \quad
% \left| \Phi^{'}\tilde{\mathbf{x}} - \mathbf{g} \right| < \varepsilon,
% \end{equation}
% où $\varepsilon$ est un petit facteur d'erreur tenant compte du bruit dans les mesures, et où la norme $\ell_1$ est définie par $\left\| \mathbf{x} \right\|_{1} = \sum_{i=1}^{N} |x_i|$.
% \end{proposition}

\begin{definition}[Mise à jour par descente de gradient pour la norme $\ell_1$ du gradient]
    Pour minimiser la norme $\ell_{1}$ de l'image de gradient, une méthode de descente de gradient est employée. La mise à jour de l'image $f$ s'effectue itérativement selon :
    \begin{equation}
    f^{\text{suivant}} = f^{\text{courant}} - \alpha \,\vec{\Delta}^{\,\text{courant}},
    \end{equation}
    où $\alpha$ est une constante contrôlant la vitesse de descente. Le terme $\vec{\Delta}$ est une image dont la valeur de chaque pixel $(h,w)$ est donnée par la dérivée partielle de la norme $\ell_1$ du gradient :
    \begin{equation}
        \begin{array}{l l l}
            \nu_{h,w} & = &
            \dfrac{\partial \lVert \nabla f_{h,w} \rVert_{1}}{\partial f_{h,w}} \\[1.2ex]
            & = &
            \dfrac{2f_{h,w} - f_{h+1,w} - f_{h,w+1}}
            {\sqrt{\varepsilon + (f_{h+1,w} - f_{h,w})^{2} + (f_{h,w+1} - f_{h,w})^{2}}} \\[2ex]
            & + &
            \dfrac{f_{h,w} - f_{h-1,w}}
            {\sqrt{\varepsilon + (f_{h,w} - f_{h-1,w})^{2} + (f_{h-1,w+1} - f_{h-1,w})^{2}}} \\[2ex]
            & + &
            \dfrac{f_{h,w} - f_{h,w-1}}
            {\sqrt{\varepsilon + (f_{h+1,w-1} - f_{h,w-1})^{2} + (f_{h,w} - f_{h,w-1})^{2}}}
        \end{array}
    \end{equation}
\end{definition}

\begin{definition}[Matrice système $\Phi$ et poids $\varphi_{i,j}$]
Dans le cadre discret, le vecteur de données de projection à faisceau parallèle $\vec{g}$ est modélisé par une somme pondérée sur les pixels traversés par le rayon X :
\begin{equation}
g_i = \sum_{j=1}^{N} \varphi_{i,j} \cdot f_j, \quad \text{où } i = 1, 2, \cdots, M.
\end{equation}
Le coefficient de pondération $\varphi_{i,j}$ de la matrice système $\Phi$ est égal à la longueur d'intersection du $i$-ème rayon à travers le $j$-ème pixel.
\end{definition}

\begin{figure}[H]
    \centering
    \includegraphics[width=0.8\textwidth]{./images/projection à faisceau parallèle.png}
    \caption{Calcul du coefficient de poids $\varphi_{i,j}$ de la matrice système $\Phi$ à partir de la longueur d'intersection du $i$-ème rayon à travers le $j$-ème pixel.}
    \label{fig:phi}
\end{figure}

Le calcul direct de chaque $\varphi_{i,j}$ est coûteux. Pour accélérer la reconstruction, on peut pré-calculer et stocker ces poids, et exploiter les propriétés de symétrie des projections à faisceau parallèle pour réduire le nombre de calculs nécessaires.

\begin{figure}[H]
    \centering
    \includegraphics[width=0.8\textwidth]{./images/projection à faisceau parallèle-2.png}
    \caption{Mesures des rayons-\textbf{X} $a$, $b$, $c$ et $d$ pour des angles de rotation $\alpha$, $90-\alpha$, $90+\alpha$ et $180-\alpha$. Les propriétés de symétrie permettent de déduire les poids d'un rayon à partir d'un autre.}
    \label{fig:mesure-rayons-X}
\end{figure}


\subsubsection{Pseudo-code}
\begin{algorithm}[H]
\caption{Méthode de reconstruction hybride (SART + Descente de gradient)}
\label{alg:hybrid-reconstruction}
\begin{algorithmic}[1]
\Require $\varphi$ - matrice de projection, $g$ - données d'acquisition,
$M$ - nombre d'itérations SART, $\lambda$ - paramètre de relaxation,
$\alpha$ - pas d'apprentissage
\Ensure $\hat{f}$ - image reconstruite

\Statex \textbf{(1) Initialisation de l'image}
\State $f^{(0)} \gets 0$

\Statex
\Statex \textbf{(2) Processus itératif de type SART}
\For{$k = 1$ \textbf{à} $M$} \Comment{Une période complète d'itération}
    \For{$j = 1$ \textbf{à} $N$}
        \State $f_j^{(k)} \gets f_j^{(k-1)}
        + \lambda
        \cdot
        \frac{
            g_i - \sum_{n=1}^{N} \varphi_{i,n} f_n^{(k-1)}
        }{
            \sum_{n=1}^{N} \varphi_{i,n}^2
        }
        \cdot \varphi_{i,j}$
    \EndFor
\EndFor

\Statex
\Statex \textbf{(3) Initialisation pour la descente de gradient}
\State $\hat{f}^{(0)} \gets f^{(M)}$

\Statex
\Statex \textbf{(4) Descente de gradient (contrainte de parcimonie)}
\For{$l = 1$ \textbf{à} $5$}
    \State $\vec{\Delta}_l \gets
    \left| \hat{f}^{(0)} - f^{(0)} \right|
    \cdot
    \frac{\nu_{x,y}}{\left| \nu_{x,y} \right|}$
    \State $\hat{f}^{(l)} \gets \hat{f}^{(l-1)} - \alpha \cdot \vec{\Delta}_l$
\EndFor

\Statex
\Statex \textbf{(5) Initialisation de l'étape itérative suivante}
\State $f^{(0)} \gets \hat{f}^{(5)}$
\State \Return $\hat{f}^{(5)}$

\end{algorithmic}
\end{algorithm}

% \textbf{(1) Initialisation de l'image \(f\) :}
% \[
% f^{(0)} = 0 ;
% \]

% \medskip
% \noindent
% \textbf{(2) Processus itératif (type SART) :}
% Pour \(k\) variant de \(1\) à \(M\) (une période complète d'itération) :
% \[
% f_j^{(k)} = f_j^{(k-1)} +
% \lambda \,
% \frac{
% g_i - \sum_{n=1}^{N} \varphi_{i,n}\, f_n^{(k-1)}
% }{
% \sum_{i=1}^{N} \varphi_{i,n}^{2}
% }
% \, \varphi_{i,j} ;
% \]
% où le paramètre de relaxation \(\lambda\) est un nombre réel positif.

% \medskip
% \noindent
% \textbf{(3) Initialisation de l'image pour la descente de gradient :}
% \[
% \hat{f}^{(0)} = f^{(M)} ;
% \]

% \medskip
% \noindent
% \textbf{(4) Itération de descente de gradient (contrainte de parcimonie) :}
% Pour \(l = 1\) jusqu'à \(5\) :
% \[
%     \hat{f}^{(l)} = \hat{f}^{(l-1)} - \alpha \cdot \vec{\Delta}_l ,
% \]
% avec
% \[
%     \vec{\Delta}_l =
%     \left| \hat{f}^{(0)} - f^{(0)} \right|
%     \cdot
%     \frac{\nu_{x,y}}{\left| \nu_{x,y} \right|}.
% \]

% \medskip
% \noindent
% \textbf{(5) Initialisation de l'étape itérative suivante :}
% \[
% f^{(0)} = \hat{f}^{(\text{end})} ;
% \]

Les étapes (2) à (5) sont répétées jusqu'à ce que la différence entre deux images successives $f^{(M)}$ soit inférieure à un seuil (e.g., $0.001$) ou que le nombre d'itérations dépasse une limite (e.g., $1000$). Les paramètres typiques sont $\lambda = 1.0$, $\varepsilon = 0.0001$, $\alpha = 0.5$.
% =======================================================================================================



\subsection{Métriques de performance}
\begin{definition}[Métriques de similarité d'image]
Soient $f_r$ et $f_o$ les vecteurs représentant respectivement l'image reconstruite et l'image originale, composées de $N$ pixels. On définit les métriques suivantes :
\begin{itemize}
    \item \textbf{Erreur quadratique moyenne (RMSE)} :
    $\displaystyle \mathrm{RMSE} = \sqrt{\frac{\sum_{i=1}^{N} \left( f_{r_i} - f_{o_i} \right)^2}{N}}$
    \item \textbf{Indice universel de qualité (UQI)} :
    $\displaystyle \mathrm{UQI} =
    \frac{2\,\mathrm{Cov}\{f_r,f_o\}}{D(f_r)+D(f_o)}
    \cdot
    \frac{2\,\bar{f}_r\,\bar{f}_o}{\bar{f}_r^{\,2}+\bar{f}_o^{\,2}}$
    \item \textbf{Coefficient de corrélation (CC)} :
    $\displaystyle \mathrm{CC} =
    \frac{2\,\mathrm{Cov}\{f_r,f_o\}}
    {\sqrt{D(f_r)\cdot D(f_o)}}$
\end{itemize}
avec $\bar{f}_o = \frac{1}{N}\sum_{i=1}^{N} f_{o_i}$, $\bar{f}_r = \frac{1}{N}\sum_{i=1}^{N} f_{r_i}$, $D(f) = \frac{1}{N-1}\sum_{i=1}^{N} \left(f_{i}-\bar{f}\right)^2$,\\ et \\$\mathrm{Cov}\{f_r,f_o\} = \frac{1}{N-1}\sum_{i=1}^{N} \left(f_{r_i}-\bar{f}_r\right) \left(f_{o_i}-\bar{f}_o\right)$.
\end{definition}

% ============================================== DRAFT ==============================================
% Ce cadre permet de résoudre plusieurs limitations pratiques :
% \paragraph{Réduction du nombre de mesures.}\text{}\\ 
% De nombreux systèmes d'acquisition sont limités par le coût, le temps ou l'énergie. Le compressed sensing permet :
% \begin{itemize}
%     \item[-] une acquisition plus rapide des données,
%     \item[-] une réduction de la complexité matérielle,
%     \item[-] une diminution de la dose de radiation (par exemple en tomodensitométrie),
%     \item[-] une réduction des coûts de stockage et de transmission.
% \end{itemize}

% \paragraph{Problèmes inverses mal posés (ill-posed inverse problems).}\text{}\\
% Lorsque le nombre de mesures est insuffisant pour garantir une solution unique, le CS introduit une régularisation fondée sur la parcimonie, permettant une reconstruction stable. Les principales applications incluent :
% \begin{itemize}
%     \item[-] la tomographie (CT, IRM, PET),
%     \item[-] l'imagerie à super-résolution,
%     \item[-] la déconvolution,
%     \item[-] les inversions géophysiques et les essais non destructifs.
% \end{itemize}

% \paragraph{Robustesse au bruit et aux données incomplètes.} \text{}\\
% Le CS garantit une reconstruction stable même en présence de bruit, de corruptions ou d'observations manquantes.

% \subsection{Reconstruction de signaux par Compressed Sensing}

% \subsection{Reconstruction par optimisation}
% La formulation canonique de la reconstruction est
% \[
% \min_{\alpha} \|\alpha\|_{1} \quad \text{s.c.} \quad y = A \Psi \alpha,
% \]
% ou, en présence de bruit,
% \[
% \min_{\alpha} \|\alpha\|_{1} \quad \text{s.c.} \quad \|A \Psi \alpha - y\|_{2} \le \epsilon.
% \]
% Cela correspond aux formulations de type \emph{Basis Pursuit} ou \emph{LASSO}. La minimisation de la norme \(\ell_1\) favorise la parcimonie tout en conservant un problème d'optimisation convexe et calculable efficacement.

% \subsection{Algorithmes gloutons}

% Des alternatives plus rapides incluent :
% \begin{itemize}
%     \item l'\emph{Orthogonal Matching Pursuit} (OMP),
%     \item le \emph{Compressive Sampling Matching Pursuit} (CoSaMP),
%     \item l'\emph{Iterative Hard Thresholding} (IHT).
% \end{itemize}
% Ces méthodes échangent une partie de la précision contre un coût computationnel réduit.

% \subsection{Applications du Compressed Sensing}

% \paragraph{Imagerie médicale.}
% \begin{itemize}
%     \item acquisition IRM accélérée,
%     \item CT à dose réduite,
%     \item échographie à haute cadence d'images.
% \end{itemize}

% \paragraph{Imagerie computationnelle.}
% \begin{itemize}
%     \item caméras à pixel unique,
%     \item imagerie à ouverture codée,
%     \item reconstruction hyperspectrale.
% \end{itemize}

% \paragraph{Télédétection et géophysique.}
% \begin{itemize}
%     \item inversion sismique parcimonieuse,
%     \item imagerie radar et radar à synthèse d'ouverture (SAR).
% \end{itemize}

% \paragraph{Communications sans fil.}
% \begin{itemize}
%     \item estimation parcimonieuse de canaux,
%     \item réduction des pilotes dans les systèmes MIMO massifs.
% \end{itemize}

% \paragraph{Apprentissage automatique et traitement du signal.}
% \begin{itemize}
%     \item régression parcimonieuse (LASSO),
%     \item apprentissage de dictionnaires,
%     \item ACP robuste et modèles de rang faible apparentés.
% \end{itemize}


% % Le compressed sensing (CS) est un cadre mathématique et algorithmique qui permet de reconstruire des signaux de grande dimension à partir d'un nombre de mesures bien inférieur à celui requis par les approches traditionnelles. Il exploite la parcimonie (sparsity) comme principal a priori structurel.
% % \subsection{Hypothèse de parcimonie}
% % Si un signal est parcimonieux ou compressible dans une certaine base, alors il peut être reconstruit exactement (ou avec une erreur contrôlée) à partir d'un nombre de mesures linéaires bien inférieur à sa dimension ambiante.
% % \subsection{Incohérence et propriété de RIP}
% % \subsection{Basis Pursuit et LASSO}
% % \subsection{OMP et algorithmes gloutons}


    \chapter{SIMULATION, DÉVELOPPEMENT ET APPLICATIONS INNOVANTES EN RECONSTRUCTION D'IMAGES CT}
\section{Collecte des données}
\subsection{Transformée de Radon et de la Génération de Sinogrammes}
La génération d'un sinogramme \Ref{fig:sinogram_generation} constitue une étape centrale en tomographie, notamment en imagerie médicale et industrielle. Elle repose sur un outil mathématique fondamental appelé transformée de Radon, qui permet de représenter une image bidimensionnelle sous forme d'un ensemble de projections unidimensionnelles. Ce processus consiste à mesurer l'intégrale de l'intensité de l'image le long de droites orientées selon différents angles, simulant ainsi le principe physique de l'acquisition tomographique. Cette transformation permet de passer de l'espace spatial de l'image à un espace de projections, appelé espace de Radon, qui contient l'information nécessaire à la reconstruction de l'image originale. L'implémentation numérique complète de ce processus, incluant la discrétisation, le calcul des projections et la construction du sinogramme, est présentée dans le code fourni en Annexe A.\vspace{5pt}\\
La transformée de Radon d'une fonction continue $f(x,y)$, représentant l'intensité de l'image, est définie comme l'intégrale de cette fonction le long d'une droite caractérisée par un angle $\theta$ et une distance $s$ par rapport à l'origine. Mathématiquement, elle s'écrit :

$$\mathcal{R}f(\theta, s) = \int_{-\infty}^{\infty} \int_{-\infty}^{\infty} f(x,y),\delta(x\cos\theta + y\sin\theta - s),dx,dy$$

où $\delta$ désigne la fonction delta de Dirac, qui permet de sélectionner uniquement les points appartenant à la droite considérée. Cette formulation exprime le fait que chaque valeur de la transformée correspond à une projection de l'image selon une direction donnée. D'un point de vue géométrique, pour un angle fixé, la transformée calcule la somme des intensités de tous les points situés le long de droites parallèles. En faisant varier l'angle, on obtient un ensemble complet de projections décrivant l'image sous différents points de vue.

\begin{figure}[H]
    \centering
    
    \begin{subfigure}[b]{0.32\textwidth}
        \centering
        \includegraphics[width=\textwidth]{images/Radon-transform-of-projections-at-different-angles2.png}
        % \caption{Image 1}
        % \label{fig:img1}
    \end{subfigure}
    \hfill
    \begin{subfigure}[b]{0.32\textwidth}
        \centering
        \includegraphics[width=\textwidth]{images/A-2-D-illustration-of-Radon-transform-geometry-On-the-right-Radon-transform-of-a-pixel.png}
        % \caption{Image 2}
        % \label{fig:img2}
    \end{subfigure}
    \hfill
    \begin{subfigure}[b]{0.32\textwidth}
        \centering
        \includegraphics[width=\textwidth]{images/Radon_transform.png}
        % \caption{Image 3}
        % \label{fig:img3}
    \end{subfigure}
    
    \caption{Processus de transformation de Radon}
    % \label{fig:three_images}
    
\end{figure}

\begin{figure}[H]
    \centering
    \includegraphics[width=1\textwidth, height=0.6\textwidth]{images/creation_de_sinogramme.png}
    \caption{Diagramme de génération d'un sinogramme}
    \label{fig:sinogram_generation}
\end{figure}

Dans la pratique, cette transformation continue doit être discrétisée afin d'être implémentée numériquement. Les angles de projection sont échantillonnés uniformément dans l'intervalle $[0,180^\circ[$ selon la relation :
$$\theta_k = \frac{180^\circ}{N} \cdot k, \quad k = 0, 1, ..., N-1$$
où $N$ représente le nombre total de projections. Pour chaque angle discret, l'image est projetée sur un ensemble de détecteurs qui échantillonnent la variable spatiale $s$. Le résultat de ce processus est organisé sous forme d'une matrice appelée sinogramme, notée $S \in \mathbb{R}^{M \times N}$, où $M$ correspond au nombre de détecteurs et $N$ au nombre d'angles. Chaque colonne du sinogramme contient la projection de l'image pour un angle donné, tandis que chaque ligne correspond à la variation des projections pour un détecteur donné lorsque l'angle change. L'algorithme détaillé de cette discrétisation ainsi que la structure matricielle utilisée pour stocker les données sont explicitement décrits dans le code de l'Annexe A, afin d'assurer la reproductibilité des résultats présentés.

\begin{figure}[H]
    \centering
    
    \begin{subfigure}[b]{0.32\textwidth}
        \centering
        \includegraphics[width=\textwidth]{images/A-Shepp-Logan-Phantom-and-reconstructed-Image-Sinogram-a-Original-image-b-radon.png}
        % \caption{Image 1}
        % \label{fig:img1}
    \end{subfigure}
    \hfill
    \begin{subfigure}[b]{0.32\textwidth}
        \centering
        \includegraphics[width=\textwidth]{images/Examples-of-reconstructed-images-and-sinograms-with-different-labels-for-a-body-part.png}
        % \caption{Image 2}
        % \label{fig:img2}
    \end{subfigure}
    \hfill
    \begin{subfigure}[b]{0.32\textwidth}
        \centering
        \includegraphics[width=\textwidth]{images/Examples-of-reconstructed-images-and-sinograms-with-different-labels-for-a-body-part.png}
        % \caption{Image 3}
        % \label{fig:img3}
    \end{subfigure}
    
    \caption{Images-Sinogrammes pairs}
    \label{fig:three_images}
    
\end{figure}

Dans les systèmes réels, les mesures sont inévitablement affectées par le bruit, provenant principalement des limitations électroniques des détecteurs, des fluctuations statistiques du signal et des perturbations environnementales. Ce bruit est généralement modélisé comme un bruit additif gaussien, ce qui conduit à l'expression :

$$S_{\text{bruité}}(s, \theta) = S(s, \theta) + \eta(s, \theta)$$

où $\eta(s,\theta)$ est une variable aléatoire suivant une distribution normale de moyenne nulle et de variance $\sigma^2$. Cette modélisation permet de reproduire de manière réaliste les conditions expérimentales et constitue une étape essentielle pour l'évaluation des algorithmes de reconstruction. Le mécanisme d'ajout de bruit, paramétré par un niveau de variance contrôlé, est également intégré dans l'implémentation fournie en Annexe A.

La transformée de Radon possède plusieurs propriétés mathématiques fondamentales qui expliquent son importance en tomographie. Elle est linéaire, ce qui signifie que la projection d'une combinaison linéaire de fonctions est égale à la combinaison linéaire de leurs projections. Elle présente également une propriété de symétrie qui réduit la plage angulaire nécessaire à l'acquisition. Plus important encore, elle est étroitement liée à la transformée de Fourier par le théorème de la coupe centrale, selon lequel la transformée de Fourier d'une projection correspond à une coupe radiale de la transformée de Fourier bidimensionnelle de l'image. Cette propriété constitue le fondement théorique de nombreux algorithmes de reconstruction, notamment la rétroprojection filtrée.

D'un point de vue physique, la génération d'un sinogramme correspond à la mesure de l'atténuation d'un rayonnement traversant un objet selon différentes directions. Chaque projection représente l'accumulation des interactions du rayonnement avec la matière le long de sa trajectoire. Cette représentation permet de transformer un problème de reconstruction bidimensionnelle en un ensemble structuré de mesures unidimensionnelles, facilitant ainsi l'analyse, la modélisation et le traitement numérique.

\subsection{Sous-Échantillonnage Angulaire en Tomographie pour le Compressed Sensing}

L'introduction de ce rapport présente le sous-échantillonnage angulaire comme une technique centrale en tomographie moderne, notamment dans le cadre du Compressed Sensing (CS). Cette approche permet de réduire la dose d'irradiation lors des acquisitions médicales tout en conservant une qualité d'image satisfaisante, grâce à des algorithmes de reconstruction avancés. Elle répond aux besoins croissants de sécurité des patients et d'acquisition rapide, tout en exploitant les propriétés structurelles des images médicales.\vspace{7pt}\\

En tomographie conventionnelle, le critère de Nyquist définit le nombre minimal d'angles nécessaires pour éviter les artefacts de repliement spectral dans le sinogramme. Formellement, on a :
\[
N_{\text{angles}} \geq \frac{\pi}{2} \cdot N_{\text{pixels}}
\]
Le sous-échantillonnage consiste à acquérir volontairement moins d'angles que cette limite. Ainsi, le nombre d'angles sous-échantillonnés s'écrit :
\[
N_{\text{sous-échantillonné}} = \alpha \cdot N_{\text{complet}}
\]
où \(\alpha \in [0,1]\) représente le facteur de sous-échantillonnage. Cette réduction entraîne une perte d'information, mais le Compressed Sensing permet de compenser cette limitation grâce à des hypothèses de parcimonie sur l'image.\vspace{7pt}\\

La sélection des angles pour le sous-échantillonnage suit généralement un processus aléatoire uniforme, sans remise, afin de garantir l'incohérence nécessaire au CS :
\[
\Theta_{\text{sous}} = \{\theta_{i_1}, \theta_{i_2}, ..., \theta_{i_M}\}
\]
où \(\{i_1, i_2, ..., i_M\}\) est un sous-ensemble aléatoire de \(\{0, 1, ..., N-1\}\), et \(M = \lfloor \alpha \cdot N \rfloor\) correspond au nombre d'angles acquis. Les indices sont triés pour préserver l'ordre angulaire, ce qui facilite l'interprétation du sinogramme (Voir Annexe A pour l'implémentation).\vspace{7pt}\\

Le Compressed Sensing repose sur trois concepts fondamentaux : la parcimonie, qui veut que l'image soit parcimonieuse dans un domaine de transformation \(\Psi\) (c'est-à-dire que seule une fraction des coefficients est significative) ; l'incohérence, qui exige que la matrice de mesure \(\Phi\) soit incohérente par rapport à la base dans laquelle l'image est parcimonieuse ; et la reconstruction non-linéaire, où le signal est reconstruit via une optimisation qui favorise la parcimonie, typiquement par la minimisation de la norme \(L_1\).  

Pour une image \(x \in \mathbb{R}^n\), le sinogramme sous-échantillonné \(y\) est obtenu par :
\[
y = \Phi x + \epsilon
\]
où \(y \in \mathbb{R}^m\) est le sinogramme (\(m \ll n\)), \(\Phi \in \mathbb{R}^{m \times n}\) représente l'opérateur de mesure (transformée de Radon partielle), et \(\epsilon\) est le bruit de mesure. Cette formulation traduit un problème inverse sous-déterminé.\vspace{7pt}\\

Pour garantir une reconstruction fiable, la matrice de mesure \(\Phi\) doit respecter la propriété d'isométrie restreinte (RIP) :
\[
(1-\delta_k)\|x\|_2^2 \leq \|\Phi x\|_2^2 \leq (1+\delta_k)\|x\|_2^2
\]
pour tous les vecteurs \(k\)-parcimonieux \(x\), avec \(\delta_k \in [0,1)\). Cette condition garantit que les distances entre vecteurs parcimonieux sont approximativement conservées après projection.\vspace{7pt}\\

Le théorème de la coupe centrale (Fourier Slice Theorem) établit que la transformée de Radon d'une image correspond à des lignes radiales dans son espace de Fourier. Ainsi, le sous-échantillonnage angulaire entraîne un sous-échantillonnage radial :
\[
\mathcal{F}_{1D}[\mathcal{R}f(\theta, s)](\omega) = \mathcal{F}_{2D}[f](\omega \cos\theta, \omega \sin\theta)
\]
En pratique, seules certaines lignes radiales sont acquises, ce qui crée des zones vides dans le plan de Fourier (Voir Annexe A pour l'implémentation).\vspace{7pt}\\

L'opérateur de Radon partiel \(\mathcal{R}_{\Theta}\) ne conserve qu'un sous-ensemble des angles :
\[
\mathcal{R}_{\Theta}f(s) = \{\mathcal{R}f(\theta, s) : \theta \in \Theta_{\text{sous}}\}
\]
On obtient ainsi un problème inverse sous-déterminé :
\[
y = \mathcal{R}_{\Theta}x
\]
Le taux de sous-échantillonnage est simplement le ratio entre le nombre d'angles acquis et le nombre total d'angles complets :
\[
\text{Taux} = \frac{M}{N} = \alpha
\]
où \(M\) est le nombre d'angles réellement acquis.\vspace{7pt}\\

La reconstruction consiste à résoudre l'optimisation :
\[
\hat{x} = \arg\min_{x} \|\Psi x\|_1 \quad \text{sous contrainte} \quad \|\mathcal{R}_{\Theta}x - y\|_2 \leq \epsilon
\]
où \(\Psi\) est un opérateur de transformation vers un domaine parcimonieux (i.e Bi-orthogonal Wavelet Transform, gradient, DCT), \(\|\cdot\|_1\) favorise la parcimonie, et \(\epsilon\) représente le niveau de bruit (Voir Annexe A pour l'implémentation).  

Dans la plupart des applications, la parcimonie est obtenue dans le domaine des ondelettes :
\[
x = \Psi c
\]
avec \(\|c\|_0 \ll n\) où \(\|c\|_0\) est le nombre de coefficients significatifs.\vspace{7pt}\\

Le pipeline complet (Voir Figure \ref{fig:sinogram_undersampling_pipeline}) du processus peut être visualisé comme suit : l'image originale \(x\) subit une transformée de Radon complète, puis un sous-échantillonnage angulaire aléatoire basé sur le facteur \(\alpha\) pour produire un sinogramme sous-échantillonné \(y\). Ce sinogramme est ensuite traité par un algorithme de reconstruction CS, qui utilise la minimisation L1 dans un domaine de parcimonie \(\Psi\) pour finalement produire l'image reconstruite \(\hat{x}\) (Voir Annexe A pour l'implémentation).

\begin{figure}[H]
    \centering
    \includegraphics[width=1\textwidth, height=1.2\textwidth]{images/undersample_sinogram.png}
    \caption{Pipeline de sous-échantillonnage angulaire}
    \label{fig:sinogram_undersampling_pipeline}
\end{figure}

Le nombre minimal d'angles pour une reconstruction fidèle est donné par :
\[
M \geq C \cdot k \cdot \log\left(\frac{n}{k}\right)
\]
où \(k\) est la parcimonie de l'image, \(n\) sa dimension et \(C\) une constante dépendant de l'opérateur de mesure. L'erreur de reconstruction est bornée par :
\[
\|\hat{x} - x\|_2 \leq C_1 \frac{\|x - x_k\|_1}{\sqrt{k}} + C_2 \epsilon
\]
avec \(x_k\) la meilleure approximation \(k\)-parcimonieuse de \(x\). Le processus reste robuste au bruit tant que :
\[
\frac{\|\epsilon\|_2}{\|y\|_2} \ll \frac{1}{\sqrt{\log n}}.
\]\vspace{7pt}\\

Dans les applications cliniques, le sous-échantillonnage angulaire permet de diminuer la dose d'irradiation proportionnellement au facteur \(\alpha\) :
\[
\text{Dose}_{\text{réduite}} = \alpha \cdot \text{Dose}_{\text{complète}}
\]
En imagerie cardiaque, l'acquisition sous-échantillonnée peut être synchronisée avec le cycle cardiaque :
\[
\Theta_{\text{cardio}} = \{\theta \in [0,180] : \text{phase cardiaque} = \text{cible}\}
\]
Pour l'imagerie dynamique (tomographie 4D), les angles peuvent varier dans le temps :
\[
y(t) = \mathcal{R}_{\Theta(t)}x(t)
\]



\subsection{Opérateur de Transformée en Ondelettes}

La transformation par ondelettes est un outil fondamental en traitement d'images et en optimisation, permettant de représenter une image dans une base multi-résolution. Elle établit une correspondance linéaire entre l'espace des pixels et l'espace des coefficients d'ondelettes. Le diagramme présenté à la Figure \ref{fig:wt_creation} illustre la construction de cet opérateur ainsi que ses transformations directe et adjointe. L'implémentation correspondante est fournie en Annexe~B.

\begin{figure}[H]
    \centering
    \includegraphics[width=1\textwidth, height=1.2\textwidth]{images/wt_creation.png}
    \caption{Opérateur de Transformée en Ondelettes}
    \label{fig:wt_creation}
\end{figure}

Mathématiquement, une image discrète de taille \(H \times W\) peut être représentée sous forme vectorielle \(x \in \mathbb{R}^n\), où \(n = H \times W\). L'opérateur de transformation en ondelettes est défini comme un opérateur linéaire :

\[
W : \mathbb{R}^n \rightarrow \mathbb{R}^m
\]

qui associe à chaque image \(x\) un vecteur de coefficients d'ondelettes :

\[
w = W x
\]

Ces coefficients représentent l'image dans une base multi-échelle composée de :

\[
w = \left\{ a_J, d_J^H, d_J^V, d_J^D, \dots, d_1^H, d_1^V, d_1^D \right\}
\]

où :

\begin{itemize}
\item \(a_J\) représente les coefficients d'approximation à basse résolution,
\item \(d_j^H, d_j^V, d_j^D\) représentent respectivement les coefficients de détail horizontal, vertical et diagonal au niveau \(j\).
\end{itemize}

Cette transformation correspond à une projection de l'image sur une base d'ondelettes :

\[
w_i = \langle x, \psi_i \rangle
\]

où \(\psi_i\) représente les fonctions de base d'ondelettes.

L'opérateur adjoint \(W^T\) permet la reconstruction de l'image à partir des coefficients :

\[
x = W^T w
\]

Dans le cas d'ondelettes orthogonales, on a la propriété importante :

\[
W^T = W^{-1}
\]

et donc :

\[
W^T W = I
\]

où \(I\) est l'opérateur identité.

Dans le cas général, \(W \in \mathbb{R}^{m \times n}\), avec :

\[
n = H \times W
\]
\[
m = \text{nombre total de coefficients}
\]

La transformation directe correspond à l'application :

\[
x \mapsto W x
\]

tandis que la transformation adjointe correspond à :

\[
w \mapsto W^T w
\]

Le diagramme de la Figure~\ref{fig:wt_creation} montre les différentes étapes de construction de cet opérateur :

\begin{itemize}
\item la définition de l'espace image \(\mathbb{R}^n\),
\item la construction de la structure des coefficients,
\item la définition de l'opérateur direct \(W\),
\item la définition de l'opérateur adjoint \(W^T\),
\item la création de l'opérateur linéaire complet.
\end{itemize}

Les coefficients sont organisés sous forme hiérarchique, permettant une représentation multi-échelle efficace. Pour assurer la reconstruction correcte, deux éléments essentiels sont conservés :

\begin{itemize}
\item la structure des coefficients,
\item les dimensions associées à chaque niveau de décomposition.
\end{itemize}

Cet opérateur possède plusieurs propriétés importantes :

\begin{itemize}
\item Linéarité :
\[
W(ax + by) = aWx + bWy
\]

\item Structure matricielle implicite :
\[
W \in \mathbb{R}^{m \times n}
\]

\item Adjoint bien défini :
\[
\langle Wx, w \rangle = \langle x, W^T w \rangle
\]

\item Conservation de l'énergie pour les bases orthogonales :
\[
\|Wx\|_2 = \|x\|_2
\]
\end{itemize}

En pratique, l'opérateur n'est pas construit explicitement sous forme matricielle, mais implémenté comme un opérateur linéaire implicite, comme illustré dans l'Annexe~B, afin d'éviter le stockage d'une matrice potentiellement très grande.

Cette structure permet d'intégrer efficacement la transformation en ondelettes dans des problèmes inverses, tels que :

\begin{itemize}
\item la reconstruction d'images,
\item le débruitage,
\item la compression,
\item les méthodes d'optimisation régularisées.
\end{itemize}
La transformation fournit ainsi une représentation compacte et multi-résolution de l'image, facilitant l'analyse et la manipulation des structures à différentes échelles.

\subsection{Reconstruction d'Images en Tomographie par Transformée de Radon}
La tomographie est une technique d'imagerie qui permet de visualiser l'intérieur d'un objet en trois dimensions sans avoir à le découper physiquement. Elle est largement utilisée en imagerie médicale, notamment dans les scanners à rayons X, mais aussi dans le contrôle non destructif des matériaux et en géophysique. Le principe fondamental de la tomographie repose sur l'acquisition d'un ensemble de projections de l'objet sous différents angles. Une projection peut être vue comme une ombre de l'objet, obtenue en mesurant l'atténuation d'un rayonnement qui traverse cet objet. Lorsque ces projections sont collectées pour de nombreux angles, elles forment un ensemble de données appelé sinogramme. Le défi principal consiste alors à reconstruire l'image originale de l'objet à partir de ce sinogramme.

La transformée de Radon est l'outil mathématique qui modélise ce processus d'acquisition des projections, et correspond à ce que l'on appelle la projection avant. Concrètement, si l'on considère une image bidimensionnelle représentant une coupe de l'objet, la transformée de Radon calcule, pour chaque angle donné, l'intégrale des valeurs de l'image le long de droites parallèles correspondant aux trajectoires des rayons. Chaque intégrale constitue un point dans la projection associée à cet angle. En répétant ce processus pour un ensemble d'angles, on obtient un sinogramme, dont l'axe horizontal représente la position des rayons sur le détecteur et l'axe vertical représente les angles de projection. D'un point de vue algébrique, cette opération peut être modélisée comme l'application d'un opérateur linéaire A qui transforme un vecteur x représentant l'image en un vecteur y représentant le sinogramme, selon la relation $y = A x$. Une implémentation pratique de cet opérateur sous forme sans matrice, utilisant la transformée de Radon, est présentée en Annexe C.

Pour reconstruire l'image originale à partir du sinogramme, il faut effectuer une opération inverse. Une première approche consiste à utiliser la rétroprojection, appelée aussi backprojection. Cette opération consiste à redistribuer les valeurs du sinogramme dans l'espace de l'image en suivant les mêmes trajectoires que celles empruntées par les rayons lors de l'acquisition. En pratique, chaque projection est projetée en retour sur l'image, et la somme de toutes ces contributions fournit une estimation de l'image originale. Mathématiquement, cette opération correspond à l'application de l'opérateur adjoint $A^T$ au sinogramme, ce qui donne une estimation de l'image selon la relation $x_{estimé} = A^T y$. Il est important de noter que l'opérateur adjoint n'est pas l'inverse exact de l'opérateur de projection, mais seulement son adjoint. Une implémentation de cet opérateur adjoint basée sur la rétroprojection non filtrée est également fournie en Annexe C.

Cependant, la rétroprojection simple produit généralement une image floue. Ce flou apparaît parce que le processus de projection favorise les basses fréquences de l'image, c'est-à-dire les variations lentes d'intensité, tandis que les hautes fréquences, correspondant aux détails fins et aux contours, sont atténuées. Lorsque l'on applique uniquement la rétroprojection simple, ce déséquilibre est amplifié, ce qui donne une image aux contours flous. Pour corriger ce problème, on utilise la rétroprojection filtrée, qui est la méthode standard en tomographie. Cette méthode consiste d'abord à appliquer un filtre aux projections du sinogramme, généralement un filtre passe-haut comme le filtre de Ram-Lak, afin de compenser la perte des hautes fréquences et de restaurer les détails. Ensuite, les projections filtrées sont rétroprojetées pour reconstruire l'image.
\begin{figure}[H]
\centering
\includegraphics[width=1\textwidth]{images/operateur_et_adjoint_CT_reconstruction.png}
\caption{Opérateur $A$ et son Adjoint $A^T$ pour la Reconstruction CT}
\end{figure}
Dans les applications réelles, la taille des images et des sinogrammes est très grande, ce qui rend impossible la construction explicite de la matrice $A$ représentant l'opérateur de Radon, car elle contiendrait des millions de lignes et de colonnes. Pour résoudre ce problème, on utilise une approche dite sans matrice, ou matrix-free. Dans cette approche, on ne construit pas la matrice elle-même, mais on définit uniquement deux fonctions : l'une qui calcule la projection avant d'une image, correspondant à l'application de $A x$, et l'autre qui calcule la rétroprojection d'un sinogramme, correspondant à l'application de $A^T$ y. Ces fonctions agissent comme une boîte noire qui encapsule le comportement de l'opérateur et de son adjoint, permettant de manipuler efficacement des problèmes de très grande taille sans stocker explicitement la matrice. Une implémentation complète de cet opérateur linéaire sous forme de LinearOperator est présentée en Annexe C.

Ainsi, la transformée de Radon modélise le processus d'acquisition des données, la rétroprojection simple correspond à l'opérateur adjoint et produit une image floue, tandis que la rétroprojection filtrée constitue la méthode standard de reconstruction en combinant filtrage et rétroprojection. Enfin, l'approche sans matrice permet de mettre en œuvre ces opérations de manière efficace, même pour des systèmes de très grande dimension, en évitant les limitations liées à la mémoire, comme illustré dans l'implémentation fournie en Annexe C.

\subsection{La Reconstruction Tomographique par Séparation de Variables (ADMM)}
La tomographie constitue un outil fondamental en imagerie médicale et dans diverses applications industrielles, permettant de reconstruire des structures internes à partir de mesures de projections acquises sous différents angles. La transformation de ces projections, regroupées sous forme de sinogramme, en une image cohérente représente un problème inverse complexe, particulièrement lorsque les données sont incomplètes, bruitées ou limitées en nombre. Les méthodes analytiques classiques atteignent rapidement leurs limites dans ces conditions, ce qui rend les approches itératives indispensables pour obtenir des reconstructions fiables.

Les méthodes itératives reposent sur un principe conceptuel simple mais efficace : partir d'une estimation initiale de l'image et l'affiner progressivement en comparant les projections simulées avec les mesures expérimentales. L'objectif est de déterminer l'image qui minimise la différence entre ses projections et les données observées tout en respectant certaines contraintes structurelles. Cette démarche s'inscrit naturellement dans un cadre d'optimisation, où l'on cherche à concilier fidélité aux données et régularisation, afin de garantir la stabilité et la qualité de la reconstruction. L'implémentation complète de cet algorithme itératif est présentée en Annexe D, offrant un exemple concret de mise en œuvre dans le contexte tomographique.

Parmi les approches itératives, la méthode de séparation de variables par directions alternées, connue sous le nom d'ADMM (Alternating Direction Method of Multipliers), constitue une solution particulièrement élégante et efficace. L'algorithme décompose un problème d'optimisation complexe en sous-problèmes plus simples, résolus de manière alternée. Dans le cadre de la reconstruction tomographique, cette séparation permet de traiter distinctement la fidélité aux données et la régularisation par parcimonie dans un domaine transformé, tel que les ondelettes. La mise en œuvre de ces mécanismes, incluant la mise à jour des variables primales et duales ainsi que la gestion de la représentation parcimonieuse, est explicitement détaillée dans l'implémentation fournie en Annexe D.

Le fonctionnement de l'algorithme repose sur trois étapes principales, illustrées dans la Figure~\ref{fig:ADMM_diagram_algorithm}, qui présente le flux complet de reconstruction par ADMM. La première consiste à mettre à jour l'image, en résolvant un système linéaire intégrant à la fois l'opérateur de projection et son adjoint, ainsi que l'opérateur de transformation. Cette étape garantit que l'image reconstruite est cohérente avec les mesures tout en restant proche d'une représentation transformée contrôlée. La deuxième étape consiste en la mise à jour de la représentation parcimonieuse, qui applique un seuillage doux afin de favoriser les coefficients significatifs et de réduire l'effet du bruit. La troisième étape met à jour la variable duale, qui assure la cohérence entre l'image reconstruite et sa représentation parcimonieuse, jouant un rôle crucial dans la stabilité et la convergence de l'algorithme. Toutes ces étapes sont implémentées et illustrées dans l'Annexe D.

La parcimonie constitue un a priori majeur en reconstruction d'images. En effet, les images naturelles et médicales possèdent une structure particulière qui permet de représenter l'information de manière concise dans un domaine approprié, tel que celui des ondelettes. L'encouragement de la parcimonie lors de la reconstruction conduit à des images plus nettes, avec des contours préservés, moins bruitées et capables d'être reconstruites de manière fiable même à partir d'un nombre limité de projections. Le choix du paramètre régulant l'équilibre entre fidélité aux données et régularisation parcimonieuse est déterminant pour obtenir des reconstructions à la fois précises et stables.

\begin{figure}[H]
\centering
\includegraphics[width=\textwidth]{images/admm_diagram.png}
\caption{Flux d'exécution de la reconstruction tomographique par ADMM}
\label{fig:ADMM_diagram_algorithm}
\end{figure}

Enfin, la convergence de l'algorithme est évaluée à l'aide de critères basés sur les résidus primal et dual, garantissant que les itérations successives atteignent un état stable et que l'image obtenue respecte à la fois les mesures et la parcimonie attendue. Cette approche itérative robuste et flexible est applicable dans de nombreux domaines, tels que la tomographie médicale (scanners et radiographies), le contrôle non destructif industriel, l'imagerie sismique et la microscopie électronique.

\section{Evaluation}
\section{Discussion}
\section{Applications}
\section{Innovation} % NOTRE MODELE
    \appendix
% \stepcounter{chapter}
\annexechapter{Génération de sinogrammes}

\begin{minted}{python}
from PIL import Image
import numpy as np
from skimage.data import shepp_logan_phantom
from skimage.transform import radon, iradon, resize, iradon_sart, rescale
from skimage.metrics import peak_signal_noise_ratio as psnr
from skimage.metrics import structural_similarity as ssim
\end{minted}

\textbf{Description :}  
Le code ci-dessous montre deux fonctions principales pour la génération de sinogrammes en tomographie.  
Un sinogramme est la représentation de projections d'une image 2D selon différents angles, obtenue via la transformée de Radon. Ces sinogrammes sont utilisés pour reconstruire l'image originale, par exemple en tomographie par rayons X.  

\begin{minted}{python}
    def generate_sinogram(
        image: np.ndarray,
        num_angles: int,
        add_noise: bool = False,
        noise_level: float = 0.05
    ):
        """
        Génère un sinogramme à partir d'une image en utilisant la transformée de Radon.

        Paramètres :
        - image : np.ndarray, image 2D à projeter
        - num_angles : int, nombre de directions de projection (angles uniformes)
        - add_noise : bool, si True, ajoute du bruit gaussien pour simuler des mesures bruitées
        - noise_level : float, amplitude relative du bruit ajouté

        Retour :
        - sinogram : np.ndarray, sinogramme obtenu
        - theta : np.ndarray, angles utilisés pour la projection
        """
        # Crée un vecteur d'angles également espacés de 0 à 180°
        theta = np.linspace(0., 180., num_angles, endpoint=False)

        # Calcule la transformée de Radon (projections de l'image)
        sinogram = radon(image, theta=theta, circle=True)

        # Ajoute du bruit si demandé
        if add_noise:
            noise = np.random.normal(0, noise_level * np.max(sinogram), sinogram.shape)
            sinogram = sinogram + noise

        return sinogram, theta
\end{minted}

\paragraph{Explication :}  
Cette fonction est utilisée pour générer le sinogramme complet d'une image. Elle est utile pour des simulations de reconstruction classique où toutes les projections sont disponibles. Le paramètre add\_noise permet de simuler des mesures réalistes avec bruit, ce qui est important pour tester la robustesse des algorithmes de reconstruction.

\begin{minted}{python}
    def generate_undersampled_sinogram(
        image: np.ndarray,
        undersampling_factor: float = 0.5
    ):
        """
        Génère un sinogramme sous-échantillonné pour la reconstruction par Compressed Sensing.

        Paramètres :
        - image : np.ndarray, image 2D à projeter
        - undersampling_factor : float, fraction des angles à utiliser (0 < factor <= 1)

        Retour :
        - sinogram : np.ndarray, sinogramme sous-échantillonné
        - theta_undersampled : np.ndarray, angles sélectionnés pour la projection
        - selected_indices : np.ndarray, indices des angles sélectionnés
        """
        # Ensemble complet d'angles de 0 à 180°
        full_angles = np.linspace(0., 180., 360, endpoint=False)

        # Sélectionne un sous-ensemble d'angles en fonction du facteur d'undersampling
        num_angles = int(len(full_angles) * undersampling_factor)
        selected_indices = np.sort(np.random.choice(len(full_angles), num_angles, replace=False))
        theta_undersampled = full_angles[selected_indices]

        # Génère le sinogramme uniquement aux angles sélectionnés
        sinogram = radon(image, theta=theta_undersampled, circle=True)

        return sinogram, theta_undersampled, selected_indices
\end{minted}

\paragraph{Explication :}  
Cette fonction est conçue pour simuler un scénario de Compressed Sensing (CS) où toutes les projections ne sont pas disponibles (sparse-view CT). 
\begin{itemize} 
    \item[-] undersampling\_factor contrôle combien d'angles sont conservés.  
    \item[-] Cela permet d'évaluer la performance des algorithmes de reconstruction CS sur des données réalistes avec moins d'informations.
\end{itemize}
\paragraph{Exemple d'utilisation pour simulation :}  
\begin{minted}{python}
    if __name__ == "__main__":
        # Crée un fantôme Shepp-Logan (image test standard en tomographie)
        image = shepp_logan_phantom()

        # Sinogramme complet
        sinogram_full, theta_full = generate_sinogram(image, num_angles=180)

        # Sinogramme sous-échantillonné (50% des angles)
        sinogram_cs, theta_cs, selected_indices = generate_undersampled_sinogram(image, undersampling_factor=0.5)
\end{minted}

\annexechapter{Opérateur de Transformée en Ondelettes}
\begin{minted}{python}
from typing import Annotated
from scipy import sparse
from scipy.sparse.linalg import LinearOperator
import pywt  # PyWavelets for wavelet transform
import time


def create_wavelet_operator(
    image_shape: Annotated[Tuple[int, int], "Height x Width"],
    wavelet: str ='db1',
    level: int|None =None
):
    """
    Create wavelet transform operator W and its adjoint W^T
    """

    height, width = image_shape
    n_pixels = height * width

    # ---- Create reference coefficient structure ----
    dummy = np.zeros(image_shape)
    coeffs = pywt.wavedec2(dummy, wavelet, level=level)
    coeff_array_ref, coeff_slices = pywt.coeffs_to_array(coeffs)

    coeff_shape = coeff_array_ref.shape
    n_coeffs = coeff_array_ref.size

    # ---- Forward: W x ----
    def wavelet_forward(x):
        img = x.reshape(image_shape)
        coeffs = pywt.wavedec2(img, wavelet, level=level)
        coeff_array, _ = pywt.coeffs_to_array(coeffs)
        return coeff_array.flatten()

    # ---- Adjoint / Inverse: W^T w ----
    def wavelet_adjoint(w):
        coeff_array = w.reshape(coeff_shape)
        coeffs = pywt.array_to_coeffs(
            coeff_array,
            coeff_slices,
            output_format='wavedec2'
        )
        img = pywt.waverec2(coeffs, wavelet)
        return img[:height, :width].flatten()

    Wop = LinearOperator(
        shape=(n_coeffs, n_pixels),
        matvec=wavelet_forward,
        rmatvec=wavelet_adjoint,
        dtype=np.float64
    )

    return Wop
\end{minted}

\annexechapter{Reconstruction d'Images en Tomographie par Transformée de Radon}
\begin{minted}{python}
    from scipy.sparse.linalg import LinearOperator
    from skimage.transform import radon, iradon
    import numpy as np

    def create_system_matrix(image_shape, theta):
        """
        Matrix-free CT system operator A and its adjoint A^T
        (true Radon transform and unfiltered backprojection)
        """
        height, width = image_shape
        n_pixels = height * width
        n_rays = height * len(theta)

        def matvec(x):
            """
            Forward projection: y = A x
            """
            image = x.reshape(image_shape)
            sinogram = radon(image, theta=theta, circle=True)
            return sinogram.flatten()

        def rmatvec(y):
            """
            Adjoint operator: x = A^T y
            (unfiltered backprojection)
            """
            sino = y.reshape((height, len(theta)))
            backproj = iradon(
                sino,
                theta=theta,
                filter_name=None,  # CRITICAL
                circle=True,
                output_size=height
            )
            return backproj.flatten()

        A = LinearOperator(
            shape=(n_rays, n_pixels),
            matvec=matvec,
            rmatvec=rmatvec,
            dtype=np.float64
        )

        return A
\end{minted}
\section{Explication détaillée du code : opérateur de tomographie (CT scan)}

\subsection{Objectif du code}

Ce code crée un opérateur linéaire qui modélise un système de tomographie (scanner médical) en utilisant la transformée de Radon. Il permet de simuler l'acquisition de données CT (projections) et la rétroprojection non filtrée.

\subsection{Explication ligne par ligne}

\subsubsection{Importations}

\begin{minted}{python}
from scipy.sparse.linalg import LinearOperator
from skimage.transform import radon, iradon
import numpy as np
\end{minted}

\begin{itemize}
\item \texttt{LinearOperator} : classe de SciPy pour créer des opérateurs linéaires sans matrice explicite.
\item \texttt{radon}, \texttt{iradon} : fonctions de scikit-image pour la transformée de Radon directe et inverse.
\item \texttt{numpy} : manipulation de tableaux.
\end{itemize}

\subsubsection{Paramètres de la fonction}

\begin{minted}{python}
def create_system_matrix(image_shape, theta):
    """
    Matrix-free CT system operator A and its adjoint A^T
    (true Radon transform and unfiltered backprojection)
    """
    height, width = image_shape
    n_pixels = height * width
    n_rays = height * len(theta)
\end{minted}

\begin{itemize}
\item \texttt{image\_shape} : dimensions (hauteur, largeur) de l'image à reconstruire.
\item \texttt{theta} : liste des angles de projection (en degrés) pour lesquels on acquiert des données.
\item \texttt{n\_pixels} : nombre total de pixels dans l'image.
\item \texttt{n\_rays} : nombre total de rayons mesurés (hauteur du détecteur $\times$ nombre d'angles).
\end{itemize}

\subsubsection{Fonction de projection avant ($A$)}

\begin{minted}{python}
def matvec(x):
    """
    Forward projection: y = A x
    """
    image = x.reshape(image_shape)
    sinogram = radon(image, theta=theta, circle=True)
    return sinogram.flatten()
\end{minted}

\begin{itemize}
\item Remet le vecteur \texttt{x} en forme 2D.
\item Applique la transformée de Radon pour simuler l'acquisition CT.
\item Retourne le sinogramme aplati en un vecteur 1D.
\end{itemize}

\textbf{Interprétation physique :} Cette fonction simule le passage des rayons X à travers l'objet. Chaque pixel du sinogramme représente l'atténuation totale le long d'un rayon.

\subsubsection{Fonction de rétroprojection ($A^T$)}

\begin{minted}{python}
def rmatvec(y):
    """
    Adjoint operator: x = A^T y
    (unfiltered backprojection)
    """
    sino = y.reshape((height, len(theta)))
    backproj = iradon(
        sino,
        theta=theta,
        filter_name=None,  # CRITICAL
        circle=True,
        output_size=height
    )
    return backproj.flatten()
\end{minted}

\begin{itemize}
\item Remet le sinogramme aplati en forme 2D.
\item Applique la rétroprojection non filtrée via \texttt{iradon} avec \texttt{filter\_name=None}.
\item Recadre l'image à la taille originale et l'aplatit.
\end{itemize}

\textbf{Interprétation physique :} La rétroprojection "étale" chaque mesure le long du trajet du rayon correspondant. C'est l'opération mathématiquement adjointe à la projection avant.

\subsubsection{Création de l'opérateur linéaire}

\begin{minted}{python}
A = LinearOperator(
    shape=(n_rays, n_pixels),
    matvec=matvec,
    rmatvec=rmatvec,
    dtype=np.float64
)

return A
\end{minted}

\subsection{Utilisation typique}

\begin{minted}{python}
# Définir la géométrie d'acquisition
image_size = 256
angles = np.linspace(0, 180, 180)  # 180 projections sur 180°

# Créer l'opérateur CT
A = create_system_matrix((image_size, image_size), angles)

# Simuler l'acquisition à partir d'une image fantôme
phantom = np.zeros((256, 256))
# ... remplir le fantôme ...
projections = A @ phantom.flatten()  # Équivalent à A * phantom

# Reconstruire par rétroprojection (non filtrée)
reconstruction = A.H @ projections    # Équivalent à A^T * projections
\end{minted}

\subsection{Applications}

\begin{enumerate}
\item Simulation d'acquisitions CT pour générer des données synthétiques.
\item Reconstruction itérative (ART, SIRT, méthodes de gradient).
\item Résolution de problèmes inverses avec régularisation (TV, ondelettes, etc.).
\item Compressed sensing en tomographie.
\end{enumerate}

\subsection{Notes importantes}

\begin{itemize}
\item L'opérateur est "matrix-free" : aucune matrice n'est stockée explicitement.
\item Mémoire utilisée : $O(n\_pixels)$ au lieu de $O(n\_pixels \times n\_rays)$.
\item L'opérateur adjoint $A^T$ est différent de l'inverse $A^{-1}$.
\item Pour une reconstruction réelle, appliquer un filtre (ramp) ou utiliser des méthodes itératives.
\end{itemize}


\annexechapter{La Reconstruction Tomographique par Séparation de Variables (ADMM)}
\begin{minted}{python}
    from tqdm import tqdm
    import numpy as np
    from scipy.sparse.linalg import cg, LinearOperator


    def cs_reconstruction_admm(
        sinogram, theta, image_shape,
        rho=1.0, lambda_l1=0.01, max_iter=100
    ):

        height, width = image_shape
        n_pixels = height * width

        A = create_system_matrix(image_shape, theta)
        Wop = create_wavelet_operator(image_shape, wavelet="bior4.4")


        # Initialize
        x = np.zeros(n_pixels)
        z = np.zeros(Wop.shape[0])
        u = np.zeros_like(z)

        for k in tqdm(range(max_iter)):

            # ---- x-update ----
            def lhs(v):
                return (
                    A.rmatvec(A.matvec(v)) +
                    rho * Wop.rmatvec(Wop.matvec(v))
                )

            rhs = (
                A.rmatvec(sinogram.flatten()) +
                rho * Wop.rmatvec(z - u)
            )

            x, _ = cg(
                LinearOperator((n_pixels, n_pixels), matvec=lhs),
                rhs, x0=x, maxiter=50
            )

            # ---- z-update (soft thresholding) ----
            Wx = Wop.matvec(x)
            v = Wx + u
            z_old = z.copy()
            z = np.sign(v) * np.maximum(np.abs(v) - lambda_l1 / rho, 0)

            # ---- u-update ----
            u = u + Wx - z

            # ---- convergence check ----
            r_norm = np.linalg.norm(Wx - z)
            s_norm = rho * np.linalg.norm(z - z_old)

            if r_norm < 1e-4 and s_norm < 1e-4:
                print(f"ADMM converged at iteration {k}")
                break

        return x.reshape(image_shape)
\section{Reconstruction CT par Compressed Sensing avec ADMM}

\subsection{Objectif du code}

Ce code reconstruit une image CT à partir d'un sinogramme incomplet ou bruité en utilisant la régularisation par ondelettes (compressed sensing) avec l'algorithme ADMM (Alternating Direction Method of Multipliers).

\subsection{Contexte mathématique}

On cherche à résoudre : 
\[
\min_x \frac{1}{2}\|Ax - y\|_2^2 + \lambda \|Wx\|_1
\]
où :
\begin{itemize}
\item $A$ est l'opérateur de projection CT
\item $y$ est le sinogramme mesuré
\item $W$ est la transformée en ondelettes
\item $\lambda$ contrôle la force de la régularisation parcimonieuse
\end{itemize}

\subsection{Explication ligne par ligne}

\subsubsection{Paramètres de la fonction}

\begin{minted}{python}
def cs_reconstruction_admm(
    sinogram, theta, image_shape,
    rho=1.0, lambda_l1=0.01, max_iter=100
):
\end{minted}

\begin{itemize}
\item \texttt{sinogram} : données CT acquises (projections)
\item \texttt{theta} : angles de projection utilisés
\item \texttt{image\_shape} : dimensions de l'image à reconstruire
\item \texttt{rho} : paramètre de pénalité ADMM
\item \texttt{lambda\_l1} : poids de la régularisation L1
\item \texttt{max\_iter} : nombre maximal d'itérations
\end{itemize}

\subsubsection{Initialisation des opérateurs}

\begin{minted}{python}
A = create_system_matrix(image_shape, theta)
Wop = create_wavelet_operator(image_shape, wavelet="bior4.4")
\end{minted}

\begin{itemize}
\item $A$ : opérateur de projection CT
\item $Wop$ : opérateur de transformée en ondelettes biorthogonales 4.4
\end{itemize}

\subsubsection{Initialisation des variables ADMM}

\begin{minted}{python}
x = np.zeros(n_pixels)           # Image à reconstruire
z = np.zeros(Wop.shape[0])       # Variable auxiliaire (coefficients d'ondelettes)
u = np.zeros_like(z)             # Multiplicateur de Lagrange (dual)
\end{minted}

ADMM décompose le problème en sous-problèmes plus simples avec trois variables :
\begin{itemize}
\item $x$ : image dans l'espace des pixels
\item $z$ : coefficients d'ondelettes (parcimonieux)
\item $u$ : variable duale qui force $Wx \approx z$
\end{itemize}

\subsubsection{Boucle principale ADMM}

\begin{minted}{python}
for k in tqdm(range(max_iter)):
\end{minted}

\paragraph{Mise à jour de x (sous-problème quadratique)}

\begin{minted}{python}
def lhs(v):
    return (
        A.rmatvec(A.matvec(v)) +
        rho * Wop.rmatvec(Wop.matvec(v))
    )

rhs = (
    A.rmatvec(sinogram.flatten()) +
    rho * Wop.rmatvec(z - u)
)

x, _ = cg(
    LinearOperator((n_pixels, n_pixels), matvec=lhs),
    rhs, x0=x, maxiter=50
)
\end{minted}

On résout : $(A^TA + \rho W^TW)x = A^Ty + \rho W^T(z - u)$

\begin{itemize}
\item \texttt{lhs(v)} : application de $(A^TA + \rho W^TW)$
\item \texttt{rhs} : membre droit
\item Algorithme du gradient conjugué (CG), 50 itérations max par itération ADMM
\end{itemize}

\paragraph{Mise à jour de z (seuillage doux)}

\begin{minted}{python}
Wx = Wop.matvec(x)
v = Wx + u
z_old = z.copy()
z = np.sign(v) * np.maximum(np.abs(v) - lambda_l1 / rho, 0)
\end{minted}

Opération de \textbf{proximal operator} pour la norme L1 :
\begin{itemize}
\item Calcule les coefficients d'ondelettes $Wx$
\item Applique le seuillage doux avec seuil $\lambda/\rho$
\item Rend les coefficients parcimonieux
\end{itemize}

\paragraph{Mise à jour de u (variable duale)}

\begin{minted}{python}
u = u + Wx - z
\end{minted}

Met à jour le multiplicateur de Lagrange.

\paragraph{Vérification de convergence}

\begin{minted}{python}
r_norm = np.linalg.norm(Wx - z)           # Résidu primal
s_norm = rho * np.linalg.norm(z - z_old)  # Résidu dual

if r_norm < 1e-4 and s_norm < 1e-4:
    print(f"ADMM converged at iteration {k}")
    break
\end{minted}

Critères classiques d'arrêt :
\begin{itemize}
\item Résidu primal : $\|Wx - z\|$
\item Résidu dual : $\rho \|z - z_\text{old}\|$
\end{itemize}

\subsection{Utilisation typique}

\begin{minted}{python}
# Données CT (peu de projections)
angles = np.linspace(0, 180, 30)
sinogram = radon(phantom, theta=angles, circle=True)

# Reconstruction CS
reconstructed = cs_reconstruction_admm(
    sinogram, angles, (256, 256),
    rho=1.0, lambda_l1=0.01, max_iter=50
)

# Visualisation
plt.imshow(reconstructed, cmap='gray')
\end{minted}

\subsection{Avantages}

\begin{enumerate}
\item Compressed sensing : reconstruction avec peu de projections
\item Régularisation parcimonieuse : exploite la parcimonie en ondelettes
\item ADMM : décomposition en sous-problèmes simples
\item Matrix-free : adaptée aux grandes images sans mémoire excessive
\end{enumerate}

\subsection{Paramètres clés à ajuster}

\begin{itemize}
\item \texttt{rho} : équilibre fidélité/régularisation (0.1 à 10)
\item \texttt{lambda\_l1} : force de la régularisation L1
\item \texttt{max\_iter} : nombre d'itérations ADMM (50–200 selon la difficulté)
\end{itemize}
    % %===========================================================
% Chapitre 3 — Méthodes variationnelles, parcimonieuses et Compressive Sensing
%===========================================================
\chapter{Reconstruction d'images : problèmes inverses et compressive sensing}

\section{Cadre des problèmes inverses}
\subsection{Le problème inverse : formulation générale}
Un problème inverse consiste à estimer une quantité inconnue (ici, une image) à partir d'observations indirectes, bruitées et souvent incomplètes. Formellement, ce processus peut être modélisé par :

\begin{equation}
    \mathbf{y} = \mathcal{A}\mathbf{x} + \mathbf{n}
    \label{eq:inverse_problem}
\end{equation}

où :
\begin{itemize}
    \item[-] $\mathbf{x} \in \mathbb{R}^n$ représente l'image à reconstruire (inconnue),
    \item[-] $\mathbf{y} \in \mathbb{R}^m$ correspond aux données acquises (observations),
    \item[-] $\mathcal{A} : \mathbb{R}^n \rightarrow \mathbb{R}^m$ est l'opérateur direct modélisant le processus de dégradation (flou, projection, sous-échantillonnage, etc.),
    \item[-] $\mathbf{n} \in \mathbb{R}^m$ désigne le bruit de mesure additif.
\end{itemize}

L'opérateur $\mathcal{A}$ est typiquement \textit{mal conditionné} voire non inversible ($m < n$ dans le cas sous-déterminé), rendant la reconstruction de $\mathbf{x}$ à partir de $\mathbf{y}$ intrinsèquement difficile.


\begin{definition}
    Un problème est dit \textbf{bien posé} au sens de Hadamard s'il vérifie simultanément trois conditions :
    \begin{enumerate}
        \item \textbf{Existence :} Il existe au moins un $\mathbf{x} \in \mathbb{R}^n$ tel que $\mathcal{A} \mathbf{x} = \mathbf{y}$.
        
        \item \textbf{Unicité :} Cette solution est unique ; c'est-à-dire que si $\mathbf{z} \in \mathbb{R}^n$ vérifie $\mathcal{A} \mathbf{z} = \mathbf{y}$, alors nécessairement $\mathbf{z} = \mathbf{x}$.
        
        \item \textbf{Stabilité :} La solution dépend continûment des données $\mathbf{y}$. Plus précisément, pour toute suite $\{\mathbf{x}_n\}_{n\in\mathbb{N}} \subset \mathbb{R}^n$ telle que $\mathcal{A} \mathbf{x}_n \to \mathbf{y}$ (convergence dans $\mathbb{R}^m$), on a également $\mathbf{x}_n \to \mathbf{x}$ (convergence dans $\mathbb{R}^n$).
    \end{enumerate}
    Si l'une de ces trois conditions n'est pas vérifiée, le problème \eqref{eq:inverse_problem} est dit \emph{mal posé} au sens de Hadamard.\vspace{5pt}
\end{definition}

La majorité des problèmes inverses en imagerie violent au moins une de ces conditions, les rendant \textbf{mal posés}. La mal-positude se manifeste notamment par une grande sensibilité au bruit : de infimes variations des données $\mathbf{y}$ peuvent provoquer des changements arbitrairement grands dans la solution estimée.\vspace{5pt}

En pratique, les observations $\mathbf{y}$ sont corrompues par différents types de bruit (photonique, électronique, quantique, etc.) et sont souvent acquises de manière incomplète ($m \ll n$) pour des raisons techniques ou temporelles. Cette sous-détermination aggrave la mal-positude et rend le problème \textit{indéterminé} (multiples solutions possibles).\vspace{5pt}

La résolution d'un problème inverse mal posé nécessite l'injection d'\textit{information a priori} sur la solution recherchée. Ces contraintes peuvent être de différentes natures :
\begin{itemize}
    \item[-] Contraintes physiques (positivité, support limité, etc.)
    \item[-] Propriétés statistiques (distribution du bruit, régularité spatiale)
    \item[-] Structures spécifiques (parcimonie dans une base appropriée, bas rang, etc.)
\end{itemize}
L'intégration judicieuse de ces informations constitue le cœur des méthodes modernes de reconstruction.

% ================================
\section{Approches variationnelles} 
% ================================
Lors de la résolution de problèmes inverses, l'objectif est de retrouver une quantité inconnue \(\mathbf{x}\) à partir de données mesurées \(\mathbf{y}\). Comme les exemples précédents l'ont montré, cette tâche peut s'avérer difficile. Même si l'opérateur \(\mathcal{A}\) admet un inverse bien défini sur son image, c'est-à-dire que \(\mathcal{A}^{-1}: \mathbb{R}^m \rightarrow \mathbb{R}^n\) existe, rien ne garantit que les données bruitées appartiennent encore à l'image de l'opérateur. On peut également n'avoir accès qu'à des mesures partielles, ce qui mène à des systèmes sous-déterminés et rend l'inversion directe impossible, même sur des données non bruitées.\vspace{5pt}\\
\textbf{Régularisation.} Une approche pour obtenir des solutions significatives dans les scénarios décrits est appelée \textit{régularisation}. Une régularisation \(\mathfrak{R}_{\alpha}:\mathbb{R}^m \rightarrow \mathbb{R}^n\) associe à tout point de données une solution favorable. Intuitivement, on espère que la régularisation étend approximativement la notion d'inverse au cadre bruité et potentiellement mal posé, c'est-à-dire
\begin{equation}
    \mathfrak{R}_{\alpha}(\mathcal{A}\mathbf{x}+\varepsilon)\approx \mathbf{x}.
\end{equation}
Une stratégie typique est la régularisation dite \textit{de type Tikhonov} ou \textit{variationnelle}, où la sortie de l'application de régularisation est définie comme la solution du problème suivant

\begin{equation}
    \underset{\mathbf{x}\in\mathbb{R}^n}{\arg\min}
    \underbrace{\left\|{\mathcal{A}\mathbf{x}-\mathbf{y}} \right\|_{L^{2}}^{2}}_{\text{fidélité aux données}}+\alpha\underbrace{{\mathcal{R}}(\mathbf{x}) }_{\text{régularisant}}.
\end{equation}


Les termes s'interprètent comme suit :

\begin{itemize}
    \item \textbf{Fidélité aux données :} Ce terme indique à quel point notre estimation \(\mathbf{x}\) correspond aux données observées \(\mathbf{y}\). Bien que nous souhaitions minimiser la fidélité aux données, il n'est pas toujours pertinent de l'annuler dans le cas bruité, car nous cherchons \(\hat{\mathbf{x}}\) tel que \({\mathcal{A}}\hat{\mathbf{x}}=\mathbf{y}-\varepsilon\), où potentiellement \({\mathcal{A}}\hat{\mathbf{x}}\neq \mathbf{y}\).

    \item \textbf{Régularisant :} Le régularisant nous permet d'incorporer des informations supplémentaires sur la solution recherchée. Dans le cas classique, nous savons déjà que \(\mathbf{x}\) doit être proche d'un certain point \(\mu\), c'est-à-dire que nous voulons pénaliser la distance entre la solution \(\mathbf{x}\) et \(\mu\). De plus, on souhaite souvent pénaliser certaines directions plus que d'autres, pour lesquelles nous considérons un opérateur unitaire \(\mathcal{Q}:\mathbb{R}^{n}\to\mathbb{R}^{n}\) et choisissons ensuite
        \[{\mathcal{R}}(\mathbf{x})=\|\mathbf{x}-\mu\|_{L^{2},\mathcal{Q}}^{2}:=\langle \mathbf{x}-\mu,\mathcal{Q}(\mathbf{x}-\mu)\rangle\,,\] 
        ce qui est appelé \textit{régularisation de Tikhonov}\footnote{parfois appelée « régularisation aux moindres carrés »}.

    \item \textbf{Paramètre de régularisation \(\alpha\) :} Le paramètre \(\alpha>0\) contrôle la force du régularisant. Dans certaines formulations, ce paramètre est inclus dans la définition du régularisant.
\end{itemize}

Le terme de régularisation $\mathcal{R}(\mathbf{x})$ joue un rôle central dans la qualité de reconstruction obtenue. Il encode des hypothèses structurelles sur l'image recherchée et conditionne à la fois l'aspect visuel de la solution et la complexité numérique de l'algorithme utilisé.\newpage

Le choix du régularisateur dépend :
\begin{itemize}
    \item[-] du type d'images traitées,
    \item[-] de la nature du bruit,
    \item[-] de la structure de l'opérateur $\mathbf{A}$,
    \item[-] des contraintes de temps de calcul.
\end{itemize}
Des régularisations hybrides, combinant par exemple variation totale et parcimonie, sont fréquemment utilisées pour améliorer la qualité de reconstruction.

%===========================================================
\section{Régularisation classique}
%===========================================================
La régularisation classique désigne l'ensemble des méthodes visant à stabiliser la résolution des problèmes inverses en introduisant des contraintes supplémentaires sur la solution. Ces contraintes imposent des propriétés de l'image reconstruite telles que la régularité, la parcimonie des gradients ou la similarité avec des voisins. Les principales familles comprennent la régularisation quadratique de Tikhonov, la variation totale, les approches non locales et les méthodes multi-échelles.

% -------------------------------
\subsection{Régularisation de Tikhonov ($L^2$)}
% -------------------------------

Une méthode bien connue et efficace est la \textit{régularisation de Tikhonov}. Dans cette approche, une solution du problème \eqref{eq:inverse_problem} est approchée par une solution du problème de minimisation

\begin{equation}
    \min \|\mathcal{A}\mathbf{x} - \mathbf{y}\|_2^2 + \alpha \|\mathbf{x} - \mathbf{x}^*\|_2^2 \quad 
    \label{eq:tikhonov_problem}
\end{equation}

où \(\alpha > 0\) est un petit paramètre, \(\mathbf{y} \in \mathbb{R}^m\) est une approximation du membre de gauche exact \(\mathbf{y}\) du problème \eqref{eq:inverse_problem}, et \(x^*\) est une estimation a priori de la solution inconnue.\\
% Dans le cas d'un opérateur linéaire \(\mathcal{A}\), les aspects de stabilité, de convergence et de vitesses de convergence (quand \(\delta \to 0\)) ont été largement étudiés [2]. 
% Le rôle de la régularisation de Tikhonov pour stabiliser les problèmes d'estimation de paramètres a été examiné dans [3].
% Dans cette étude, nous montrons qu'une solution de l'équation
% \begin{equation}
%     (\mathcal{A}^*\mathcal{A} + \alpha \mathbf{I})x(\alpha) = \mathcal{A}^*\mathbf{y}
%     \label{eq:tikhonov_solution}
% \end{equation}
% existe et est unique, et que \(\mathcal{R}_{xy}\) défini par \(x(\alpha)\) est un régularisateur de l'équation \eqref{eq:inverse_problem} sur \(\mathbb{R}^n\), à condition que l'équation \(Ax = 0\) n'ait que la solution nulle.

D'après l'article \cite{7}, la démonstration de l'existence, de l'unicité et de la stabilité de la solution du problème de minimisation de Tikhonov est principalement détaillée dans les sections 2 et 3.

\subsubsection{Existence de la solution}
L'existence d'une solution au problème de minimisation est établie à la fois par l'analyse fonctionnelle générale et par une construction opératorielle spécifique :
\begin{itemize}
    \item \textbf{Lemme 1 :} Les auteurs affirment qu'une solution $R_\alpha y$ du problème de minimisation $\min (\|Av - y\|^2_Y + \alpha M(v))$ \textbf{existe} . La preuve repose sur le fait que $M$ est une fonctionnelle semi-continue inférieurement et que l'ensemble $\{v \in X_M : M(v) \leq r\}$ est compact .
    \item \textbf{Système de fonctions propres :} Dans la section 3, le texte déclare explicitement : « Tout d'abord, nous prouvons l'\textbf{existence} et l'unicité ». Cela est démontré en montrant que la solution $x(\alpha)$ peut être représentée à l'aide d'un système complet de fonctions propres $\{L_k\}$ de l'opérateur autoadjoint $\mathcal{A}^*\mathcal{A}$ .
\end{itemize}

\subsubsection{Unicité de la solution}
L'article fournit deux justifications mathématiques distinctes pour l'unicité de la solution :
\begin{itemize}
    \item \textbf{Détermination des composantes :} Dans la section 3, les composantes de la solution $x_k(\alpha)$ sont montrées être \textbf{déterminées de manière unique} par la formule :
    \[ x_k(\alpha) = \frac{y^A_k}{\lambda_k + \alpha} \]
    où $\lambda_k$ sont les valeurs propres de $\mathcal{A}^*\mathcal{A}$
    \item \textbf{Opérateur injectif :} Les références prouvent en outre l'unicité en montrant que l'opérateur $(\mathcal{A}^*\mathcal{A} + \alpha I)$ est positif. Plus précisément, ils démontrent que le \textbf{noyau} de $(\mathcal{A}^*\mathcal{A} + \alpha I)$ est $\{0\}$, ce qui garantit que toute solution existante est nécessairement unique.
\end{itemize}

\subsubsection{Stabilité de la solution}
La stabilité est abordée comme la dépendance continue de la solution par rapport aux données et au paramètre $\alpha$ :
\begin{itemize}
    \item \textbf{Preuve de continuité :} Dans la section 3, les auteurs déclarent explicitement : « Nous montrons maintenant que $R_\alpha$, c'est-à-dire $x(\alpha)$, est \textbf{continu} » [5]. Ils fournissent une preuve montrant que lorsque la différence entre les paramètres $|\alpha' - \alpha|$ tend vers zéro, la distance entre les solutions correspondantes $\|x(\alpha) - x(\alpha')\|$ s'annule également.
    \item \textbf{Propriété de régularisation :} Par définition, la famille d'opérateurs $R_\alpha$ est montrée être un \textbf{régulariseur}, ce qui signifie qu'elle fournit une approximation stable qui converge ponctuellement vers la solution exacte lorsque $\alpha \to 0$, sous réserve que les données soient cohérentes ($y = \mathcal{A}x$).
\end{itemize}

% % -------------------------------
\subsection{Variation Totale (TV)}
% -------------------------------

Afin de pallier les limitations de la régularisation quadratique, la régularisation par variation totale a été introduite pour préserver les discontinuités. Dans sa forme isotrope, elle est définie par :
\begin{equation}
    \mathcal{R}_{\text{TV}}(\mathbf{x}) 
    = \sum_{i,j} \sqrt{(\nabla_x x_{i,j})^2 + (\nabla_y x_{i,j})^2}
    = \|\nabla \mathbf{x}\|_1.
\end{equation}

L'emploi de la norme $L^1$ du gradient conduit à :
\begin{itemize}
    \item la préservation des bords nets,
    \item la suppression efficace du bruit,
    \item la production d'images par morceaux quasi-constants.
\end{itemize}

La minimisation associée est cependant non lisse et nécessite des méthodes itératives de type primal-dual, ADMM ou Split-Bregman \cite{8}. Un inconvénient notable est l'apparition du phénomène dit de \emph{staircasing}, caractérisé par des paliers artificiels.

% -------------------------------
\subsection{Méthodes non locales}
% -------------------------------

Les méthodes non locales exploitent la redondance statistique présente dans l'image en considérant des similarités entre pixels spatialement éloignés. Plutôt que de ne considérer que le voisinage local, elles s'appuient sur une mesure de similarité entre patchs.

Une formulation typique repose sur la pénalisation :
\begin{equation}
    \mathcal{R}_{\text{NL}}(\mathbf{x}) =
    \sum_{i,j} w_{ij} \, (x_i - x_j)^2,
\end{equation}
où $w_{ij}$ représente un poids mesurant la similarité entre les régions autour des pixels $i$ et $j$.

Ces approches permettent :
\begin{itemize}
    \item meilleure préservation des textures,
    \item réduction du bruit sans sur-lissage,
    \item exploitation de structures répétitives.
\end{itemize}

Elles sont à la base des méthodes telles que \textit{Non-Local Means} et des approches basées patchs pour la débruitage et la défloutage.

% -------------------------------
\subsection{Approches multi-échelles (ondelettes, curvelets)}
% -------------------------------

Les approches multi-échelles s'appuient sur le fait que les images naturelles présentent des structures à plusieurs résolutions. Les représentations dans des bases telles que les ondelettes, les curvelets ou les contourlets permettent de capturer efficacement :

\begin{itemize}
    \item les singularités orientées,
    \item les contours,
    \item les textures fines.
\end{itemize}

La régularisation consiste alors à imposer la parcimonie des coefficients transformés :
\begin{equation}
    \mathcal{R}_{\text{MS}}(\mathbf{x}) = \|\mathbf{\Psi x}\|_1,
\end{equation}
où $\mathbf{\Psi}$ désigne une transformée multi-résolution.

Les ondelettes sont particulièrement adaptées aux singularités ponctuelles, tandis que les curvelets et shearlets offrent une meilleure représentation des structures anisotropes et courbes. Ces méthodes constituent un lien direct avec le \textit{Compressed Sensing} et les modèles parcimonieux modernes.


%===========================================================
\section{Modèles parcimonieux}
%===========================================================

Les modèles parcimonieux reposent sur l'hypothèse qu'une image ou un signal peut être représenté par un nombre réduit de coefficients significatifs dans une base appropriée ou un dictionnaire sur-complet. Cette propriété est à la base de nombreuses techniques modernes de reconstruction d'images, de compression et de \textit{Compressed Sensing}. L'objectif est de promouvoir des représentations compactes permettant de régulariser les problèmes inverses mal posés.

% -------------------------------
\subsection{Bases orthogonales vs dictionnaires}
% -------------------------------

Une représentation parcimonieuse peut être obtenue soit dans une \textbf{base orthogonale}, soit dans un \textbf{dictionnaire sur-complet}.

\paragraph{Bases orthogonales}

Une base orthogonale $\{\mathbf{\phi}_k\}_{k=1}^{N}$ de $\mathbb{R}^n$ permet d'écrire :
\begin{equation}
    \mathbf{x} = \sum_{k=1}^{N} \alpha_k \mathbf{\phi}_k,
\end{equation}
où les coefficients $\alpha_k$ sont uniques. Exemples courants :
\begin{itemize}
    \item ondelettes orthogonales,
    \item base de Fourier,
    \item transformée discrète du cosinus (DCT).
\end{itemize}

Les bases orthogonales ont l'avantage de garantir l'unicité des coefficients et de permettre des calculs rapides via des transformées rapides (FFT, DWT, DCT).

\paragraph{Dictionnaires sur-complets}

Un dictionnaire $\mathbf{D} \in \mathbb{R}^{N \times K}$ avec $K > n$ est dit sur-complet. Dans ce cas, un signal peut avoir plusieurs décompositions possibles :
\begin{equation}
    \mathbf{x} = \mathbf{D}\mathbf{\alpha},
\end{equation}
où $\mathbf{\alpha} \in \mathbb{R}^{K}$ est un vecteur de coefficients.

Les dictionnaires sur-complets offrent :
\begin{itemize}
    \item une meilleure capacité d'adaptation aux structures complexes,
    \item des représentations plus parcimonieuses,
    \item la possibilité d'apprentissage à partir des données.
\end{itemize}
Cependant, la décomposition n'est plus unique et nécessite la résolution de problèmes d'optimisation.

% -------------------------------
\subsection{Modèles sparse : $L^0$ et $L^1$}
% -------------------------------

La parcimonie consiste à rechercher une représentation comportant le moins de coefficients non nuls possible. Le problème fondamental est :
\begin{equation}
    \min_{\mathbf{\alpha}} \|\mathbf{\alpha}\|_0 
    \quad \text{sous la contrainte} \quad 
    \mathbf{x} = \mathbf{D}\mathbf{\alpha},
\end{equation}
où $\|\mathbf{\alpha}\|_0$ désigne le nombre de coefficients non nuls. Ce problème est combinatoire et NP-difficile.

Pour rendre la résolution praticable, on remplace la norme $L^0$ par la norme convexe $L^1$ :
\begin{equation}
    \min_{\mathbf{\alpha}} \|\mathbf{\alpha}\|_1 
    \quad \text{sous} \quad 
    \mathbf{x} = \mathbf{D}\mathbf{\alpha},
\end{equation}
ou, en présence de bruit,
\begin{equation}
    \min_{\mathbf{\alpha}} 
    \left\{
    \frac{1}{2}\|\mathbf{x} - \mathbf{D}\mathbf{\alpha}\|_2^2 + 
    \alpha \|\mathbf{\alpha}\|_1
    \right\}.
\end{equation}

La relaxation $L^1$ constitue la base mathématique des approches de \textit{Basis Pursuit}, \textit{LASSO} et du \textit{Compressed Sensing}.

% -------------------------------
\subsection{Algorithmes d'approximation parcimonieuse}
% -------------------------------

Plusieurs classes d'algorithmes permettent de résoudre les problèmes parcimonieux.

\paragraph{Méthodes gloutonnes}

Ces méthodes sélectionnent itérativement les atomes du dictionnaire qui expliquent au mieux le signal résiduel. Parmi elles :
\begin{itemize}
    \item Matching Pursuit (MP),
    \item Orthogonal Matching Pursuit (OMP),
    \item Stagewise OMP (StOMP).
\end{itemize}

Elles présentent un faible coût de calcul et sont adaptées aux dictionnaires de grande taille.

\paragraph{Méthodes par seuillage}

Ces méthodes reposent sur le seuillage doux ou dur des coefficients :
\begin{itemize}
    \item ISTA (Iterative Shrinkage-Thresholding Algorithm),
    \item FISTA (version accélérée),
    \item algorithmes proximal-gradient.
\end{itemize}

Elles sont particulièrement adaptées à la minimisation de fonctionnelles $L^2$–$L^1$ convexes.

% -------------------------------
\subsection{K-SVD : apprentissage de dictionnaire}
% -------------------------------

Le K-SVD est un algorithme d'apprentissage de dictionnaire visant à construire un dictionnaire sur-complet directement à partir d'un ensemble d'images ou de patchs. L'objectif est de résoudre :
\begin{equation}
    \min_{\mathbf{D},\mathbf{\alpha}_i}
    \sum_{i}
    \left\|
    \mathbf{x}_i - \mathbf{D}\mathbf{\alpha}_i
    \right\|_2^2
    \quad
    \text{sous la contrainte}
    \quad
    \|\mathbf{\alpha}_i\|_0 \leq T_0,
\end{equation}
où $\{\mathbf{x}_i\}$ sont les signaux d'apprentissage et $T_0$ fixe le niveau de parcimonie.

L'algorithme alterne :
\begin{enumerate}
    \item une étape de \textbf{codage parcimonieux} des coefficients,
    \item une étape de \textbf{mise à jour du dictionnaire} atome par atome via la décomposition en valeurs singulières (SVD).
\end{enumerate}

Le K-SVD permet de construire des dictionnaires adaptés aux structures réelles des images, conduisant à d'excellentes performances en :
\begin{itemize}
    \item débruitage,
    \item défloutage,
    \item inpainting,
    \item super-résolution.
\end{itemize}


% ============================================================
\section{Théorie du Compressive Sensing}
% ============================================================

Le compressed sensing fournit un cadre rigoureux pour la reconstruction de signaux parcimonieux ou compressibles à partir de mesures linéaires fortement sous-échantillonnées. Il offre une solution fondée sur des principes solides aux problèmes inverses sous-déterminés, via des méthodes d'optimisation favorisant la parcimonie ou des algorithmes d'approximation gloutons. Cette approche a un impact majeur en imagerie médicale, imagerie computationnelle, résolution de problèmes inverses et systèmes de communication modernes.
\begin{definition}
    Le \emph{compressed sensing} (CS) est un cadre mathématique et algorithmique qui permet la reconstruction de signaux de grande dimension à partir d'un nombre de mesures significativement inférieur à celui requis par les méthodes traditionnelles.
\end{definition}
\begin{definition}
    Soit $x\in \mathbb{R}^{n}$ un signal inconnu. On dit que $x$  est \(k\)-parcimonieux dans une base (ou dictionnaire) \(\Psi\) (ex: Fourier, wavelet, DCT) si
    \[
        x=\Psi \alpha, \qquad \text{où } \alpha \text{ possède au plus } k \ll n \text{ coefficients non nuls}.   
    \]
\end{definition}
% Le principe fondamental repose sur la \textbf{parcimonie}. 
En pratique, de nombreux signaux ne sont pas parcimonieux dans leur domaine original (canonique), mais le deviennent après l'application d'une transformation linéaire. 
On observe des mesures linéaires de la forme
\[
y = A x,
\]
où \(A \in \mathbb{R}^{m \times n}\) avec \(m \ll n\). La théorie classique de l'échantillonnage exige \(m = n\) mesures indépendantes pour une reconstruction exacte, tandis que le compressed sensing montre que
\[
m \gtrsim k \log(n/k)
\]
est suffisant pour une reconstruction exacte ou stable, sous des conditions appropriées sur \(A\), telles que la propriété d'isométrie restreinte (\emph{Restricted Isometry Property}, RIP) ou l'incohérence.\\

Les résultats classiques de l'échantillonnage, tels que le théorème de Nyquist--Shannon, imposent qu'un signal soit échantillonné à une fréquence proportionnelle à sa bande passante. Le compressed sensing remet en cause ce paradigme en observant que de nombreux signaux réels (images, données médicales, spectres) sont parcimonieux ou compressibles dans une base de transformation (par exemple ondelettes, Fourier, DCT). Par conséquent, leur dimension effective est bien plus faible que le nombre d'échantillons disponibles. Le compressed sensing exploite cette redondance pour réduire drastiquement les coûts d'acquisition. Le compressed sensing s'attaque au problème général suivant :
\begin{center}
    \vspace*{\fill}
        Comment reconstruire un signal parcimonieux de grande dimension à partir d'un ensemble sous-déterminé de mesures linéaires ?
    \vspace*{\fill}
\end{center}
Ce cadre permet de résoudre plusieurs limitations pratiques :
\paragraph{Réduction du nombre de mesures.}\text{}\\ 
De nombreux systèmes d'acquisition sont limités par le coût, le temps ou l'énergie. Le compressed sensing permet :
\begin{itemize}
    \item[-] une acquisition plus rapide des données,
    \item[-] une réduction de la complexité matérielle,
    \item[-] une diminution de la dose de radiation (par exemple en tomodensitométrie),
    \item[-] une réduction des coûts de stockage et de transmission.
\end{itemize}

\paragraph{Problèmes inverses mal posés (ill-posed inverse problems).}\text{}\\
Lorsque le nombre de mesures est insuffisant pour garantir une solution unique, le CS introduit une régularisation fondée sur la parcimonie, permettant une reconstruction stable. Les principales applications incluent :
\begin{itemize}
    \item[-] la tomographie (CT, IRM, PET),
    \item[-] l'imagerie à super-résolution,
    \item[-] la déconvolution,
    \item[-] les inversions géophysiques et les essais non destructifs.
\end{itemize}

\paragraph{Robustesse au bruit et aux données incomplètes.} \text{}\\
Le CS garantit une reconstruction stable même en présence de bruit, de corruptions ou d'observations manquantes.

\subsection{Reconstruction de signaux par Compressed Sensing}

\subsubsection{Reconstruction par optimisation}
La formulation canonique de la reconstruction est
\[
\min_{\alpha} \|\alpha\|_{1} \quad \text{s.c.} \quad y = A \Psi \alpha,
\]
ou, en présence de bruit,
\[
\min_{\alpha} \|\alpha\|_{1} \quad \text{s.c.} \quad \|A \Psi \alpha - y\|_{2} \le \epsilon.
\]
Cela correspond aux formulations de type \emph{Basis Pursuit} ou \emph{LASSO}. La minimisation de la norme \(\ell_1\) favorise la parcimonie tout en conservant un problème d'optimisation convexe et calculable efficacement.

\subsubsection{Algorithmes gloutons}

Des alternatives plus rapides incluent :
\begin{itemize}
    \item l'\emph{Orthogonal Matching Pursuit} (OMP),
    \item le \emph{Compressive Sampling Matching Pursuit} (CoSaMP),
    \item l'\emph{Iterative Hard Thresholding} (IHT).
\end{itemize}
Ces méthodes échangent une partie de la précision contre un coût computationnel réduit.

\subsection{Applications du Compressed Sensing}

\paragraph{Imagerie médicale.}
\begin{itemize}
    \item acquisition IRM accélérée,
    \item CT à dose réduite,
    \item échographie à haute cadence d'images.
\end{itemize}

\paragraph{Imagerie computationnelle.}
\begin{itemize}
    \item caméras à pixel unique,
    \item imagerie à ouverture codée,
    \item reconstruction hyperspectrale.
\end{itemize}

\paragraph{Télédétection et géophysique.}
\begin{itemize}
    \item inversion sismique parcimonieuse,
    \item imagerie radar et radar à synthèse d'ouverture (SAR).
\end{itemize}

\paragraph{Communications sans fil.}
\begin{itemize}
    \item estimation parcimonieuse de canaux,
    \item réduction des pilotes dans les systèmes MIMO massifs.
\end{itemize}

\paragraph{Apprentissage automatique et traitement du signal.}
\begin{itemize}
    \item régression parcimonieuse (LASSO),
    \item apprentissage de dictionnaires,
    \item ACP robuste et modèles de rang faible apparentés.
\end{itemize}


% Le compressed sensing (CS) est un cadre mathématique et algorithmique qui permet de reconstruire des signaux de grande dimension à partir d'un nombre de mesures bien inférieur à celui requis par les approches traditionnelles. Il exploite la parcimonie (sparsity) comme principal a priori structurel.
% \subsection{Hypothèse de parcimonie}
% Si un signal est parcimonieux ou compressible dans une certaine base, alors il peut être reconstruit exactement (ou avec une erreur contrôlée) à partir d'un nombre de mesures linéaires bien inférieur à sa dimension ambiante.
% \subsection{Incohérence et propriété de RIP}
% \subsection{Basis Pursuit et LASSO}
% \subsection{OMP et algorithmes gloutons}


%===========================================================
\section{Méthodes d'optimisation}
%===========================================================

Les problèmes inverses régularisés et les modèles parcimonieux conduisent le plus souvent à la minimisation de fonctionnelles non différentiables, voire contraintes. Le choix d'une méthode d'optimisation adaptée conditionne la qualité de la reconstruction, la rapidité de convergence et la robustesse numérique. Les approches modernes reposent notamment sur les méthodes proximales, les schémas de décomposition de type ADMM, ainsi que les méthodes de gradient projeté.

% -------------------------------
\subsection{Méthodes proximales}
% -------------------------------

Les méthodes proximales constituent le cadre de référence pour l'optimisation de fonctionnelles comportant des termes non lisses. Soit une fonction convexe propre et semi-continue inférieurement $f : \mathbb{R}^n \rightarrow \mathbb{R}\cup\{+\infty\}$. L'opérateur proximal associé est défini par :
\begin{equation}
    \mathrm{prox}_{\alpha f}(\mathbf{x}) =
    \arg\min_{\mathbf{z}} 
    \left\{
        f(\mathbf{z}) + \frac{1}{2\alpha}\|\mathbf{z} - \mathbf{x}\|_2^2
    \right\}.
\end{equation}

Cet opérateur permet de traiter naturellement des pénalités non différentiables telles que :
\begin{itemize}
    \item la norme $\ell_1$ (seuilage doux),
    \item la variation totale,
    \item les contraintes indicatrices de convexes fermés.
\end{itemize}

Les algorithmes emblématiques incluent :
\begin{itemize}
    \item Forward–Backward Splitting,
    \item FISTA (accéléré de Nesterov),
    \item Primal–Dual de Chambolle–Pock.
\end{itemize}

Ils sont particulièrement adaptés aux problèmes de la forme :
\begin{equation}
    \min_{\mathbf{x}} \; f(\mathbf{x}) + g(\mathbf{x}),
\end{equation}
où $f$ est différentiable à gradient lipschitzien et $g$ est convexe éventuellement non lisse.

% -------------------------------
\subsection{Méthodes ADMM}
% -------------------------------

L'Alternating Direction Method of Multipliers (ADMM) est une méthode de décomposition permettant de résoudre des problèmes séparables en introduisant des variables auxiliaires. On considère typiquement le problème :
\begin{equation}
    \min_{\mathbf{x},\mathbf{z}} \;
    f(\mathbf{x}) + g(\mathbf{z})
    \quad \text{sous la contrainte} \quad
    \mathbf{Kx} = \mathbf{z}.
\end{equation}

Le schéma itératif repose sur :
\begin{enumerate}
    \item minimisation alternée sur $\mathbf{x}$ et $\mathbf{z}$,
    \item mise à jour des multiplicateurs de Lagrange.
\end{enumerate}

Les avantages majeurs d'ADMM sont :
\begin{itemize}
    \item traitement naturel des contraintes linéaires,
    \item parallélisation possible des sous-problèmes,
    \item robustesse pour les grands problèmes mal conditionnés.
\end{itemize}

ADMM est aujourd'hui une référence pour la résolution de problèmes de variation totale, de débruitage parcimonieux et d'apprentissage de dictionnaire.

% -------------------------------
\subsection{Méthodes de gradient projeté}
% -------------------------------

Les méthodes de gradient projeté visent la résolution de problèmes d'optimisation sous contraintes convexes :
\begin{equation}
    \min_{\mathbf{x} \in C} f(\mathbf{x}),
\end{equation}
où $C$ est un ensemble convexe fermé. L'itération générique s'écrit :
\begin{equation}
    \mathbf{x}^{k+1} =
    \Pi_{C}\left(
        \mathbf{x}^k - \alpha_k \nabla f(\mathbf{x}^k)
    \right),
\end{equation}
où $\Pi_{C}$ désigne l'opérateur de projection sur $C$.

Ces méthodes sont notamment utilisées pour :
\begin{itemize}
    \item l'imposition de contraintes de positivité,
    \item la borne supérieure ou inférieure sur des intensités,
    \item les contraintes de norme sur des coefficients parcimonieux.
\end{itemize}

Elles constituent des schémas simples, peu coûteux en mémoire, et très utilisés en traitement d'images et en reconstruction tomographique lorsque la fonction coût est différentiable.

% \subsection{Unrolling des algorithmes (vers deep learning)}

% \section{Synthèse critique}
% \subsection{Avantages et limites}
% \subsection{Cas où le compressive sensing excelle}
% \subsection{Motivation des approches apprises}

    % %===========================================================
% Chapitre 4 — Méthodes basées sur l'apprentissage
%===========================================================
\chapter{Méthodes basées sur l'apprentissage}

\section{Fondements théoriques du deep learning appliqué à la reconstruction}
\subsection{Apprentissage de régularisation}
\subsection{Approches data-driven vs physics-based}
\subsection{Architectures CNN, U-Net, ResNet}

\section{Approches supervisées}
\subsection{Reconstruction directe image-à-image}
\subsection{Modèles guidés par sinogrammes (CT)}
\subsection{Méthodes itératives apprises}
\subsubsection{Networks unrollés}
\subsubsection{ISTA-Net}
\subsubsection{ADMM-Net}

\section{Approches non supervisées et auto-supervisées}
\subsection{Modèles génératifs (GAN, VAE)}
\subsection{Méthodes Noise2Noise, Noise2Void}
\subsection{Auto-supervision pour MRI sous-échantillonnée}

\section{Régularisation neuronale et méthodes hybrides}
\subsection{Deep Image Prior}
\subsection{Plug-and-Play (PnP)}
\subsection{Regularization by Denoising (RED)}
\subsection{Physics-Informed Neural Networks (PINNs)}

\section{Applications}
\subsection{Tomographie (CT)}
\subsubsection{Low-dose CT}
\subsubsection{Sparse-view CT}
\subsection{IRM (MRI)}
\subsubsection{Reconstruction accélérée}
\subsubsection{Super-résolution}
\subsection{OCT, microscopie, hyperspectral}

\section{Analyse expérimentale}
\subsection{Métriques (PSNR, SSIM, NMSE, FID)}
\subsection{Comparaison avec l'état de l'art}
\subsection{Analyse de robustesse}

    % ================================================================

    % ================================================================
    % ADD BIBILIOGRAPHY HERE
    \bibliographystyle{unsrt}
    \begin{thebibliography}{100} % 100 is a random guess of the total number of references
        % \bibitem{1} Wei Zou, Jiajun Wang, David Dagan Feng, \emph{Image reconstruction of fluorescent molecular
        %     tomography based on the tree structured Schur complement decomposition}
        % \bibitem{2}RANDRIANARISON N.Tsirilala, RANDRIAMITANTSOA P. Auguste, RAMAFIARISON H. Malalatiana, \emph{Performance of K-SVD algorithm in digital Optical Coherence Tomography}
        % \bibitem{3} Linwei Fan, Fan Zhang, Hui Fan, and Caiming Zhang, \emph{Brief review of image denoising techniques}
        % \bibitem{4} Michael S. Hansen, and Peter Kellman, \emph{Image reconstruction: an overview from clinicians}
        % \bibitem{5} Joshua Trzasko*, Member, IEEE, and Armando Manduca, Member, IEEE, \emph{Highly Undersampled Magnetic Resonance Image Reconstruction via Homotopic `0-Minimization}
        % \bibitem{6} Helmholtz Imaging, Deutsches Elektronen-Synchrotron DESY, Notkestr. 85, 22607 Hamburg, Germany, \emph{ntroduction to Regularization and Learning Methods for Inverse Problems}
        % \bibitem{7} Bunyamin Yildiz *, Murat Subasi, Ali Sever, \emph{On a regularization problem}
        % \bibitem{8} J. P. Bregman, \emph{Bregman Iteration for Correspondence Problems}
        % \bibitem{15} Gao, Hongxia, Luo, Yinghao, Chen, Kewei, Ma, Ge, and Wu, Lixuan, \emph{An Image Reconstruction Model and Hybrid Algorithm for Limited-angle Projection Data}.
        % \bibitem{16} INGRID DAUBECHIES, A. COHEN \emph{Biorthogonal Bases of Compactly Supported Wavelets} 
        % \bibitem{22} Sarobidy Nomenjanahary Razafitsalama Fin Luc1 , Marie Emile Randrianandrasana2 , Hariony Bienvenu Rakotonirina2, \emph{Image Reconstruction in Compressive Sensing Using Daubechies 7 (db7) and Lifting Wavelet Transforms with SP, CoSaMP, and ALISTA Algorithms}
        \bibitem{17} \emph{COMPARISON BETWEEN ORTHOGONAL AND BI-ORTHOGONAL WAVELETS}
        \bibitem{14}  Engl, Heinz W., and R. Ramlau, \emph{Regularization of inverse problems}. Kluwer Academic Publishers, 2000.
        \bibitem{9} Jen Beatty Colby College, \emph{The Radon Transform and the Mathematics of Medical Imaging}
        \bibitem{10} R. Snieder and J. Trampert. \emph{Linear and nonlinear inverse problems}. Department of Geophysics, Utrecht University, P.O. Box 80.021, 3508 TA Utrecht, The Netherlands. Email: snieder@geo.uu.nl.
        \bibitem{11} I. Bloch. \emph{Reconstruction d’images de tomographie}. LTCI, Télécom Paris, Isabelle.Bloch@telecom-paris.fr.
        \bibitem{12} S. Qin. \emph{Simple algorithm for L1-norm regularisation-based compressed sensing and image restoration}. Third Institute of Physics, Georg-August-Universität Göttingen, Friedrich-Hund-Platz 1, Göttingen, Germany. Email: shun.qin@outlook.com.
        \bibitem{13} M. Piening, F. Altekrüger, J. Hertrich, P. Hagemann, A. Walther and G. Steidl. \emph{Learning from small data sets: Patch-based regularizers in inverse problems for image reconstruction}. (Information manquante pour l’affiliation complète). Note: Les affiliations sont indiquées par les chiffres 1 et 2, mais ne sont pas détaillées dans le texte fourni.
        \bibitem{15} Stephen Boyd , Neal Parikh , Eric Chu Borja Peleato and Jonathan Eckstein \emph{Distributed Optimization and Statistical Learning via the Alternating Direction Method of Multipliers}
        \bibitem{18}CANDITIIS, D.D. and VIDAKOVIC, B. (2004) \emph { Wavelet Bayesian Block Shrinkage via Mixtures of Normal-Inverse Gamma Priors}. Journal of Computational and Graphical Statistics, 13 (2), pp. 383-398
        \bibitem{19} LIU, C.-L. (2010) \emph{A Tutorial of the Wavelet Transform}.
        
        \bibitem{20}MEEN, R.S. and SHARMA, A. (2014) \emph{Comparison and Analysis of Orthogonal and Biorthogonal Wavelets for ECG Compression}. International Journal of Research in Engineering and Technology, 3 (3), pp. 242-247.
        \bibitem{21} WIM SWELDENS \emph{THE LIFTING SCHEME: A CONSTRUCTION OF SECOND GENERATION WAVELETS}

        \bibitem{22} Y. Wang, Z. Liu, and H. Chen, \emph{Accurate Image Quality Assessment in Compressive Sensing: Beyond PSNR and MSE}, IEEE Transactions on Image Processing, vol. 33, pp. 2105--2118, 2024. DOI: 10.1109/TIP.2024.3372109.

      \bibitem{23} A. Gupta and R. Singh, \emph{Efficient Error Metrics for Sparse Signal Recovery in Medical Imaging}, Signal Processing, vol. 212, pp. 109145, 2023. DOI: 10.1016/j.sigpro.2023.109145.

      \bibitem{24} Y. Liu, H. Zhang, and Q. Wang, \emph{High-Fidelity Image Recovery in Compressive Sensing: A PSNR-Driven Optimization Framework}, IEEE Transactions on Multimedia, 2024.

      \bibitem{25} W. Simoes and M. De Sa, \emph{PSNR and SSIM: Evaluation of the Imperceptibility Quality of Images Transmitted over Wireless Networks}, Procedia Computer Science, 2024. DOI: 10.1016/j.procs.2024.11.134.


      \bibitem{26} Z. Wang and A. C. Bovik, \emph{Advances in Structural Similarity Metrics for Image Quality Assessment}, IEEE Transactions on Pattern Analysis and Machine Intelligence, vol. 45, no. 8, pp. 10212--10227, 2023. DOI: 10.1109/TPAMI.2023.3267890.

      \bibitem{27} H. Li, Y. Liu, and J. Zhang, \emph{SSIM-Based Optimization for Compressive Sensing Reconstruction in Medical Imaging}, Medical Image Analysis, vol. 92, pp. 102987, 2024. DOI: 10.1016/j.media.2023.102987.
      \bibitem{28} Jonathan Richard Shewchuk \emph{An Introduction to the Conjugate Gradient Method Without the Agonizing Pain}
      \bibitem{29} Jeremy E. Cohen \emph{Computing the proximal operator of the $\ell_1$ induced matrix norm}
    \end{thebibliography}

    % ================================================================

\end{document}